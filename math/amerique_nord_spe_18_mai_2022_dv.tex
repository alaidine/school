\documentclass[10pt]{article}
\usepackage[T1]{fontenc}
\usepackage[utf8]{inputenc}
\usepackage{fourier}
\usepackage[scaled=0.875]{helvet}
\renewcommand{\ttdefault}{lmtt}
\usepackage{makeidx}
\usepackage{amsmath,amssymb}
\usepackage{fancybox}
\usepackage[normalem]{ulem}
\usepackage{pifont}
\usepackage{lscape}
\usepackage{multicol}
\usepackage{mathrsfs}
\usepackage{tabularx}
\usepackage{multirow}
\usepackage{enumitem}
\usepackage{textcomp} 
\newcommand{\euro}{\eurologo{}}
%Tapuscrit : Denis Vergès
\usepackage{pst-plot,pst-tree,pstricks,pst-node,pst-text}
\usepackage{pst-eucl}
\usepackage{pstricks-add}
\newcommand{\R}{\mathbb{R}}
\newcommand{\N}{\mathbb{N}}
\newcommand{\D}{\mathbb{D}}
\newcommand{\Z}{\mathbb{Z}}
\newcommand{\Q}{\mathbb{Q}}
\newcommand{\C}{\mathbb{C}}
\usepackage[left=3.5cm, right=3.5cm, top=3cm, bottom=3cm]{geometry}
\newcommand{\vect}[1]{\overrightarrow{\,\mathstrut#1\,}}
\renewcommand{\theenumi}{\textbf{\arabic{enumi}}}
\renewcommand{\labelenumi}{\textbf{\theenumi.}}
\renewcommand{\theenumii}{\textbf{\alph{enumii}}}
\renewcommand{\labelenumii}{\textbf{\theenumii.}}
\def\Oij{$\left(\text{O}~;~\vect{\imath},~\vect{\jmath}\right)$}
\def\Oijk{$\left(\text{O}~;~\vect{\imath},~\vect{\jmath},~\vect{k}\right)$}
\def\Ouv{$\left(\text{O}~;~\vect{u},~\vect{v}\right)$}
\usepackage{fancyhdr}
\usepackage[dvips]{hyperref}
\hypersetup{%
pdfauthor = {APMEP},
pdfsubject = {Baccalauréat S},
pdftitle = {Amérique du Nord mai 2021},
allbordercolors = white,
pdfstartview=FitH} 
\usepackage[frenchb]{babel}
\usepackage[np]{numprint}
\begin{document}
\setlength\parindent{0mm}
\rhead{\textbf{A. P{}. M. E. P{}.}}
\lhead{\small Baccalauréat S}
\lfoot{\small{Amérique du Nord}}
\rfoot{\small{mai 2021}}
\pagestyle{fancy}
\thispagestyle{empty}

\begin{center}{\Large\textbf{\decofourleft~Baccalauréat Amérique du Nord Jour 1 18 mai 2021~\decofourright\\[6pt] ÉPREUVE D'ENSEIGNEMENT DE SPÉCIALITÉ}}
\end{center}

\vspace{0,25cm}

Le sujet propose 4 exercices

Le candidat choisit 3 exercices parmi les 4 exercices et \textbf{ne doit traiter que ces 3 exercices}

Chaque exercice est noté sur \textbf{7 points (le total sera ramené sur 20 points)}.

Les traces de recherche, même incomplètes ou infructueuses, seront prises en compte.

\bigskip

\textbf{\textsc{Exercice 1} \quad (7 points)\hfill Thème : probabilités}

\medskip

Chaque chaque jour où il travaille, Paul doit se rendre à la gare pour rejoindre son lieu de travail en train. Pour cela, il prend son vélo deux fois sur trois et, si il ne prend pas son vélo, il prend sa voiture.

\medskip

\begin{enumerate}
\item lorsqu'il prend son vélo pour rejoindre la gare, Paul ne rate le train qu'une fois sur 50 alors que, lorsqu'il prend sa voiture pour rejoindre la gare Paul rate son train une fois sur 10.

On considère une journée au hasard lors de laquelle Paul sera à la gare pour prendre le train qui le conduira au travail.

On note:

\setlength\parindent{8mm}
\begin{itemize}
\item[$\bullet~~$] $V$ l'évènement \og Paul prend son vélo pour rejoindre la gare \fg{} ; 
\item[$\bullet~~$] $R$ l'évènement \og Paul rate son train \fg.
\end{itemize}
\setlength\parindent{0mm}

	\begin{enumerate}
		\item Faire un arbre pondéré résumant la situation.
		\item Montrer que la probabilité que Paul rate son train est égale à $\dfrac{7}{150}$.
		\item Paul a raté son train. Déterminer la valeur exacte de la probabilité qu'il ait pris son vélo pour rejoindre la gare.
	\end{enumerate}
\item On choisit au hasard un mois pendant lequel Paul s'est rendu $20$ jours à la gare pour rejoindre son lieu de travail selon les modalités décrites en préambule. 

On suppose que, pour chacun de ces 20 jours, le choix entre le vélo et
la voiture est indépendant des choix des autres jours.

On note $X$ la variable aléatoire donnant le nombre de jours où Paul prend son vélo sur ces $20$ jours.
	\begin{enumerate}
		\item Déterminer la loi suivie par la variable aléatoire $X$. Préciser ses paramètres.
		\item Quelle est la probabilité que Paul prenne son vélo exactement $10$ jours sur ces $20$ jours pour se rendre à la gare ? On arrondira la probabilité cherchée à $10^{-3}$.
		\item Quelle est la probabilité que Paul prenne son vélo au moins $10$ jours sur ces 20 jours pour se rendre à la gare ?
On arrondira la probabilité cherchée à $10^{-3}$.
		\item En moyenne, combien de jours sur une période choisie au hasard de 20 jours pour se rendre à la gare, Paul prend-il son vélo ? On arrondira la réponse à l'entier.
	\end{enumerate}
\item Dans le cas où Paul se rend à la gare en voiture, on note $T$ la variable aléatoire donnant le temps de trajet nécessaire pour se rendre à la gare. La durée du trajet est donnée en minutes, arrondie à la minute. La loi de probabilité de $T$ est donnée par le tableau ci-dessous:

\begin{center}
\begin{tabularx}{\linewidth}{|m{2.3cm}|*{9}{>{\centering \arraybackslash}X|}}\hline
$k$ (en minutes)&10 &11&12 &13 &14 &15 &16 &17 &18\\ \hline
$P(T = k)$&0,14&0,13 &0,13&0,12 &0,12&0,11 &0,10 &0,08&0,07\\ \hline
\end{tabularx}
\end{center}

Déterminer l'espérance de la variable aléatoire $T$ et interpréter cette valeur dans le contexte de l'exercice.
\end{enumerate}

\bigskip

\textbf{\textsc{Exercice 2} \quad (7 points)\hfill Thème : suites}

\medskip

Dans cet exercice, on considère la suite $\left(T_n\right)$ définie par :

\[T_0 = 180 \:\text{et, \: pour tout entier naturel} \:n, \: T_{n+1} = 0,955T_n + 0,9\]

\begin{enumerate}
\item 
	\begin{enumerate}
		\item Démontrer par récurrence que, pour tout entier naturel $n$,\: $T_n \geqslant 20$.
		\item Vérifier que pour tout entier naturel $n$,\: $T_{n+1} -  T_n  = - 0,045\left(T_n - 20\right)$. En déduire le sens de variation de la suite $\left(T_n\right)$.
		\item Conclure de ce qui précède que la suite $\left(T_n\right)$ est convergente. Justifier.
	\end{enumerate}	
\item Pour tout entier naturel $n$, on pose : $u_n =  T_n - 20$.
	\begin{enumerate}
		\item Montrer que la suite $\left(u_n\right)$ est une suite géométrique dont on précisera la raison.
		\item En déduire que pour tout entier naturel $n$, \:$T_n =  20 + 160 \times  0,955^n$.
		\item Calculer la limite de la suite $\left(T_n\right)$.
		\item Résoudre l'inéquation $T_n \leqslant 120$ d'inconnue $n$ entier naturel.
	\end{enumerate}	
\item Dans cette partie, on s'intéresse à l'évolution de la température au centre d'un gâteau après sa sortie du four. 

On considère qu'à la sortie du four, la température au centre du gâteau est de $180 \degres$ C et celle de l'air ambiant de $20 \degres$ C.

La loi de refroidissement de Newton permet de modéliser la température au centre du gâteau par la suite précédente $\left(T_n\right)$. Plus précisément, $T_n$ représente la température au centre du gâteau, exprimée en degré Celsius, n minutes après sa sortie du four.
	\begin{enumerate}
		\item Expliquer pourquoi la limite de la suite $\left(T_n\right)$ déterminée à la question 2. c. était prévisible dans le contexte de l'exercice.
		\item On considère la fonction Python ci-dessous:
		\begin{center}
		\begin{tabular}{|l|}\hline
def temp(x) :\\
\quad T = 180\\
\quad n = 0\\
\quad while T $>$ x :\\
\qquad T=0.955*T+0.9\\
\qquad n=n+1\\
\quad return n\\ \hline
\end{tabular}
\end{center}
Donner le résultat obtenu en. exécutant la commande temp(120).

Interpréter le résultat dans le contexte de l'exercice.
	\end{enumerate}
\end{enumerate}

\bigskip

\textbf{\textsc{Exercice 3} \quad (7 points)\hfill Thème : géométrie dans l'espace}

\medskip

Dans l'espace muni d'un repère orthonormé \Oijk{} d'unité 1 cm, on considère les points suivants :

\[\text{J}(2~;~0~;~1), \quad \text{K}( 1~;~2~;~1)\:\text{et} \quad \text{L}(-2~;~-2~;~-2)\]

\begin{enumerate}
\item 
	\begin{enumerate}
		\item Montrer que le triangle JKL est rectangle en J.
		\item Calculer la valeur exacte de l'aire du triangle JKL en cm$^2$.
		\item Déterminer une valeur approchée au dixième près de l'angle géométrique $\widehat{\text{JKL}}$.
	\end{enumerate}
		
\item
	\begin{enumerate}
		\item Démontrer que le vecteur $\vect{n}$ de coordonnées $\begin{pmatrix}6\\3\\-10\end{pmatrix}$ est un vecteur normal au plan (JKL).
		\item En déduire une équation cartésienne du plan (JKL).
	\end{enumerate}
\end{enumerate}

Dans la suite, T désigne le point de coordonnées $(10~;~9~;~-6)$.

\begin{enumerate}[resume]
\item
	\begin{enumerate}
		\item Déterminer une représentation paramétrique de la droite $\Delta$
orthogonale au plan (JKL) et passant par T.
		\item Déterminer les coordonnées du point H, projeté orthogonal du point T sur le plan (JKL).
		\item On rappelle que le volume $V$ d'un tétraèdre est donné par la formule :
		
		\[V  = \dfrac13 \mathcal{B} \times h\:\: \text{où }  \mathcal{B}\:\text{désigne l'aire d'une base et } \: h \: \text{la hauteur correspondante}\]
		
Calculer la valeur exacte du volume du tétraèdre JKLT en cm$^3$.
	\end{enumerate}
\end{enumerate}

\bigskip

\textbf{\textsc{Exercice 4} \quad (7 points)\hfill Thème : fonction exponentielle}

\medskip

Pour chacune des affirmations suivantes, indiquer si elle est vraie ou fausse. Justifier chaque réponse.

\medskip

\begin{enumerate}
\item \textbf{Affirmation 1} : Pour tout réel $x$  : $1 - \dfrac{1 - \text{e}^x}{1 + \text{e}^x} = \dfrac{2}{1 + \text{e}^{-x}}$.
\item On considère la fonction$g$ définie sur $\R$ par $g(x) = \dfrac{\text{e}^x}{\text{e}^x + 1}$.

\textbf{Affirmation 2 : } L'équation $g(x) = \dfrac12$ admet une unique solution dans $\R$.
\item On considère la fonction $f$ définie sur $\R$ par $f(x) = x^2\text{e}^{-x}$ et on note 
$\mathcal{C}$ sa courbe dans un repère orthonormé.

\textbf{Affirmation 3 : } L'axe des abscisses est tangent à la courbe $\mathcal{C}$ en un seul point.
\item On considère la fonction $h$ définie sur $\R$ par $h(x) = \text{e}^x\left(1 -  x^2\right)$.

\textbf{Affirmation 4 :} Dans le plan muni d'un repère orthonormé, la courbe représentative de la fonction $h$ n'admet pas de point d'inflexion.

\item \textbf{Affirmation 5 :} $\displaystyle\lim_{x \to + \infty} \dfrac{\text{e}^x}{\text{e}^x + x} = 0$.

\item \textbf{Affirmation 6 :} Pour tout réel $x,\: 1 + \text{e}^{2x} \geqslant 2\text{e}^x$.
\end{enumerate}
\end{document}
