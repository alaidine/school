\documentclass[11pt,a4paper,french]{article}
\usepackage[T1]{fontenc}
\usepackage[utf8]{inputenc}
\usepackage{fourier}
\usepackage[scaled=0.875]{helvet}
\renewcommand{\ttdefault}{lmtt}
\usepackage{makeidx}
\usepackage{amsmath,amssymb}
\usepackage{fancybox}
\usepackage[normalem]{ulem}
\usepackage{pifont}
\usepackage{lscape}
\usepackage{multicol}
\usepackage{mathrsfs}
\usepackage{tabularx}
\usepackage{multirow}
\usepackage{enumitem}
\usepackage{textcomp} 
\newcommand{\euro}{\eurologo{}}
%Tapuscrit : Denis Vergès
\usepackage{pst-plot,pst-tree,pstricks,pst-node,pst-text}
\usepackage{pst-eucl,pst-3dplot,pstricks-add}
\usepackage{esvect}
\newcommand{\R}{\mathbb{R}}
\newcommand{\N}{\mathbb{N}}
\newcommand{\D}{\mathbb{D}}
\newcommand{\Z}{\mathbb{Z}}
\newcommand{\Q}{\mathbb{Q}}
\newcommand{\C}{\mathbb{C}}
\usepackage[left=2.5cm, right=2.5cm, top=2cm, bottom=3cm]{geometry}
\headheight15 mm
\newcommand{\vect}[1]{\overrightarrow{\,\mathstrut#1\,}}
\renewcommand{\theenumi}{\textbf{\arabic{enumi}}}
\renewcommand{\labelenumi}{\textbf{\theenumi.}}
\renewcommand{\theenumii}{\textbf{\alph{enumii}}}
\renewcommand{\labelenumii}{\textbf{\theenumii.}}
\def\Oij{$\left(\text{O}~;~\vect{\imath},~\vect{\jmath}\right)$}
\def\Oijk{$\left(\text{O}~;~\vect{\imath},~\vect{\jmath},~\vect{k}\right)$}
\def\Ouv{$\left(\text{O}~;~\vect{u},~\vect{v}\right)$}
\newcommand{\e}{\text{e}}
\usepackage{fancyhdr}
\usepackage[dvips]{hyperref}
\hypersetup{%
pdfauthor = {APMEP},
pdfsubject = {Baccalauréat Spécialité},
pdftitle = {Centres étrangers groupe 1 Sujet 1 18  mai 2022},
allbordercolors = white,
pdfstartview=FitH} 
\usepackage{babel}
\usepackage[np]{numprint}
\renewcommand\arraystretch{1.3}
\frenchsetup{StandardLists=true}
\begin{document}
\setlength\parindent{0mm}
\rhead{\textbf{A. P{}. M. E. P{}.}}
\lhead{\small Baccalauréat spécialité sujet 1}
\lfoot{\small{Centres étrangers}}
\rfoot{\small{18 mai 2022}}
\pagestyle{fancy}
\thispagestyle{empty}

\begin{center}{\Large\textbf{\decofourleft~Baccalauréat Centres étrangers Groupe 1  (D)\footnote{Afrique du Sud, Bulgarie, Comores, Djibouti, Kenya, Liban, Lituanie, Madagascar, Mozambique et Ukraine} 18 mai 2022~\decofourright\\[7pt]  Sujet 1\\[7pt] ÉPREUVE D'ENSEIGNEMENT DE SPÉCIALITÉ}}
\end{center}

\vspace{0,25cm}

Le sujet propose 4 exercices

Le candidat choisit 3 exercices parmi les 4 exercices et \textbf{ne doit traiter que ces 3 exercices}

Chaque exercice est noté sur 6 points (le total sera ramené sur 20 points)\footnote{Chaque exercice est noté sur 7 points au Liban.}

La clarté et la précision de l'argumentation ainsi que la qualité de la rédaction sont notées sur 2 points.

Les traces de recherche, même incomplètes ou infructueuses, seront prises en compte.

\bigskip

\textbf{\textsc{Exercice 1} \quad 6 points\hfill }

\medskip

\begin{tabularx}{\linewidth}{|X|}\hline
\textbf{Principaux domaines abordés :} Probabilités\\ \hline
\end{tabularx}

\medskip

Dans une station de ski, il existe deux types de forfait selon l'âge du skieur :

\begin{itemize}
\item un forfait JUNIOR pour les personnes de moins de 25 ans ;
\item un forfait SENIOR pour les autres.
\end{itemize}

 Par ailleurs, un usager peut choisir, en plus du forfait correspondant à son âge l'\emph{option  coupe-file} qui permet d'écourter le temps d'attente aux remontées mécaniques.

On admet que:
\begin{itemize}
\item[$\bullet~~$] 20\,\% des skieurs ont un forfait JUNIOR ;
\item[$\bullet~~$] 80\,\% des skieurs ont un forfait SENIOR ;
\item[$\bullet~~$] parmi les skieurs ayant un forfait JUNIOR, 6\,\% choisissent l'option coupe-file ;
\item[$\bullet~~$] parmi les skieurs ayant un forfait SENIOR, 12,5\,\% choisissent l'option coupe-file.
\end{itemize}

On interroge un skieur au hasard et on considère les évènements :

\begin{itemize}
\item[$\bullet~~$] $J$ : \og le skieur a un forfait JUNIOR \fg ;
\item[$\bullet~~$] $C$ : \og le skieur choisit l'option coupe-file \fg.
\end{itemize}

\medskip

\emph{Les deux parties peuvent être traitées de manière indépendante}

\bigskip

\textbf{Partie A}

\medskip

\begin{enumerate}
\item Traduire la situation par un arbre pondéré.
\item Calculer la probabilité $P(J \cap C)$.
\item Démontrer que la probabilité que le skieur choisisse l'option coupe-file est égale à $0,112$.
\item Le skieur a choisi l'option coupe-file. Quelle est la probabilité qu'il s'agisse d'un
skieur ayant un forfait SENIOR ? Arrondir le résultat à $10^{-3}$.
\item Est-il vrai que les personnes de moins de vingt-cinq ans représentent moins de 15\,\% des skieurs ayant choisi l'option coupe-file ? Expliquer.
\end{enumerate}

\bigskip

\textbf{Partie B}

\medskip

On rappelle que la probabilité qu'un skieur choisisse l'option coupe-file est égale à 0,112.

On considère un échantillon de $30$ skieurs choisis au hasard.

Soit $X$ la variable aléatoire qui compte le nombre des skieurs de l'échantillon ayant
choisi t'option coupe-file.

\medskip

\begin{enumerate}
\item On admet que la variable aléatoire $X$ suit une loi binomiale. 

Donner les paramètres de cette loi.
\item Calculer la probabilité qu'au moins un des $30$ skieurs ait choisi l'option coupe-file. Arrondir le résultat à $10^{-3}$.
\item Calculer la probabilité qu'au plus un des $30$ skieurs ait choisi l'option coupe-file. Arrondir le résultat à $10^{-3}$.
\item Calculer l'espérance mathématique de la variable aléatoire $X$.
\end{enumerate}

\bigskip

\textbf{\textsc{Exercice 2} \quad 6 points\hfill Thème: Fonction exponentielle }

\medskip

\begin{tabularx}{\linewidth}{|X|}\hline
\textbf{Principaux domaines abordés:} Suites ; Fonctions, Fonction logarithme.\\ \hline
\end{tabularx}

\medskip

\emph{Cet exercice est un questionnaire à choix multiples.\\
Pour chacune des questions suivantes, une seule des quatre réponses proposées
est exacte.\\
Une réponse exacte rapporte un point. Une réponse fausse, une réponse multiple ou
l'absence de réponse à une question ne rapporte ni n'enlève de point.\\
Pour répondre, indiquer sur la copie le numéro de la question et la lettre de la réponse choisie. Aucune justification n'est demandée.}

\bigskip
\begin{enumerate}
\item Un récipient contenant initialement 1 litre d'eau est laissé au soleil.

Toutes les heures, le volume d'eau diminue de 15\,\%.

Au bout de quel nombre entier d'heures le volume d'eau devient-il inférieur à un quart de litre ? 

\begin{center}
\begin{tabularx}{\linewidth}{*{4}{X}}
\textbf{a.~~} 2 heures &\textbf{b.~~}  8 heures .&\textbf{c.~~}  9 heures
&\textbf{d.~~}  13 heures 
\end{tabularx}
\end{center}

\item  On considère la fonction $f$ définie sur l'intervalle $]0~;~+\infty[$ par $f(x) = 4\ln (3x)$.

Pour tout réel $x$ de l'intervalle $]0~;~+\infty[$ , on a :

\begin{center}
\begin{tabularx}{\linewidth}{*{2}{X}}
\textbf{a.~~} $f(2x) = f(x) + \ln (24) - \ln \left(\frac32\right)$&\textbf{b.~~}  $f(2x) = f(x) + \ln (16)$\\
\textbf{c.~~} $f(2x) = \ln (2) + f(x)$& \textbf{d.~~} $f(2x) = 2f(x)$
\end{tabularx}
\end{center}
\item  On considère la fonction $g$ définie sur l'intervalle $]1~;~+\infty[$ par : 

\[g(x) = \dfrac{\ln (x)}{x - 1}.\]


On note $\mathcal{C}_g$ la courbe représentative de la fonction $g$ dans un repère orthogonal. La courbe $\mathcal{C}_g$ admet :

\begin{center}
\begin{tabularx}{\linewidth}{*{2}{X}}
\textbf{a.~~} une asymptote verticale
et une asymptote horizontale.&\textbf{b.~~} une asymptote verticale
et aucune asymptote horizontale.\\
\textbf{c.~~} aucune asymptote verticale et une asymptote horizontale.&\textbf{d.~~} 
aucune asymptote verticale .et aucune asymptote horizontale.
\end{tabularx}
\end{center}
\end{enumerate}

Dans la suite de l'exercice, on considère la fonction $h$ définie sur l'intervalle ]0~;~2] par:

\[h(x) = x^2(1 + 2\ln (x)).\]

On note $\mathcal{C}_h$ la courbe représentative de $h$ dans un repère du plan. 

On admet que $h$ est deux fois dérivable sur l'intervalle ]0~;~2].

On note $h'$ sa dérivée et $h''$ sa dérivée seconde.

On admet que, pour tout réel $x$ de l'intervalle ]0~;~2], on a : 

\[h'(x) = 4x(1 + \ln (x)).\]

\begin{enumerate}[resume]
\item Sur l'intervalle $\left]\dfrac{1}{\text{e}}~;~2\right]$, la fonction $h$ s'annule :

\begin{center}
\begin{tabularx}{\linewidth}{*{2}{X}}
\textbf{a.~~} exactement 0 fois. &\textbf{b.~~}  exactement 1 fois.\\
 \textbf{c.~~} exactement 2 fois. &\textbf{d.~~}  exactement 3 fois.
\end{tabularx}
\end{center}

\item Une équation de la tangente à $\mathcal{C}_h$ au point d'abscisse $\sqrt{\text{e}}$ est:

\begin{center}
\begin{tabularx}{\linewidth}{*{2}{X}}
\textbf{a.~~} $y = \left(6\text{e}^{\frac12}\right) .{} x $&
\textbf{b.~~} $y = \left(6\sqrt{\text{e}}\right).{} x + 2\text{e}$\\
\textbf{c.~~} $y = 6\text{e}^{\frac{x}{2}}$&\textbf{d.~~} $y = \left(6\text{e}^{\frac12}\right) .{} x - 4\text{e}$.
\end{tabularx}
\end{center}

\item Sur l'intervalle ]0~;~2], le nombre de points d'inflexion de la courbe $\mathcal{C}_h$ est égal à :

\begin{center}
\begin{tabularx}{\linewidth}{*{4}{X}}
\textbf{a.~~} 0&\textbf{b.~~} 1 &\textbf{c.~~}  2 &\textbf{d.~~}  3
\end{tabularx}
\end{center}

\item \footnote{Uniquement au Liban} On considère la suite $\left(u_n\right)$ définie pour tout entier naturel $n$ par 

\[u_{n+1} = \dfrac12u_n + 3\quad \text{et}\quad u_0 = 6.\]

On peut affirmer que :

\begin{center}
\begin{tabularx}{\linewidth}{*{2}{X}}
\textbf{a.~~} la suite $\left(u_n\right)$ est strictement croissante.&\textbf{b.~~} la suite $\left(u_n\right)$ est strictement décroissante.\\
\textbf{c.~~} la suite $\left(u_n\right)$ n'est pas monotone. &\textbf{d.~~} la suite $\left(u_n\right)$ est constante.
\end{tabularx}
\end{center}
\end{enumerate}

\bigskip

\textbf{\textsc{Exercice 3} \quad 6 points\hfill Thème: Fonction exponentielle }

\medskip

\begin{tabularx}{\linewidth}{|X|}\hline
\textbf{Principaux domaines abordés :} Suites;
Fonctions, Fonction exponentielle.\\ \hline
\end{tabularx}

\bigskip

\textbf{Partie A}

\medskip

On considère la fonction $f$ définie pour tout réel $x$ par:

\[ f(x) = 1+x - \text{e}^{0,5x - 2}.\]

On admet que la fonction $f$ est dérivable sur $\R$. On note $f'$ sa dérivée.

\medskip

\begin{enumerate}
\item 
	\begin{enumerate}
		\item Déterminer la limite de la fonction $f$ en $- \infty$.
		\item Démontrer que, pour tout réel $x$ non nul, $f(x) = 1+ 0,5x\left(2 - \dfrac{\text{e}^{0,5x}}{0,5x} \times \text{e}^{-2}\right)$.
		
En déduire la limite de la fonction $f$ en $+\infty$. 
	\end{enumerate}
\item 
	\begin{enumerate}
		\item Déterminer $f'(x)$ pour tout réel $x$.
		\item Démontrer que l'ensemble des solutions de l'inéquation $f'(x) < 0$ est l'intervalle 
		
		$]4 + 2\ln (2)~;~+\infty[$.
	\end{enumerate}	
\item Déduire des questions précédentes le tableau de variation de la fonction $f$ sur $\R$. On fera figurer la valeur exacte de l'image de $4 + 2\ln (2)$ par $f$.
\item Montrer que l'équation $f(x) = 0$ admet une unique solution sur l'intervalle $[-1~;~0]$.
\end{enumerate}

\bigskip

\textbf{Partie B}

\medskip

On considère la suite $\left(u_n\right)$ définie par $u_0 = 0$ et, pour tout entier naturel $n$,
\: $u_{n+1} = f\left(u_n\right)$ où $f$ est la fonction définie à la partie A.

\medskip

\begin{enumerate}
\item 
	\begin{enumerate}
		\item Démontrer par récurrence que, pour tout entier naturel $n$, on a : 
		
		\[u_n \leqslant u_{n+1} \leqslant 4.\]
		
		\item En déduire que la suite $\left(u_n\right)$ converge. On notera $\ell$ la limite.
	\end{enumerate}	
\item 
	\begin{enumerate}
		\item On rappelle que $f$ vérifie la relation $\ell = f(\ell)$.
		
Démontrer que $\ell  = 4$.
		\item ~
		
\begin{minipage}{0.48\linewidth}On considère la fonction \texttt{valeur} écrite ci-contre dans le langage Python :
		\end{minipage}\hfill 
\begin{minipage}{0.48\linewidth}
\begin{tabular}{|l|}\hline
\texttt{def valeur (a) :}\\
\quad \texttt{u = 0}\\
\quad \texttt{n = 0}\\
\quad \texttt{while u $\leqslant$ a:}\\
\qquad \texttt{u=1 + u - exp(0.5*u - 2)}\\
\qquad \texttt{n = n+1}\\ 
\quad return \texttt{n}\\ \hline
\end{tabular}
\end{minipage}

L'instruction \texttt{valeur}(3.99) renvoie la valeur 12.

Interpréter ce résultat dans le contexte de l'exercice.
	\end{enumerate}
\end{enumerate}

\bigskip

\textbf{\textsc{Exercice 4} \quad 6 points\hfill Thème: Fonction exponentielle }

\medskip

\begin{tabularx}{\linewidth}{|X|}\hline
\textbf{Principaux domaines abordés :} Géométrie dans l'espace\\ \hline
\end{tabularx}

\medskip

L'espace est muni d'un repère orthonormé \Oijk.

On considère les points A$(5~;~0~;~-1)$, B$(1~;~4~;~-1)$, C(1~;~0~;~3), D(5~;~4~;~3) et E(10~;~9~;~8).

\medskip

\begin{enumerate}
\item 
	\begin{enumerate}
		\item Soit R le milieu du segment [AB].
		
Calculer les coordonnées du point R ainsi que les coordonnées du vecteur $\vect{\text{AB}}$.
		\item %Soit $\mathcal{P}_1$ le plan passant par le point R et dont $\vect{\text{AB}}$ est un vecteur normal. Démontrer qu'une équation cartésienne du plan $\mathcal{P}_1$ est:

%\[x - y - 1 = 0.\]




		\item Démontrer que le point E appartient au plan $\mathcal{P}_1$ et que EA = EB .
	\end{enumerate}	
\item  On considère le plan $\mathcal{P}_2$ d'équation cartésienne $x - z - 2 = 0$. 
	\begin{enumerate}
		\item Justifier que les plans $\mathcal{P}_1$ et $\mathcal{P}_2$ sont sécants.
		\item On note $\Delta$ la droite d'intersection de $\mathcal{P}_1$ et $\mathcal{P}_2$.
		
Démontrer qu'une représentation paramétrique de la droite $\Delta$ est :

\[\left\{\begin{array}{l c r}
x&=&2+t\\
y&=&1+t \\
z&=&t\end{array}\right.(t \in \R).\]

	\end{enumerate}
\item  On considère le plan $\mathcal{P}_3$ d'équation cartésienne $y + z - 3 = 0$.

Justifier que la droite $\Delta$ est sécante au plan $\mathcal{P}_3$ en un point $\Omega$ dont on déterminera les coordonnées.
\end{enumerate}

\medskip

Si S et T sont deux points distincts de l'espace, on rappelle que l'ensemble des points M de l'espace tels que MS = MT est un plan, appelé plan médiateur du segment [ST]. On admet que les plans $\mathcal{P}_1$,\: $\mathcal{P}_2$ et $\mathcal{P}_3$ sont les plans médiateurs respectifs des
segments [AB], [AC] et [AD].

\begin{enumerate}[resume]
\item
	\begin{enumerate}
		\item Justifier que $\Omega \text{A} = \Omega \text{B} = \Omega \text{C} = \Omega \text{D}$.
		\item En déduire que les points A, B, C et D appartiennent à une même sphère dont
on précisera le centre et le rayon.
	\end{enumerate}
\end{enumerate}
\end{document}