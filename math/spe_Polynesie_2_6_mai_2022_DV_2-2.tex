\documentclass[11pt]{article}
\usepackage[T1]{fontenc}
\usepackage[utf8]{inputenc}
\usepackage{fourier}
\usepackage[scaled=0.875]{helvet}
\renewcommand{\ttdefault}{lmtt}
\usepackage{makeidx}
\usepackage{amsmath,amssymb}
\usepackage{fancybox}
\usepackage[normalem]{ulem}
\usepackage{pifont}
\usepackage{lscape}
\usepackage{multicol}
\usepackage{mathrsfs}
\usepackage{tabularx}
\usepackage{multirow}
\usepackage{enumitem}
\usepackage{textcomp} 
\newcommand{\euro}{\eurologo{}}
%Tapuscrit : Denis Vergès
%Relecture : François Hache
\usepackage{pst-plot,pst-tree,pstricks,pst-node,pst-text}
\usepackage{pst-eucl}
\usepackage{pstricks-add}
\newcommand{\R}{\mathbb{R}}
\newcommand{\N}{\mathbb{N}}
\newcommand{\D}{\mathbb{D}}
\newcommand{\Z}{\mathbb{Z}}
\newcommand{\Q}{\mathbb{Q}}
\newcommand{\C}{\mathbb{C}}
\usepackage[left=3.5cm, right=3.5cm, top=3cm, bottom=3cm]{geometry}
\newcommand{\vect}[1]{\overrightarrow{\,\mathstrut#1\,}}
\renewcommand{\theenumi}{\textbf{\arabic{enumi}}}
\renewcommand{\labelenumi}{\textbf{\theenumi.}}
\renewcommand{\theenumii}{\textbf{\alph{enumii}}}
\renewcommand{\labelenumii}{\textbf{\theenumii.}}
\def\Oij{$\left(\text{O}~;~\vect{\imath},~\vect{\jmath}\right)$}
\def\Oijk{$\left(\text{O}~;~\vect{\imath},~\vect{\jmath},~\vect{k}\right)$}
\def\Ouv{$\left(\text{O}~;~\vect{u},~\vect{v}\right)$}
\usepackage{fancyhdr}
\usepackage[dvips]{hyperref}
\hypersetup{%
pdfauthor = {APMEP},
pdfsubject = {Baccalauréat spécialité},
pdftitle = {Polynésie 6 mai 2022 sujet 1},
allbordercolors = white,
pdfstartview=FitH} 
\usepackage[french]{babel}
\DecimalMathComma
\usepackage[np]{numprint}
\begin{document}
\setlength\parindent{0mm}
\rhead{\textbf{A. P{}. M. E. P{}.}}
\lhead{\small Baccalauréat spécialité}
\lfoot{\small{Polynésie}}
\rfoot{\small{5 mai 2022}}
\pagestyle{fancy}
\thispagestyle{empty}

\begin{center}{\Large\textbf{\decofourleft~Baccalauréat Polynésie 6 mai 2022~\decofourright\\[6pt] ÉPREUVE D'ENSEIGNEMENT DE SPÉCIALITÉ sujet \no 2}}

\bigskip

Durée de l'épreuve : \textbf{4 heures}

\medskip

L'usage de la calculatrice avec mode examen actif est autorisé

\medskip

Le sujet propose 4 exercices

Le candidat choisit 3 exercices parmi les 4 et \textbf{ne doit traiter que ces 3 exercices}
\end{center}

\bigskip

\textbf{\textsc{Exercice 1} \quad 7 points\hfill Thèmes : fonctions, primitives, probabilités}

\medskip

\emph{Cet exercice est un questionnaire à choix multiples. \\
Pour chacune des six questions suivantes, une seule des quatre réponses proposées est exacte.\\
Une réponse fausse, une réponse multiple ou l'absence de réponse à une question ne rapporte ni n'enlève de point.\\
Pour répondre, indiquer sur la copie le numéro de la question et la lettre de la réponse choisie.\\
Aucune justification n'est demandée.}

\medskip

\begin{enumerate}
\item On considère la fonction $f$ définie et dérivable sur $]0~;~+\infty[$ par:
\[f(x) = x \ln(x) - x + 1.\]

Parmi les quatre expressions suivantes, laquelle est celle de la fonction dérivée de $f$?

\begin{center}
\begin{tabularx}{\linewidth}{|*{4}{X|}}\hline
\textbf{a.~~} $\ln (x)$&\textbf{b.~~}$\dfrac{1}{x} - 1$&\textbf{c.~~} $\ln (x) - 2$&\textbf{d.~~}$\ln (x) - 1$\rule[-3mm]{0mm}{9mm}\\ \hline
\end{tabularx}
\end{center}

\item On considère la fonction $g$ définie sur $]0~;~+\infty[$ par $g(x) = x^2[1 - \ln (x)]$.

 Parmi les quatre affirmations suivantes, laquelle est correcte ?
 
\begin{center}
\begin{tabularx}{\linewidth}{|*{4}{X|}}\hline
\textbf{a.~~}$\displaystyle\lim_{x \to 0} g(x) = +\infty$&\textbf{b.~~} 
$\displaystyle\lim_{x \to 0} g(x) = - \infty$&\textbf{c.~~} $\displaystyle\lim_{x \to 0} g(x) = 0$&\textbf{d.~~} La fonction $g$ n'admet pas de limite en 0.\\ \hline
\end{tabularx}
\end{center}

\item On considère la fonction $f$ définie sur $\R$ par $f(x) = x^3 - 0,9x^2 -0,1x$. Le nombre de solutions de l'équation $f(x) = 0$ sur $\R$ est :

\begin{center}
\begin{tabularx}{\linewidth}{|*{4}{X|}}\hline
\textbf{a.~~} $0$&\textbf{b.~~}$1$&\textbf{c.~~} $2$&\textbf{d.~~}$3$\\ \hline
\end{tabularx}
\end{center}

\item Si $H$ est une primitive d'une fonction $h$ définie et continue sur $\R$, 
et si $k$ est la fonction définie sur $\R$ par $k(x) = h(2x)$, 
alors, une primitive $K$ de $k$ est définie sur $\R$ par :

\begin{center}
\begin{tabularx}{\linewidth}{|*{4}{X|}}\hline
\textbf{a.~~} $K(x) =H(2x)$&\textbf{b.~~}$K(x) =2H(2x)$&\textbf{c.~~} $ K(x) =\dfrac{1}{2}H(2x)$&\textbf{d.~~}$K(x) =2H(x)$\rule[-3mm]{0mm}{9mm}\\ \hline
\end{tabularx}
\end{center}

\item L'équation réduite de la tangente au point d'abscisse 1 de la courbe de la fonction $f$ définie sur $\R$ par $f(x) = x\text{e}^x$ est:

\begin{center}
\begin{tabularx}{\linewidth}{|*{4}{X|}}\hline
\textbf{a.~~} $y = \text{e}x + \text{e}$&\textbf{b.~~}$y =2\text{e}x - \text{e}$&\textbf{c.~~} $y = 2\text{e}x +  \text{e}$&\textbf{d.~~}$y = \text{e}x$\\ \hline
\end{tabularx}
\end{center}

\item Les nombres entiers $n$ solutions de l'inéquation $(0,2)^n < 0,001$ sont tous les
nombres entiers $n$ tels que :

\begin{center}
\begin{tabularx}{\linewidth}{|*{4}{X|}}\hline
\textbf{a.~~} $n \leqslant 4$&\textbf{b.~~}$n\leqslant 5$&\textbf{c.~~} $n \geqslant 4$ &\textbf{d.~~}$n \geqslant 5$\\ \hline
\end{tabularx}
\end{center}

\end{enumerate}

\bigskip

\textbf{\textsc{Exercice 2} \quad 7 points\hfill Thèmes : probabilités}

\medskip

Les douanes s'intéressent aux importations de casques audio portant le logo d'une certaine marque. Les saisies des douanes permettent d'estimer que:

\setlength\parindent{10mm}
\begin{itemize}
\item[$\bullet~~$] 20\,\% des casques audio portant le logo de cette marque sont des contrefaçons ;
\item[$\bullet~~$] 2\,\% des casques non contrefaits présentent un défaut de conception ;\item[$\bullet~~$] 10\,\% des casques contrefaits présentent un défaut de conception.
\end{itemize}
\setlength\parindent{0mm}

L'agence des fraudes commande au hasard sur un site internet un casque affichant le logo de la marque. On considère les évènements suivants:

\setlength\parindent{10mm}
\begin{itemize}
\item[$\bullet~~$] $C$: \og le casque est contrefait \fg{} ;
\item[$\bullet~~$] $D$: \og le casque présente un défaut de conception \fg{} ;
\item[$\bullet~~$] $\overline{C}$ et $\overline{D}$ désignent respectivement les évènements contraires de $C$ et $D$.
\end{itemize}
\setlength\parindent{0mm}

Dans l'ensemble de l'exercice, les probabilités seront arrondies à $10^{-3}$ si nécessaire.

\bigskip

\textbf{Partie 1}

\medskip

\begin{enumerate}
\item Calculer $P(C \cap D)$. On pourra s'appuyer sur un arbre pondéré.
\item Démontrer que $P(D) = 0,036$.
\item Le casque a un défaut. Quelle est la probabilité qu'il soit contrefait ?
\end{enumerate}

\bigskip

\textbf{Partie 2}

\medskip

On commande $n$ casques portant le logo de cette marque. On assimile cette expérience à
un tirage aléatoire avec remise. On note $X$ la variable aléatoire qui donne le nombre de casques présentant un défaut de conception dans ce lot.

\medskip

\begin{enumerate}
\item Dans cette question, $n = 35$.
	\begin{enumerate}
		\item Justifier que $X$ suit une loi binomiale $\mathcal{B}(n,~p)$ où $n = 35$ et $p = 0,036$.
		\item Calculer la probabilité qu'il y ait parmi les casques commandés, exactement un casque présentant un défaut de conception.
		\item Calculer $P(X \leqslant 1)$.
	\end{enumerate}	
\item Dans cette question, $n$ n'est pas fixé.

Quel doit être le nombre minimal de casques à commander pour que la probabilité 
 qu'au moins un casque présente un défaut soit supérieur à $0,99$ ?
\end{enumerate}

\bigskip

\textbf{\textsc{Exercice 3} \quad 7 points\hfill Thèmes : suites, fonctions}

\medskip

Au début de l'année 2021, une colonie d'oiseaux comptait $40$ individus. L'observation conduit à modéliser l'évolution de la population par la suite $\left(u_n\right)$ définie pour tout entier naturel $n$ par:

\[\left\{\begin{array}{l c l}
u_0&=&40\\
u_{n+1}&=&0,008u_n\left(200 - u_n\right)
\end{array}\right.\]

où $u_n$ désigne le nombre d'individus au début de l'année $(2021+n)$.

\medskip


\begin{enumerate}
\item Donner une estimation, selon ce modèle, du nombre d'oiseaux dans la colonie au
début de l'année 2022.
\end{enumerate}

On considère la fonction $f$ définie sur l'intervalle [0~;~100] par $f(x) = 0,008x(200 - x)$.

\begin{enumerate}[resume]
\item Résoudre dans l'intervalle [0~;~100] l'équation $f(x) = x$.
\item
	\begin{enumerate}
		\item Démontrer que la fonction $f$ est croissante sur l'intervalle [0~;~100] et dresser son tableau de variations.
		\item En remarquant que, pour tout entier naturel $n$,\, $u_{n+1} = f\left(u_n\right)$ démontrer par récurrence que, pour tout entier naturel $n$ :

\[0 \leqslant u_n \leqslant u_{n+1} \leqslant 100.\]

		\item En déduire que la suite $\left(u_n\right)$ est convergente.
		\item Déterminer la limite $\ell$ de la suite $\left(u_n\right)$. Interpréter le résultat dans le contexte de l'exercice.
	\end{enumerate}
\item On considère l'algorithme suivant:

\begin{center}
\fbox{\begin{tabular}{l}
def seuil(p) :\\
\qquad n=0\\
\qquad u = 40\\
\qquad while u < p :\\
\quad \qquad n =n+1\\
\quad \qquad u = 0.008*u*(200-u)\\
\qquad return(n+2021)\\ 
\end{tabular}}
\end{center}

L'exécution de seuil(100) ne renvoie aucune valeur. Expliquer pourquoi à l'aide de la question 3.
\end{enumerate}

\bigskip

\textbf{\textsc{Exercice 4} \quad 7 points\hfill Thèmes : géométrie dans le plan et dans l'espace}

\medskip

On considère le cube ABCDEFGH d'arête de longueur 1.

L'espace est muni du repère orthonormé $\left(\text{A}~;\, \vect{\text{AB}}, \vect{\text{AD}}, \vect{\text{AE}}\right)$. Le point I est le milieu du
segment [EF], K le centre du carré ADHE et O le milieu du segment [AG].

\begin{center}
\psset{unit=0.9cm,radius=0pt}
\begin{pspicture}(0,-1)(9,7)
%%%%%%%%%%%%%%%%%
\Cnode*(0.5,0.4){A} \Cnode*(5.5,0){B} 
\Cnode*(7.5,1.4){C} \Cnode*(2.5,1.8){D}
\Cnode*(0.5,5.4){E} \Cnode*(5.5,5){F} 
\Cnode*(7.5,6.4){G} \Cnode*(2.5,6.8){H}
\Cnode*[radius=2pt](3,5.2){I}% milieu de [EF]
\Cnode*[radius=2pt](1.5,3.6){K}% milieu de [ED]
\Cnode*[radius=2pt](4,3.4){O}% centre du cube
%%%%%%%%%%%%%%%
\uput[dl](A){A} \uput[dr](B){B} \uput[r](C){C} 
\uput[ur](D){D} \uput[ul](E){E} \uput[u](F){F} 
\uput[ur](G){G} \uput[u](H){H} \uput[u](I){I}
\uput[dr](O){O}
\uput[l](K){K}
%%%%%%%%%%%%%%%
\pspolygon(A)(B)(F)(E)
\psline(B)(C)(G)(F)
\psline(G)(H)(E)
\psline(A)(I)(G)
\psline[linestyle=dashed](A)(G)
\psline[linestyle=dashed](A)(D)(H)
\psline[linestyle=dashed](D)(C)
\end{pspicture}
\end{center}

\emph{Le but de l'exercice est de calculer de deux manières différentes, la distance du point B au plan (AIG).}

\bigskip

\textbf{Partie 1. Première méthode}

\medskip

\begin{enumerate}
\item Donner, sans justification, les coordonnées des points A, B, et G. 

On admet que les points I et K ont pour coordonnées I$\left(\dfrac{1}{2}~;~0~;~1\right)$ et K$\left(0~;~\dfrac{1}{2}~;~\dfrac{1}{2}\right)$.
\item Démontrer que la droite (BK) est orthogonale au plan (AIG).
\item Vérifier qu'une équation cartésienne du plan (AIG) est : $2x - y - z = 0$.
\item Donner une représentation paramétrique de la droite (BK).
\item En déduire que le projeté orthogonal L du point B sur le plan (AIG) a pour
coordonnées L$\left(\dfrac{1}{3}~;~\dfrac{1}{3}~;~\dfrac{1}{3}\right)$.
\item Déterminer la distance du point B au plan (AIG).
\end{enumerate}

\bigskip

\textbf{Partie 2. Deuxième méthode}

\medskip

\emph{On rappelle que le volume $V$ d'une pyramide est donné par la formule $V = \dfrac{1}{3} \times  b \times h$, où $b$ est l'aire d'une base et $h$ la hauteur associée à cette base.}

\medskip

\begin{enumerate}
\item 
	\begin{enumerate}
		\item Justifier que dans le tétraèdre ABIG, [GF] est la hauteur relative à la base AIB. 
		\item En déduire le volume du tétraèdre ABIG.
	\end{enumerate}
\item On admet que AI = IG $= \dfrac{\sqrt{5}}{2}$ et que AG $= \sqrt 3$.

Démontrer que l'aire du triangle isocèle AIG est égale à $\dfrac{\sqrt{6}}{4}$ unité d'aire.
\item En déduire la distance du point B au plan (AIG).
\end{enumerate}
\end{document}