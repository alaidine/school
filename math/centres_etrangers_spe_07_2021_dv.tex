\documentclass[11pt,a4paper]{article}
\usepackage[T1]{fontenc}
\usepackage{fourier}
\usepackage[scaled=0.875]{helvet}
\renewcommand{\ttdefault}{lmtt}
\usepackage{makeidx}
\usepackage{amsmath,amssymb}
\usepackage{fancybox}
\usepackage[normalem]{ulem}
\usepackage{pifont}
\usepackage{lscape}
\usepackage{multicol}
\usepackage{mathrsfs}
\usepackage{tabularx}
\usepackage{multirow}
\usepackage{enumitem}
\usepackage{textcomp}
\newcommand{\euro}{\eurologo{}}
%Tapuscrit : Denis Vergès
\usepackage{pst-plot,pst-tree,pstricks,pst-node,pst-text}
\usepackage{pst-eucl,pst-3dplot,pstricks-add}
\usepackage{esvect}
\newcommand{\R}{\mathbb{R}}
\newcommand{\N}{\mathbb{N}}
\newcommand{\D}{\mathbb{D}}
\newcommand{\Z}{\mathbb{Z}}
\newcommand{\Q}{\mathbb{Q}}
\newcommand{\C}{\mathbb{C}}
\usepackage[left=3.5cm, right=3.5cm, top=2cm, bottom=3cm]{geometry}
\headheight15 mm
\newcommand{\vect}[1]{\overrightarrow{\,\mathstrut#1\,}}
\renewcommand{\theenumi}{\textbf{\arabic{enumi}}}
\renewcommand{\labelenumi}{\textbf{\theenumi.}}
\renewcommand{\theenumii}{\textbf{\alph{enumii}}}
\renewcommand{\labelenumii}{\textbf{\theenumii.}}
\def\Oij{$\left(\text{O}~;~\vect{\imath},~\vect{\jmath}\right)$}
\def\Oijk{$\left(\text{O}~;~\vect{\imath},~\vect{\jmath},~\vect{k}\right)$}
\def\Ouv{$\left(\text{O}~;~\vect{u},~\vect{v}\right)$}
\newcommand{\e}{\text{e}}
\usepackage{fancyhdr}
\usepackage[dvips]{hyperref}
\hypersetup{%
pdfauthor = {APMEP},
pdfsubject = {Baccalauréat spécialité},
pdftitle = {Métropole 10 juin 2021},
allbordercolors = white,
pdfstartview=FitH}
\usepackage[french]{babel}
\usepackage[np]{numprint}
\renewcommand\arraystretch{1.2}
\frenchsetup{StandardLists=true}
\begin{document}
\setlength\parindent{0mm}
\rhead{\textbf{A. P{}. M. E. P{}.}}
\lhead{\small Baccalauréat spécialité}
\lfoot{\small{Centres étrangers candidats libres}}
\rfoot{\small{10 juin 2021}}
\pagestyle{fancy}
\thispagestyle{empty}

\begin{center}{\Large\textbf{\decofourleft~Baccalauréat Centres étrangers 10 juin 2021~\decofourright\\[6pt] Candidats libres  Sujet 2\\[6pt] ÉPREUVE D'ENSEIGNEMENT DE SPÉCIALITÉ}}
\end{center}

\vspace{0,25cm}

Le candidat traite 4 exercices : les exercices 1, 2 et 3 communs à tous les candidats et un seul des deux exercices A ou B.

\bigskip

\textbf{\textsc{Exercice 1} \hfill 5 points}

\textbf{Commun à tous les candidats}

\medskip

Cet exercice est un questionnaire à choix multiples.

Pour chacune des cinq questions, quatre
réponses sont proposées; une seule de ces réponses est exacte.

\smallskip

\textbf{Indiquer sur la copie le numéro de la question et recopier la réponse exacte sans justifier le choix effectué.}

\textbf{Barème :} une bonne réponse rapporte un point. Une réponse inexacte ou une absence de réponse n'apporte ni n'enlève aucun point.

\bigskip

\textbf{Question 1 :}

On considère la fonction $g$ définie sur $]0~;~+\infty[$  par $g(x)= x^2+ 2x - \dfrac{3}{x}$.

Une équation de la tangente à la courbe représentative de $g$ au point d'abscisse 1 est:
\begin{center}
\begin{tabularx}{\linewidth}{|*{4}{X|}}\hline
\textbf{a.~~} $y=7(x - 1)$&\textbf{b.~~} $y = x - 1$&\textbf{c.~~} $y = 7x + 7$&\textbf{d.~~}$ y = x +1$\\ \hline
\end{tabularx}
\end{center}

\medskip

\textbf{Question 2 :}

On considère la suite $\left(v_n\right)$ définie sur $\N$ par $v_n = \dfrac{3n}{n + 2}$. On cherche à déterminer la limite de $v_n$ lorsque $n$ tend vers $+\infty$.

\begin{center}
\begin{tabularx}{\linewidth}{|*{4}{X|}}\hline
\textbf{a.~~}$\displaystyle\lim_{n \to + \infty}v_n = 1$&
\textbf{b.~~} $\displaystyle\lim_{n \to + \infty}v_n = 3$ &  \textbf{c.~~} $\displaystyle\lim_{n \to + \infty}v_n = \dfrac{3}{2}$ &\textbf{d.~~} On ne peut pas la déterminer\\ \hline
\end{tabularx}
\end{center}

\medskip

\textbf{Question 3 :}
Dans une urne il y a 6 boules noires et 4 boules rouges. On effectue successivement 10 tirages aléatoires avec remise. Quelle est la probabilité (à $10^{-4}$ près) d'avoir 4 boules noires et 6 boules rouges?

\begin{center}
\begin{tabularx}{\linewidth}{|*{4}{X|}}\hline
\textbf{a.~~}\np{0,1662}&\textbf{b.~~} 0,4&\textbf{c.~~} \np{0,1115}&\textbf{d.~~} \np{0,8886}\\ \hline
\end{tabularx}
\end{center}

\medskip

\textbf{Question 4 :}

On considère la fonction $f$ définie sur $\R$ par $f(x) = 3\e^x - x$.

\begin{center}
\begin{tabularx}{\linewidth}{|*{4}{>{\small}X|}}\hline
\textbf{a.~~}$\displaystyle\lim_{x \to + \infty} f(x) = 3$
&\textbf{b.~~}$\displaystyle\lim_{x \to + \infty} f(x) = +\infty $&\textbf{c.~~} $\displaystyle\lim_{x \to + \infty} f(x) =  -\infty$
&\raggedright\textbf{d.~~} On ne peut pas déterminer la limite de la fonction $f$ lorsque $x$ tend vers $+\infty$\tabularnewline
\hline
\end{tabularx}
\end{center}

\medskip

\textbf{Question 5 :}

Un code inconnu est constitué de 8 signes.

Chaque signe peut être une lettre ou un chiffre. Il y a
donc 36 signes utilisables pour chacune des positions.

Un logiciel de cassage de code teste environ cent millions de codes par seconde. En combien de temps au maximum le logiciel peut-il découvrir le code ?

\begin{center}
\begin{tabularx}{\linewidth}{|*{4}{X|}}\hline
\textbf{a.~~}\raggedright environ 0,3 seconde&\textbf{b.~~}environ 8 heures&\textbf{c.~~}environ 3 heures&
\raggedright\textbf{d.~~}environ 470 heures\tabularnewline \hline
\end{tabularx}
\end{center}

\bigskip

\textbf{\textsc{Exercice 2} \hfill 5 points}

\textbf{Commun à tous les candidats}

\medskip

Au 1\up{er} janvier 2020, la centrale solaire de Big Sun possédait \np{10560} panneaux solaires.

On observe, chaque année, que 2\,\% des panneaux se sont détériorés et nécessitent d'être retirés tandis que 250 nouveaux panneaux solaires sont installés.

\bigskip

\textbf{Partie A - Modélisation à l'aide d'une suite}

\medskip

On modélise l'évolution du nombre de panneaux solaires par la suite $\left(u_n\right)$ définie par $u_0 = \np{10560}$ et, pour tout entier naturel $n$,\, $u_{n+1} = 0,98u_n +250$, où $u_n$  est le nombre de panneaux solaires au 1\up{er} janvier de l'année $2020 +n$.

\medskip

\begin{enumerate}
\item
	\begin{enumerate}
		\item  Expliquer en quoi cette modélisation correspond à la situation étudiée.
		\item  On souhaite savoir au bout de combien d'années le nombre de panneaux solaires sera strictement supérieur à \np{12000}.
		
À l'aide de la calculatrice, donner la réponse à ce
problème.
		\item  Recopier et compléter le programme en Python ci-dessous de sorte que la valeur
cherchée à la question précédente soit stockée dans la variable n à l'issue de l'exécution de ce dernier.
\begin{center}
\begin{tabular}{|l|}\hline
u $= \np{10560}$\\
n$ =0$\\
while  \ldots \ldots :\\
\qquad u $= \ldots \ldots$ \\
\qquad n $= \ldots \ldots$\\ \hline
\end{tabular}
\end{center}

	\end{enumerate}
\item Démontrer par récurrence que, pour tout entier naturel $n$, on a $u_n \leqslant \np{12500}$.
\item Démontrer que la suite $\left(u_n\right)$ est croissante.
\item En déduire que la suite $\left(u_n\right)$ converge. Il n'est pas demandé, ici, de calculer sa limite.
\item On définit la suite $\left(v_n\right)$  par $v_n = u_n - \np{12500}$, pour tout entier naturel $n$.
	\begin{enumerate}
		\item Démontrer que la suite $\left(v_n\right)$ est une suite géométrique de raison $0,98$ dont on précisera le premier terme.
		\item Exprimer, pour tout entier naturel $n$,\, $v_n$ en fonction de $n$.
		\item En déduire, pour tout entier naturel $n$,\, $u_n$ en fonction de $n$.
		\item Déterminer la limite de la suite $\left(u_n\right)$.
		
Interpréter ce résultat dans le contexte du modèle.
	\end{enumerate}
\end{enumerate}

\bigskip

\textbf{Partie B - Modélisation à l'aide d'une fonction}

\medskip

Une modélisation plus précise a permis d'estimer le nombre de panneaux solaires de la centrale à l'aide de la fonction $f$ définie pour tout $x \in [0~;~ +\infty[$ par

\[f(x) = \np{12500} - 500 \e^{-0,02x+1,4},\]

 où $x$ représente le nombre d'années écoulées depuis le 1\up{er} janvier 2020.

\medskip

\begin{enumerate}
\item Étudier le sens de variation de la fonction $f$.
\item Déterminer la limite de la fonction $f$ en $+\infty$.
\item En utilisant ce modèle, déterminer au bout de combien d'années le nombre de panneaux
solaires dépassera \np{12000}.
\end{enumerate}

\bigskip

\textbf{\textsc{Exercice 3} \hfill 5 points}

\textbf{Commun à tous les candidats}

\medskip

ABCDEFGH est un cube. I est le centre de la face ADHE et J est un point du segment [CG].

Il existe donc $a \in [0~;~1]$ tel que $\vect{\text{CJ}} =a \vect{\text{CG}}$.

On note $(d)$ la droite passant par I et parallèle à (FJ).

On note K et L les points d'intersection de la droite $(d)$ et des droites (AE) et (DH).

On se place dans le repère $\left(\text{A}~;~\vect{\text{AB}},\,\vect{\text{AD}},\, \vect{\text{AE}}\right)$.

\bigskip

\textbf{Partie A : Dans cette partie} \boldmath $a = \dfrac{2}{3}$ \unboldmath

\begin{center}
\psset{unit=1cm}
\begin{pspicture}(-0.5,0)(9.5,7.5)
\pspolygon[fillstyle=solid,fillcolor=lightgray](2,3.65)(4.2,4.05)(8.9,5.6)(6.8,5.2)%KLJF
\psframe(2,0.5)(6.8,5.2)%ABFE
\psline(6.8,0.5)(8.9,2.4)(8.9,7.2)(6.8,5.2)%BCGF
\psline(8.9,7.2)(4.2,7.2)(2,5.1)%GHE
\psline[linestyle=dashed](2,0.5)(4.2,2.4)(4.2,7.2)%ADH
\psline[linestyle=dashed](8.9,2.4)(4.2,2.4)
\psdots(2,0.5)(6.8,0.5)(8.9,2.4)(4.2,2.4)%ABCD
\uput[dl](2,0.5){A} \uput[d](6.8,0.5){B} \uput[r](8.9,2.4){C} \uput[ul](4.2,2.4){D}
\psdots(2,5.2)(6.8,5.2)(8.9,7.2)(4.2,7.2)(8.9,2.4)(8.9,4)(8.9,5.6)(3.1,3.85)%EFGHPJI
\uput[l](2,5.2){E} \uput[u](6.8,5.2){F} \uput[ur](8.9,7.2){G} \uput[u](4.2,7.2){H} \uput[r](8.9,4){P}
\uput[r](8.9,5.6){J}\uput[d](3.1,3.85){I}
\psline(2,3.65)(6.8,5.2)(8.9,5.6)(4.2,4.05)%KFJL
\psline(-0.2,3.25)(2,3.65)(4.2,4.05)%KL
\psdots(2,3.65)(4.2,4.05)%KL
\uput[ul](2,3.65){K}\uput[dr](4.2,4.05){L}
\psline[linestyle=dashed](4.2,4.05)(8.9,4.9)
\psdots[dotstyle=+,dotangle=45,dotscale=1.8](8.9,3.2)(8.9,4.8)(8.9,6.4)
\end{pspicture}
\end{center}
\medskip

\begin{enumerate}
\item Donner les coordonnées des points F{}, I et J.
\item Déterminer une représentation paramétrique de la droite $(d)$.
\item
	\begin{enumerate}
		\item Montrer que le point de coordonnées $\left(0~;~ 0~;~\dfrac{2}{3}\right)$ est le point K.
		\item Déterminer les coordonnées du point L, intersection des droites $(d)$ et (DH).

	\end{enumerate}
\item
	\begin{enumerate}
		\item Démontrer que le quadrilatère FJLK est un parallélogramme.
		\item Démontrer que le quadrilatère FJLK est un losange.
		\item Le quadrilatère FJLK est-il un carré ?

	\end{enumerate}
\end{enumerate}

\bigskip

\textbf{Partie B : Cas général}

\medskip

On admet que les coordonnées des points K et L sont : K$\left(0~;~0~;~1- \dfrac{a}{2}\right)$ et L$\left(0~;~1~;~\dfrac{a}{2}\right)$.

On rappelle que $a \in [0~;~1]$.

\medskip

\begin{enumerate}
\item Déterminer les coordonnées de J en fonction de $a$.
\item Montrer que le quadrilatère FJLK est un parallélogramme.
\item Existe-t-il des valeurs de $a$ telles que le quadrilatère FJLK soit un losange ? Justifier.
\item Existe-t-il des valeurs de $a$ telles que le quadrilatère FJLK soit un carré ? Justifier.
\end{enumerate}

\bigskip

\textbf{\textsc{EXERCICE au choix du candidat} \hfill 5 points}

\textbf{Le candidat doit traiter un seul des deux exercices A ou B}

\textbf{Il indique sur sa copie l'exercice choisi : exercice A ou exercice B.}

\medskip

\textbf{EXERCICE A - Fonction ln}

\bigskip

\textbf{Partie A :}

\medskip

Dans un pays, une maladie touche la population avec une probabilité de $0,05$.

On possède un test de dépistage de cette maladie.

On considère un échantillon de $n$ personnes ($n \geqslant 20$) prises au hasard dans la population assimilé à un tirage avec remise.

On teste l'échantillon suivant cette méthode : on mélange le sang de ces $n$ individus, on teste le mélange.

Si le test est positif, on effectue une analyse individuelle de chaque personne.

Soit $X_n$ la variable aléatoire qui donne le nombre d'analyses effectuées.

\medskip

\begin{enumerate}
\item Montrer $X_n$ prend les valeurs 1 et $(n + 1)$.
\item Prouver que $P\left(X_n = 1\right) = 0,95^n$.

Établir la loi de $X_n$ en recopiant sur la copie et en complétant le tableau suivant:

\begin{center}
\begin{tabularx}{0.6\linewidth}{|*{3}{>{\centering \arraybackslash}X|}}\hline
$x_i$& 1 &$n + 1$\\ \hline
$P\left(X_n = x_i\right)$&&\\ \hline
\end{tabularx}
\end{center}

\item Que représente l'espérance de $X_n$ dans le cadre de l'expérience ?

Montrer que $E\left(X_n\right) =n + 1 - n \times  0,95^n$.
\end{enumerate}

\bigskip

\textbf{Partie B :}

\medskip

\begin{enumerate}
\item On considère la fonction $f$ définie sur $[20~;~ +\infty[$ par $f(x) = \ln (x) + x \ln (0,95)$.

Montrer que $f$ est décroissante sur $[20~;~ +\infty[$.
\item On rappelle que $\displaystyle\lim_{x \to + \infty} \dfrac{\ln x}{x} = 0$. Montrer que $\displaystyle\lim_{x \to + \infty} f(x) = - \infty$.
\item Montrer que $f(x) = 0$ admet une unique solution $a$ sur $[20~;~ +\infty[$.

Donner un encadrement à 0,1 près de cette solution.
\item En déduire le signe de $f$ sur $[20~;~ +\infty[$.
\end{enumerate}

\bigskip

\textbf{Partie C :}

\medskip

On cherche à comparer deux types de dépistages.

La première méthode est décrite dans la partie A, la seconde, plus classique, consiste à tester tous les individus.

La première méthode permet de diminuer le nombre d'analyses dès que $E\left(X_n\right)  < n$.

En utilisant la partie B, montrer que la première méthode diminue le nombre d'analyses pour des échantillons comportant $87$ personnes maximum.

\bigskip

\textbf{EXERCICE B - Équation différentielle}

\medskip

\textbf{Partie A : Détermination d'une fonction $f$ et résolution d'une équation différentielle}

\medskip

On considère la fonction $f$ définie sur $\R$ par :

\[f(x) = \e^x+ ax + b\e^{-x}\]

où $a$ et $b$ sont des nombres réels que l'on propose de déterminer dans cette partie.

Dans le plan muni d'un repère d'origine O, on a représenté ci-dessous la courbe $\mathcal{C}$, représentant la fonction $f$, et la tangente $(T)$ à la courbe $\mathcal{C}$ au point d'abscisse $0$.

\begin{center}
\psset{unit=1.25cm}
\begin{pspicture*}(-2.2,-0.6)(3.2,6.6)
\psgrid[gridlabels=0pt,subgriddiv=5,gridwidth=0.25pt,subgridwidth=0.1pt]
\psaxes[linewidth=1.25pt,labelFontSize=\scriptstyle]{->}(0,0)(-2.2,-0.4)(3.2,6.6)
\psplot[plotpoints=2000,linewidth=1.25pt,linecolor=red]{-2}{2.5}{2.71828 x exp x sub 2 2.71828 x exp div add} \uput[r](2,5){\red $\mathcal{C}$}
\psplot[linewidth=1.25pt]{-2}{2}{3 2 x mul sub}
\end{pspicture*}
\end{center}

\smallskip

\begin{enumerate}
\item Par lecture graphique, donner les valeurs de $f(0)$ et de $f'(0)$.
\item En utilisant l'expression de la fonction $f$, exprimer $f(0)$ en fonction de $b$ et en déduire la valeur de $b$.
\item On admet que la fonction $f$ est dérivable sur $\R$ et on note $f'$ sa fonction dérivée.
	\begin{enumerate}
		\item Donner, pour tout réel $x$, l'expression de $f'(x)$.
		\item Exprimer $f'(0)$ en fonction de $a$.
		\item En utilisant les questions précédentes, déterminer $a$, puis en déduire l'expression de $f(x)$.
	\end{enumerate}
\item On considère l'équation différentielle :

\[(E):\quad  y' +y =2\e^x - x - 1\]

	\begin{enumerate}
		\item Vérifier que la fonction $g$ définie sur $\R$ par :
		
\[g(x) = \e^x - x + 2\e^{-x}.\]

est solution de l'équation $(E)$.
		\item Résoudre l'équation différentielle $y' + y = 0$.
		\item En déduire toutes les solutions de l'équation $(E)$.
	\end{enumerate}
\end{enumerate}

\bigskip

\textbf{Partie B : Étude de la fonction $g$ sur $[1~;~+\infty[$}

\medskip

\begin{enumerate}
\item Vérifier que pour tout réel $x$, on a :

\[\e^{2x} - \e^x - 2 = \left(\e^x - 2\right)\left(\e^x + 1\right)\]

\item En déduire une expression factorisée de $g'(x)$, pour tout réel $x$.
\item On admettra que, pour tout $x \in  [1~;~ +\infty[$,\, $\e^x - 2 > 0$.

 Étudier le sens de variation de la fonction $g$ sur $[1~;~ +\infty[$.
\end{enumerate}
\end{document}
