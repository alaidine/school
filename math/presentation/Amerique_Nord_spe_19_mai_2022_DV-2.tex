\documentclass[10pt]{article}
\usepackage[T1]{fontenc}
\usepackage[utf8]{inputenc}

\usepackage[scaled=0.875]{helvet}
\renewcommand{\ttdefault}{lmtt}
\usepackage{makeidx}
\usepackage{amsmath,amssymb}
\usepackage{fancybox}
\usepackage[normalem]{ulem}
\usepackage{pifont}
\usepackage{lscape}
\usepackage{multicol}
\usepackage{mathrsfs}
\usepackage{tabularx}
\usepackage{multirow}
\usepackage{enumitem}
\usepackage{textcomp} 
\newcommand{\euro}{\eurologo{}}
%Tapuscrit : Denis Vergès
\usepackage{pst-plot,pst-tree,pstricks,pst-node,pst-text}
\usepackage{pst-eucl}
\usepackage{pstricks-add}
\newcommand{\R}{\mathbb{R}}
\newcommand{\N}{\mathbb{N}}
\newcommand{\D}{\mathbb{D}}
\newcommand{\Z}{\mathbb{Z}}
\newcommand{\Q}{\mathbb{Q}}
\newcommand{\C}{\mathbb{C}}
\usepackage[left=3.5cm, right=3.5cm, top=3cm, bottom=3cm]{geometry}
\newcommand{\vect}[1]{\overrightarrow{\,\mathstrut#1\,}}
\renewcommand{\theenumi}{\textbf{\arabic{enumi}}}
\renewcommand{\labelenumi}{\textbf{\theenumi.}}
\renewcommand{\theenumii}{\textbf{\alph{enumii}}}
\renewcommand{\labelenumii}{\textbf{\theenumii.}}
\def\Oij{$\left(\text{O}~;~\vect{\imath},~\vect{\jmath}\right)$}
\def\Oijk{$\left(\text{O}~;~\vect{\imath},~\vect{\jmath},~\vect{k}\right)$}
\def\Ouv{$\left(\text{O}~;~\vect{u},~\vect{v}\right)$}
\usepackage{fancyhdr}
\usepackage[dvips]{hyperref}
\hypersetup{%
pdfauthor = {APMEP},
pdfsubject = {Baccalauréat Spécialité},
pdftitle = {Amérique du Nord 19 mai }2022,
allbordercolors = white,
pdfstartview=FitH} 
\usepackage[frenchb]{babel}
\usepackage[np]{numprint}
\begin{document}
\setlength\parindent{0mm}
\rhead{\textbf{A. P{}. M. E. P{}.}}
\lhead{\small Baccalauréat Spécialité}
\lfoot{\small{Amérique du Nord}}
\rfoot{\small{19 mai 2022}}
\pagestyle{fancy}
\thispagestyle{empty}

\begin{center}{\Large\textbf{\decofourleft~Baccalauréat Amérique du Nord Jour 2 19 mai 2022~\decofourright\\[6pt] ÉPREUVE D'ENSEIGNEMENT DE SPÉCIALITÉ}}
\end{center}

\vspace{0,25cm}

Le sujet propose 4 exercices

Le candidat choisit 3 exercices parmi les 4 exercices et \textbf{ne doit traiter que ces 3 exercices}

Chaque exercice est noté sur \textbf{7 points (le total sera ramené sur 20 points)}.

Les traces de recherche, même incomplètes ou infructueuses, seront prises en compte.

\bigskip

\textbf{\textsc{Exercice 1} \quad (7 points)\hfill Thème : probabilités, suites}

\medskip

Dans une région touristique, une société propose un service de location de vélos pour la journée.

\smallskip

La société dispose de deux points de location distinctes, le point A et le point B. Les vélos peuvent être empruntés et restitués indifféremment dans l'un où l'autre des deux points de location.

\smallskip

On admettra que le nombre total de vélos est constant et que tous les matins, à l'ouverture du service, chaque vélo se trouve au point A ou au point B.

\smallskip

D'après une étude statistique :

\smallskip

\setlength\parindent{1cm}
\begin{itemize}
\item[$\bullet~~$] Si un vélo se trouve au point A un matin, la probabilité qu'il se trouve au point A le matin suivant est égale à $0,84$ ;
\item[$\bullet~~$] Si un vélo se trouve au point B un matin la probabilité qu'il se trouve au point B le matin suivant est égale à $0,76$.
\end{itemize}
\setlength\parindent{0cm}

À l'ouverture du service le premier matin, la société a disposé la moitié de ses vélos au point A, l'autre moitié au point B.

\smallskip

On considère un vélo de la société pris au hasard.

\smallskip

Pour tout entier naturel non nul $n$, on définit les évènements suivants :

\setlength\parindent{1cm}
\begin{itemize}
\item[$\bullet~~$]$A_n$ : \og le vélo se trouve au point A le $n$-ième matin \fg{}
\item[$\bullet~~$]$B_n$ : \og le vélo se trouve au point B le $n$-ième matin \fg.
\end{itemize}
\setlength\parindent{0cm}

Pour tout entier naturel non nul $n$, on note $a_n$ la probabilité de l'évènement $A_n$ et $b_n$ la probabilité de l'évènement $B_n$. Ainsi $a_1 = 0,5$ et $b_1 =  0,5$.

\medskip

\begin{enumerate}
\item Recopier et compléter l'arbre pondéré ci-dessous qui modélise la situation pour les deux premiers matins :

\begin{center}
\pstree[treemode=R,nodesepB=3pt,levelsep=3cm]{\TR{}}
{\pstree{\TR{$A_1~~$}\taput{\ldots}}
	{\TR{$A_2$}\taput{\ldots}
	\TR{$B_2$}\tbput{\ldots}
	}
\pstree{\TR{$B_1~~$}\tbput{\ldots}}
	{\TR{$A_2$}\taput{\ldots}
	\TR{$B_2$}\tbput{\ldots}
	}	
}
\end{center}

\item
	\begin{enumerate}
		\item Calculer $a_2$.
		\item Le vélo se trouve au point A le deuxième matin. Calculer la probabilité qu'il se soit trouvé au point B le premier matin. La probabilité sera arrondie au millième.
	\end{enumerate}
\item
	\begin{enumerate}
		\item Recopier et compléter l'arbre pondéré ci-dessous qui modélise la situation pour les $n$-ième et $n + 1$-ième  matins.
		
\begin{center}
\pstree[treemode=R,nodesepB=3pt,levelsep=3cm]{\TR{}}
{\pstree{\TR{$A_n~~$}\taput{$a_n$}}
	{\TR{$A_{n+1}$}\taput{\ldots}
	\TR{$B_{n+1}$}\tbput{\ldots}
	}
\pstree{\TR{$B_n~~$}\tbput{\ldots}}
	{\TR{$A_{n+1}$}\taput{\ldots}
	\TR{$B_{n+1}$}\tbput{\ldots}
	}	
}
\end{center}
		\item Justifier que pour tout entier naturel non nul $n$,\: $a_{n+1} = 0,6a_n + 0,24$.
	\end{enumerate}
\item Montrer par récurrence que, pour tout entier naturel non nul $n$, \: $a_n = 0,6 - 0,1 \times 0,6^{n - 1}$.
\item Déterminer la limite de la suite $\left(a_n\right)$ et interpréter cette limite dans le contexte de l'exercice. 
\item Déterminer le plus petit entier naturel $n$ tel que $a_n \geqslant 0,599$ et interpréter le résultat obtenu dans le contexte de l'exercice.
\end{enumerate}

\bigskip

\textbf{\textsc{Exercice 2} \quad (7 points)\hfill Thème : fonctions, fonction exponentielle}

\bigskip

\textbf{Partie A}

\medskip 

Soit $p$ la fonction définie sur l'intervalle $[-3~;~4]$ par :

\[p(x) = x^3 - 3x^2 + 5x + 1\]

\begin{enumerate}
\item Déterminer les variations de la fonction $p$ sur l'intervalle $[-3~;~4]$.
\item Justifier que l'équation $p(x) = 0$ admet dans l'intervalle $[-3~;~4]$ une unique solution qui sera notée $\alpha$.
\item Déterminer une valeur approchée du réel $\alpha$ au dixième près.
\item Donner le tableau de signes de la fonction $p$ sur l'intervalle $[-3~;~4]$. 
\end{enumerate}

\bigskip

\textbf{Partie B}

\medskip 

Soit $f$ la fonction définie sur l'intervalle $[-3~;~4]$ par :

\[f(x) = \dfrac{\text{e}^x}{1 + x^2}\]

On note $\mathcal{C}_f$ sa courbe représentative dans un repère orthogonal.

\medskip

\begin{enumerate}
\item 
	\begin{enumerate}
		\item Déterminer la dérivée de la fonction $f$ sur l'intervalle $[-3~;~4]$.
		\item Justifier que la courbe $\mathcal{C}_f$ admet une tangente horizontale au point d'abscisse 1.
	\end{enumerate}
\item  Les concepteurs d'un toboggan utilisent la courbe $\mathcal{C}_f$ comme profil d'un toboggan. Ils estiment que le toboggan assure de bonnes sensations si le profil possède au moins deux points d'inflexion.

\medskip

\begin{minipage}{0.48\linewidth}
\psset{unit=0.75cm}
\begin{pspicture*}(-4,-1.95)(4.25,4.1)
\psgrid[gridlabels=0pt,subgriddiv=1,gridwidth=0.15pt](-4,0)(4,4)
\psaxes[linewidth=1.25pt,labelFontSize=\scriptstyle](0,0)(-4,-0)(4.25,4.1)
\psplot[plotpoints=2000,linewidth=1.25pt,linecolor=blue]{-3}{4}{2.71828 x exp x dup mul 1 add div}
\rput(0,-1){Représentation de la courbe $\mathcal{C}_f$}
\uput[ul](3,2){\blue \small $\mathcal{C}_f$}
\end{pspicture*}
\end{minipage} \hfill
\begin{minipage}{0.48\linewidth}
\psset{unit=0.75cm}
\begin{pspicture}(-4,-1.95)(4.25,4.1)
%\psgrid
\def\tobo{\psplot[plotpoints=2000,linewidth=1.25pt]{-3}{4}{2.71828 x exp x dup mul 1 add div}}
\psplot[plotpoints=2000,linewidth=1.25pt]{-3}{4}{2.71828 x exp x dup mul 1 add div}
\rput(-0.6,0.6){\tobo}
\psline(-3,0)(-3.6,0.6)\psline(4,3.2)(3.4,3.8)
\pscustom[fillstyle=solid,fillcolor=lightgray]
{\psplot[plotpoints=2000,linewidth=1.25pt,linecolor=red]{-3}{4}{2.71828 x exp x dup mul 1 add div}
\psline(4,3.2)(4,0)(-2,0)}
\rput(0,-1){Vue de profil du toboggan}
\end{pspicture}
\end{minipage}

	\begin{enumerate}
		\item D'après le graphique ci-dessus, le toboggan semble-t-il assurer de bonnes sensations ? Argumenter.
		\item On admet que la fonction $f''$, dérivée seconde de la fonction $f$, a pour expression pour tout réel $x$ de l'intervalle $[-3~;~4]$ :

		\[f''(x) = \dfrac{p(x)(x - 1)\text{e}^x}{\left(1 + x^2\right)^3}\]

où $p$ est la fonction définie dans la partie A.

En utilisant l'expression précédente de $f''$, répondre à la question : \og le toboggan assure-t-il de bonnes sensations ? \fg. Justifier.
	\end{enumerate}
\end{enumerate}

\bigskip

\textbf{\textsc{Exercice 3} \quad (7 points)\hfill Thème : géométrie dans l'espace}

\bigskip


Une exposition d'art contemporain a lieu dans une salle en forme de pavé droit de largeur 6 m, de longueur 8 m et de hauteur 4 m. 

Elle est représentée par le parallélépipède rectangle OBCDEFGH où
OB = 6 m, OD = 8 m et OE = 4 m.

On utilise le repère orthonormé \Oijk{} tel que $\vect{\imath} = \dfrac16\vect{\text{OB}}, \vect{\jmath} = \dfrac18\vect{\text{OD}}$ et $\vect{k} =\dfrac14\vect{\text{OE}}$.

\begin{center}
\psset{unit=1cm}
\begin{pspicture}(7.2,5.3)
%\psgrid
\pspolygon(0.2,1.2)(4.5,0.2)(4.5,2.7)(0.2,3.7)%BCGF
\psline(4.5,0.2)(6.9,1.2)(6.9,3.7)(4.5,2.7)%CDHG
\psline(6.9,3.7)(2.6,4.7)(0.2,3.7)%HEF
\psline(0.2,2.45)(1.8,3.33)(1.4,4.2)%ART
\psline[linestyle=dashed,linewidth=1.5pt](0.2,2.45)(1.4,4.2)
\psline[linestyle=dotted,linewidth=1.5pt](0.2,1.2)(2.6,2.2)(2.6,4.7)%BOE
\psline[linestyle=dotted,linewidth=1.5pt](2.6,2.2)(6.9,1.2)%OD
\psline{->}(2.6,2.2)(2.2,2.02)%vect i
\psline{->}(2.6,2.2)(3.4,2.03)%vect j
\psline{->}(2.6,2.2)(2.6,2.82)%\vect k
\psdot(2.7,1.5)%S
\uput[l](0.2,2.45){A} \uput[dl](0.2,1.2){B} \uput[d](4.5,0.2){C} \uput[dr](6.9,1.2){D}
\uput[u](2.6,4.7){E} \uput[ul](0.2,3.7){F} \uput[dr](4.5,2.7){G} \uput[ur](6.9,3.7){H}
\uput[d](2.6,2.2){O} \uput[ur](1.8,3.33){R} \uput[d](2.7,1.5){S} \uput[u](1.4,4.2){T}
\uput[ul](2.2,2){$\vect{\imath}$}\uput[ur](3.4,2.03){$\vect{\jmath}$}\uput[l](2.6,2.82){$\vect{k}$}
\end{pspicture}
\end{center}

Dans ce repère on a, en particulier C(6~;~8~;~0), F(6~;~0~;~4) et G(6~;~8~;~4).

Une des œuvres exposées est un triangle de verre représenté par le triangle ART qui a pour

sommets A(6~;~0~;~2), R(6~;~3~;~4) et T(3~;~0~;~4), Enfin, S est le point de coordonnées $\left(3~;~\dfrac52~;~0\right)$.

\begin{enumerate}
\item 
	\begin{enumerate}
		\item Vérifier que le triangle ART est isocèle en A.
		\item Calculer le produit scalaire $\vect{\text{AR}} \cdot \vect{\text{AT}}$.
		\item En déduire une valeur approchée à $0,1$ degré près de l'angle $\widehat{\text{RAT}}$.
	\end{enumerate}
\item 
	\begin{enumerate}
		\item Justifier que le vecteur $\vect{n}\begin{pmatrix}2\\-2\\3\end{pmatrix}$ est un vecteur normal au plan (ART).
		\item En déduire une équation cartésienne du plan (ART).
	\end{enumerate}
\item Un rayon laser dirigé vers le triangle ART est émis du plancher à partir du point S. On admet que ce rayon est orthogonal au plan (ART).
	\begin{enumerate}
		\item Soit $\Delta$ la droite orthogonale au plan (ART) et passant par le point S.
		
Justifier que le système ci-dessous est une représentation paramétrique de la droite $\Delta$ :

\[\left\{\begin{array}{l c r}
x&=&3+2k\\
y&=& \dfrac52 - 2k\\
z &=& 3k
\end{array}\right.,\: \text{avec }\:k \in \R.\]

		\item Soit L le point d'intersection de la droite $\Delta$, avec le plan (ART).

Démontrer que L a pour coordonnées $\left(5~;~\dfrac12~;~3\right)$.
	\end{enumerate}
\item L'artiste installe un rail représenté  par le segment  [DK] ou K est le milieu du segment [EH].

Sur ce rail, il positionne une source lumineuse laser en un point N du segment [DK] et il oriente ce
second rayon laser vers le point S.

\begin{center}
\psset{unit=1cm,arrowsize=2pt 3}
\begin{pspicture}(7.2,5.3)
%\psgrid
\pspolygon(0.2,1.2)(4.5,0.2)(4.5,2.7)(0.2,3.7)%BCGF
\psline(4.5,0.2)(6.9,1.2)(6.9,3.7)(4.5,2.7)%CDHG
\psline(6.9,3.7)(2.6,4.7)(0.2,3.7)%HEF
\psline(0.2,2.45)(1.8,3.33)(1.4,4.2)%ART
\psline[linestyle=dashed,linewidth=1.5pt](0.2,2.45)(1.4,4.2)
\psline[linestyle=dotted,linewidth=1.5pt](0.2,1.2)(2.6,2.2)(2.6,4.7)%BOE
\psline[linestyle=dotted,linewidth=1.5pt](2.6,2.2)(6.9,1.2)%OD
\psline[linestyle=dotted,linewidth=1.5pt](4.75,4.2)(6.9,1.2)%KD
\psline{->}(2.6,2.2)(2.2,2.)%vect i
\psline{->}(2.6,2.2)(3.4,2.03)%vect j
\psline{->}(2.6,2.2)(2.6,2.82)%\vect k
\psline[ArrowInside=->](5.2,3.6)(2.7,1.5)(0.8,3.1)%NSL
\psdots(5.2,3.6)(2.7,1.5)(0.8,3.1)%NSL
\uput[l](0.2,2.45){A} \uput[dl](0.2,1.2){B} \uput[d](4.5,0.2){C} \uput[dr](6.9,1.2){D}
\uput[u](2.6,4.7){E} \uput[ul](0.2,3.7){F} \uput[dr](4.5,2.7){G} \uput[ur](6.9,3.7){H}
\uput[d](2.6,2.2){O} \uput[ur](1.8,3.33){R} \uput[d](2.7,1.5){S} \uput[u](1.4,4.2){T}
\uput[u](2.2,2){\small $\vect{\imath}$}\uput[u](3.4,2.03){\small $\vect{\jmath}$}\uput[u](2.6,2.82){\small $\vect{k}$}
\uput[ur](5.2,3.6){N}\uput[u](4.75,4.2){K}\uput[d](0.8,3.1){L}
\end{pspicture}
\end{center}

	\begin{enumerate}
		\item Montrer que, pour tout réel $t$ de l'intervalle [0~;~1], le point N de coordonnées $(0~;~8 - 4t~;~4t)$ est un point du segment [DK].
		\item  Calculer les coordonnées exactes du point N tel que les deux rayons laser représentés par les segments [SL] et [SN] soient perpendiculaires.
	\end{enumerate}
\end{enumerate}

\bigskip

\textbf{\textsc{Exercice 4} \quad (7 points)\hfill Thème : fonction logarithme népérien, probabilités}

\bigskip

\emph{Cet exercice est un questionnaire à choix multiples (QCM)
 qui comprend six questions. Les six questions sont indépendantes. Pour chacune des questions, \textbf{une seule des quatre réponses est exacte}. Le candidat indiquera sur sa copie le numéro de la question suivi de la lettre correspondant à la réponse exacte.\\
Aucune justification n'est demandée.\\
Une réponse fausse, une réponse multiple ou une absence de réponse ne rapporte ni n'enlève aucun point}

\medskip

\textbf{Question 1}

\medskip

Le réel $a$ est définie par $a = \ln (9) + \ln \left(\dfrac{\sqrt{3}}{3} \right)  + \ln \left(\dfrac19 \right)$ est égal à :

\begin{center}
\begin{tabularx}{\linewidth}{*{4}{X}}
\textbf{a.~~}$1 - \dfrac12 \ln (3)$&\textbf{b.~~} $\dfrac12 \ln (3)$&\textbf{c.~~} $3 \ln (3) + \dfrac12$ &\textbf{d.~~} $- \dfrac12 \ln (3)$
\end{tabularx}
\end{center}
 
\medskip

\textbf{Question 2}

\medskip

On note $(E)$ l'équation suivante $\ln x + \ln (x - 10) = \ln 3 + \ln 7$ d'inconnue le réel $x$.

\medskip
\textbf{a.~~}3 est solution de $(E)$.

\textbf{b.~~}$5 - \sqrt{46}$ est solution de $(E)$.

\textbf{c.~~}L'équation $(E)$ admet une unique solution réelle.

\textbf{d.~~}L'équation $(E)$ admet deux solutions réelles.

\medskip

\textbf{Question 3}

\medskip

La fonction $f$ est définie sur l'intervalle $]0~;~+ \infty[$ par l'expression $f(x) = x^2(- 1 + \ln x)$.

On note $\mathcal{C}_f$ sa courbe représentative dans le plan muni d'un repère.

\textbf{a.~~} Pour tout réel $x$ de l'intervalle $]0~;~+ \infty[$, \: $f'(x) = 2x + \dfrac1x$.

\textbf{b.~~}La fonction $f$ est croissante sur l'intervalle $]0~;~+ \infty[$.

\textbf{c.~~}$f'\left(\sqrt{\text{e}} \right)$ est différent de $0$.

\textbf{d.~~}La droite d'équation $y = - \dfrac12 \text{e}$ est tangente à la courbe $\mathcal{C}_f$ au point d'abscisse $\sqrt{\text{e}}$.

\medskip

\textbf{Question 4}

\medskip

Un sac contient 20 jetons jaunes et 30 jetons bleus. On tire successivement et avec remise 5 jetons du sac.

La probabilité de tirer exactement 2 jetons jaunes, arrondie au millième, est :

\begin{center}
\begin{tabularx}{\linewidth}{*{4}{X}}
\textbf{a.~~}0,683&\textbf{b.~~}0,346 &\textbf{c.~~}0,230&\textbf{d.~~} 0,165
\end{tabularx}
\end{center}

\medskip

\textbf{Question 5}

\medskip

Un sac contient 20 jetons jaunes et 30 jetons bleus. On tire successivement et avec remise 5 jetons du sac.

La probabilité de tirer au moins un jeton jaune, arrondie au milllième, est :

\begin{center}
\begin{tabularx}{\linewidth}{*{4}{X}}
\textbf{a.~~}0,078&\textbf{b.~~}0,259 &\textbf{c.~~}0,337&\textbf{d.~~} 0,922
\end{tabularx}
\end{center}

\medskip

\textbf{Question 6}

\medskip

Un sac contient $20$ jetons jaunes et $30$ jetons bleus.

On réalise l'expérience aléatoire suivante : on tire successivement et avec remise cinq jetons du sac. 

On note le nombre de jetons jaunes obtenus après ces cinq tirages. 

Si on répète cette expérience aléatoire un très grand nombre de fois alors, en moyenne, le nombre de jetons jaunes est égal à :

\begin{center}
\begin{tabularx}{\linewidth}{*{4}{X}}
\textbf{a.~~}0,4&\textbf{b.~~}1,2 &\textbf{c.~~}2&\textbf{d.~~} 2,5
\end{tabularx}
\end{center}
\end{document}
