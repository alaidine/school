\documentclass[11pt]{article}
\usepackage[T1]{fontenc}
\usepackage[utf8]{inputenc}
\usepackage{fourier}
\usepackage[scaled=0.875]{helvet}
\renewcommand{\ttdefault}{lmtt}
\usepackage{makeidx}
\usepackage{amsmath,amssymb}
\usepackage{fancybox}
\usepackage[normalem]{ulem}
\usepackage{pifont}
\usepackage{lscape}
\usepackage{multicol}
\usepackage{mathrsfs}
\usepackage{tabularx}
\usepackage{multirow}
\usepackage{enumitem}
\usepackage{textcomp} 
\newcommand{\euro}{\eurologo{}}
%Tapuscrit : Denis Vergès
%Relecture : 
\usepackage{pst-plot,pst-tree,pstricks,pst-node,pst-text}
\usepackage{pst-eucl}
\usepackage{pstricks-add}
\newcommand{\R}{\mathbb{R}}
\newcommand{\N}{\mathbb{N}}
\newcommand{\D}{\mathbb{D}}
\newcommand{\Z}{\mathbb{Z}}
\newcommand{\Q}{\mathbb{Q}}
\newcommand{\C}{\mathbb{C}}
\usepackage[left=3.5cm, right=3.5cm, top=3cm, bottom=3cm]{geometry}
\newcommand{\vect}[1]{\overrightarrow{\,\mathstrut#1\,}}
\renewcommand{\theenumi}{\textbf{\arabic{enumi}}}
\renewcommand{\labelenumi}{\textbf{\theenumi.}}
\renewcommand{\theenumii}{\textbf{\alph{enumii}}}
\renewcommand{\labelenumii}{\textbf{\theenumii.}}
\def\Oij{$\left(\text{O}~;~\vect{\imath},~\vect{\jmath}\right)$}
\def\Oijk{$\left(\text{O}~;~\vect{\imath},~\vect{\jmath},~\vect{k}\right)$}
\def\Ouv{$\left(\text{O}~;~\vect{u},~\vect{v}\right)$}
\usepackage{fancyhdr}
\usepackage[dvips]{hyperref}
\hypersetup{%
pdfauthor = {APMEP},
pdfsubject = {Baccalauréat spécialité},
pdftitle = {Métropole 13 septembre 2021},
allbordercolors = white,
pdfstartview=FitH} 
\usepackage[french]{babel}
\usepackage[np]{numprint}
\begin{document}
\setlength\parindent{0mm}
\rhead{\textbf{A. P{}. M. E. P{}.}}
\lhead{\small Baccalauréat spécialité}
\lfoot{\small{Métropole}}
\rfoot{\small{13 septembre 2021}}
\pagestyle{fancy}
\thispagestyle{empty}

\begin{center}{\Large\textbf{\decofourleft~Baccalauréat Métropole 13 septembre 2021 J2~\decofourright\\[6pt] ÉPREUVE D'ENSEIGNEMENT DE SPÉCIALITÉ}\\[7pt]Candidats libres}
\end{center}

\vspace{0,25cm}

\textbf{Le candidat traite 4 exercices : les exercices 1, 2 et 3 communs à tous les candidats et un seul des deux exercices A ou B.}

\bigskip

\textbf{\textsc{Exercice 1} \hfill 4 points}

\textbf{Commun à tous les candidats}

\medskip

Une entreprise reçoit quotidiennement de nombreux courriels (courriers électroniques).

Parmi ces courriels, 8\,\% sont du \og spam \fg, c'est-à-dire des courriers à intention publicitaire, voire malveillante, qu'il est souhaitable de ne pas ouvrir.

On choisit au hasard un courriel reçu par l'entreprise.

Les propriétés du logiciel de messagerie utilisé dans l'entreprise permettent d'affirmer que :

\setlength\parindent{1cm}
\begin{itemize}
\item[$\bullet~~$]La probabilité que le courriel choisi soit classé comme \og indésirable\fg{} sachant que c'est un spam est égale à $0,9$.
\item[$\bullet~~$]La probabilité que le courriel choisi soit classé comme \og indésirable\fg{} sachant que ce n'est pas un spam est égale à $0,01$.
\end{itemize}
\setlength\parindent{0cm}

On note :

\setlength\parindent{1cm}
\begin{itemize}
\item[$\bullet~~$]$S$ l'évènement \og le courriel choisi est un spam \fg ;
\item[$\bullet~~$]$I$ l'évènement \og le courriel choisi est classé comme indésirable par le logiciel de messagerie \fg.
\item[$\bullet~~$]$\overline{S}$ et $\overline{I}$ les évènements contraires de $S$ et $I$ respectivement.
\end{itemize}
\setlength\parindent{0cm}

\medskip

\begin{enumerate}
\item Modéliser la situation étudiée par un arbre pondéré, sur lequel on fera apparaître les probabilités associées à chaque branche.
\item 
	\begin{enumerate}
		\item Démontrer que la probabilité que le courriel choisi soit un message de spam et qu'il soit classé indésirable est égale à $0,072$.
		\item Calculer la probabilité que le message choisi soit classé indésirable.
		\item Le message choisi est classé comme indésirable. Quelle est la probabilité que ce soit effectivement un message de spam ? On donnera un résultat arrondi au centième.
	\end{enumerate}
\item On choisit au hasard $50$ courriels parmi ceux reçus par l'entreprise. On admet que ce choix se ramène à un tirage au hasard avec remise de $50$ courriels parmi l'ensemble des courriels reçus par l'entreprise.

On appelle $Z$ la variable aléatoire dénombrant les courriels de spam parmi les $50$ choisis.
	\begin{enumerate}
		\item Quelle est la loi de probabilité suivie par la variable aléatoire $Z$, et quels sont ses paramètres ?
		\item Quelle est la probabilité que, parmi les 50 courriels choisis, deux au moins soient du spam ? On donnera un résultat arrondi au centième.
	\end{enumerate}
\end{enumerate}

\bigskip

\textbf{\textsc{Exercice 2} \hfill 5 points}

\textbf{Commun à tous les candidats}

\medskip

\emph{Cet exercice est un questionnaire à choix multiples. Pour chacune des questions suivantes, une seule des quatre réponses proposées est exacte. Une réponse exacte rapporte un point. Une réponse fausse, une réponse multiple ou l'absence de réponse à une question ne rapporte ni n'enlève de point.\\[5pt]
 Pour répondre, indiquer sur la copie le numéro de la question et la lettre de la réponse choisie. Aucune justification n'est demandée.}

\medskip

Dans l'espace rapporté à un repère orthonormé \Oijk, on considère les points A(1~;~0~;~2), B(2~;~1~;~0), C(0~;~1~;~2) et la droite $\Delta$ dont une représentation paramétrique est :

$\left\{\begin{array}{l c r}
x &=& 1 + 2t\\
y &=& -2 + t\\ 
z&=&4 - t
\end{array}\right., t\, \in \R$.

\medskip

\begin{enumerate}
\item Parmi les points suivants, lequel appartient à la droite $\Delta$ ?
\begin{center}
\begin{tabularx}{\linewidth}{X X}
\textbf{Réponse A :} M$(2~;~1~;~-1)$; & \textbf{Réponse B :} N$(-3~;~-4~;~6)$ ;\\
\textbf{Réponse C :} P$(-3~;~-4~;~2)$ ; & \textbf{Réponse D :} Q$(-5~;~-5~;~1)$.
\end{tabularx}
\end{center}

\item Le vecteur $\vect{\text{AB}}$ admet pour coordonnées : 

\begin{center}
\begin{tabularx}{\linewidth}{X X}
\textbf{Réponse A :} $\begin{pmatrix}1,5\\0,5\\1\end{pmatrix}$;& \textbf{Réponse B :}$\begin{pmatrix}-1\\-1\\2\end{pmatrix}$ ;\\
\textbf{Réponse C :} $\begin{pmatrix}1\\1\\-2\end{pmatrix}$& \textbf{Réponse D :} $\begin{pmatrix}3\\1\\2\end{pmatrix}$.
\end{tabularx}
\end{center}

\item Une représentation paramétrique de la droite (AB) est :

\begin{center}
\begin{tabularx}{\linewidth}{X X}
\textbf{Réponse A :}$\left\{\begin{array}{l c l}x=1+2t\\y = t\\z = 2\end{array}\right., t \in \R$&
\textbf{Réponse B :} $\left\{\begin{array}{l c l}x =2 - t\\y = 1 - t\\z = 2t\end{array}\right., t \in \R$\\
\textbf{Réponse C :} $\left\{\begin{array}{l c l}x = 2 + t\\y = 1 + t\\z = 2t\end{array}\right., t \in \R$&
\textbf{Réponse D :} $\left\{\begin{array}{l c l}x = 1 + t\\y = 1 + t\\z = 2 - 2t\end{array}\right., t \in \R$
\end{tabularx}
\end{center}


\item Une équation cartésienne du plan passant par le point C et orthogonal à la droite $\Delta$ est :

\begin{center}
\begin{tabularx}{\linewidth}{X X}
\textbf{Réponse A :} $x - 2y + 4z - 6 = 0$ ;& \textbf{Réponse B :} $2x + y - z + 1 = 0$ ;\\
\textbf{Réponse C :} $2x + y - z- 1 = 0$ ;& \textbf{Réponse D :} $y + 2z - 5 = 0$.
\end{tabularx}
\end{center}


\item On considère le point D défini par la relation vectorielle $\vect{\text{OD}} = 3\vect{\text{OA}} - \vect{\text{OB}} - \vect{\text{OC}}$.

\begin{center}
\begin{tabularx}{\linewidth}{X X}
\textbf{Réponse A :} \parbox[t]{4cm}{$\vect{\text{AD}},~ \vect{\text{AB}},~ \vect{\text{AC}}$ sont\\ coplanaires ;} &\textbf{Réponse B :} $\vect{\text{AD}} = \vect{\text{BC}}$ ;\\
\textbf{Réponse C :} \parbox[t]{4cm}{D a pour coordonnées\\ $(3~;~-1~;~-1)$ ;} &\textbf{Réponse D :} \parbox[t]{4cm}{les points A, B, C et D\\ sont alignés.}
\end{tabularx}
\end{center}

\end{enumerate}

\bigskip

\textbf{\textsc{Exercice 3} \hfill 6 points}

\textbf{Commun à tous les candidats}

\medskip

\begin{center}\textbf{Partie I}\end{center}

On considère la fonction $f$ définie sur $\R$ par 

\[f(x) = x - \text{e}^{-2x}.\]

On appelle $\Gamma$ la courbe représentative de la fonction $f$ dans un repère orthonormé \Oij.

\medskip

\begin{enumerate}
\item Déterminer les limites de la fonction $f$ en $- \infty$ et en $+ \infty$.
\item Étudier le sens de variation de la fonction $f$ sur $\R$ et dresser son tableau de variation.
\item Montrer que l'équation $f(x) = 0$ admet une unique solution $\alpha$ sur $\R$, dont on donnera une valeur approchée à $10^{-2}$ près.
\item Déduire des questions précédentes le signe de $f(x)$ suivant les valeurs de $x$.
\end{enumerate}

\begin{center}\textbf{Partie II}\end{center}


Dans le repère orthonormé \Oij, on appelle $\mathcal{C}$ la courbe représentative de la fonction $g$ définie sur $\R$ par:

\[g(x) = \text{e}^{-x}.\]

 La courbes $\mathcal{C}$ et la courbe $\Gamma$ (qui représente la fonction $f$ de la Partie I) sont tracées sur le \textbf{graphique donné en annexe qui est à compléter et à rendre avec la copie.}

\smallskip

Le but de cette partie est de déterminer le point de la courbe $\mathcal{C}$ le plus proche de l'origine O du repère et d'étudier la tangente à $\mathcal{C}$ en ce point.

\medskip

\begin{enumerate}
\item Pour tout nombre réel $t$, on note $M$ le point de coordonnées $\left(t~;~\text{e}^{-t}\right)$ de la courbe $\mathcal{C}$.

On considère la fonction $h$ qui, au nombre réel $t$, associe la distance O$M$.

On a donc: $h(t) = \text{O}M$, c'est-à-dire :

\[h(t) = \sqrt{t^2 + \text{e}^{-2t}}\]

	\begin{enumerate}
		\item Montrer que, pour tout nombre réel $t$,

\[h'(t) = \dfrac{f(t)}{\sqrt{t^2 + \text{e}^{-2t}}}.\]

où $f$ désigne la fonction étudiée dans la \textbf{Partie I}.
		\item Démontrer que le point A de coordonnées $\left(\alpha~;~\text{e}^{-\alpha}\right)$ est le point de la courbe $\mathcal{C}$ pour lequel la longueur O$M$ est minimale.
		

Placer ce point sur le \textbf{graphique donné en annexe, à rendre avec la copie}.
	\end{enumerate}
\item On appelle $T$ la tangente en A à la courbe $\mathcal{C}$.
	\begin{enumerate}
		\item Exprimer en fonction de $\alpha$ le coefficient directeur de la tangente $T$.
		
On rappelle que le coefficient directeur de la droite (OA) est égal à $\dfrac{\text{e}^{-\alpha}}{\alpha}$.

On rappelle également le résultat suivant qui pourra être utilisé sans démonstration:

\emph{Dans un repère orthonormé du plan, deux droites $D$ et $D'$ de coefficients directeurs respectifs $m$ et $m'$ sont perpendiculaires si, et seulement si le produit $mm'$ est égal à $-1$.}

		\item Démontrer que la droite (OA) et la tangente $T$ sont perpendiculaires. 
		
Tracer ces droites sur le \textbf{graphique donné en annexe, à rendre avec la copie.}
	\end{enumerate}
\end{enumerate}

\bigskip

\textbf{EXERCICE au choix du candidat \hfill 5 points}

\medskip

\textbf{Le candidat doit traiter un seul des deux exercices A ou B.\\[5pt]
Il indique sur sa copie l'exercice choisi: exercice A ou exercice B.\\[5pt]Pour éclairer son choix, les principaux domaines abordés dans chaque exercice sont indiqués dans un encadré.}

\medskip

\textbf{Exercice A}

\medskip

\begin{tabular}{|l|}\hline
Principaux domaines abordés :\\
Suites numériques ; raisonnement par récurrence.\\ \hline
\end{tabular}

\medskip

On considère les suites $\left(u_n\right)$ et $\left(v_n\right)$ définies par:

\[u_0 = 16 \quad ;\quad  v_0 = 5 \;;\]

et pour tout entier naturel $n$ :

\renewcommand\arraystretch{2}
\[\left\{\begin{array}{l c l}
u_{n+1}&=&\dfrac{3u_n + 2v_n}{5}\\
v_{n+1}&=& \dfrac{u_n + v_n}{2}
\end{array}\right.\]
\renewcommand\arraystretch{1}

\smallskip

\begin{enumerate}
\item Calculer $u_1$ et $v_1$.
\item On considère la suite $\left(w_n\right)$ définie pour tout entier naturel $n$ par : $w_n = u_n - v_n$.
	\begin{enumerate}
		\item Démontrer que la suite $\left(w_n\right)$ est géométrique de raison 0,1.
		
En déduire, pour tout entier naturel $n$, l'expression de $w_n$ en fonction de $n$.
		\item Préciser le signe de la suite $\left(w_n\right)$  et la limite de cette suite.
	\end{enumerate}
\item 
	\begin{enumerate}
		\item Démontrer que, pour tout entier naturel $n$, on a : $u_{n+1} - u_n = - 0,4 w_n$.
		\item En déduire que la suite $\left(u_n\right)$ est décroissante.

\end{enumerate}
		
On peut démontrer de la même manière que la suite $\left(v_n\right)$ est croissante. On admet ce résultat, et on remarque qu'on a alors: pour tout entier naturel $n$,\, $v_n \geqslant v_0 = 5$.
\begin{enumerate}[resume]
		\item Démontrer par récurrence que, pour tout entier naturel $n$, on a : $u_n \geqslant 5$. 
		
En déduire que la suite $\left(u_n\right)$ est convergente. On appelle $\ell$ la limite de $\left(u_n\right)$.
	\end{enumerate}
\end{enumerate}
	
On peut démontrer de la même manière que la suite $\left(v_n\right)$ est convergente. On admet ce résultat, et on appelle $\ell'$ la limite de $\left(v_n\right)$.
\begin{enumerate}[resume]
\item 
	\begin{enumerate}
		\item Démontrer que $\ell = \ell'$.
		\item On considère la suite $\left(c_n\right)$ définie pour tout entier naturel $n$ par : $c_n = 5u_n + 4v_n$.
		
Démontrer que la suite $\left(c_n\right)$ est constante, c'est-à-dire que pour tout entier naturel $n$, on a : $c_{n+1} = c_n$. 

En déduire que, pour tout entier naturel $n$ ,\, $c_n = 100$.
		\item Déterminer la valeur commune des limites  $\ell$ et $\ell'$.
	\end{enumerate}
\end{enumerate}

\bigskip

\textbf{Exercice B}

\medskip

\begin{tabular}{|l|}\hline
Principaux domaines abordés :\\
Fonction logarithme, limites, dérivation.\\ \hline
\end{tabular}

\medskip

\textbf{Partie 1}

\medskip

Le graphique ci-dessous donne la représentation graphique dans un repère orthonormé de la fonction $f$
définie sur l'intervalle $]0~;~+\infty[$ par:

\[f(x) = \dfrac{2\ln (x) - 1}{x}.\]

\begin{center}
\psset{unit=1.75cm}
\begin{pspicture*}(-0.6,-2.2)(5,1)
\psaxes[linewidth=1.25pt,labelFontSize=\scriptstyle]{->}(0,0)(0,-2.1)(5,1)
\psplot[plotpoints=2000,linewidth=1.25pt,linecolor=red]{0.1}{5}{x ln 2 mul 1 sub x div}
\end{pspicture*}
\end{center}

\medskip

\begin{enumerate}
\item Déterminer par le calcul l'unique solution $\alpha$ de l'équation $f(x) = 0$.

On donnera la valeur exacte de $\alpha$ ainsi que la valeur arrondie au centième.
\item Préciser, par lecture graphique, le signe de $f(x)$ lorsque $x$ varie dans l'intervalle $]0~;~+\infty[$.
\end{enumerate}

\bigskip

\textbf{Partie II}

\medskip
On considère la fonction $g$ définie sur l'intervalle $]0~;~+\infty[$ par:

\[g(x) = [\ln (x)]^2 - \ln (x).\]

\begin{enumerate}
\item 
	\begin{enumerate}
		\item Déterminer la limite de la fonction $g$ en $0$.
		\item Déterminer la limite de la fonction $g$ en $+ \infty$.
	\end{enumerate}
\item On note $g'$ la fonction dérivée de la fonction $g$ sur l'intervalle $]0~;~+\infty[$.

Démontrer que, pour tout nombre réel $x$ de $]0~;~+\infty[$, on a : $g'(x) = f(x)$, où $f$ désigne la fonction définie dans la partie I.
\item Dresser le tableau de variations de la fonction $g$ sur l'intervalle $]0~;~+\infty[$.

On fera figurer dans ce tableau les limites de la fonction $g$ en $0$ et en $+\infty$, ainsi que la valeur du minimum de $g$ sur $]0~;~+\infty[$.
\item Démontrer que, pour tout nombre réel $m > - 0,25$, l'équation $g(x) = m$ admet exactement deux solutions.
\item Déterminer par le calcul les deux solutions de l'équation $g(x) = 0$.
\end{enumerate}

\newpage

\begin{center}

\textbf{\large Annexe à compléter et à rendre avec la copie}

\vspace{1.5cm}

\textbf{Exercice 3}

\vspace{1.5cm}

\psset{unit=3.25cm,comma=true}
\begin{pspicture*}(-1,-0.75)(3.25,2.6)
\psgrid[gridlabels=0pt,subgriddiv=2,gridwidth=0.08pt](-1,-1)(4,3)
\psaxes[linewidth=1.25pt,Dx=0.5,Dy=0.5]{->}(0,0)(-0.99,-0.75)(3.25,2.6)
\uput[l](1.5,1.5){\red $\Gamma$}\uput[ur](-0.75,2){\blue $\mathcal{C}$}
\psplot[plotpoints=2000,linecolor=red,linewidth=1.25pt]{-1}{3.25}{x 1 2.71828 2 x mul exp div sub}
\psplot[plotpoints=2000,linecolor=blue,linewidth=1.25pt]{-1}{3.25}{1 2.71828  x  exp div}
\end{pspicture*}

\end{center}
\end{document}