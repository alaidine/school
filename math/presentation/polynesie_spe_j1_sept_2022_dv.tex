\documentclass[11pt]{article}
\usepackage[T1]{fontenc}
\usepackage[utf8]{inputenc}
\usepackage{fourier}
\usepackage[scaled=0.875]{helvet}
%\renewcommand{\ttdefault}{lmtt}
\usepackage{makeidx}
\usepackage{amsmath,amssymb}
\usepackage{fancybox}
\usepackage[normalem]{ulem}
\usepackage{pifont}
\usepackage{lscape}
\usepackage{multicol}
\usepackage{mathrsfs}
\usepackage{tabularx}
\usepackage{multirow}
\usepackage{enumitem}
\usepackage{textcomp}
\newcommand{\euro}{\eurologo{}}
%Tapuscrit : Denis Vergès
%Relecture : François Hache
\usepackage{pst-plot,pst-tree,pst-all,pst-bezier,pst-node,pst-text,pst-eucl,pstricks-add}
\usepackage{pst-infixplot} 
\usepackage{twoopt}
\newcommand{\R}{\mathbb{R}}
\newcommand{\N}{\mathbb{N}}
\newcommand{\D}{\mathbb{D}}
\newcommand{\Z}{\mathbb{Z}}
\newcommand{\Q}{\mathbb{Q}}
\newcommand{\C}{\mathbb{C}}
\usepackage[left=3.5cm, right=3.5cm, top=2.5cm, bottom=3cm]{geometry}
\newcommand{\vect}[1]{\overrightarrow{\,\mathstrut#1\,}}
\renewcommand{\theenumi}{\textbf{\arabic{enumi}}}
\renewcommand{\labelenumi}{\textbf{\theenumi.}}
\renewcommand{\theenumii}{\textbf{\alph{enumii}}}
\renewcommand{\labelenumii}{\textbf{\theenumii.}}
\def\Oij{$\left(\text{O}~;~\vect{\imath},~\vect{\jmath}\right)$}
\def\Oijk{$\left(\text{O}~;~\vect{\imath},~\vect{\jmath},~\vect{k}\right)$}
\def\Ouv{$\left(\text{O}~;~\vect{u},~\vect{v}\right)$}
\usepackage{fancyhdr}
\usepackage[dvips]{hyperref}
\hypersetup{%
pdfauthor = {APMEP},
pdfsubject = {Baccalauréat spécialité},
pdftitle = {Polynésie 30 août 2022 sujet 1},
allbordercolors = white,
pdfstartview=FitH} 
\usepackage[french]{babel}
\usepackage[np]{numprint}
\begin{document}
\setlength\parindent{0mm}
\rhead{\textbf{A. P{}. M. E. P{}.}}
\lhead{\small Baccalauréat spécialité}
\lfoot{\small{Polynésie}}
\rfoot{\small{30 août 2022}}
\pagestyle{fancy}
\thispagestyle{empty}

\begin{center}{\Large\textbf{\decofourleft~Baccalauréat Polynésie 30 août 2022~\decofourright\\[6pt] ÉPREUVE D'ENSEIGNEMENT DE SPÉCIALITÉ sujet \no 1}}

\bigskip

Durée de l'épreuve : \textbf{4 heures}

\medskip

L'usage de la calculatrice avec mode examen actif est autorisé

\medskip

Le sujet propose 4 exercices

Le candidat choisit 3 exercices parmi les 4 et \textbf{ne doit traiter que ces 3 exercices}
\end{center}

\bigskip

\textbf{\textsc{Exercice 1} \quad 7 points\hfill probabilités}

\medskip

Parmi les angines, un quart nécessite la prise d'antibiotiques, les autres non.

Afin d'éviter de prescrire inutilement des antibiotiques, les médecins disposent d'un test de diagnostic ayant les caractéristiques suivantes :

\begin{itemize}
\item[$\bullet~~$]lorsque l'angine nécessite la prise d'antibiotiques, le test est positif dans 90\,\% des cas;
\item[$\bullet~~$]lorsque l'angine ne nécessite pas la prise d'antibiotiques, le test est négatif dans 95\,\% des cas.
\end{itemize}

\medskip

Les probabilités demandées dans la suite de l'exercice seront arrondies à $10^{-4}$ près si nécessaire.

\bigskip

\textbf{Partie 1}

\medskip

Un patient atteint d'angine et ayant subi le test est choisi au hasard. 

On considère les évènements suivants :

\begin{itemize}
\item[$\bullet~~$]$A$ : « le patient est atteint d'une angine nécessitant la prise d'antibiotiques »{} ; 
\item[$\bullet~~$]$T$ : « le test est positif »{} ;
\item[$\bullet~~$]$\overline{A}$ et $\overline{T}$ sont respectivement les évènements contraires de $A$ et $T$.
\end{itemize}

\medskip

\begin{enumerate}
\item Calculer $P(A \cap T)$. On pourra s'appuyer sur un arbre pondéré.
\item Démontrer que $P(T) = \np{0,2625}$.
\item On choisit un patient ayant un test positif. Calculer la probabilité qu'il soit atteint d'une angine nécessitant la prise d'antibiotiques.
\item
	\begin{enumerate}
		\item Parmi les évènements suivants, déterminer ceux qui correspondent à un résultat erroné du test :  $A \cap T,\: \overline{A} \cap T,\:A \cap \overline{T},\: \overline{A} \cap \overline{T}$.
		\item On définit l'évènement $E$ : « le test fournit un résultat erroné ». 
		
Démontrer que $p(E) = \np{0,0625}$.
	\end{enumerate}
\end{enumerate}

\bigskip

\textbf{Partie 2}

\medskip

On sélectionne au hasard un échantillon de $n$ patients qui ont été testés.

On admet que l'on peut assimiler ce choix d'échantillon à un tirage avec remise.

On note $X$ la variable aléatoire qui donne le nombre de patients de cet échantillon ayant un test erroné.

\medskip

\begin{enumerate}
\item On suppose que $n = 50$.
	\begin{enumerate}
		\item Justifier que la variable aléatoire $X$ suit une loi binomiale $\mathcal{B}(n,\: p)$ de paramètres
$n = 50$ et $p = \np{0,0625}.$
		\item Calculer $P(X = 7)$.
		\item Calculer la probabilité qu'il y ait au moins un patient dans l'échantillon dont le test est erroné.
	\end{enumerate}	
\item Quelle valeur minimale de la taille de l'échantillon faut-il choisir pour que $P(X \geqslant 10)$ soit supérieure à $0,95$ ?
\end{enumerate}

\bigskip

\textbf{\textsc{Exercice 2} \quad 7 points\hfill suites, fonctions}

\medskip

Soit $k$ un nombre réel.

On considère la suite $\left(u_n\right)$ définie par son premier terme $u_0$ et pour tout entier naturel $n$,

\[u_{n+1} = ku_n\left(1- u_n\right).\]

Les deux parties de cet exercice sont indépendantes. 

On y étudie deux cas de figure selon les valeurs de $k$.

\bigskip

\textbf{Partie 1}

\medskip

Dans cette partie, $k = 1,9$ et $u_0 = 0,1$.

On a donc,  pour tout entier naturel $n$,\: $u_{n+1} = 1,9u_n\left(1 - u_n\right)$.

\medskip

\begin{enumerate}
\item On considère la fonction $f$ définie sur [0~;~1] par $f(x) = 1,9x(1 - x)$.
	\begin{enumerate}
		\item Étudier les variations de $f$ sur l'intervalle [0~;~1].
		\item En déduire que si $x \in [0~;~1]$ alors $f(x) \in [0~;~1]$.
	\end{enumerate}	
\item Ci-dessous sont représentés les premiers termes de la suite $\left(u_n\right)$ construits à partir de la courbe $\mathcal{C}_f$ de la fonction $f$ et de la droite $D$ d'équation $y = x$.

Conjecturer le sens de variation de la suite $\left(u_n\right)$ et sa limite éventuelle.

\medskip

%%début
\psset{unit=10cm,arrowsize=2pt 3,comma=true}
\def\xmin{-0.3} \def\xmax{1.05}
\def\ymin{-0.2} \def\ymax{0.65} 
\begin{pspicture*}(\xmin,\ymin)(\xmax,\ymax)

%%% grille 1cm x 1cm
\psgrid[unit=1cm,gridlabels=0,subgriddiv=1,gridcolor=gray](0,0)(1,1)
%%% axes
\psaxes[linewidth=1.5pt,Dx=0.1,Dy=0.1,labelFontSize=\scriptstyle]{->}(0,0)(0,0)(\xmax,\ymax)
%%% origine
\uput[dl](0,0){O}
%%% définition de la fonction
\def\f{1.9 x mul  1 x sub mul}
%%% tracé de la représentation graphique de la fonction
\psplot[plotpoints=2000,linewidth=1.25pt,linecolor=blue]{0}{1}{\f}
%%%% nom de la courbe
\uput[10](0.8,0.3){\blue $\mathscr{C}_f$}
%%% droite d'équation y = x
\psplot[plotpoints=1000,linecolor=rb!80]{0}{1}{x}
\uput[d]{45}(0.65,0.65){\rb $D : y=x$}
%%% \pstfixpoint[options]{abscisse début}{fonction}{nb itérations}
\psFixpoint[linecolor=red,ArrowInside=-<]{0.1}{\f}{3}
\psline[linestyle=dashed](0.1,0.171)(0,0.171)\uput[l](0,0.171){\small $u_1 = f\left(u_0\right)$}
\uput[d](0.1,-0.075){\small $u_0$}
\psline[linestyle=dashed](0.171,0.171)(0.171,0)\uput[d](0.171,-0.075){\small $u_1$}
\psline[linestyle=dashed](0.269,0.269)(0.269,0)\uput[d](0.269,-0.075){\small $u_2$}
\psline[linestyle=dashed](0.374,0.374)(0.374,0)\uput[d](0.374,-0.075){\small $u_3$}
\psline[linecolor=red](0.374,0.374)(0.374,0.445)
\end{pspicture*}
%%fin
\item 
	\begin{enumerate}
		\item En utilisant les résultats de la question 1, démontrer par récurrence que pour tout entier naturel $n$ :
		
\[0 \leqslant u_n \leqslant u_{n+1} \leqslant \dfrac12.\]

		\item En déduire que la suite $\left(u_n\right)$ converge.
		\item Déterminer sa limite.
	\end{enumerate}
\end{enumerate}

\bigskip

\textbf{Partie 2}

\medskip

 Dans cette partie, $k= \dfrac12$  et $u_0 = \dfrac14$.

On a donc, pour tout entier naturel $n$,\: $u_{n+1} = \dfrac12 u_n \left(1 - u_n\right)$ et $u_0 = \dfrac14$.

On admet que pour tout entier naturel $n$ :\: 0 \leqslant u_n \leqslant \left(\dfrac12\right)^n$.

\medskip

\begin{enumerate}
\item Démontrer que la suite $\left(u_n\right)$ converge et déterminer sa limite.
\item On considère la fonction Python \texttt{algo (p)} où \texttt{p} désigne un entier naturel non nul:

\begin{center}
\begin{tabular}{|l|}\hline
\texttt{def algo(p) :}\\
\quad  \texttt{u =1/4}\\
\quad \texttt{n = 0}\\
\quad \texttt{while u > 10**(-p):}\\
\quad \quad \texttt{u = 1/2*u*(1 - u)}\\
\quad \quad \texttt{n = n+1}\\
\quad \texttt{ return(n)}\\ \hline
\end{tabular}
\end{center}
Expliquer pourquoi, pour tout entier naturel non nul \texttt{p}, la boucle while ne tourne pas indéfiniment, ce qui permet à la commande \texttt{algo (p)} de renvoyer une valeur.
\end{enumerate}

\bigskip

\textbf{\textsc{Exercice 3} \quad 7 points\hfill fonctions}

\bigskip

\textbf{Partie 1}

\medskip

Soit $g$ la fonction définie pour tout nombre réel $x$ de l'intervalle ]0~;~+ \infty[$ par:

\[g(x) = \dfrac{2 \ln x}{ x}.\]

\smallskip

\begin{enumerate}
\item On note $g'$ la dérivée de $g$. Démontrer que pour tout réel $x$ strictement positif : 

\[g'(x) = \dfrac{2 - 2\ln x}{x^2}.\]

\item On dispose de ce tableau de variations de la fonction $g$ sur l'intervalle $]0~;~+ \infty[$ : 

\begin{center}
\psset{unit=1cm}
\begin{pspicture}(9,3)
%\psgrid
\psframe(9,3)\psline(0,2.5)(9,2.5)
\psline(2.5,0)(2.5,3)
\uput[u](1.25,2.4){$x$}\uput[u](3,2.4){$0$}\uput[u](5.5,2.4){1}\uput[u](7,2.4){e}\uput[u](8.5,2.4){$+ \infty$}
\psline(2.95,0)(2.95,2.5)\psline(3.05,0)(3.05,2.5)
\uput[u](3.5,0){$- \infty$}\rput(5.5,1.25){0}\uput[d](7,2.5){$\frac{2}{\text{e}}$}\uput[u](8.8,0){0}
\rput(1.25,1.5){Variations }
\rput(1.25,1.){de $g$}
\psline(4,0.32)(5.3,1.1)\psline{->}(5.6,1.3)(6.8,2.)\psline{->}(7.2,2)(8.6,0.4)
\end{pspicture}
\end{center}

\smallskip

Justifier les informations suivantes lues dans ce tableau:

	\begin{enumerate}
		\item la valeur $\dfrac{2}{\text{e}}$ ;
		\item les variations de la fonction $g$ sur son ensemble de définition ;
		\item les limites de la fonction $g$ aux bornes de son ensemble de définition.
	\end{enumerate}
\item En déduire le tableau de signes de la fonction $g$ sur l'intervalle $]0~;~+ \infty[$.
\end{enumerate}

\bigskip

\textbf{Partie 2}

\medskip

Soit $f$ la fonction définie sur l'intervalle ]0~;~ + \infty[$ par 

\[f(x) = [\ln (x)]^2.\]

 Dans cette partie, chaque étude est effectuée sur l'intervalle $]0~;~+ \infty[$.

\medskip

\begin{enumerate}
\item Démontrer que sur l'intervalle $]0~;~+ \infty[$, la fonction $f$ est une primitive de la fonction $g$.
\item À l'aide de la partie 1, étudier :
	\begin{enumerate}
		\item la convexité de la fonction $f$ ;
		\item les variations de la fonction $f$.
\end{enumerate}
\item 
	\begin{enumerate}
		\item Donner une équation de la tangente à la courbe représentative de la fonction $f$ au point d'abscisse e.
		\item En déduire que, pour tout réel $x$ dans $]0~;~\text{e}]$ :
		
\[[\ln (x)]^2 \geqslant \dfrac{2}{\text{e}}x - 1.\]

	\end{enumerate}
\end{enumerate}
\bigskip

\textbf{\textsc{Exercice 4} \quad 7 points\hfill géométrie dans le plan et dans l'espace}

\medskip

On considère le cube ABCDEFGH.

On note I le milieu du segment[EH] et on considère le triangle CFI.

L'espace est muni du repère orthonormé $\left(\text{A}~;~ \vect{\text{AB}},\: \vect{\text{AD}},\: \vect{\text{AE}}\right)$ et on admet que le point I a
pour coordonnées $\left(0~;~\dfrac12;~1\right)$ dans ce repère.

\begin{center}
\psset{unit=1cm}
\begin{pspicture}(7.4,8)
\pspolygon(0.2,0.8)(4.6,0.2)(4.6,4.6)(0.2,5.2)%BCGF
\psline(4.6,0.2)(7.2,2.6)(7.2,7)(4.6,4.6)%CDHG
\psline(7.2,7)(2.8,7.6)(0.2,5.2)%HEF
\psline[linestyle=dashed](0.2,0.8)(2.8,3.2)(7.2,2.6)%BAD
\psline[linestyle=dashed](2.8,3.2)(2.8,7.6)%AE
\psline(5,7.3)(0.2,5.2)(4.6,0.2)%IFC
\psline[linestyle=dashed](5,7.3)(4.6,0.2)%IC
\psdots(2.8,3.2)(0.2,0.8)(4.6,0.2)(7.2,2.6)(2.8,7.6)(0.2,5.2)(4.6,4.6)(7.2,7)(5,7.3)(2.4,4.9)
\uput[l](2.8,3.2){A} \uput[l](0.2,0.8){B} \uput[dr](4.6,0.2){C} \uput[r](7.2,2.6){D}
\uput[u](2.8,7.6){E} \uput[l](0.2,5.2){F} \uput[dl](4.6,4.6){G} \uput[ur](7.2,7){H}
\uput[u](5,7.3){I} \uput[u](2.4,4.9){J}
\end{pspicture}

\smallskip

\begin{enumerate}
\item 
	\begin{enumerate}
		\item Donner sans justifier les coordonnées des points C, F et G.
		\item Démontrer que le vecteur $\vect{n}\begin{pmatrix}1\\2\\2\end{pmatrix}$ est normal au plan (CFI).
		\item Vérifier qu'une équation cartésienne du plan (CFI) est : $x + 2y + 2z - 3 = 0$.
	\end{enumerate}	
\item  On note $d$ la droite passant par G et orthogonale au plan (CFI).
	\begin{enumerate}
		\item Déterminer une représentation paramétrique de la droite $d$.
		\item Démontrer que le point K$\left(\dfrac79~;~\dfrac59~;~\dfrac59\right)$ est le projeté orthogonal du point G sur le plan (CFI).
		\item Déduire des questions précédentes que la distance du point G au plan (CFI) est
égale à $\dfrac23$.
	\end{enumerate}
\item On considère la pyramide GCFI.

\emph{On rappelle que le volume $V$ d'une pyramide est donné par la formule} 

\[V = \frac13 \times b \times h,\]

\emph{où $b$ est l'aire d'une base et $h$ la hauteur associée à cette base}.
	\begin{enumerate}
		\item Démontrer que le volume de la pyramide GCFI est égal à $\dfrac16$, exprimé en unité de volume.
		\item En déduire l'aire du triangle CFI, en unité d'aire.
	\end{enumerate}
\end{enumerate}
\end{document}

%#1 lettre désignant la suite, par défaut u 
%#2 indice initial, par défaut 0 
%#3 valeur du terme initial 
%#4 expression de f 
%#5 nombre de termes 
\newcounter{INDICE} 
\newcommandtwoopt{\SuiteRec}[5][u][0]{% 
\getcoor % récupère les coordonnées de la fenêtres, merci JCC 
\infixtoRPN{#4} 
\psplot{\xmin}{\xmax}{\RPN} 
\SpecialCoor 
%\psaxes{->}(0,0)(\xmin,\ymin)(\xmax,\ymax) 
%\psplot[linewidth=1pt,algebraic,VarStep,VarStepEpsilon=0.0001]{\xmin}{\xmax}{#4} 
\psline[linestyle=dashed](\xmin,\xmin)(\xmax,\xmax)  
\pstVerb{ /x #3 def } 
\psline[ArrowInside=->,ArrowInsideNo=1](! x 0 )(! x \RPN ) 
\psline[ArrowInside=->,ArrowInsideNo=1](! x \RPN )(! \RPN\space dup) 
\psline{-}(! x 0.03 )(! x 0.03 neg) 
\rput(!x 0.08 neg ){$#1_{#2}$} 
\pstVerb{ /x \RPN\space def} 
\setcounter{INDICE}{#2} 
\addtocounter{INDICE}{1} 
\multido{\I=\theINDICE+1}{#5}{% 
\psline{-}(! x 0.03 )(! x 0.03 neg) 
\rput(!x 0.08 neg ){$#1_{\I}$} 
\psline[linestyle=dotted](! x 0 )(! x x) 
\psline[ArrowInside=->,ArrowInsideNo=1](! x x )(! x \RPN ) 
\psline[ArrowInside=->,ArrowInsideNo=1](! x \RPN )(! \RPN\space dup) 
\pstVerb{ /x \RPN\space def} 
}
}
\newcommandtwoopt{\SuiteRecbis}[5][u][0]{% 
\getcoor % récupère les coordonnées de la fenêtres, merci JCC 
\infixtoRPN{1.9 x mul 1 x sub mul} 
\SpecialCoor 
\psline[linestyle=dashed](\xmin,\xmin)(\xmax,\xmax)  
\pstVerb{ /x #3 def } 
\psline[ArrowInside=->,ArrowInsideNo=1](! x 0 )(! x \RPN ) 
\psline[ArrowInside=->,ArrowInsideNo=1](! x \RPN )(! \RPN\space dup) 
\psline{-}(! x 0.03 )(! x 0.03 neg) 
\rput(!x 0.08 neg ){$#1_{#2}$} 
\pstVerb{ /x \RPN\space def} 
\setcounter{INDICE}{#2} 
\addtocounter{INDICE}{1} 
\multido{\I=\theINDICE+1}{4}{% 
\psline{-}(! x 0.03 )(! x 0.03 neg) 
\rput(!x 0.08 neg ){$#1_{\I}$} 
\psline[linestyle=dotted](! x 0 )(! x x) 
\psline[ArrowInside=->,ArrowInsideNo=1](! x x )(! x \RPN ) 
\psline[ArrowInside=->,ArrowInsideNo=1](! x \RPN )(! \RPN\space dup) 
\pstVerb{ /x \RPN\space def} 
}
}
