\documentclass[11pt]{article}
\usepackage[T1]{fontenc}
\usepackage[utf8]{inputenc}
\usepackage{fourier}
\usepackage[scaled=0.875]{helvet}
%\renewcommand{\ttdefault}{lmtt}
\usepackage{makeidx}
\usepackage{amsmath,amssymb}
\usepackage{fancybox}
\usepackage[normalem]{ulem}
\usepackage{pifont}
\usepackage{lscape}
\usepackage{multicol}
\usepackage{mathrsfs}
\usepackage{tabularx}
\usepackage{multirow}
\usepackage{enumitem}
\usepackage{textcomp}
\newcommand{\euro}{\eurologo{}}
%Merci pour le sujet à Florence Vigne et Sébastien Dibos
%Tapuscrit : Denis Vergès
\usepackage{pst-plot,pst-tree,pst-all,pst-node,pst-text,pst-eucl,pstricks-add}%,pst-bezier
\usepackage{pst-infixplot} 
\usepackage{twoopt}
\newcommand{\R}{\mathbb{R}}
\newcommand{\N}{\mathbb{N}}
\newcommand{\D}{\mathbb{D}}
\newcommand{\Z}{\mathbb{Z}}
\newcommand{\Q}{\mathbb{Q}}
\newcommand{\C}{\mathbb{C}}
\usepackage[left=3.5cm, right=3.5cm, top=2.5cm, bottom=3cm]{geometry}
\newcommand{\vect}[1]{\overrightarrow{\,\mathstrut#1\,}}
\newcommand{\vectt}[1]{\overrightarrow{\,\mathstrut\text{#1}\,}}
\renewcommand{\theenumi}{\textbf{\arabic{enumi}}}
\renewcommand{\labelenumi}{\textbf{\theenumi.}}
\renewcommand{\theenumii}{\textbf{\alph{enumii}}}
\renewcommand{\labelenumii}{\textbf{\theenumii.}}
\def\Oij{$\left(\text{O}~;~\vect{\imath},~\vect{\jmath}\right)$}
\def\Oijk{$\left(\text{O}~;~\vect{\imath},~\vect{\jmath},~\vect{k}\right)$}
\def\Ouv{$\left(\text{O}~;~\vect{u},~\vect{v}\right)$}
\usepackage{fancyhdr}
\usepackage[dvips]{hyperref}
\hypersetup{%
pdfauthor = {APMEP},
pdfsubject = {Baccalauréat spécialité},
pdftitle = {Nouvelle-Calédonie 26 octobre 2022 sujet 1},
allbordercolors = white,
pdfstartview=FitH}
\usepackage[french]{babel}
\DecimalMathComma
\usepackage[np]{numprint}

\newcommand{\ds}{\displaystyle}
\newcommand{\e}{\text{\,e\,}}
\setlength\parindent{0mm}
\setlength\parskip{3pt}

\begin{document}

\rhead{\textbf{A. P{}. M. E. P{}.}}
\lhead{\small Baccalauréat spécialité}
\lfoot{\small{Nouvelle-Calédonie Jour 1}}
\rfoot{\small{26 octobre 2022}}
\pagestyle{fancy}
\thispagestyle{empty}

\begin{center}{\Large\textbf{\decofourleft~Baccalauréat Nouvelle-Calédonie 26 octobre 2022 Jour 1~\decofourright\\[7pt] ÉPREUVE D'ENSEIGNEMENT DE SPÉCIALITÉ }}

\bigskip

Durée de l'épreuve : \textbf{4 heures}

\medskip

L'usage de la calculatrice avec mode examen actif est autorisé

\medskip

Le sujet propose 4 exercices

Le candidat choisit 3 exercices parmi les 4 et \textbf{ne doit traiter que ces 3 exercices}
\end{center}

\bigskip

\textbf{\textsc{Exercice 1} \quad 7 points\hfill}

\medskip

\textbf{Principaux domaines abordés :} fonctions, fonction logarithme; convexité.

\medskip

On considère la fonction $f$ définie sur l'intervalle $]0~;~+\infty[$ par

\[f(x) = x^2 - 6x + 4\ln (x).\]

On admet que la fonction $f$ est deux fois dérivable sur l'intervalle $]0~;~+ \infty[$.

On note $f$' sa dérivée et $f''$ sa dérivée seconde.

On note $\mathcal{C}_f$ la courbe représentative de la fonction $f$ dans un repère orthogonal.

\medskip

\begin{enumerate}
\item 
	\begin{enumerate}
		\item Déterminer $\displaystyle\lim_{x \to 0} f(x)$.
	
Interpréter graphiquement ce résultat.
		\item Déterminer $\displaystyle\lim_{x \to + \infty} f(x)$.
	\end{enumerate}	
\item
	\begin{enumerate}
		\item Déterminer $f'(x)$ pour tout réel $x$ appartenant à $]0~;~+ \infty[$.

		\item Étudier le signe de $f'(x)$ sur l'intervalle $]0~;~+ \infty[$.
		
En déduire le tableau de variations de $f$.
	\end{enumerate}
\item Montrer que l'équation $f(x) = 0$ admet une unique solution dans l'intervalle [4~;~5].
\item On admet que, pour tout $x$ de $]0~;~+ \infty[$, on a :

\[f''(x) = \dfrac{2x^2 - 4}{x^2}.\]

	\begin{enumerate}
		\item Étudier la convexité de la fonction $f$ sur $]0~;~+ \infty[$.
		
On précisera les valeurs exactes des coordonnées des éventuels points d'inflexion de $\mathcal{C}_f$.
		\item On note A le point de coordonnées $\left(\sqrt 2~;~f\left(\sqrt 2~\right)\right)$.
		
Soit $t$ un réel strictement positif tel que $t \ne \sqrt 2$. Soit $M$ le point de coordonnées $(t~;~ f(t))$.

En utilisant la question 4. a, indiquer, selon la valeur de $t$, les positions relatives du segment [A$M$] et de la courbe $\mathcal{C}_f$.
	\end{enumerate}
\end{enumerate}

\bigskip

\textbf{\textsc{Exercice 2} \quad 7 points\hfill}

\medskip

\textbf{Principaux domaines abordés :}
suites ;
fonctions, fonction exponentielle.

\medskip

On considère la fonction $f$ définie sur $\R$ par

\[f(x) = x^3\e^x.\]

On admet que la fonction $f$ est dérivable sur $\R$ et on note $f'$ sa fonction dérivée.

\medskip

\begin{enumerate}
\item On définit la suite $\left(u_n\right)$ par $u_0 = - 1$ et, pour tout entier naturel $n,\: u_{n+1} = f\left(u_n\right)$.
	\begin{enumerate}
		\item Calculer $u_1$ puis $u_2$.
		
On donnera les valeurs exactes, puis les valeurs approchées à $10^{-3}$.
		\item On considère la fonction \texttt{fonc}, écrite en langage Python ci-dessous.

\smallskip

\begin{minipage}{0.524\linewidth}
On rappelle qu'en langage Python, 

\og \texttt{i in range (n)}\fg{} signifie que 

\texttt{i} varie de 0 à \texttt{n -1}.
\end{minipage}\hfill
\begin{minipage}{0.40\linewidth}
\begin{tabular}{|l l|}\hline
\texttt{def}& \texttt{fonc (n)} :\\
&\texttt{u =- 1}\\
&\texttt{for i in range(n) :}\\
&\quad \texttt{u=u**3*exp(u)}\\
&\texttt{return u}\\\hline
\end{tabular}
\end{minipage}

Déterminer, sans justifier, la valeur renvoyée par \texttt{fonc (2)} arrondie à $10^{-3}$.
	\end{enumerate}
\item 
	\begin{enumerate}
		\item Démontrer que, pour tout $x$ réel, on a $f'(x) = x^2\e^x(x + 3)$.
		\item Justifier que le tableau de variations de $f$ sur $\R$ est celui représenté
ci-dessous :

\begin{center}
\psset{unit=1cm,arrowsize=2pt 3}
\begin{pspicture}(7,2.5)
\psframe(7,2.5)
\psline(0,2)(7,2)\psline(1,0)(1,2.5)
\uput[u](0.5,1.9){$x$} \uput[u](1.2,1.9){$- \infty$} \uput[u](4,1.9){$-3$} \uput[u](6.5,1.9){$+ \infty$} 
\rput(0.5,1){$f$}\uput[d](1.2,2){0}\uput[u](4,0){$- 27\e^{-3}$}\uput[d](6.5,2){$+ \infty$}
\psline{->}(1.5,1.5)(3.5,0.5)\psline{->}(4.5,0.52)(6.5,1.5)
\end{pspicture}
\end{center}

		\item Démontrer, par récurrence, que pour tout entier naturel $n$, on a :
		
\[- 1 \leqslant u_n \leqslant u_{n+1} \leqslant 0.\]
		
		\item En déduire que la suite $\left(u_n\right)$ est convergente.
		\item On note $\ell$ la limite de la suite $\left(u_n\right)$.

On rappelle que $\ell$ est solution de l'équation $f(x) = x$.

Déterminer $\ell$. (Pour cela, on admettra que l'équation $x^2\e^x - 1 = 0$ possède une seule solution dans $\R$ et que celle-ci est strictement supérieure à $\dfrac12$).
	\end{enumerate}
\end{enumerate}

\bigskip

\textbf{\textsc{Exercice 3} \quad 7 points\hfill }

\medskip

\textbf{Principaux domaines abordés :} géométrie dans l'espace.

\medskip

Une maison est constituée d'un parallélépipède rectangle ABCDEFGH surmonté d'un prisme EFIHGJ dont une base est le triangle EIF isocèle en I.

Cette maison est représentée ci-dessous.

\begin{center}
\psset{unit=1cm,arrowsize=2pt 3}
\begin{pspicture}(12.2,6)
\pspolygon(0.4,3.4)(0.4,1.5)(1.9,0.4)(1.9,2.3)%HDAE
\psline(1.9,0.4)(7.9,0.8)(7.9,2.7)(1.9,2.3)%ABFE
\psline[linestyle=dashed](0.4,1.5)(6.4,1.9)(6.4,3.8)(0.4,3.4)%DCGH
\psline[linestyle=dashed](7.9,0.8)(6.4,1.9)%BC
\pspolygon(7.9,2.7)(6.4,3.8)(3.4,5.5)(4.9,4.4)%FGJI
\psline(4.9,4.4)(1.9,2.3)%IE
\psline(3.4,5.5)(0.4,3.4)%JH
\psline[linewidth=1.5pt]{->}(1.9,0.4)(3.9,0.533)
\psline[linewidth=1.5pt]{->}(1.9,0.4)(1.15,0.95)
\psline[linewidth=1.5pt]{->}(1.9,0.4)(1.9,2.3)
\psline(11.6,0.3)(7.9,1.2)
\psline[linestyle=dashed](7.9,1.2)(7.15,1.35)
\uput[d](1.9,0.4){A} \uput[dr](7.9,0.8){B} \uput[r](6.4,1.9){C} \uput[l](0.4,1.5){D}
\uput[u](1.9,2.3){E} \uput[r](7.9,2.7){F} \uput[ur](6.4,3.8){G} \uput[ul](0.4,3.4){H}
\uput[l](4.9,4.4){I} \uput[u](3.4,5.5){J} \uput[r](11.6,0.3){R} \uput[dl](1.7,0.6){$\vect{\jmath}$}
\uput[dr](2.9,0.45){$\vect{\imath}$} \uput[r](1.9,1.35){$\vect{k}$}
\psdots(11.6,0.3)
\end{pspicture}
\end{center}

On a AB $= 3$,\quad AD $= 2$,\quad AE $= 1$.

On définit les vecteurs $\vect{\imath}= \dfrac13\vect{\text{AB}},\:\vect{\jmath}= \dfrac12\vect{\text{AD}}, \:\vect{k} = \vect{\text{AE}}$.

On munit ainsi l'espace du repère orthonormé $\left(\text{A}~;~\vect{\imath},~\vect{\jmath},~\vect{k}\right)$.

\medskip

\begin{enumerate}
\item Donner les coordonnées du point G.
\item Le vecteur $\vect{n}$ de coordonnées $(2~;~0~;~-3)$ est vecteur normal au plan (EHI).

Déterminer une équation cartésienne du plan (EHI).
\item Déterminer les coordonnées du point I.
\item Déterminer une mesure au degré près de l'angle $\widehat{\text{EIF}}$.
\item Afin de raccorder la maison au réseau électrique, on souhaite creuser une tranchée rectiligne depuis un relais électrique situé en contrebas de la maison.

Le relais est représenté par le point R de coordonnées $(6~;~- 3~;~- 1)$.

La tranchée est assimilée à un segment d'une droite $\Delta$ passant par R et dirigée par le vecteur $\vect{u}$ de coordonnées $(-3~;~4~;~1)$. On souhaite vérifier que la tranchée atteindra la maison au niveau de l'arête [BC].
	\begin{enumerate}
		\item Donner une représentation paramétrique de la droite $\Delta$.
		\item On admet qu'une équation du plan (BFG) est $x = 3$.
		
Soit K le point d'intersection de la droite $\Delta$ avec le plan (BFG).

Déterminer les coordonnées du point K.
		\item Le point K appartient-il bien à l'arête [BC] ?
	\end{enumerate}
\end{enumerate}

\bigskip

\textbf{\textsc{Exercice 4} \quad 7 points\hfill}

\medskip

\textbf{Principaux domaines abordés :} 
probabilités.

\medskip

\emph{Cet exercice est un questionnaire à choix multiples.\\
Pour chacune des questions suivantes, une seule des quatre réponses proposées est exacte.\\
Une réponse fausse, une réponse multiple ou l'absence de réponse à une question ne rapporte ni n'enlève de point.\\
Pour répondre, indiquer sur la copie le numéro de la question et la lettre de la réponse choisie. Aucune justification n'est demandée.}

\medskip

On considère un système de communication binaire transmettant des $0$
et des $1$.

Chaque $0$ ou $1$ est appelé bit.

En raison d'interférences, il peut y avoir des erreurs de transmission:

un $0$ peut être reçu comme un $1$ et, de même, un $1$ peut être reçu comme un $0$.

Pour un bit choisi au hasard dans le message, on note les évènements :

\begin{minipage}{0.48\linewidth}

\begin{itemize}
\item[$\bullet~~$] $E_0$ : \og le bit envoyé est un $0$ \fg{} ;
\item[$\bullet~~$] $E_1$ : \og le bit envoyé est un 1 \fg{} ;
\item[$\bullet~~$] $R_0$ : \og le bit reçu est un $0$\fg{} 
\item[$\bullet~~$] $R_1$ : \og le bit reçu est un $1$ \fg.
\end{itemize}
\end{minipage}\hfill
\begin{minipage}{0.48\linewidth}
\begin{center}
\pstree[treemode=R,nodesepA=0pt,nodesepB=2.5pt,treesep = 1cm,levelsep=2.5cm]{\TR{}}
{\pstree{\TR{$E_0$~}\taput{0,4}}
{\TR{$R_0$}\taput{\ldots}
\TR{$R_1$}\tbput{0,01}
}
\pstree{\TR{$E_1$~}\tbput{\ldots}}
{\TR{$R_0$}\taput{0,02}
\TR{$R_1$}\tbput{\ldots}
}}
\end{center}
\end{minipage}

On sait que:

$p\left(E_0\right) = 0,4 \:;\quad p_{E_0}\left(R_1\right) = 0,01 \:;\quad p_{E_1}\left(R_0\right) = 0,02$.

On rappelle que la probabilité conditionnelle de $A$ sachant $B$ est notée $p_B(A)$.

On peut ainsi représenter la situation par l'arbre de probabilités ci-dessus.

\medskip

\begin{enumerate}
\item La probabilité que le bit envoyé soit un $0$ et que le bit reçu soit un $0$ est égale à :

\begin{center}
\begin{tabularx}{\linewidth}{*{4}{X}}
\textbf{a.~~}0,99 &\textbf{b.~~}0,396 &\textbf{c.~~}0,01 &\textbf{d.~~} 0,4
\end{tabularx}
\end{center}

\item La probabilité  $p\left(R_0\right)$ est égale à :
\begin{center}
\begin{tabularx}{\linewidth}{*{4}{X}}
\textbf{a.~~}0,99 &\textbf{b.~~}0,02 &\textbf{c.~~}0,408 &\textbf{d.~~}0,931
\end{tabularx}
\end{center}
\item Une valeur, approchée au millième, de la probabilité $p_{R_1}\left(E_0\right)$ est égale 
\begin{center}
\begin{tabularx}{\linewidth}{*{4}{X}}
\textbf{a.~~}0,004 &\textbf{b.~~}0,001 &\textbf{c.~~}0,007 &\textbf{d.~~}0,010
\end{tabularx}
\end{center}
\item La probabilité de l'évènement \og  il y a une erreur de transmission \fg{} est égale à :
\begin{center}
\begin{tabularx}{\linewidth}{*{4}{X}}
\textbf{a.~~}0,03 &\textbf{b.~~}0,016 &\textbf{c.~~}0,16 &\textbf{d.~~}0,015
\end{tabularx}
\end{center}
\end{enumerate}

Un message de longueur huit bits est appelé un octet.

On admet que la probabilité qu'un octet soit transmis sans erreur est égale à 0,88.

\begin{enumerate}[resume]
\item On transmet successivement 10 octets de façon indépendante.

La probabilité, à $10^{-3}$ près, qu'exactement 7 octets soient transmis sans erreur est égale à :

\begin{center}
\begin{tabularx}{\linewidth}{*{4}{X}}
\textbf{a.~~}0,915 &\textbf{b.~~}0,109 &\textbf{c.~~}0,976 &\textbf{d.~~}0,085
\end{tabularx}
\end{center}
\item On transmet successivement 10 octets de façon indépendante.

La probabilité qu'au moins 1 octet soit transmis sans erreur est égale à :

\begin{center}
\begin{tabularx}{\linewidth}{*{4}{X}}
\textbf{a.~~} $1 - 0,12^{10}$ &\textbf{b.~~}$0,12^{10}$ &\textbf{c.~~}$0,88^{10}$ &\textbf{d.~~} $1- 0,88^{10}$
\end{tabularx}
\end{center}
\item Soit $N$ un entier naturel. On transmet successivement $N$ octets de façon indépendante. 

Soit $N_0$ la plus grande valeur de $N$ pour laquelle la probabilité que les $N$ octets soient tous transmis sans erreur est supérieure ou égale à $0,1$.

On peut affirmer que:

\begin{center}
\begin{tabularx}{\linewidth}{*{4}{X}}
\textbf{a.~~}$N_0 = 17$ &\textbf{b.~~}$N_0 = 18$ &\textbf{c.~~}$N_0 = 19$ &\textbf{d.~~}$N_0 = 20$
\end{tabularx}
\end{center}
\end{enumerate}
\end{document}