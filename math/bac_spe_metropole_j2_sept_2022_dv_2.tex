\documentclass[11pt]{article}
\usepackage[T1]{fontenc}
\usepackage[utf8]{inputenc}
\usepackage{fourier}
\usepackage[scaled=0.875]{helvet}
\renewcommand{\ttdefault}{lmtt}
\usepackage{amsmath,amssymb,makeidx}
\usepackage{fancybox}
\usepackage{tabularx}
\usepackage{graphicx}
\usepackage[normalem]{ulem}
\usepackage{pifont}
\usepackage{lscape}
\usepackage{multicol}
\usepackage{diagbox}
\usepackage{multirow} 
\usepackage{textcomp} 
\newcommand{\euro}{\eurologo{}}
\usepackage{enumitem}
%Tapuscrit : Denis Vergès
\usepackage{pstricks,pst-plot,pst-text,pst-tree,pstricks-add}
\newcommand{\R}{\mathbb{R}}
\newcommand{\N}{\mathbb{N}}
\newcommand{\D}{\mathbb{D}}
\newcommand{\Z}{\mathbb{Z}}
\newcommand{\Q}{\mathbb{Q}}
\newcommand{\C}{\mathbb{C}}
\usepackage[left=3.5cm, right=3.5cm, top=3cm, bottom=3cm,headheight=14pt]{geometry}
\newcommand{\vect}[1]{\overrightarrow{\,\mathstrut#1\,}}
\renewcommand{\theenumi}{\textbf{\arabic{enumi}}}
\renewcommand{\labelenumi}{\textbf{\theenumi.}}
\renewcommand{\theenumii}{\textbf{\alph{enumii}}}
\renewcommand{\labelenumii}{\textbf{\theenumii.}}
\def\Oij{$\left(\text{O},~\vect{\imath},~\vect{\jmath}\right)$}
\def\Oijk{$\left(\text{O},~\vect{\imath},~\vect{\jmath},~\vect{k}\right)$}
\def\Ouv{$\left(\text{O},~\vect{u},~\vect{v}\right)$}
\usepackage{fancyhdr}
%\usepackage[colorlinks=true,pdfstartview=FitV,linkcolor=blue,citecolor=blue,urlcolor=blue]{hyperref}
%\usepackage[dvips]{hyperref}
%\hypersetup{%
%pdfauthor = {APMEP},
%pdfsubject = {Brevet},
%pdftitle = {Métropole septembre 2022},
%allbordercolors = white,
%pdfstartview=FitH}
\usepackage[french]{babel}
\AtBeginDocument{\DecimalMathComma}
\usepackage[np]{numprint}
\begin{document}
\setlength\parindent{0mm}
\rhead{}
\lhead{\small L'année 2022}
\rfoot{\small Métropole, Antilles-Guyane}
\lfoot{\small 9 septembre 2022}
\pagestyle{fancy}
\thispagestyle{empty}
\begin{center} {\huge \textbf{\decofourleft~Baccalauréat spécialité Jour 2 \decofourright\\[7pt]
Métropole Antilles-Guyane 9 septembre 2022}}
\end{center}

\bigskip

\textbf{Exercice 1 \quad 7 points\hfill Thèmes : probabilités}

\medskip

Dans le magasin d'Hugo, les clients peuvent louer deux types de vélos : vélos de route ou bien vélos tout terrain. 

Chaque type de vélo peut être loué dans sa version électrique ou non.

On choisit un client du magasin au hasard, et on admet que:
\begin{itemize}
\item Si le client loue un vélo de route, la probabilité que ce soit un vélo électrique est de 0,4 ;
\item Si le client loue un vélo tout terrain, la probabilité que ce soit un vélo électrique est de 0,7 ; 
\item La probabilité que le client loue un vélo électrique est de 0,58.
\end{itemize}

\medskip

On appelle $\alpha$ la probabilité que le client loue un vélo de route, avec $0 \leqslant \alpha \leqslant 1$. 

On considère les évènements suivants:

\begin{itemize}
\item[$\bullet~~$] $R$ : \og le client loue un vélo de route \fg{} ;
\item[$\bullet~~$] $E$ : \og le client loue un vélo électrique \fg{} ;
\item[$\bullet~~$] $\overline{R}$ et $\overline{E}$ , évènements contraires de $R$ et $E$.
\end{itemize}

\medskip

\begin{minipage}{0.48\linewidth}

On modélise cette situation aléatoire à l'aide de l'arbre reproduit ci-contre:

Si $F$ désigne un évènement quelconque, on notera $p(F)$ la probabilité de $F$.
\end{minipage} \hfill
\begin{minipage}{0.48\linewidth}
\begin{center}
\pstree[treemode=R,nodesepA=0pt,nodesepB=2.5pt,treesep = 1cm,levelsep=2.5cm]{\TR{}}
{\pstree{\TR{$R$~}\taput{$\alpha$}}
{\TR{$E$}\taput{\ldots}
\TR{$\overline{E}$}\tbput{\ldots}
}
\pstree{\TR{$\overline{R}$~}\tbput{$1 - \alpha$}}
{\TR{$E$}\taput{\ldots}
\TR{$\overline{E}$}\tbput{\ldots}
}}
\end{center}
\end{minipage} 

\medskip

\begin{enumerate}
\item Recopier cet arbre sur la copie et le compléter.
\item 
	\begin{enumerate}
		\item Montrer que $p(E) = 0,7 - 0,3\alpha$.
		\item En déduire que : $\alpha = 0,4$.
	\end{enumerate}
\item On sait que le client a loué un vélo électrique. 

Déterminer la probabilité qu'il ait loué un vélo tout terrain. On donnera le résultat arrondi au centième.
\item Quelle est la probabilité que le client loue un vélo tout terrain électrique ?
\item Le prix de la location à la journée d'un vélo de route non électrique est de $25$ euros, celui d'un vélo tout terrain non électrique de $35$~euros. 

Pour chaque type de vélo, le choix de la version électrique augmente le prix de location à la journée de $15$~euros. 

On appelle $X$ la variable aléatoire modélisant le prix de location d'un vélo à la journée. 
	\begin{enumerate}
		\item Donner la loi de probabilité de $X$. On présentera les résultats sous forme d'un tableau.
		\item Calculer l'espérance mathématique de $X$ et interpréter ce résultat.
	\end{enumerate}	
\item Lorsqu'on choisit $30$ clients d'Hugo au hasard, on assimile ce choix à un tirage avec remise. 

On note $Y$ la variable aléatoire associant à un échantillon de 30 clients choisis au hasard le nombre de clients qui louent un vélo électrique.

On rappelle que la probabilité de l'événement $E$ est : $p(E) = 0,58$.
	\begin{enumerate}
		\item Justifier que $Y$ suit une loi binomiale dont on précisera les paramètres.
		\item Déterminer la probabilité qu'un échantillon contienne exactement $20$ clients qui louent un vélo électrique. On donnera le résultat arrondi au millième.
		\item Déterminer la probabilité qu'un échantillon contienne au moins $15$ clients qui louent un vélo électrique. On donnera le résultat arrondi au millième.
	\end{enumerate}
\end{enumerate}

\bigskip

\textbf{Exercice 2 \quad 7 points\hfill Thèmes : suites, fonctions}

\medskip

\emph{Cet exercice est un questionnaire à choix multiples. Pour chacune des questions suivantes, une seule des quatre réponses proposées est exacte.\\
Une réponse fausse, une réponse multiple ou l'absence de réponse à une question ne rapporte ni n'enlève de point.\\
Pour répondre, indiquer sur la copie le numéro de la question et la lettre de la réponse choisie. Aucune justification n'est demandée.}

\medskip

\begin{enumerate}
\item On considère les suites $\left(a_n\right)$ et $\left(b_n\right)$ définie par $a_0 = 1$ et, pour tout entier naturel $n$,\: $a_{n+1} = 0,5a_n + 1$ et $b_n = a_n - 2$.

On peut affirmer que:

\begin{center}
\begin{tabularx}{\linewidth}{*{2}{X}}
\textbf{a.~} $\left(a_n\right)$ est arithmétique ;&\textbf{b.~} $\left(b_n\right)$ est géométrique;\\
\textbf{c.~} $\left(a_n\right)$ est géométrique;&\textbf{d.~} $\left(b_n\right)$ est arithmétique. 
\end{tabularx}
\end{center}

Dans les questions 2. et 3., on considère les suites $\left(u_n\right)$ et $\left(v_n\right)$ définies par :

\[u_0 = 2,\: v_0 = 1 \:\: \text{et, pour tout entier naturel }\:\:n : \left\{\begin{array}{l c l}
u_{n+1}&=&u_n + 3v_n\\
v_{n+1}&=&u_n + v_n.
\end{array}\right.\]

\item On peut affirmer que :

\begin{center}
\begin{tabularx}{\linewidth}{*{4}{X}}
\textbf{a.~}$\left\{\begin{array}{l c l}
u_2& =& 5\\ v_2 &=& 3
\end{array}\right.$&\textbf{b.~} $u_2^2 - 3v_2^2 = - 2^2$&\textbf{c.~} $\dfrac{u_2}{v_2} = 1,75$&\textbf{d.~} $5u_1 = 3v_1$.\\
\end{tabularx}
\end{center}

\item On considère le programme ci-dessous écrit en langage Python :

\begin{center}
\begin{tabular}{|>{\texttt}l|}\hline
\texttt{def valeurs() :}\\
\quad \texttt{u = 2}\\
\quad \texttt{v = 1}\\
\quad \texttt{for k in range(1,11)}\\
\quad \quad \texttt{c = u}\\
\quad \quad \texttt{u = u + 3*v}\\
\quad \quad \texttt{v = c + v}\\
\quad \texttt{return (u, v)}\\ \hline
\end{tabular}
\end{center}

Ce programme renvoie :

\begin{center}
\begin{tabularx}{\linewidth}{*{4}{X}}
\textbf{a.~} $u_{11}$ et $v_{11}$ ;&
\textbf{b.~} $u_{10}$ et $v_{11}$ ;&
\textbf{c.~} les valeurs de $u_n$ et $v_n$ pour $n$ allant de 1 à 10;&
\textbf{d.~} $u_{0 }$ et $v_{10}$.
\end{tabularx}
\end{center}
\end{enumerate}

Pour les questions 4. et 5., on considère une fonction $f$ deux fois dérivable sur l'intervalle
$[-4~;~2]$. On note $f'$ la fonction dérivée de $f$ et $f''$  la dérivée seconde de $f$.

On donne ci-dessous la courbe représentative $\mathcal{C}'$ de la fonction dérivée $f'$ dans un repère du plan. On donne de plus les points A$(-2~;~0)$, B(1~;~0) et C(0~;~5).

\begin{minipage}{0.52\linewidth}
\begin{enumerate}[resume]
\item La fonction $f$ est :

\begin{center}
\begin{tabularx}{\linewidth}{*{2}{X}}
\textbf{a.~} concave sur $[-2~;~1]$ ; 	&\textbf{b.~} convexe sur $[-4~;~0]$;\\
\textbf{c.~}convexe sur $[-2~;~1]$;		&\textbf{d.~} convexe sur [0~;~2].
\end{tabularx}
\end{center}

\item On admet que la droite (BC) est la tangente à la courbe $\mathcal{C}'$ au point B.
On a :

\begin{center}
\begin{tabularx}{\linewidth}{*{2}{X}}
\textbf{a.~} $f'(1) < 0$ ;	&\textbf{b.~} $f'(1) = 5$ ;\\
\textbf{c.~} $f''(1) > 0$;	&\textbf{d.~} $f''(1) = - 5$.
\end{tabularx}
\end{center}
\end{enumerate}
\end{minipage}\hfill
\begin{minipage}{0.45\linewidth}
\begin{center}
\psset{unit=1cm,arrowsize=2pt 3}
\begin{pspicture*}(-4,-3)(2,6)
\psaxes[linewidth=1.25pt,Dx=10,Dy=10]{->}(0,0)(-4,-3)(2,6)
\psaxes[linewidth=1.25pt](0,0)(1,1)
\psplot[plotpoints=2000,linewidth=1.25pt,linecolor=blue]{-4}{2}{x 2 add 1 x  sub mul 2.71828 x 0.5 mul exp mul}\uput[dr](-4,-1.5){\blue $\mathcal{C}'$}
\psline(-4,0.1)(-4,-0.1)
\uput[dr](-2,0){A} \uput[ur](1,0){B} \uput[l](0,5){C} 
\psplot[plotpoints=2000,linewidth=1.25pt]{-4}{2}{5 5 x  mul sub}
\uput[d](-4,0){$-4$}
\end{pspicture*}
\end{center}

\end{minipage}

\begin{enumerate}[resume,start=6]
\item Soit $f$ la fonction définie sur $\R$ par $f(x) = \left(x^2 + 1\right)\text{e}^x$.

La primitive $F$ de $f$ sur $\R$  telle que $F(0) = 1$ est définie par:

\begin{center}
\begin{tabularx}{\linewidth}{*{2}{X}}
\textbf{a.~} $F(x) = \left(x^2 - 2x +3\right)\text{e}^x$ ;&\textbf{b.~} $F(x) = \left(x^2 - 2x + 3\right)\text{e}^x - 2;$\\
\textbf{c.~} $F(x) = \left(\dfrac13 x^3 + x\right)\text{e}^x + 1$ ;&\textbf{d.~} $F(x) = \left(\dfrac13 x^3 + x \right) \text{e}^x$.\\
\end{tabularx}
\end{center}
\end{enumerate}

\bigskip

\textbf{Exercice 3 \quad 7 points\hfill Thèmes : fonction logarithme, suites}

\medskip

\textbf{Les parties B et C sont indépendantes}

\medskip

On considère la fonction $f$  définie sur $]0~;~+\infty[$ par 

\[f(x) = x - x \ln x,\]

où ln désigne la fonction logarithme népérien. 

\bigskip

\textbf{Partie A}

\medskip

\begin{enumerate}
\item Déterminer la limite de $f(x)$ quand $x$ tend vers $0$.
\item Déterminer la limite de $f(x)$ quand $x$ tend vers $+\infty$.
\item On admet que la fonction $f$ est dérivable sur $]0~;~+\infty[$ et on note $f'$ sa fonction dérivée.
	\begin{enumerate}
		\item Démontrer que, pour tout réel $x > 0$, on a : $f'(x) = - \ln x$.
		\item En déduire les variations de la fonction $f$ sur $]0~;~+\infty[$ et dresser son tableau de
variations.
	\end{enumerate}
\item Résoudre l'équation $f(x) = x$ sur $]0~;~+\infty[$.
\end{enumerate}

\bigskip

\textbf{Partie B}

\medskip

Dans cette partie, on pourra utiliser avec profit certains résultats de la partie A. 

On considère la suite $\left(u_n\right)$ définie par:

\[\left\{\begin{array}{l c l}
u_0 &=& 0,5\\
u_{n+1} &=& u_n - u_n \ln u_n \:\text{ pour tout entier naturel } n,
\end{array}\right.\]

Ainsi, pour tout entier naturel $n$, on a : $u_{n+1} = f\left(u_n\right)$.

\medskip

\begin{enumerate}
\item On rappelle que la fonction $f$ est croissante sur l'intervalle [0,5~;~1].

Démontrer par récurrence que, pour tout entier naturel $n$, on a : $0,5 \leqslant  u_n \leqslant u_{n+1} \leqslant 1$.
\item 
	\begin{enumerate}
		\item Montrer que la suite $\left(u_n\right)$ est convergente.
		\item On note $\ell$ la limite de la suite $\left(u_n\right)$. Déterminer la valeur de $\ell$.
	\end{enumerate}
\end{enumerate}

\bigskip

\textbf{Partie C}

\medskip

Pour un nombre réel $k$ quelconque, on considère la fonction $f_k$ définie sur $]0~;~+\infty[$ par:

\[f_k(x) = kx - x \ln x.\]

\begin{enumerate}
\item Pour tout nombre réel $k$, montrer que $f_k$ admet un maximum $y_k$ atteint en $x_k = \text{e}^{k- 1}$.
\item  Vérifier que, pour tout nombre réel $k$, on a : $x_k = y_k$.
\end{enumerate}

\bigskip

\textbf{Exercice 4 \quad 7 points\hfill Thèmes : géométrie dans l'espace }

\medskip

Dans l'espace rapporté à un repère orthonormé \Oijk, on considère:

\begin{itemize}
\item[$\bullet~~$] la droite $\mathcal{D}$ passant par le point A(2~;~4~;~0) et dont un vecteur directeur est $\vect{u}\begin{pmatrix}1\\2\\0\end{pmatrix}$ ;
\item[$\bullet~~$] la droite $\mathcal{D}'$ dont une représentation paramétrique est : $\left\{\begin{array}{l c l}
x&=&3\\
y&=& 3 + t\\
z&=&3 + t
\end{array}\right. ,\: t \in \R$.
\end{itemize}

\medskip

\begin{enumerate}
\item 

	\begin{enumerate}
		\item Donner les coordonnées d'un vecteur directeur $\vect{u'}$ de la droite $\mathcal{D}'$.
		\item Montrer que les droites $\mathcal{D}$ et $\mathcal{D}'$ ne sont pas parallèles.
		\item Déterminer une représentation paramétrique de la droite $\mathcal{D}$.
	\end{enumerate}
\end{enumerate}

On admet dans la suite de cet exercice qu'il existe une unique droite $\Delta$ perpendiculaire aux droites $\mathcal{D}$ et $\mathcal{D}'$. Cette droite $\Delta$ coupe chacune des droites $\mathcal{D}$ et $\mathcal{D}'$. On appellera M le point d'intersection de $\Delta$ et $\mathcal{D}$, et M$'$ le point d'intersection de $\Delta$ et $\mathcal{D}'$.

On se propose de déterminer la distance MM$'$ appelée \og distance entre les droites $\mathcal{D}$ et $\mathcal{D}'$ \fg.
\begin{enumerate}[resume]
\item Montrer que le vecteur $\vect{v}\begin{pmatrix}2\\- 1\\1\end{pmatrix}$ est un vecteur directeur de la droite $\Delta$.
\item On note $\mathcal{P}$ le plan contenant les droites $\mathcal{D}$ et $\Delta$, c'est-à-dire le plan passant par le point A et de vecteurs directeurs $\vect{u}$ et $\vect{v}$.
	\begin{enumerate}
		\item Montrer que le vecteur $\vect{n}\begin{pmatrix}2\\-1\\-5 \end{pmatrix}$ est un vecteur normal au plan $\mathcal{P}$.
		\item En déduire qu'une équation du plan $\mathcal{P}$ est : $2x - y - 5z = 0$.
		\item On rappelle que M$'$ est le point d'intersection des droites $\Delta$ et $\mathcal{D}'$.
		
Justifier que M$'$ est également le point d'intersection de $\mathcal{D}'$ et du plan $\mathcal{P}$.
		
En déduire que les coordonnées du point M$'$ sont (3~;~1~;~1).
	\end{enumerate}
\item 
	\begin{enumerate}
		\item Déterminer une représentation paramétrique de la droite $\Delta$.
		\item Justifier que le point M a pour coordonnées (1~;~2~;~0).
		\item Calculer la distance MM$'$.
	\end{enumerate}
\item On considère la droite $d$ de représentation paramétrique $\left\{\begin{array}{l c l}
x &=& 5t\\y &=& 2 + 5t \\z&=&1 + t\end{array}\right.$ avec $t \in \R$.
	\begin{enumerate}
		\item Montrer que la droite $d$ est parallèle au plan $\mathcal{P}$.
		\item On note $\ell$ la distance d'un point N de la droite $d$ au plan $\mathcal{P}$ . 
		
Exprimer le volume du  tétraèdre ANMM$'$ en fonction de $\ell$.

On rappelle que le volume d'un tétraèdre est donné par : $V = \dfrac13 \times B \times h$ où $B$ désigne 
l'aire d'une base et $h$ la hauteur relative à cette base.
		\item Justifier que, si N$_1$ et N$_2$ sont deux points quelconques de la droite $d$, les tétraèdres AN$_1$MM$'$ et AN$_2$MM$'$ ont le même volume.
	\end{enumerate}
\end{enumerate}
\end{document}