\documentclass[11pt]{article}
\usepackage[T1]{fontenc}
\usepackage[utf8]{inputenc}
\usepackage{fourier}
\usepackage[scaled=0.875]{helvet}
\renewcommand{\ttdefault}{lmtt}
\usepackage{makeidx}
\usepackage{amsmath,amssymb}
\usepackage{fancybox}
\usepackage[normalem]{ulem}
\usepackage{pifont}
\usepackage{lscape}
\usepackage{multicol}
\usepackage{mathrsfs}
\usepackage{tabularx}
\usepackage{multirow}
\usepackage{enumitem}
\usepackage{textcomp} 
\newcommand{\euro}{\eurologo{}}
%Merci à David Hamet
%Tapuscrit : Denis Vergès
%Relecture : 
\usepackage{pst-plot,pst-tree,pstricks,pst-node,pst-text}
\usepackage{pst-eucl}
\usepackage{pstricks-add}
\newcommand{\R}{\mathbb{R}}
\newcommand{\N}{\mathbb{N}}
\newcommand{\D}{\mathbb{D}}
\newcommand{\Z}{\mathbb{Z}}
\newcommand{\Q}{\mathbb{Q}}
\newcommand{\C}{\mathbb{C}}
\usepackage[left=3.5cm, right=3.5cm, top=3cm, bottom=3cm]{geometry}
\newcommand{\vect}[1]{\overrightarrow{\,\mathstrut#1\,}}
\renewcommand{\theenumi}{\textbf{\arabic{enumi}}}
\renewcommand{\labelenumi}{\textbf{\theenumi.}}
\renewcommand{\theenumii}{\textbf{\alph{enumii}}}
\renewcommand{\labelenumii}{\textbf{\theenumii.}}
\def\Oij{$\left(\text{O}~;~\vect{\imath},~\vect{\jmath}\right)$}
\def\Oijk{$\left(\text{O}~;~\vect{\imath},~\vect{\jmath},~\vect{k}\right)$}
\def\Ouv{$\left(\text{O}~;~\vect{u},~\vect{v}\right)$}
\newcommand{\e}{\text{e}}
\usepackage{fancyhdr}
\usepackage[dvips]{hyperref}
\hypersetup{%
pdfauthor = {APMEP},
pdfsubject = {Baccalauréat spécialité},
pdftitle = {Asie 18 mai 2022},
allbordercolors = white,
pdfstartview=FitH} 
\usepackage[french]{babel}
\usepackage[np]{numprint}
\begin{document}
\setlength\parindent{0mm}
\rhead{\textbf{A. P{}. M. E. P{}.}}
\lhead{\small Baccalauréat spécialité}
\lfoot{\small{Asie}}
\rfoot{\small{18 mai 2022}}
\pagestyle{fancy}
\thispagestyle{empty}

\begin{center}{\Large\textbf{\decofourleft~Baccalauréat Asie 18 mai 2022 Jour 2~\decofourright\\[6pt] ÉPREUVE D'ENSEIGNEMENT DE SPÉCIALITÉ}}

\vspace{0,25cm}

Le sujet propose 4 exercices.

Le candidat choisit 3 exercices parmi les 4 exercices et \textbf{ne doit traiter que ces 3 exercices.}

\medskip

Chaque exercice est noté sur 7 points (le total sera ramené sur 20 points). 

\medskip

Les traces de recherche, même incomplètes ou infructueuses, seront prises en compte.

\hrule
\end{center}

\bigskip

\textbf{\textsc{Exercice 1} \hfill 7 points}

\medskip

\emph{Principaux domaines abordés}: Manipulation des vecteurs, des droites et des plans de l'espace. Orthogonalité et distances dans l'espace.
Représentations paramétriques et équations cartésiennes.

\bigskip

Dans un repère orthonormé \Oijk{} de l'espace,
on considère les points 

\[\text{A}(-3~;~1~;~3), \text{B}(2~;~2~;~3),\, \text{C}(1~;~7~;~-1), \,\text{D}(-4~;~6~;~-1)\: \text{et 
K}(-3~;~14~;~14).\]

\medskip

\begin{enumerate}
\item 
	\begin{enumerate}
		\item Calculer les coordonnées des vecteurs $\vect{\text{AB}},\: \vect{\text{DC}}$ et 
		$\vect{\text{AD}}$.
		\item Montrer que le quadrilatère ABCD est un rectangle.
		\item Calculer l'aire du rectangle ABCD.
	\end{enumerate}
\item 
	\begin{enumerate}
		\item Justifier que les points A, B et D définissent un plan.
		\item Montrer que le vecteur $\vect{n}(-2~;~10~;~13)$ est un vecteur normal au plan (ABD). 
		\item  En déduire une équation cartésienne du plan (ABD).
	\end{enumerate}
\item
	\begin{enumerate}
		\item Donner une représentation paramétrique de la droite $\Delta$ orthogonale au plan (ABD) et qui passe par le point K.
		\item Déterminer les coordonnées du point I, projeté orthogonal du point K sur le plan (ABD).
		\item Montrer que la hauteur de la pyramide KABCD de base ABCD et de sommet K
vaut $\sqrt{273}$.
	\end{enumerate}
\item Calculer le volume $V$ de la pyramide KABCD.

On rappelle que le volume V d'une pyramide est donné par la formule:

\[V= \dfrac13 \times \text{aire de la base} \times \text{hauteur}.\]
\end{enumerate}

\newpage

\textbf{\textsc{Exercice 2} \hfill 7 points}

\medskip

\emph{Principaux domaines abordés}: Étude des fonctions.
Fonction logarithme.

\bigskip

\textbf{Partie A}

\begin{center}
\psset{unit=0.6cm,arrowsize=2pt 3}
\begin{pspicture*}(-4,-5)(15.5,8)
\psgrid[gridlabels=0pt,subgriddiv=1,gridwidth=0.15pt]
\psaxes[linewidth=1.25pt,labelFontSize=\scriptstyle]{->}(0,0)(-4,-5)(15.5,8)
\psplot[plotpoints=2000,linewidth=1.25pt,linecolor=blue]{3.001}{15.5}{x dup mul x sub 6 sub ln}
\psplot[plotpoints=2000,linewidth=1.25pt,linecolor=red,linestyle=dashed]{3.001}{15.5}{x 2 mul 1 sub x dup mul x sub 6 sub div}
\uput[r](3.1,7.7){\red $\mathcal{C}_2$}\uput[r](3.1,-4){\blue $\mathcal{C}_1$}
\end{pspicture*}
\end{center}

Dans le repère orthonormé ci-dessus, sont tracées les courbes représentatives d'une fonction $f$ et de sa fonction dérivée, notée $f'$, toutes deux définies sur $]3~;~+\infty[$.

\medskip

\begin{enumerate}
\item Associer à chaque courbe la fonction qu'elle représente. Justifier.
\item Déterminer graphiquement la ou les solutions éventuelles de l'équation $f(x) = 3$. 
\item Indiquer, par lecture graphique, la convexité de la fonction $f$.
\end{enumerate}

\bigskip

\textbf{Partie B}

\medskip

\begin{enumerate}
\item Justifier que la quantité $\ln \left(x^2- x- 6\right)$ est bien définie pour les valeurs $x$ de l'intervalle $]3~;~ +\infty[$, que l'on nommera $I$ dans la suite.
\item On admet que la fonction $f$ de la Partie A est définie par $f(x) = \ln \left(x^2- x- 6\right)$ sur $I$. 

Calculer les limites de la fonction $f$ aux deux bornes de l'intervalle $I$.

En déduire une équation d'une asymptote à la courbe représentative de la fonction $f$ sur $I$.
\item 
	\begin{enumerate}
		\item Calculer $f'(x)$ pour tout $x$ appartenant à $I$.
		\item Étudier le sens de variation de la fonction $f$ sur $I$.
		
Dresser le tableau des variations de la fonction $f$ en y faisant figurer les limites aux bornes de $I$.
	\end{enumerate}
\item 
	\begin{enumerate}
		\item Justifier que l'équation $f(x) = 3$ admet une unique solution $\alpha$ sur l'intervalle ]5~;~ 6[.
		\item Déterminer, à l'aide de la calculatrice, un encadrement de $\alpha$ à $10^{-2}$ près.
	\end{enumerate}
\item 
	\begin{enumerate}
		\item Justifier que $f''(x) = \dfrac{- 2x^2 + 2x - 13}{\left(x^2 - x - 6\right)^2}$.
		\item Étudier la convexité de la fonction $f$ sur $I$.
	\end{enumerate}
\end{enumerate}

\bigskip

\textbf{\textsc{Exercice 3} \hfill 7 points}

\medskip

\emph{Principaux domaines abordés}: Probabilités conditionnelles et indépendance. Variables aléatoires.

\medskip

\begin{center}\emph{Les deux parties de cet exercice sont indépendantes.}\end{center}

\textbf{Partie 1}

\medskip

Julien doit prendre l'avion; il a prévu de prendre le bus pour se rendre à l'aéroport. 

S'il prend le bus de $8$~h, il est sûr d'être à l'aéroport à temps pour son vol.

Par contre, le bus suivant ne lui permettrait pas d'arriver à temps à l'aéroport.

Julien est parti en retard de son appartement et la probabilité qu'il manque son bus est de $0,8$.

S'il manque son bus, il se rend à l'aéroport en prenant une compagnie de voitures privées; il a alors une probabilité de $0,5$ d'être à l'heure à l'aéroport.

\smallskip

On notera :

\setlength\parindent{12mm}
\begin{itemize}
\item[$\bullet~~$] $B$ l'évènement: \og Julien réussit à prendre son bus \fg ;
\item[$\bullet~~$] $V$ l'évènement: \og Julien est à l'heure à l'aéroport pour son vol \fg.
\end{itemize}
\setlength\parindent{0mm}

\medskip

\begin{enumerate}
\item Donner la valeur de $P_B(V)$.
\item Représenter la situation par un arbre pondéré.
\item Montrer que $P(V) = 0,6$.
\item Si Julien est à l'heure à l'aéroport pour son vol, quelle est la probabilité qu'il soit arrivé à l'aéroport en bus ? Justifier.
\end{enumerate}

\bigskip

\textbf{Partie 2}

\medskip

Les compagnies aériennes vendent plus de billets qu'il n'y a de places dans les avions car certains passagers ne se présentent pas à l'embarquement du vol sur lequel ils ont réservé.
On appelle cette pratique le surbooking.

Au vu des statistiques des vols précédents, la compagnie aérienne estime que chaque passager a 5\,\% de chance de ne pas se présenter à l'embarquement.

Considérons un vol dans un avion de $200$~places pour lequel $206$~billets ont été vendus.
On suppose que la présence à l'embarquement de chaque passager est indépendante des autres passagers et on appelle $X$ la variable aléatoire qui compte le nombre de passagers se présentant à l'embarquement.

\medskip

\begin{enumerate}
\item Justifier que $X$ suit une loi binomiale dont on précisera les paramètres.
\item En moyenne, combien de passagers vont-ils se présenter à l'embarquement ?
\item Calculer la probabilité que $201$ passagers se présentent à l'embarquement. Le résultat sera arrondi à $10^{-3}$ près.
\item Calculer $P(X \leqslant 200)$, le résultat sera arrondi à $10^{-3}$ près. Interpréter ce résultat dans le contexte de l'exercice.
\item La compagnie aérienne vend chaque billet à $250$ euros.

Si plus de $200$ passagers se présentent à l'embarquement, la compagnie doit rembourser le billet d'avion et payer une pénalité de $600$ euros à chaque passager lésé.

On appelle :

\setlength\parindent{12mm}
\begin{description}
\item[ ] $Y$ la variable aléatoire égale au nombre de passagers qui ne peuvent pas embarquer
bien qu'ayant acheté un billet;
\item[ ] $C$ la variable aléatoire qui totalise le chiffre d'affaire de la compagnie aérienne sur ce vol.
\end{description}

On admet que $Y$ suit la loi de probabilité donnée par le tableau suivant:

\begin{center}
\begin{tabularx}{\linewidth}{|c|*{7}{>{\centering \arraybackslash}X|}}\hline
$y_i$					&0	&1	&2	&3	&4	&5	&6\\ \hline
$P\left(Y = y_i\right)$	&\np{0,94775}&\np{0,03063}&\np{0,01441}&\np{0,00539}&\np{0,00151}&\np{0,00028}& \\ \hline
\end{tabularx}
\end{center}

	\begin{enumerate}
		\item Compléter la loi de probabilité donnée ci-dessus en calculant $P(Y = 6)$.
		\item Justifier que: $C = \np{51500} - 850Y$.
		\item Donner la loi de probabilité de la variable aléatoire $C$ sous forme d'un tableau.
		
Calculer l'espérance de la variable aléatoire $C$ à l'euro près.
		\item Comparer le chiffre d'affaires obtenu en vendant exactement $200$ billets et le chiffre d'affaires moyen obtenu en pratiquant le surbooking.
	\end{enumerate}
\end{enumerate}

\bigskip

\textbf{\textsc{Exercice 4} \hfill 7 points}

\medskip

\emph{Principaux domaines abordés} : Suites numériques.
Algorithmique et programmation.

\bigskip

On s'intéresse au développement d'une bactérie.

Dans cet exercice, on modélise son développement avec les hypothèses suivantes : cette bactérie a une probabilité $0,3$ de mourir sans descendance et une probabilité $0,7$ de se diviser en deux bactéries filles.

Dans le cadre de cette expérience, on admet que les lois de reproduction des bactéries sont les mêmes pour toutes les générations de bactéries qu'elles soient mère ou fille.

Pour tout entier naturel $n$, on appelle $p_n$ la probabilité d'obtenir au plus $n$ descendances pour une bactérie. 

On admet que, d'après ce modèle, la suite $\left(p_n\right)$ est définie de la façon suivante :

$p_0 = 0,3$ et, pour tout entier naturel $n$,

\[p_{n+1} = 0,3 + 0,7p_n^2.\]

\begin{minipage}{0.68\linewidth}
\begin{enumerate}
\item La feuille de calcul ci-dessous donne des valeurs approchées de la suite $\left(p_n\right)$
	\begin{enumerate}
		\item Déterminer les valeurs exactes de $p_1$ et $p_2$ (masquées dans la feuille de calcul) et interpréter ces valeurs dans le contexte de l'énoncé.
		\item Quelle est la probabilité, arrondie à $10^{-3}$ près, d'obtenir au moins 11 générations de bactéries à partir d'une bactérie de ce type ?
		\item Formuler des conjectures sur les variations et la convergence de la suite $\left(p_n\right)$.
	\end{enumerate}
\item 
	\begin{enumerate}
		\item Démontrer par récurrence sur $n$ que,
pour tout entier naturel $n,\: 0 \leqslant p_n \leqslant p_{n+1} \leqslant0,5$.
		\item Justifier que la suite $\left(p_n\right)$ est convergente.
	\end{enumerate}
\item On appelle $L$ la limite de la suite $\left(p_n\right)$.
	\begin{enumerate}
		\item Justifier que $L$ est solution de l'équation 
		
		\[0,7x^2 - x  + 0,3 = 0\]
		\item Déterminer alors la limite de la suite $\left(p_n\right)$.
	\end{enumerate}
\end{enumerate}
\end{minipage}\hfill
\begin{minipage}{0.31\linewidth}
\begin{tabularx}{\linewidth}{|c|c|>{\centering \arraybackslash}X|}\hline
&A &B\\ \hline
1&$n$&$p_n$\\ \hline
2& 0 &0,3\\ \hline
3&1&\\ \hline
4&2&\\ \hline
5&3& \np{0,40769562}\\ \hline 
6&4& \np{0,416351} \\ \hline 
7&5 &\np{0,42134371} \\ \hline 
8&6 &\np{0,42427137} \\ \hline 
9&7& \np{0,42600433}\\ \hline 
10&8& \np{0,42703578} \\ \hline 
11&9& \np{0,42765169} \\ \hline 
12& 10& \np{0,42802018} \\ \hline 
13& 11& \np{0,42824089} \\ \hline 
14& 12& \np{0,42837318} \\ \hline 
15& 13& \np{0,42845251} \\ \hline 
16& 14& \np{0,42850009} \\ \hline 
17& 15& \np{0,42852863} \\ \hline 
18& 16& \np{0,42854575} \\ \hline 
19& 17& \np{0,42855602}\\ \hline 
\end{tabularx}
\end{minipage}

\begin{enumerate}[resume]
\item La fonction suivante, écrite en langage Python, a pour objectif de renvoyer les $n$ premiers termes de la suite $\left(p_n\right)$.

\begin{center}
\begin{tabularx}{0.4\linewidth}{c |l|}\cline{2-2}
1 &\texttt{\textbf{def} suite(n) :}\\
2 &\quad \texttt{p= \ldots}\\
3 &\quad \texttt{s=[p]}\\
4 &\quad \texttt{for i in range (\ldots) :}\\
5 &\quad \qquad \texttt{p=\ldots}\\
6 &\quad \qquad \texttt{s.append(p)}\\
7 &\quad \texttt{\textbf{return} (s)}\\ \cline{2-2}
\end{tabularx}
\end{center}

Recopier, sur votre copie, cette fonction en complétant les lignes 2, 4 et 5 de façon à ce que la fonction \texttt{suite (n)} retourne, sous forme de liste, les $n$ premiers termes de la suite.
\end{enumerate}
\end{document}