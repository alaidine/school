\documentclass[11pt]{article}
\usepackage[T1]{fontenc}
\usepackage[utf8]{inputenc}
\usepackage{fourier}
\usepackage[scaled=0.875]{helvet}
\renewcommand{\ttdefault}{lmtt}
\usepackage{makeidx}
\usepackage{amsmath,amssymb}
\usepackage{fancybox}
\usepackage[normalem]{ulem}
\usepackage{pifont}
\usepackage{lscape}
\usepackage{multicol}
\usepackage{mathrsfs}
\usepackage{tabularx}
\usepackage{multirow}
\usepackage{enumitem}
\usepackage{textcomp}
\newcommand{\euro}{\eurologo{}}
%Tapuscrit : Denis Vergès
%Relecture : François Hache
\usepackage{pst-plot,pst-tree,pst-node,pst-text,pst-bezier,pst-all}%
\usepackage{pst-eucl}
%\usepackage{pstricks-add}
\newcommand{\R}{\mathbb{R}}
\newcommand{\N}{\mathbb{N}}
\newcommand{\D}{\mathbb{D}}
\newcommand{\Z}{\mathbb{Z}}
\newcommand{\Q}{\mathbb{Q}}
\newcommand{\C}{\mathbb{C}}
\usepackage[left=3.5cm, right=3.5cm, top=3cm, bottom=3cm]{geometry}
\newcommand{\vect}[1]{\overrightarrow{\,\mathstrut#1\,}}
\renewcommand{\theenumi}{\textbf{\arabic{enumi}}}
\renewcommand{\labelenumi}{\textbf{\theenumi.}}
\renewcommand{\theenumii}{\textbf{\alph{enumii}}}
\renewcommand{\labelenumii}{\textbf{\theenumii.}}
\def\Oij{$\left(\text{O}~;~\vect{\imath},~\vect{\jmath}\right)$}
\def\Oijk{$\left(\text{O}~;~\vect{\imath},~\vect{\jmath},~\vect{k}\right)$}
\def\Ouv{$\left(\text{O}~;~\vect{u},~\vect{v}\right)$}
\usepackage{fancyhdr}
\usepackage[dvips]{hyperref}
\hypersetup{%
pdfauthor = {APMEP},
pdfsubject = {Baccalauréat spécialité},
pdftitle = {Polynésie 4 mai 2022 sujet 1},
allbordercolors = white,
pdfstartview=FitH} 
\usepackage[french]{babel}
\DecimalMathComma
\usepackage[np]{numprint}
\begin{document}
\setlength\parindent{0mm}
\rhead{\textbf{A. P{}. M. E. P{}.}}
\lhead{\small Baccalauréat spécialité}
\lfoot{\small{Polynésie}}
\rfoot{\small{4 mai 2022}}
\pagestyle{fancy}
\thispagestyle{empty}

\begin{center}{\Large\textbf{\decofourleft~Baccalauréat Polynésie 4 mai 2022~\decofourright\\[6pt] ÉPREUVE D'ENSEIGNEMENT DE SPÉCIALITÉ sujet \no 1}}

\bigskip

Durée de l'épreuve : \textbf{4 heures}

\medskip

L'usage de la calculatrice avec mode examen actif est autorisé

\medskip

Le sujet propose 4 exercices

Le candidat choisit 3 exercices parmi les 4 et \textbf{ne doit traiter que ces 3 exercices}
\end{center}

\bigskip

\textbf{\textsc{Exercice 1} \quad 7 points\hfill Thèmes : fonctions, suites}

\medskip

\emph{Cet exercice est un questionnaire à choix multiples. Pour chacune des six questions suivantes, une seule des quatre réponses proposées est exacte.\\
Une réponse fausse, une réponse multiple ou l'absence de réponse à une question ne rapporte ni n'enlève de point.\\
Pour répondre, indiquer sur la copie le numéro de la question et la lettre de la réponse choisie. Aucune justification n'est demandée.}

\medskip

\begin{enumerate}
\item On considère  la fonction $g$ définie et dérivable sur $]0~;~+ \infty[$ par :

\[g(x) = \ln \left(x^2 + x + 1\right).\]

Pour tout nombre réel $x$ strictement positif :

\begin{center}
\begin{tabularx}{\linewidth}{X X}
\textbf{a.~~}$g'(x) = \dfrac{1}{2x + 1}$&\textbf{b.~~}$g'(x) = \dfrac{1}{x^2 + x + 1}$\\[8pt]
\textbf{c.~~}$g'(x) = \ln (2x + 1)$&\textbf{d.~~}$g'(x) = \dfrac{2x + 1}{x^2 + x + 1}$
\end{tabularx}
\end{center}
\item La fonction $x \longmapsto \ln (x)$ admet pour primitive sur $]0~;~+ \infty[$ la fonction :

\begin{center}
\begin{tabularx}{\linewidth}{*{4}{X} }
\textbf{a.~~}$x \longmapsto \ln (x)$&\textbf{b.~~}$x \longmapsto \dfrac{1}{x}$&\textbf{c.~~}$x \longmapsto x \ln (x) - x$&\textbf{d.~~}$x \longmapsto \dfrac{\ln (x)}{x}$
\end{tabularx}
\end{center}
\item On considère la suite $\left(a_n\right)$ définie  pour tout $n$ dans $\N$ par :

\[a_n = \dfrac{1 - 3^n}{1 + 2^n}.\]

La limite de la suite $\left(a_n\right)$ est égale à :

\begin{center}
\begin{tabularx}{\linewidth}{*{4}{X} }
\textbf{a.~~}$- \infty$&\textbf{b.~~}$- 1$&\textbf{c.~~}$1$&\textbf{d.~~}$+ \infty$
\end{tabularx}
\end{center}
\item 
On considère une fonction $f$ définie et dérivable sur $[-2~;~2]$. Le tableau de variations de la fonction $f'$ dérivée de la fonction $f$ sur l'intervalle $[2~;~2]$ est donné par :

\begin{center}
{\renewcommand{\arraystretch}{1.2}
\psset{nodesep=3pt,arrowsize=2pt 3}  % paramètres
\def\esp{\hspace*{1.5cm}}% pour modifier la largeur du tableau
\def\hauteur{0pt}% mettre au moins 20pt pour augmenter la hauteur
$\begin{array}{|c| *4{c} c|}
\hline
 x & -2 & \esp & 0 & \esp & 2 \\
% \hline
%f'(x) &  &  \pmb{-} & \vline\hspace{-2.7pt}0 & \pmb{+} & \\  
\hline
  & \Rnode{max1}{1}  &  &  &  & \Rnode{max2}{-1}   \\
\text{variations de }f' & &  & & &  \rule{0pt}{\hauteur} \\
 &  & &   \Rnode{min}{-2} & & \rule{0pt}{\hauteur}
\ncline{->}{max1}{min} \ncline{->}{min}{max2}
\rput*(-3.7,0.6){\Rnode{zero}{0}}
\rput(-3.7,1.85){\Rnode{alpha}{-1}}
%\ncline[linestyle=dotted, linecolor=blue]{alpha}{zero}
%\rput*(-1.3,0.65){\Rnode{zero2}{\red 0}}
%\rput(-1.3,1.7){\Rnode{beta}{\red \beta}}
%\ncline[linestyle=dotted, linecolor=red]{beta}{zero2}
\\
\hline
\end{array}$
}
\end{center}


La fonction $f$ est :

\begin{center}
\begin{tabularx}{\linewidth}{X X}
\textbf{a.~~} convexe sur $[- 2~;~- 1]$&\textbf{b.~~} concave sur [0~;~1]\\
\textbf{c.~~} convexe sur $[- 1~;~2]$&\textbf{d.~~}concave sur $[-2~;~0]$
\end{tabularx}
\end{center}

\item On donne ci-dessus la courbe représentative de la dérivée $f'$ d'une fonction $f$ définie sur l'intervalle $[-2~;~4]$.

\begin{center}
\psset{unit=1cm,arrowsize=2pt 3}
\begin{pspicture*}(-2.2,-3.2)(4.2,3.2)
\psgrid[gridlabels=0pt,subgriddiv=1,gridcolor=lightgray]
\psaxes[linewidth=1.25pt,labelFontSize=\scriptstyle]{->}(0,0)(-2.2,-3.2)(4.2,3.2)
\psbcurve[plotpoints=5000,linewidth=1.25pt,linecolor=blue](-2,-3)(-1,0)(0,1)L(0.1,1)(1,0)L(1.5,-1)(2,-1)L(2.1,-1)(3,0)(4,3)
%\pscspline[plotpoints=2000,linewidth=1.25pt,linecolor=red](-2,-3)(-1,0)(0,1)(1,0)(2,-1)(3,0)(4,3)
\end{pspicture*}
\end{center}

Par lecture graphique de la courbe de $f'$, déterminer l'affirmation correcte pour $f$ :

\begin{center}
\begin{tabularx}{\linewidth}{X X}
\textbf{a.~~} $f$ est décroissante sur [0~;~2]&\textbf{b.~~}$f$ est décroissante sur $[-1~;~0]$\\
\textbf{c.~~}$f$ admet un maximum en 1 sur [0~;2]&\textbf{d.~~}$f$ admet un maximum en 3 sur [2~;~4]
\end{tabularx}
\end{center}

\item Une action est cotée à 57 \euro. Sa valeur augmente de 3\,\% tous les mois.

La fonction python seuil() qui renvoie le nombre de mois à attendre pour que sa valeur dépasse 200 \euro{} est :

\begin{center}
\begin{tabularx}{\linewidth}{X X}
\textbf{a.~~}&\textbf{b.~~}\\
\fbox{\begin{tabular}{l}
def seuil() :\\
\quad m=0\\
\quad v=57\\
\quad while v < 200 :\\
\qquad m=m+1\\
\qquad v = v*1.03\\
\quad return m
\end{tabular}
} & \fbox{\begin{tabular}{l}
def seuil() :\\
\quad m=0\\
\quad v=57\\
\quad while v > 200 :\\
\qquad m=m+1\\
\qquad v = v*1.03\\
\quad return m
\end{tabular}
}  \\
\textbf{c.~~} &\textbf{d.~~} \\
\fbox{\begin{tabular}{l}
def seuil() :\\
\quad v=57\\
\quad for i in range (200) :\\
\qquad v = v*1.03\\
\quad return v
\end{tabular}
} &\fbox{\begin{tabular}{l}
def seuil() :\\
\quad m=0\\
\quad v=57\\
\quad if v < 200 :\\
\qquad m=m+1\\
\quad else :\\
\qquad v = v*1.03\\
\quad return m
\end{tabular}
}
\end{tabularx}
\end{center}
\end{enumerate}

\bigskip

\textbf{\textsc{Exercice 2}  \quad  7 points\hfill Thèmes : probabilités}

\medskip


Selon les autorités sanitaires d'un pays, 7\,\% des habitants sont affectés par une certaine maladie.

Dans ce pays, un test est mis au point pour détecter cette maladie. Ce test a les caractéristiques suivantes :
\setlength\parindent{1cm}
\begin{itemize}
\item[$\bullet~~$] Pour les individus malades, le test donne un résultat négatif dans $20 \,\%$ des cas ;
\item[$\bullet~~$] Pour les individus sains, le test donne un résultat positif dans $1\,\%$ des cas.
\end{itemize}
\setlength\parindent{0cm}

Une personne est choisie au hasard dans la population et testée.

On considère les évènements suivants :

\setlength\parindent{1cm}
\begin{itemize}
\item[$\bullet~~$] $M$ \og la personne est malade \fg{} ;
\item[$\bullet~~$]  $T$ \og le test est positif \fg{}.
\end{itemize}
\setlength\parindent{0cm}

\medskip

\begin{enumerate}
\item Calculer la probabilité de l'évènement $M \cap T$. On pourra s'appuyer sur un arbre pondéré.
\item Démontrer que la probabilité que le test  de la personne choisie au hasard soit positif, est de \np{0,0653}.
\item Dans un contexte de dépistage de la maladie, est-il plus pertinent de connaître $P_M(T)$ ou $P_T(M)$ ?
\item On considère dans cette question que la personne choisie au hasard a eu un test positif.

Quelle est la probabilité qu'elle soit malade ? On arrondira le résultat à $10^{-2}$ près.
\item On choisit des personnes au hasard dans la population. La taille de la population de ce pays permet d'assimiler ce prélèvement à un tirage avec remise.

On note $X$ la variable aléatoire qui donne le nombre d'individus ayant un test positif parmi les 10 personnes. 
	\begin{enumerate}
		\item Préciser la nature et les paramètres de la loi de probabilité suivie par $X$.
		\item Déterminer la probabilité pour qu'exactement deux personnes aient un test positif. On arrondira le résultat à $10^{-2}$ près.
	\end{enumerate}	
\item Déterminer le nombre minimum de personnes à tester dans ce pays pour que la probabilité qu'au moins l'une d'entre elle ait un test positif, soit supérieur à $99\,\%$.
\end{enumerate}

\bigskip

\textbf{\textsc{Exercice 3}  \quad  7 points\hfill Thèmes : suites}

\medskip

Soit $\left(u_n\right)$ la suite définie par $u_0 = 1$ et pour tout entier naturel $n$

\[u_{n+1} = \dfrac{u_n}{1 + u_n}\]

\begin{enumerate}
\item 
	\begin{enumerate}
		\item Calculer les termes $u_1$, $u_2$  et $u_3$. On  donnera les résultats sous forme de fractions irréductibles.
		\item Recopier le script python ci-dessous et compléter les lignes 3 et 6 pour que liste(k) prenne en paramètre un entier naturel k et renvoie la liste des premières valeurs de la suite $\left(u_n\right)$ de $u_0$ à $u_k$.

\begin{center}
\begin{tabularx}{0.5\linewidth}{|l|X|}\hline
1.&def liste(k) :\\
2.&\qquad L = []\\
3.&\qquad u = \ldots\\
4.&\qquad for i in range(0, k+1) :\\
5.&\quad \qquad L.append(u)\\
6.&\quad \qquad u = \ldots\\
7.&\qquad return(L)\\ \hline
\end{tabularx}
\end{center}
	\end{enumerate}
\item On admet que, pour tout entier naturel $n$, $u_n$ est strictement positif.

Déterminer le sens de variation de la suite $\left(u_n\right)$.
\item En déduire que la suite $\left(u_n\right)$ converge.
\item Déterminer la valeur de sa limite.
\item 
	\begin{enumerate}
		\item Conjecturer une expression de $u_n$ en fonction de $n$.
		\item Démontrer par récurrence la conjecture précédente.
	\end{enumerate}
\end{enumerate}

\bigskip

\textbf{\textsc{Exercice 4}  \quad  7 points\hfill Thèmes : géométrie dans le plan et dans l'espace}

\medskip

L'espace est rapporté un repère orthonormal où l'on considère :

\setlength\parindent{1cm}
\begin{itemize}
\item[$\bullet~~$]les points A$(2~;~-1~;~0)$ B$(1~;~0~;~- 3)$, C$(6~;~6~;~1)$  et E$(1~;~2~;~4)$ ;
\item[$\bullet~~$]Le plan $\mathcal{P}$ d'équation cartésienne $2x - y - z + 4 = 0$.
\end{itemize}
\setlength\parindent{0cm}

\medskip

\begin{enumerate}
\item
	\begin{enumerate}
		\item Démontrer que le triangle ABC est rectangle en A.
		\item Calculer le produit scalaire $\vect{\text{BA}} \cdot \vect{\text{BC}}$ puis les longueurs BA et BC.
		\item En déduire la mesure en degrés de l'angle $\widehat{\text{ABC}}$ arrondie au degré.
	\end{enumerate}
\item
	\begin{enumerate}
		\item Démontrer que le plan $\mathcal{P}$ est parallèle au plan ABC.
		\item En déduire une équation cartésienne du plan ABC.
		\item Déterminer une représentation paramétrique de la droite $\mathcal{D}$ orthogonale au plan ABC et passant par le point E.
		\item Démontrer que le projeté orthogonal H  du point E sur le plan ABC a pour coordonnées $\left(4~;~\dfrac{1}{2}~;~\dfrac{5}{2}\right)$.
	\end{enumerate}
\item On rappelle que le volume d'une pyramide est donné par $\mathcal{V} = \dfrac13 \mathcal{B}h$ où $\mathcal{B}$ désigne l'aire d'une base et $h$  la hauteur de la pyramide associée à cette base.

Calculer l'aire du triangle ABC puis démontrer que le volume de la pyramide ABCE est égal à $16,5$ unités de volume.
\end{enumerate}
\end{document}