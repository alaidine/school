\documentclass[11pt,a4paper,french]{article}
\usepackage[T1]{fontenc}
\usepackage[utf8]{inputenc}
\usepackage{fourier}
\usepackage[scaled=0.875]{helvet}
\renewcommand{\ttdefault}{lmtt}
\usepackage{makeidx}
\usepackage{amsmath,amssymb}
\usepackage{fancybox}
\usepackage[normalem]{ulem}
\usepackage{pifont}
\usepackage{lscape}
\usepackage{multicol}
\usepackage{mathrsfs}
\usepackage{tabularx}
\usepackage{multirow}
\usepackage{enumitem}
\usepackage{textcomp} 
\newcommand{\euro}{\eurologo{}}
%Tapuscrit : Denis Vergès
\usepackage{pst-plot,pst-tree,pstricks,pst-node,pst-text}
\usepackage{pst-eucl,pst-3dplot,pstricks-add}
\usepackage{esvect}
\newcommand{\R}{\mathbb{R}}
\newcommand{\N}{\mathbb{N}}
\newcommand{\D}{\mathbb{D}}
\newcommand{\Z}{\mathbb{Z}}
\newcommand{\Q}{\mathbb{Q}}
\newcommand{\C}{\mathbb{C}}
\usepackage[left=2.5cm, right=2.5cm, top=2cm, bottom=3cm]{geometry}
\headheight15 mm
\newcommand{\vect}[1]{\overrightarrow{\,\mathstrut#1\,}}
\renewcommand{\theenumi}{\textbf{\arabic{enumi}}}
\renewcommand{\labelenumi}{\textbf{\theenumi.}}
\renewcommand{\theenumii}{\textbf{\alph{enumii}}}
\renewcommand{\labelenumii}{\textbf{\theenumii.}}
\def\Oij{$\left(\text{O}~;~\vect{\imath},~\vect{\jmath}\right)$}
\def\Oijk{$\left(\text{O}~;~\vect{\imath},~\vect{\jmath},~\vect{k}\right)$}
\def\Ouv{$\left(\text{O}~;~\vect{u},~\vect{v}\right)$}
\newcommand{\e}{\text{e}}
\usepackage{fancyhdr}
\usepackage[dvips]{hyperref}
\hypersetup{%
pdfauthor = {APMEP},
pdfsubject = {Baccalauréat Spécialité},
pdftitle = {Centres étrangers groupe 1 Sujet 1 17  mai 2022},
allbordercolors = white,
pdfstartview=FitH} 
\usepackage{babel}
\usepackage[np]{numprint}
\renewcommand\arraystretch{1.3}
\frenchsetup{StandardLists=true}
\begin{document}
\setlength\parindent{0mm}
\rhead{\textbf{A. P{}. M. E. P{}.}}
\lhead{\small Baccalauréat spécialité sujet 1}
\lfoot{\small{Centres étrangers}}
\rfoot{\small{19 mai 2022}}
\pagestyle{fancy}
\thispagestyle{empty}

\begin{center}{\Large\textbf{\decofourleft~Baccalauréat Centres étrangers Groupe 1 D\footnote{Afrique du Sud, Bulgarie, Comores, Djibouti, Kenya, Liban, Lituanie, Madagascar, Mozambique et Ukraine} 19 mai 2022~\decofourright\\[7pt]  Sujet 2\\[7pt] ÉPREUVE D'ENSEIGNEMENT DE SPÉCIALITÉ}}
\end{center}

\vspace{0,25cm}

Le sujet propose 4 exercices

Le candidat choisit 3 exercices parmi les 4 exercices et \textbf{ne doit traiter que ces 3 exercices}

Chaque exercice est noté sur 7 points (le total sera ramené sur 20 points).\footnote{Madagascar : chaque exercice est noté sur 6  points. La clarté et la précision de l'argumentation ainsi que la qualité de la rédaction sont notées sur 2 points.}

Les traces de recherche, même incomplètes ou infructueuses, seront prises en compte.

\bigskip

\textbf{\textsc{Exercice 1} \quad 6 points\hfill Thème: Probabilités}

\medskip

\emph{Les résultats seront arrondis si besoin à $10^{- 4}$ près}

\medskip

Une étude statistique réalisée dans une entreprise fournit les informations suivantes : 

\setlength\parindent{1cm}
\begin{itemize}
\item[$\bullet~~$] 48\,\% des salariés sont des femmes. Parmi elles, 16,5\,\% exercent une
profession de cadre ;
\item[$\bullet~~$] 52\,\% des salariés sont des hommes. Parmi eux, 21,5\,\% exercent une
profession de cadre.
\end{itemize}
\setlength\parindent{0cm}

\medskip

On choisit une personne au hasard parmi les salariés. On considère les évènements suivants:

\setlength\parindent{1cm}
\begin{itemize}
\item[$\bullet~~$] $F$: \og la personne choisie est une femme \fg{} ;
\item[$\bullet~~$] $C$: \og la personne choisie exerce une profession de cadre \fg.
\end{itemize}
\setlength\parindent{0cm}

\medskip

\begin{enumerate}
\item Représenter la situation par un arbre pondéré.
\item Calculer la probabilité que la personne choisie soit une femme qui exerce une profession de cadre.
\item 
	\begin{enumerate}
		\item Démontrer que la probabilité que la personne choisie exerce une profession de cadre est égale à $0,191$.
		\item Les évènements $F$ et $C$ sont-ils indépendants ? Justifier.
	\end{enumerate}	
\item Calculer la probabilité de $F$ sachant $C$, notée $P_C(F)$. Interpréter le résultat dans le
contexte de l'exercice.
\item On choisit au hasard un échantillon de $15$ salariés. Le grand nombre de salariés
dans l'entreprise permet d'assimiler ce choix à un tirage avec remise. 

On note $X$ la variable aléatoire donnant le nombre de cadres au sein de
l'échantillon de 15 salariés.

On rappelle que la probabilité qu'un salarié choisi au hasard soit un cadre est
égale à $0,191$.
	\begin{enumerate}
		\item Justifier que $X$ suit une loi binomiale dont on précisera les paramètres. 
		\item Calculer la probabilité que l'échantillon contienne au plus 1 cadre.
		\item Déterminer l'espérance de la variable aléatoire $X$.
	\end{enumerate}	
\item Soit $n$ un entier naturel.

On considère dans cette question un échantillon de $n$ salariés.

Quelle doit être la valeur minimale de $n$ pour que la probabilité qu'il y ait au moins un cadre au sein de l'échantillon soit supérieure ou égale à $0,99$ ?
\end{enumerate}

\bigskip

\textbf{\textsc{Exercice 2} \quad 6 points\hfill Thème: Géométrie dans l'espace}

\medskip

On considère le cube ABCDEFGH de côté 1 représenté ci-dessous.

\begin{center}
\psset{unit=1cm}
\begin{pspicture}(7.2,6.9)
\psframe(0.2,0.2)(5.2,5.2)%ABFE
\psline(5.2,0.2)(6.8,1.5)(6.8,6.5)(5.2,5.2)%BCGF
\psline(6.8,6.5)(1.8,6.5)(0.2,5.2)%GHE
\psline[linestyle=dashed](0.2,0.2)(1.8,1.5)(6.8,1.5)%ADC
\psline[linestyle=dashed](1.8,1.5)(1.8,6.5)%DH
\uput[dl](0.2,0.2){A} \uput[dr](5.2,0.2){B} \uput[r](6.8,1.5){C} \uput[l](1.8,1.5){D}
\uput[l](0.2,5.2){E} \uput[r](5.2,5.2){F} \uput[ur](6.8,6.5){G} \uput[ul](1.8,6.5){H}
\end{pspicture}
\end{center}

On munit l'espace du repère orthonormé $\left(\text{A}~;~\vect{\text{AB}},\, \vect{\text{AD}},\, \vect{\text{AE}}\right)$.

\medskip

\begin{enumerate}
\item 
	\begin{enumerate}
		\item Justifier que les droites (AH) et (ED) sont perpendiculaires.
		\item Justifier que la droite (GH) est orthogonale au plan (EDH).
		\item En déduire que la droite (ED) est orthogonale au plan (AGH).
	\end{enumerate}	
\item Donner les coordonnées du vecteur $\vect{\text{ED}}$.

Déduire de la question 1. c. qu'une équation cartésienne du plan (AGH) est:

\[y - z = 0.\]

\item On désigne par L le point de coordonnées $\left(\dfrac23~;~1~;~0\right)$.
	\begin{enumerate}
		\item Déterminer une représentation paramétrique de la droite (EL).
		\item Déterminer l'intersection de la droite (EL) et du plan (AGH).
		\item Démontrer que le projeté orthogonal du point L sur le plan (AGH) est le
point K de coordonnées $\left(\dfrac23~;~\dfrac12~;~\dfrac12\right)$.
		\item Montrer que la distance du point L au plan (AGH) est égale à $\dfrac{\sqrt{2}}{2}$.
		\item Déterminer le volume du tétraèdre LAGH.
		
On rappelle que le volume $V$ d'un tétraèdre est donné par la formule :

\[V = \dfrac13 \times (\text{aire de la base}) \times \text{hauteur}.\]
	\end{enumerate}
\end{enumerate}

\bigskip

\textbf{\textsc{Exercice 3} \quad 6 points\hfill Thème: Fonctions; Suites}

\medskip

\emph{Cet exercice est un questionnaire à choix multiples.\\
Pour chacune des questions suivantes, une seule des quatre réponses proposées est exacte.\\
Une réponse fausse, une réponse multiple ou l'absence de réponse à une question ne rapporte ni n'enlève de point.\\
Pour répondre, indiquer sur la copie le numéro de la question et la lettre de la réponse choisie.\\Aucune justification n'est demandée.}

\medskip

\begin{enumerate}
\item Soit $g$ la fonction définie sur $\R$ par $g(x) = x^{\np{1000}} + x$. 

On peut affirmer que :
	\begin{enumerate}
		\item la fonction $g$ est concave sur $\R$.
		\item la fonction $g$ est convexe sur $\R$.
		\item la fonction $g$ possède exactement un point d'inflexion.
		\item la fonction $g$ possède exactement deux points d'inflexion.
	\end{enumerate}
\item On considère une fonction $f$ définie et dérivable sur $\R$. 

\begin{minipage}{0.48\linewidth}
On note $f'$ sa fonction dérivée.

On note $\mathcal{C}$ la courbe représentative de $f$.

On note $\Gamma$ la courbe représentative de $f'$.

On a tracé ci-contre la courbe $\Gamma$.
\end{minipage}\hfill
\begin{minipage}{0.4\linewidth}
\psset{unit=1.2cm}
\begin{pspicture*}(-1.8,-1.6)(3.2,1.4)
\psgrid[gridlabels=0pt,subgriddiv=1,gridwidth=0.15pt]
\psaxes[linewidth=1.25pt,labelFontSize=\scriptstyle]{->}(0,0)(-1.8,-1.6)(3.2,1.4)
\psplot[plotpoints=2000,linewidth=1.25pt,linecolor=red]{-1.8}{3.2}{x 1 add 2.71828 x exp div}
\uput[ur](2,0.45){\red $\Gamma$}
\end{pspicture*}
\end{minipage}

On note $T$ la tangente à la courbe $\mathcal{C}$ au point d'abscisse 0 .

On peut affirmer que la tangente $T$ est parallèle à la droite d'équation : 

\begin{center}
\begin{tabularx}{\linewidth}{*{2}{X}}
\textbf{a.~~}$y =x$ &\textbf{b.~~}$y = 0$\\
\textbf{c.~~}$y = 1$&\textbf{d.~~}$x = 0$
\end{tabularx}
\end{center}
\item On considère la suite $\left(u_n\right)$ définie pour tout entier naturel $n$ par $u_n = \dfrac{(-1)^n}{n+1}$.

On peut affirmer que la suite $\left(u_n\right)$ est : 

\begin{center}
\begin{tabularx}{\linewidth}{*{2}{X}}
\textbf{a.~~}majorée et non minorée.&\textbf{b.~~}minorée et non majorée.\\
\textbf{c.~~}bornée.				&\textbf{d.~~}non majorée et non minorée.
\end{tabularx}
\end{center}

\item Soit $k$ un nombre réel non nul.

Soit $\left(v_n\right)$  une suite définie pour tout entier naturel $n$.

On suppose que $v_0 = k$ et que pour tout $n$, on a $v_n \times v_{n+1} < 0$.

On peut affirmer que $v_{10}$ est :

\begin{center}
\begin{tabularx}{\linewidth}{*{2}{X}}
\textbf{a.~~}positif. &\textbf{a.~~} négatif.\\
\textbf{c.~~}du signe de $k$.&\textbf{d.~~}du signe de $- k$.
\end{tabularx}
\end{center}
\item On considère la suite $\left(w_n\right)$ définie pour tout entier naturel $n$ par : 

\[w_{n+1} = 2w_n - 4\quad \text{et}\quad  w_2 = 8.\]

On peut affirmer que:

\begin{center}
\begin{tabularx}{\linewidth}{*{2}{X}}
\textbf{a.~~}$w_0 = 0$&\textbf{b.~~} $w_0 = 5$.\\
\textbf{c.~~}$w_0 = 10$.&\textbf{d.~~}Il n'est pas possible de calculer $w_0$.
\end{tabularx}
\end{center}

\item \footnote{Cette question ne fait pas partie du sujet donné à Madagascar} On considère la suite $\left(a_n\right)$ définie pour tout entier naturel $n$ par :

\[a_{n+1} = \dfrac{\text{e}^n}{\text{e}^n + 1}a_n\quad \text{et} \quad a_0 = 1.\]

On peut affirmer que :

\begin{center}
\begin{tabularx}{\linewidth}{*{2}{X}}
\textbf{a.~~} la suite $\left(a_n\right)$ est strictement croissante.&\textbf{b.~~}la suite $\left(a_n\right)$ est strictement décroissante.\\
\textbf{c.~~} la suite $\left(a_n\right)$ n'est pas monotone.&\textbf{d.~~} la suite $\left(a_n\right)$ est constante.
\end{tabularx}
\end{center}
\item Une cellule se reproduit en se divisant en deux cellules identiques, qui se divisent à leur tour, et ainsi de suite. 

On appelle temps de génération le temps nécessaire pour qu'une cellule donnée se divise en deux cellules.

On a mis en culture 1 cellule. Au bout de 4 heures, il y a environ \np{4000} cellules.

On peut affirmer que le temps de génération est environ égal à :

\begin{center}
\begin{tabularx}{\linewidth}{*{2}{X}}
\textbf{a.~~}moins d'une minute.&\textbf{b.~~}12 minutes.\\
\textbf{c.~~}20 minutes.		&\textbf{d.~~}1 heure.
\end{tabularx}
\end{center}

\end{enumerate}

\bigskip

\textbf{\textsc{Exercice 4} \quad 6 points\hfill Thème: Fonctions, Fonction exponentielle, Fonction logarithme; Suites}

\medskip

\textbf{Partie A}

\medskip

On considère la fonction $f$ définie pour tout réel $x$ de ]0~;~1] par:

\[f(x) = \text{e}^{-x} + \ln (x).\]

\smallskip

\begin{enumerate}
\item Calculer la limite de $f$ en $0$.
\item On admet que $f$ est dérivable sur ]0~;~1]. On note $f'$ sa fonction dérivée.

Démontrer que, pour tout réel $x$ appartenant à ]0~;~1], on a :

\[f'(x) = \dfrac{1 - x\text{e}^{-x}}{x}\]

\item Justifier que, pour tout réel $x$ appartenant à ]0~;~1], on a $x\text{e}^{-x} < 1$.

En déduire le tableau de variation de $f$ sur ]0~;~1].
\item Démontrer qu'il existe un unique réel $\ell$ appartenant à ]0~;~1] tel que $f(\ell) = 0$.
\end{enumerate}

\medskip

\textbf{Partie B}

\medskip

\begin{enumerate}
\item On définit deux suites $\left(a_n\right)$ et $\left(b_n\right)$ par:

\[\left\{\begin{array}{l c l}
a_0& =& \dfrac{1}{10}\\
b_0& =& 1
\end{array}\right.\: \text{et, pour tout entier naturel }\: n,\left\{\begin{array}{l c l}
a_{n+1}&=&\text{e}^{-b_n}\\
b_{n+1}&=&\text{e}^{-a_n}
\end{array}\right.\]

	\begin{enumerate}
		\item Calculer $a_1$ et $b_1$. On donnera des valeurs approchées à $10^{-2}$ près.
		\item On considère ci-dessous la fonction \texttt{termes}, écrite en langage Python.
		
\begin{center}

\begin{tabular}{|l|}\hline
\texttt{def termes (n) :}\\
\quad \texttt{a=1/10}\\
\quad \texttt{b=1}\\
\quad \texttt{for k in range(0,n) :}\\
\qquad \texttt{c= \ldots}\\
\qquad \texttt{b  = \ldots}\\
\qquad \texttt{a = c}\\
\quad \texttt{return(a,b)}\\ \hline
\end{tabular}
\end{center}

Recopier et compléter sans justifier le cadre ci-dessus de telle sorte que la fonction termes calcule les termes des suites $\left(a_n\right)$ et $\left(b_n\right)$.
	\end{enumerate}
\item On rappelle que la fonction $x \longmapsto \text{e}^{-x}$ est décroissante sur $\R$.
	\begin{enumerate}
		\item Démontrer par récurrence que, pour tout entier naturel $n$, on a :
		
		\[0 < a_n \leqslant a_{n+1} \leqslant b_{n+1} \leqslant b_n \leqslant 1\]
		
		\item En déduire que les suites $\left(a_n\right)$ et $\left(b_n\right)$ sont convergentes.
	\end{enumerate}	
\item On note $A$ la limite de $\left(a_n\right)$ et $B$ la limite de $\left(b_n\right)$.

On admet que $A$ et $B$ appartiennent à l'intervalle ]0~;~1], et que $A = \text{e}^{-B}$ et $B = \text{e}^{-A}$.
	\begin{enumerate}
		\item Démontrer que $f(A) = 0$.
		\item Déterminer $A - B$.
	\end{enumerate}
\end{enumerate}
\end{document}