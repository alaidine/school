\documentclass[10pt,a4paper]{article}
\usepackage[T1]{fontenc}
\usepackage[utf8]{inputenc}
\usepackage{fourier}
\usepackage[scaled=0.875]{helvet}
\renewcommand{\ttdefault}{lmtt}
\usepackage{makeidx}
\makeindex
\usepackage{amsmath,amssymb}
\usepackage{fancybox}
\usepackage[normalem]{ulem}
\usepackage{pifont}
\usepackage{lscape}
\usepackage{multicol}
\usepackage{mathrsfs}
\usepackage{tabularx,array}
\usepackage{colortbl}
\usepackage{multirow}
\usepackage{textcomp}
\usepackage{enumitem}  
\newcommand{\euro}{\eurologo{}}
%Tapuscrit : Denis Vergès 
\usepackage{graphicx}
\usepackage{pst-bezier,pst-all,pst-func,pstricks-add}
\usepackage{pst-eucl}
%\usepackage{tikz}
\newcommand{\R}{\mathbb{R}}
\newcommand{\N}{\mathbb{N}}
\newcommand{\D}{\mathbb{D}}
\newcommand{\Z}{\mathbb{Z}}
\newcommand{\Q}{\mathbb{Q}}
\newcommand{\C}{\mathbb{C}}
\usepackage{diagbox}
\usepackage[left=3.5cm, right=3.5cm, top=2cm, bottom=3cm]{geometry}
\newcommand{\cg}{\texttt{]}}% crochet gauche
\newcommand{\cd}{\texttt{[}}% crochet droit
\newcommand{\pg}{\geqslant}%  plus grand ou égal
\newcommand{\pp}{\leqslant}%  plus petit ou égal
\newcommand{\vect}[1]{\overrightarrow{\,\mathstrut#1\,}}
\newcommand{\barre}[1]{\overline{\,\mathstrut#1\,}}
\renewcommand{\theenumi}{\textbf{\arabic{enumi}}}
\renewcommand{\labelenumi}{\textbf{\theenumi.}}
\renewcommand{\theenumii}{\textbf{\alph{enumii}}}
\renewcommand{\labelenumii}{\textbf{\theenumii.}}
\def\Oij{$\left(\text{O}~;~\vect{\imath},~\vect{\jmath}\right)$}
\def\Oijk{$\left(\text{O}~;~\vect{\imath},~\vect{\jmath},~\vect{k}\right)$}
\def\Ouv{$\left(\text{O}~;~\vect{u},~\vect{v}\right)$}
\usepackage{fancyhdr}
\usepackage[french]{babel}
\DecimalMathComma
\usepackage[np]{numprint}
\usepackage[dvips,colorlinks=true,linkcolor=blue,citecolor=blue,urlcolor=blue]{hyperref}
\hypersetup{%
pdfauthor = {APMEP},
pdfsubject = {Baccalauréat S},
pdftitle = {Baccalauréat S -  2022},
allbordercolors = white,
pdfstartview=FitH} 
\newcommand{\e}{\,\text{e\,}}%%%    le e de l'exponentielle
\renewcommand{\d}{\,\text d}%%%     le d de l'intégration
\renewcommand{\i}{\,\text{i}\,}%%%  le i des complexes
\newcommand{\ds}{\displaystyle}

\begin{document}
\setlength\parindent{0mm}
\rhead{\textbf{A. P{}. M. E. P{}.}}
\lhead{\small{Baccalauréat S : l'intégrale 2022}}
\pagestyle{fancy}
\thispagestyle{empty} 
\begin{center}
{\huge\textbf{\decofourleft~Baccalauréat spécialité  
2022~\decofourright\\ \vspace{1cm} L'intégrale de mai à novembre
 2022}}

\vspace{1cm}

Pour un accès direct cliquez sur les liens {\Large 
\textcolor{blue}{bleus}}
\end{center}

\vspace{1cm}

\phantomsection
\hypertarget{Sommaire}{}
\index{sommaire@\emph{sommaire}}
 
\begin{tabularx}{\linewidth}{>{\Large}X} 
\Large 

\hyperlink{Polynesie1}{Polynésie 4 mai 2022} \dotfill \pageref{Polynesie1}\\
\hyperlink{Polynesie2}{Polynésie 5 mai 2022} \dotfill \pageref{Polynesie2}\\
\hyperlink{Metropole1}{Métropole 11 mai 2022} \dotfill \pageref{Metropole1}\\
\hyperlink{Centresetrangers1}{Centres étrangers 11 mai 2022} \dotfill \pageref{Centresetrangers1}\\
\hyperlink{Metropole2}{Métropole  12 mai 2022} \dotfill \pageref{Metropole2}\\
\hyperlink{Centresetrangers2}{Centres étrangers 12 mai 2022} \dotfill \pageref{Centresetrangers2}\\
\hyperlink{Asie1}{Asie 17 mai 2022} \dotfill \pageref{Asie1}\\
\hyperlink{Asie2}{Asie 18 mai 2022} \dotfill \pageref{Asie2}\\
\hyperlink{GroupeI1}{Groupe I 18 mai 2022} \dotfill \pageref{GroupeI1}\\
\hyperlink{AmeriqueNord1}{Amérique du Nord 18 mai 2022} \dotfill \pageref{AmeriqueNord1}\\
\hyperlink{GroupeI2}{Groupe I 19 mai 2022} \dotfill \pageref{GroupeI2}\\
\hyperlink{AmeriqueNord2}{Amérique du Nord 19 mai 2022} \dotfill \pageref{AmeriqueNord2}\\
\hyperlink{Polynesie3}{Polynésie 30 août 2022} \dotfill \pageref{Polynesie3}\\
\hyperlink{Metropole3}{Métropole 9 septembre 2022} \dotfill \pageref{Metropole3}\\
\hyperlink{Metropole3}{Métropole 10 septembre 2022} \dotfill \pageref{Metropole4}\\
\hyperlink{AmeriSud1}{Amérique du Sud  26 septembre 2022}\dotfill \pageref{AmeriSud1}\\
\hyperlink{AmeriSud2}{Amérique du Sud  27 septembre 2022}\dotfill \pageref{AmeriSud2}\\
\hyperlink{NCaledo1}{Nouvelle-Calédonie 26 octobre 2022} \dotfill \pageref{NCaledo1}\\
\hyperlink{NCaledo2}{Nouvelle-Calédonie 27 octobre 2022} \dotfill \pageref{NCaledo2}\\
\end{tabularx}
\vspace{1cm}

\hyperlink{Index}{À la fin index des notions abordées}

%À la fin de chaque exercice cliquer sur {\blue *} pour aller à l'index
\newpage ~
\newpage
%%%%%%%%%%% Polynésie 4 mai 2022
\phantomsection
\hypertarget{Polynesie1}{}

\label{Polynesie1}

\lfoot{\small{Polynésie}}
\rfoot{\small{4 mai 2022}}
\pagestyle{fancy}
\thispagestyle{empty}

\begin{center}{\Large\textbf{\decofourleft~Baccalauréat Polynésie 4 mai 2022~\decofourright\\[6pt] ÉPREUVE D'ENSEIGNEMENT DE SPÉCIALITÉ sujet \no 1}}

\bigskip

Durée de l'épreuve : \textbf{4 heures}

\medskip

L'usage de la calculatrice avec mode examen actif est autorisé

\medskip

Le sujet propose 4 exercices

Le candidat choisit 3 exercices parmi les 4 et \textbf{ne doit traiter que ces 3 exercices}
\end{center}

\bigskip

\textbf{\textsc{Exercice 1} \quad 7 points\hfill Thèmes : fonctions, suites}

\medskip

\emph{Cet exercice est un questionnaire à choix multiples. Pour chacune des six questions suivantes, une seule des quatre réponses proposées est exacte.\\
Une réponse fausse, une réponse multiple ou l'absence de réponse à une question ne rapporte ni n'enlève de point.\\
Pour répondre, indiquer sur la copie le numéro de la question et la lettre de la réponse choisie. Aucune justification n'est demandée.}

\medskip

\begin{enumerate}
\item On considère  la fonction $g$ définie et dérivable sur $]0~;~+ \infty[$ par :

\[g(x) = \ln \left(x^2 + x + 1\right).\]

Pour tout nombre réel $x$ strictement positif :\index{fonction logarithme}

\begin{center}
\begin{tabularx}{\linewidth}{X X}
\textbf{a.~~}$g'(x) = \dfrac{1}{2x + 1}$&\textbf{b.~~}$g'(x) = \dfrac{1}{x^2 + x + 1}$\\[8pt]
\textbf{c.~~}$g'(x) = \ln (2x + 1)$&\textbf{d.~~}$g'(x) = \dfrac{2x + 1}{x^2 + x + 1}$
\end{tabularx}
\end{center}

\item La fonction $x \longmapsto \ln (x)$ admet pour primitive sur $]0~;~+ \infty[$ la fonction :\index{primitive}\index{primitive}

\begin{center}
\begin{tabularx}{\linewidth}{*{4}{X} }
\textbf{a.~~}$x \longmapsto \ln (x)$&\textbf{b.~~}$x \longmapsto \dfrac{1}{x}$&\textbf{c.~~}$x \longmapsto x \ln (x) - x$&\textbf{d.~~}$x \longmapsto \dfrac{\ln (x)}{x}$
\end{tabularx}
\end{center}
\item On considère la suite $\left(a_n\right)$ définie  pour tout $n$ dans $\N$ par :\index{limite de suite}

\[a_n = \dfrac{1 - 3^n}{1 + 2^n}.\]

La limite de la suite $\left(a_n\right)$ est égale à :

\begin{center}
\begin{tabularx}{\linewidth}{*{4}{X} }
\textbf{a.~~}$- \infty$&\textbf{b.~~}$- 1$&\textbf{c.~~}$1$&\textbf{d.~~}$+ \infty$
\end{tabularx}
\end{center}
\item 
On considère une fonction $f$ définie et dérivable sur $[-2~;~2]$. Le tableau de variations de la fonction $f'$ dérivée de la fonction $f$ sur l'intervalle $[2~;~2]$ est donné par :

\begin{center}
{\renewcommand{\arraystretch}{1.2}
\psset{nodesep=3pt,arrowsize=2pt 3}  % paramètres
\def\esp{\hspace*{1.5cm}}% pour modifier la largeur du tableau
\def\hauteur{0pt}% mettre au moins 20pt pour augmenter la hauteur
$\begin{array}{|c| *4{c} c|}
\hline
 x & -2 & \esp & 0 & \esp & 2 \\
% \hline
%f'(x) &  &  \pmb{-} & \vline\hspace{-2.7pt}0 & \pmb{+} & \\  
\hline
  & \Rnode{max1}{1}  &  &  &  & \Rnode{max2}{-1}   \\
\text{variations de }f' & &  & & &  \rule{0pt}{\hauteur} \\
 &  & &   \Rnode{min}{-2} & & \rule{0pt}{\hauteur}
\ncline{->}{max1}{min} \ncline{->}{min}{max2}
\rput*(-3.7,0.6){\Rnode{zero}{0}}
\rput(-3.7,1.85){\Rnode{alpha}{-1}}
%\ncline[linestyle=dotted, linecolor=blue]{alpha}{zero}
%\rput*(-1.3,0.65){\Rnode{zero2}{\red 0}}
%\rput(-1.3,1.7){\Rnode{beta}{\red \beta}}
%\ncline[linestyle=dotted, linecolor=red]{beta}{zero2}
\\
\hline
\end{array}$
}
\end{center}


La fonction $f$ est :

\begin{center}
\begin{tabularx}{\linewidth}{X X}
\textbf{a.~~} convexe sur $[- 2~;~- 1]$&\textbf{b.~~} concave sur [0~;~1]\\
\textbf{c.~~} convexe sur $[- 1~;~2]$&\textbf{d.~~}concave sur $[-2~;~0]$\index{convexité}
\end{tabularx}
\end{center}

\item On donne ci-dessus la courbe représentative de la dérivée $f'$ d'une fonction $f$ définie sur l'intervalle $[-2~;~4]$.

\begin{center}
\psset{unit=1cm,arrowsize=2pt 3}
\begin{pspicture*}(-2.2,-3.2)(4.2,3.2)
\psgrid[gridlabels=0pt,subgriddiv=1,gridcolor=lightgray]
\psaxes[linewidth=1.25pt,labelFontSize=\scriptstyle]{->}(0,0)(-2.2,-3.2)(4.2,3.2)
\psbcurve[plotpoints=5000,linewidth=1.25pt,linecolor=blue](-2,-3)(-1,0)(0,1)L(0.1,1)(1,0)L(1.5,-1)(2,-1)L(2.1,-1)(3,0)(4,3)
%\pscspline[plotpoints=2000,linewidth=1.25pt,linecolor=red](-2,-3)(-1,0)(0,1)(1,0)(2,-1)(3,0)(4,3)
\end{pspicture*}
\end{center}

Par lecture graphique de la courbe de $f'$, déterminer l'affirmation correcte pour $f$ :

\begin{center}
\begin{tabularx}{\linewidth}{X X}
\textbf{a.~~} $f$ est décroissante sur [0~;~2]&\textbf{b.~~}$f$ est décroissante sur $[-1~;~0]$\\
\textbf{c.~~}$f$ admet un maximum en 1 sur [0~;2]&\textbf{d.~~}$f$ admet un maximum en 3 sur [2~;~4]\index{maximum}
\end{tabularx}
\end{center}

\item Une action est cotée à 57 \euro. Sa valeur augmente de 3\,\% tous les mois.

La fonction python seuil() qui renvoie le nombre de mois à attendre pour que sa valeur dépasse 200 \euro{} est :\index{script python}

\begin{center}
\begin{tabularx}{\linewidth}{X X}
\textbf{a.~~}&\textbf{b.~~}\\
\fbox{\begin{tabular}{l}
def seuil() :\\
\quad m=0\\
\quad v=57\\
\quad while v < 200 :\\
\qquad m=m+1\\
\qquad v = v*1.03\\
\quad return m
\end{tabular}
} & \fbox{\begin{tabular}{l}
def seuil() :\\
\quad m=0\\
\quad v=57\\
\quad while v > 200 :\\
\qquad m=m+1\\
\qquad v = v*1.03\\
\quad return m
\end{tabular}
}  \\
\textbf{c.~~} &\textbf{d.~~} \\
\fbox{\begin{tabular}{l}
def seuil() :\\
\quad v=57\\
\quad for i in range (200) :\\
\qquad v = v*1.03\\
\quad return v
\end{tabular}
} &\fbox{\begin{tabular}{l}
def seuil() :\\
\quad m=0\\
\quad v=57\\
\quad if v < 200 :\\
\qquad m=m+1\\
\quad else :\\
\qquad v = v*1.03\\
\quad return m
\end{tabular}
}
\end{tabularx}
\end{center}
\end{enumerate}

\bigskip

\textbf{\textsc{Exercice 2}  \quad  7 points\hfill Thèmes : probabilités}

\medskip


Selon les autorités sanitaires d'un pays, 7\,\% des habitants sont affectés par une certaine maladie.

Dans ce pays, un test est mis au point pour détecter cette maladie. Ce test a les caractéristiques suivantes :
\setlength\parindent{1cm}
\begin{itemize}
\item[$\bullet~~$] Pour les individus malades, le test donne un résultat négatif dans $20 \,\%$ des cas ;
\item[$\bullet~~$] Pour les individus sains, le test donne un résultat positif dans $1\,\%$ des cas.
\end{itemize}
\setlength\parindent{0cm}

Une personne est choisie au hasard dans la population et testée.

On considère les évènements suivants :

\setlength\parindent{1cm}
\begin{itemize}
\item[$\bullet~~$] $M$ \og la personne est malade \fg{} ;
\item[$\bullet~~$]  $T$ \og le test est positif \fg{}.
\end{itemize}
\setlength\parindent{0cm}

\medskip

\begin{enumerate}
\item Calculer la probabilité de l'évènement $M \cap T$. On pourra s'appuyer sur un arbre pondéré.\index{arbre pondéré}
\item Démontrer que la probabilité que le test  de la personne choisie au hasard soit positif, est de \np{0,0653}.
\item Dans un contexte de dépistage de la maladie, est-il plus pertinent de connaître $P_M(T)$ ou $P_T(M)$ ?
\item On considère dans cette question que la personne choisie au hasard a eu un test positif.

Quelle est la probabilité qu'elle soit malade ? On arrondira le résultat à $10^{-2}$ près.
\item On choisit des personnes au hasard dans la population. La taille de la population de ce pays permet d'assimiler ce prélèvement à un tirage avec remise.

On note $X$ la variable aléatoire qui donne le nombre d'individus ayant un test positif parmi les 10 personnes. 
	\begin{enumerate}
		\item Préciser la nature et les paramètres de la loi de probabilité suivie par $X$.\index{loi binomiale}
		\item Déterminer la probabilité pour qu'exactement deux personnes aient un test positif. On arrondira le résultat à $10^{-2}$ près.
	\end{enumerate}	
\item Déterminer le nombre minimum de personnes à tester dans ce pays pour que la probabilité qu'au moins l'une d'entre elle ait un test positif, soit supérieur à $99\,\%$.
\end{enumerate}

\bigskip

\textbf{\textsc{Exercice 3}  \quad  7 points\hfill Thèmes : suites}

\medskip

Soit $\left(u_n\right)$ la suite définie par $u_0 = 1$ et pour tout entier naturel $n$

\[u_{n+1} = \dfrac{u_n}{1 + u_n}\]

\begin{enumerate}
\item 
	\begin{enumerate}
		\item Calculer les termes $u_1$, $u_2$  et $u_3$. On  donnera les résultats sous forme de fractions irréductibles.
		\item Recopier le script python ci-dessous et compléter les lignes 3 et 6 pour que liste(k) prenne en paramètre un entier naturel k et renvoie la liste des premières valeurs de la suite $\left(u_n\right)$ de $u_0$ à $u_k$.

\begin{center}
\begin{tabularx}{0.5\linewidth}{|l|X|}\hline
1.&def liste(k) :\\
2.&\qquad L = []\\
3.&\qquad u = \ldots\\
4.&\qquad for i in range(0, k+1) :\\
5.&\quad \qquad L.append(u)\\
6.&\quad \qquad u = \ldots\\
7.&\qquad return(L)\\ \hline
\end{tabularx}\index{script python}
\end{center}
	\end{enumerate}
\item On admet que, pour tout entier naturel $n$, $u_n$ est strictement positif.

Déterminer le sens de variation de la suite $\left(u_n\right)$.
\item En déduire que la suite $\left(u_n\right)$ converge.
\item Déterminer la valeur de sa limite.\index{limite de suite}
\item 
	\begin{enumerate}
		\item Conjecturer une expression de $u_n$ en fonction de $n$.
		\item Démontrer par récurrence la conjecture précédente.\index{récurrence}
	\end{enumerate}
\end{enumerate}

\bigskip

\textbf{\textsc{Exercice 4}  \quad  7 points\hfill Thèmes : géométrie dans le plan et dans l'espace}

\medskip

L'espace est rapporté un repère orthonormal où l'on considère :

\setlength\parindent{1cm}
\begin{itemize}
\item[$\bullet~~$]les points A$(2~;~-1~;~0)$ B$(1~;~0~;~- 3)$, C$(6~;~6~;~1)$  et E$(1~;~2~;~4)$ ;
\item[$\bullet~~$]Le plan $\mathcal{P}$ d'équation cartésienne $2x - y - z + 4 = 0$.
\end{itemize}
\setlength\parindent{0cm}

\medskip

\begin{enumerate}
\item
	\begin{enumerate}
		\item Démontrer que le triangle ABC est rectangle en A.
		\item Calculer le produit scalaire $\vect{\text{BA}} \cdot \vect{\text{BC}}$ puis les longueurs BA et BC.\index{produit scalaire}
		\item En déduire la mesure en degrés de l'angle $\widehat{\text{ABC}}$ arrondie au degré.
	\end{enumerate}
\item
	\begin{enumerate}
		\item Démontrer que le plan $\mathcal{P}$ est parallèle au plan ABC.
		\item En déduire une équation cartésienne du plan ABC.\index{equation de plan@équation de plan}
		\item Déterminer une représentation paramétrique de la droite $\mathcal{D}$ orthogonale au plan ABC et passant par le point E.\index{equation de droite@équation de droite}
		\item Démontrer que le projeté orthogonal H  du point E sur le plan ABC a pour coordonnées $\left(4~;~\dfrac{1}{2}~;~\dfrac{5}{2}\right)$.
	\end{enumerate}
\item On rappelle que le volume d'une pyramide est donné par $\mathcal{V} = \dfrac13 \mathcal{B}h$ où $\mathcal{B}$ désigne l'aire d'une base et $h$  la hauteur de la pyramide associée à cette base.

Calculer l'aire du triangle ABC puis démontrer que le volume de la pyramide ABCE est égal à $16,5$ unités de volume.
\end{enumerate}
\newpage
%%%%%%%%%%% Polynesie 5 mai 2022
\phantomsection
\hypertarget{Polynesie2}{}

\label{Polynesie2}

\lfoot{\small{Polynésie}}
\rfoot{\small{5 mai 2022}}
\pagestyle{fancy}
\thispagestyle{empty}

\begin{center}{\Large\textbf{\decofourleft~Baccalauréat Polynésie 6 mai 2022~\decofourright\\[6pt] ÉPREUVE D'ENSEIGNEMENT DE SPÉCIALITÉ sujet \no 2}}

\bigskip

Durée de l'épreuve : \textbf{4 heures}

\medskip

L'usage de la calculatrice avec mode examen actif est autorisé

\medskip

Le sujet propose 4 exercices

Le candidat choisit 3 exercices parmi les 4 et \textbf{ne doit traiter que ces 3 exercices}
\end{center}

\bigskip

\textbf{\textsc{Exercice 1} \quad 7 points\hfill Thèmes : fonctions, primitives, probabilités}

\medskip

\emph{Cet exercice est un questionnaire à choix multiples. \\\index{QCM}
Pour chacune des six questions suivantes, une seule des quatre réponses proposées est exacte.\\
Une réponse fausse, une réponse multiple ou l'absence de réponse à une question ne rapporte ni n'enlève de point.\\
Pour répondre, indiquer sur la copie le numéro de la question et la lettre de la réponse choisie.\\
Aucune justification n'est demandée.}

\medskip

\begin{enumerate}
\item On considère la fonction $f$ définie et dérivable sur $]0~;~+\infty[$ par:
\[f(x) = x \ln(x) - x + 1.\]\index{fonction logarithme}

Parmi les quatre expressions suivantes, laquelle est celle de la fonction dérivée de $f$?\index{dérivée}

\begin{center}
\begin{tabularx}{\linewidth}{|*{4}{X|}}\hline
\textbf{a.~~} $\ln (x)$&\textbf{b.~~}$\dfrac{1}{x} - 1$&\textbf{c.~~} $\ln (x) - 2$&\textbf{d.~~}$\ln (x) - 1$\rule[-3mm]{0mm}{9mm}\\ \hline
\end{tabularx}
\end{center}

\item On considère la fonction $g$ définie sur $]0~;~+\infty[$ par $g(x) = x^2[1 - \ln (x)]$.

 Parmi les quatre affirmations suivantes, laquelle est correcte ?\index{limite de fonction}
 
\begin{center}
\begin{tabularx}{\linewidth}{|*{4}{X|}}\hline
\textbf{a.~~}$\displaystyle\lim_{x \to 0} g(x) = +\infty$&\textbf{b.~~} 
$\displaystyle\lim_{x \to 0} g(x) = - \infty$&\textbf{c.~~} $\displaystyle\lim_{x \to 0} g(x) = 0$&\textbf{d.~~} La fonction $g$ n'admet pas de limite en 0.\\ \hline
\end{tabularx}
\end{center}

\item On considère la fonction $f$ définie sur $\R$ par $f(x) = x^3 - 0,9x^2 -0,1x$. Le nombre de solutions de l'équation $f(x) = 0$ sur $\R$ est :

\begin{center}
\begin{tabularx}{\linewidth}{|*{4}{X|}}\hline
\textbf{a.~~} $0$&\textbf{b.~~}$1$&\textbf{c.~~} $2$&\textbf{d.~~}$3$\\ \hline
\end{tabularx}
\end{center}

\item Si $H$ est une primitive d'une fonction $h$ définie et continue sur $\R$, 
et si $k$ est la fonction définie sur $\R$ par $k(x) = h(2x)$, 
alors, une primitive $K$ de $k$ est définie sur $\R$ par :\index{primitive}

\begin{center}
\begin{tabularx}{\linewidth}{|*{4}{X|}}\hline
\textbf{a.~~} $K(x) =H(2x)$&\textbf{b.~~}$K(x) =2H(2x)$&\textbf{c.~~} $ K(x) =\dfrac{1}{2}H(2x)$&\textbf{d.~~}$K(x) =2H(x)$\rule[-3mm]{0mm}{9mm}\\ \hline
\end{tabularx}
\end{center}

\item L'équation réduite de la tangente au point d'abscisse 1 de la courbe de la fonction $f$ définie sur $\R$ par $f(x) = x\text{e}^x$ est:\index{equation de tangente@équation de tangente}

\begin{center}
\begin{tabularx}{\linewidth}{|*{4}{X|}}\hline
\textbf{a.~~} $y = \text{e}x + \text{e}$&\textbf{b.~~}$y =2\text{e}x - \text{e}$&\textbf{c.~~} $y = 2\text{e}x +  \text{e}$&\textbf{d.~~}$y = \text{e}x$\\ \hline
\end{tabularx}
\end{center}

\item Les nombres entiers $n$ solutions de l'inéquation $(0,2)^n < 0,001$ sont tous les
nombres entiers $n$ tels que :\index{inéquation}

\begin{center}
\begin{tabularx}{\linewidth}{|*{4}{X|}}\hline
\textbf{a.~~} $n \leqslant 4$&\textbf{b.~~}$n\leqslant 5$&\textbf{c.~~} $n \geqslant 4$ &\textbf{d.~~}$n \geqslant 5$\\ \hline
\end{tabularx}
\end{center}

\end{enumerate}

\bigskip

\textbf{\textsc{Exercice 2} \quad 7 points\hfill Thèmes : probabilités}

\medskip

Les douanes s'intéressent aux importations de casques audio portant le logo d'une certaine marque. Les saisies des douanes permettent d'estimer que:

\setlength\parindent{10mm}
\begin{itemize}
\item[$\bullet~~$] 20\,\% des casques audio portant le logo de cette marque sont des contrefaçons ;
\item[$\bullet~~$] 2\,\% des casques non contrefaits présentent un défaut de conception ;\item[$\bullet~~$] 10\,\% des casques contrefaits présentent un défaut de conception.
\end{itemize}
\setlength\parindent{0mm}

L'agence des fraudes commande au hasard sur un site internet un casque affichant le logo de la marque. On considère les évènements suivants:

\setlength\parindent{10mm}
\begin{itemize}
\item[$\bullet~~$] $C$: \og le casque est contrefait \fg{} ;
\item[$\bullet~~$] $D$: \og le casque présente un défaut de conception \fg{} ;
\item[$\bullet~~$] $\overline{C}$ et $\overline{D}$ désignent respectivement les évènements contraires de $C$ et $D$.
\end{itemize}
\setlength\parindent{0mm}

Dans l'ensemble de l'exercice, les probabilités seront arrondies à $10^{-3}$ si nécessaire.

\bigskip

\textbf{Partie 1}

\medskip

\begin{enumerate}
\item Calculer $P(C \cap D)$. On pourra s'appuyer sur un arbre pondéré.\index{arbre pondéré}
\item Démontrer que $P(D) = 0,036$.
\item Le casque a un défaut. Quelle est la probabilité qu'il soit contrefait ?
\end{enumerate}

\bigskip

\textbf{Partie 2}

\medskip

On commande $n$ casques portant le logo de cette marque. On assimile cette expérience à
un tirage aléatoire avec remise. On note $X$ la variable aléatoire qui donne le nombre de casques présentant un défaut de conception dans ce lot.\index{loi binomiale}

\medskip

\begin{enumerate}
\item Dans cette question, $n = 35$.
	\begin{enumerate}
		\item Justifier que $X$ suit une loi binomiale $\mathcal{B}(n,~p)$ où $n = 35$ et $p = 0,036$.
		\item Calculer la probabilité qu'il y ait parmi les casques commandés, exactement un casque présentant un défaut de conception.
		\item Calculer $P(X \leqslant 1)$.
	\end{enumerate}	
\item Dans cette question, $n$ n'est pas fixé.

Quel doit être le nombre minimal de casques à commander pour que la probabilité 
 qu'au moins un casque présente un défaut soit supérieur à $0,99$ ?
\end{enumerate}

\bigskip

\textbf{\textsc{Exercice 3} \quad 7 points\hfill Thèmes : suites, fonctions}

\medskip

Au début de l'année 2021, une colonie d'oiseaux comptait $40$ individus. L'observation conduit à modéliser l'évolution de la population par la suite $\left(u_n\right)$ définie pour tout entier naturel $n$ par:

\[\left\{\begin{array}{l c l}
u_0&=&40\\u_{n+}&=&0,008u_n\left(200 - u_n\right)
\end{array}\right.\]

où $u_n$ désigne le nombre d'individus au début de l'année $(2021+n)$.

\medskip


\begin{enumerate}
\item Donner une estimation, selon ce modèle, du nombre d'oiseaux dans la colonie au
début de l'année 2022.
\end{enumerate}

On considère la fonction $f$ définie sur l'intervalle [0~;~100] par $f(x) = 0,008x(200 - x)$.

\begin{enumerate}[resume]
\item Résoudre dans l'intervalle [0~;~100] l'équation $f(x) = x$.
\item
	\begin{enumerate}
		\item Démontrer que la fonction $f$ est croissante sur l'intervalle [0~;~100] et dresser son tableau de variations.
		\item En remarquant que, pour tout entier naturel $n$,\, $u_{n+1} = f\left(u_n\right)$ démontrer par récurrence que, pour tout entier naturel $n$ :\index{récurrence}

\[0 \leqslant u_n \leqslant u_{n+1} \leqslant 100.\]

		\item En déduire que la suite $\left(u_n\right)$ est convergente.\index{limite de suite}
		\item Déterminer la limite $\ell$ de la suite $\left(u_n\right)$. Interpréter le résultat dans le contexte de l'exercice.\index{limite de suite}
	\end{enumerate}
\item On considère l'algorithme suivant:\index{script python}

\begin{center}
\fbox{\begin{tabular}{l}
def seuil(p) :\\
\qquad n=0\\
\qquad u = 40\\
\qquad while u < p :\\
\quad \qquad n =n+1\\
\quad \qquad u = 0.008*u*(200-u)\\
\qquad return(n+2021)\\ 
\end{tabular}}
\end{center}

L'exécution de seuil(100) ne renvoie aucune valeur. Expliquer pourquoi à l'aide de la question 3.
\end{enumerate}

\bigskip

\textbf{\textsc{Exercice 4} \quad 7 points\hfill Thèmes : géométrie dans le plan et dans l'espace}

\medskip

On considère le cube ABCDEFGH d'arête de longueur 1.

L'espace est muni du repère orthonormé $\left(\text{A}~;\, \vect{\text{AB}}, \vect{\text{AD}}, \vect{\text{AE}}\right)$. Le point I est le milieu du
segment [EF], K le centre du carré ADHE et O le milieu du segment [AG].

\begin{center}
\psset{unit=0.9cm,radius=0pt}
\begin{pspicture}(0,-1)(9,7)
%%%%%%%%%%%%%%%%%
\Cnode*(0.5,0.4){A} \Cnode*(5.5,0){B} 
\Cnode*(7.5,1.4){C} \Cnode*(2.5,1.8){D}
\Cnode*(0.5,5.4){E} \Cnode*(5.5,5){F} 
\Cnode*(7.5,6.4){G} \Cnode*(2.5,6.8){H}
\Cnode*[radius=2pt](3,5.2){I}% milieu de [EF]
\Cnode*[radius=2pt](1.5,3.6){K}% milieu de [ED]
\Cnode*[radius=2pt](4,3.4){O}% centre du cube
%%%%%%%%%%%%%%%
\uput[dl](A){A} \uput[dr](B){B} \uput[r](C){C} 
\uput[ur](D){D} \uput[ul](E){E} \uput[u](F){F} 
\uput[ur](G){G} \uput[u](H){H} \uput[u](I){I}
\uput[dr](O){O}
\uput[l](K){K}
%%%%%%%%%%%%%%%
\pspolygon(A)(B)(F)(E)
\psline(B)(C)(G)(F)
\psline(G)(H)(E)
\psline(A)(I)(G)
\psline[linestyle=dashed](A)(G)
\psline[linestyle=dashed](A)(D)(H)
\psline[linestyle=dashed](D)(C)
\end{pspicture}
\end{center}

\emph{Le but de l'exercice est de calculer de deux manières différentes, la distance du point {\rm B} au plan ({\rm AIG}).}

\bigskip

\textbf{Partie 1. Première méthode}

\medskip

\begin{enumerate}
\item Donner, sans justification, les coordonnées des points A, B, et G.

On admet que les points I et K ont pour coordonnées I$\left(\dfrac{1}{2}~;~0~;~1\right)$ et K$\left(0~;~\dfrac{1}{2}~;~\dfrac{1}{2}\right)$.
\item Démontrer que la droite (BK) est orthogonale au plan (AIG).
\item Vérifier qu'une équation cartésienne du plan (AIG) est : $2x - y - z = 0$.\index{equation de plan@équation de plan}
\item Donner une représentation paramétrique de la droite (BK).\index{equation de plan@équation de plan}
\item En déduire que le projeté orthogonal L du point B sur le plan (AIG) a pour
coordonnées L$\left(\dfrac{1}{3}~;~\dfrac{1}{3}~;~\dfrac{1}{3}\right)$.
\item Déterminer la distance du point B au plan (AIG).
\end{enumerate}

\bigskip

\textbf{Partie 2. Deuxième méthode}

\medskip

\emph{On rappelle que le volume $V$ d'une pyramide est donné par la formule $V = \dfrac{1}{3} \times  b \times h$, où $b$ est l'aire d'une base et $h$ la hauteur associée à cette base.}

\medskip

\begin{enumerate}
\item 
	\begin{enumerate}
		\item Justifier que dans le tétraèdre ABIG, [GF] est la hauteur relative à la base AIB. 
		\item En déduire le volume du tétraèdre ABIG.
	\end{enumerate}
\item On admet que AI = IG $= \dfrac{\sqrt{5}}{2}$ et que AG $= \sqrt 3$.

Démontrer que l'aire du triangle isocèle AIG est égale à $\dfrac{\sqrt{6}}{4}$ unité d'aire.
\item En déduire la distance du point B au plan (AIG).
\end{enumerate}
\newpage
%%%%%%%%%%% Métropole1 11 mai 2022
\phantomsection
\hypertarget{Metropole1}{}

\label{Metropole1}

\lfoot{\small{Métropole}}
\rfoot{\small{11 mai 2022}}
\pagestyle{fancy}
\thispagestyle{empty}

\begin{center}{\Large\textbf{\decofourleft~Baccalauréat Métropole 11 mai 2022~\decofourright\\[6pt]  Sujet 1\\[7pt] ÉPREUVE D'ENSEIGNEMENT DE SPÉCIALITÉ}}
\end{center}

\vspace{0,25cm}

Le sujet propose 4 exercices

Le candidat choisit 3 exercices parmi les 4 et \textbf{ne doit traiter que ces 3 exercices}

\bigskip

\textbf{\textsc{Exercice 1} \quad (7 points)\hfill Thèmes : fonction exponentielle, suites}

\medskip

Dans le cadre d'un essai clinique on envisage deux protocoles de traitement 
d'une maladie.

L'objectif de cet exercice est d'étudier, pour ces deux protocoles, l'évolution de la quantité de médicament présente dans le sang d'un patient en fonction du temps.

\medskip

\textbf{Les parties A et B sont indépendantes}

\medskip

\textbf{Partie A : Étude du premier protocole}

\medskip

Le premier protocole consiste à faire absorber un médicament, sous forme de comprimé, au patient.

On modélise la quantité de médicament présente dans le sang du patient, exprimée en mg, par la fonction $f$ définie sur l'intervalle [0~;~10] par 

\[f(t) = 3t\text{e}^{-0,5t+1},\]\index{fonction exponentielle}

où $t$ désigne le temps, exprimé en heure, écoulé depuis la prise du comprimé.

\medskip

\begin{enumerate}
\item 
	\begin{enumerate}
		\item On admet que la fonction $f$ est dérivable sur l'intervalle [0~;~10] et on note $f'$ sa fonction dérivée.
		
Montrer que, pour tout nombre réel $t$ de [0~;~10], on a : $f'(t) = 3(- 0,5t + 1)\text{e}^{-0,5t+1}$.\index{dérivée}
		\item En déduire le tableau de variations de la fonction $f$ sur l'intervalle [0~;~10].\index{tableau de variations}
		\item Selon cette modélisation, au bout de combien de temps la quantité de médicament présente dans le sang du patient sera-t-elle maximale ?\index{maximum}
		
Quelle est alors cette quantité maximale ?\index{maximum}
	\end{enumerate}
\item 
	\begin{enumerate}
		\item Montrer que l'équation $f(t) = 5$ admet une unique solution sur l'intervalle [0~;~2] notée $\alpha$, dont on donnera une valeur approchée à $10^{-2}$ près.\index{valeurs intermédiaires}
		
On admet que l'équation $f(t) = 5$ admet une unique solution sur l'intervalle [2~;~10], notée $\beta$, et qu'une valeur approchée de $\beta$ à $10^{-2}$ près est 3,46.
		\item On considère que ce traitement est efficace lorsque la quantité de médicament présente dans le sang du patient est supérieure ou égale à 5 mg.
		
Déterminer, à la minute près, la durée d'efficacité du médicament dans le cas de ce protocole.\index{inéquation}
	\end{enumerate}
\end{enumerate}

\bigskip

\textbf{Partie B : Étude du deuxième protocole}

\medskip

Le deuxième protocole consiste à injecter initialement au patient, par piqûre intraveineuse, une dose de $2$ mg de médicament puis à réinjecter toutes les heures une dose de $1,8$ mg.

On suppose que le médicament se diffuse instantanément dans le sang et qu'il est ensuite
progressivement éliminé.

On estime que lorsqu'une heure s'est écoulée après une injection, la quantité de médicament dans le sang a diminué de 30\,\% par rapport à la quantité présente immédiatement après cette injection.

On modélise cette situation à l'aide de la suite $\left(u_n\right)$ où, pour tout entier naturel $n$, $u_n$ désigne la quantité de médicament, exprimée en mg, présente dans le sang du patient immédiatement après l'injection de la $n$-ième heure. On a donc $u_0 = 2$.\index{suites}

\medskip

\begin{enumerate}
\item Calculer, selon cette modélisation, la quantité $u_1$, de médicament (en mg) présente dans le sang du patient immédiatement après l'injection de la première heure.
\item Justifier que, pour tout entier naturel $n$, on a : $u_{n+1} = 0,7u_n + 1,8$.
\item 
	\begin{enumerate}
		\item Montrer par récurrence que, pour tout entier naturel $n$, on a : $u_n \leqslant u_{n+1}  < 6$.\index{récurrence}
		\item En déduire que la suite $\left(u_n\right)$ est convergente. On note $\ell$ sa limite.\index{limite de suite}
		\item Déterminer la valeur de $\ell$. Interpréter cette valeur dans le contexte de l'exercice.
	\end{enumerate}	
\item On considère la suite $\left(v_n\right)$ définie, pour tout entier naturel $n$, par $v_n = 6 - u_n$.
	\begin{enumerate}
		\item Montrer que la suite $\left(v_n\right)$ est une suite géométrique de raison $0,7$ dont on précisera le premier terme.\index{suite géométrique}
		\item Déterminer l'expression de $v_n$ en fonction de $n$, puis de $u_n$ en fonction de $n$.
		\item Avec ce protocole, on arrête les injections lorsque la quantité de médicament présente dans le sang du patient est supérieure ou égale à $5,5$ mg.

Déterminer, en détaillant les calculs, le nombre d'injections réalisées en appliquant ce protocole.
	\end{enumerate}
\end{enumerate}

\bigskip

\textbf{\textsc{Exercice 2}  \quad (7 points)\hfill Thème: géométrie dans l'espace}

\medskip

Dans l'espace rapporté à un repère orthonormé \Oijk, on considère:

$\bullet~~$ le point A de coordonnées $(-1~;~1~;~3)$,

$\bullet~~$la droite $\mathcal{D}$ dont une représentation paramétrique est: $\left\{\begin{array}{l c l}
x&=&1+2t\\y &=& 2 - t,\\z&=& 2+2t
\end{array}\right.  t \in \R$.\index{equation paramétrique de droite@équation paramétrique de droite}

On admet que le point A n'appartient pas à la droite $\mathcal{D}$.

\medskip

\begin{enumerate}
\item 
	\begin{enumerate}
		\item Donner les coordonnées d'un vecteur directeur $\vect{u}$ de la droite $\mathcal{D}$.\index{vecteur directeur}
		
		\item Montrer que le point B$(-1~;~3~;~0)$ appartient à la droite $\mathcal{D}$.
		\item Calculer le produit scalaire $\vect{\text{AB}} \cdot \vect{u}$.\index{produit scalaire}
	\end{enumerate}	
\item On note $\mathcal{P}$ le plan passant par le point A et orthogonal à la droite $\mathcal{D}$, et on appelle H le point d'intersection du plan $\mathcal{P}$ et de la droite $\mathcal{D}$. Ainsi, H est le projeté orthogonal de A sur la droite $\mathcal{D}$.
	\begin{enumerate}
		\item Montrer que le plan $\mathcal{P}$ admet pour équation cartésienne: $2x - y + 2z - 3 = 0$.\index{equation de plan@équation de plan}
		\item En déduire que le point H a pour coordonnées $\left(\dfrac79~;~\dfrac{19}{9}~;~\dfrac{16}{9}\right)$.
		\item Calculer la longueur AH. On donnera une valeur exacte.
	\end{enumerate}	
\item Dans cette question, on se propose de retrouver les coordonnées du point H, projeté orthogonal du point A sur la droite $\mathcal{D}$, par une autre méthode.

On rappelle que le point B$(-1~;~3~;~0)$ appartient à la droite $\mathcal{D}$ et que le vecteur $\vect{u}$ est un vecteur directeur de la droite $\mathcal{D}$.
	\begin{enumerate}
		\item Justifier qu'il existe un nombre réel $k$ tel que $\vect{\text{HB}} = k\vect{u}$.\index{vecteurs colinéaires}
		\item Montrer que $k = \dfrac{\vect{\text{AB}} \cdot \vect{u}}{\left\|\vect{u}\right\|^2}$.
		\item Calculer la valeur du nombre réel $k$ et retrouver les coordonnées du point H.
	\end{enumerate}
\item On considère un point C appartenant au plan $\mathcal{P}$ tel que le volume du tétraèdre ABCH soit égal à $\dfrac89$.

Calculer l'aire du triangle ACH.

On rappelle que le volume d'un tétraèdre est donné par: $V = \dfrac13 \times \mathcal{B} \times h$ où $\mathcal{B}$ désigne l'aire d'une base et $h$ la hauteur relative à cette base.\index{volume tétraèdre}
\end{enumerate}

\bigskip

\textbf{\textsc{Exercice 3} \quad (7 points)\hfill Thème: probabilités}

\medskip

Le directeur d'une grande entreprise a proposé à l'ensemble de ses salariés un stage de formation à l'utilisation d'un nouveau logiciel.

Ce stage a été suivi par 25\,\% des salariés.

\medskip

\begin{enumerate}
\item Dans cette entreprise, 52\,\% des salariés sont des femmes, parmi lesquelles 
40\,\% ont suivi le stage.

On interroge au hasard un salarié de l'entreprise et on considère les évènements:

\setlength\parindent{1cm}
\begin{itemize}
\item[$\bullet~~$] $F$ : \og le salarié interrogé est une femme \fg,
\item[$\bullet~~$] $S$ : \og le salarié interrogé a suivi le stage \fg.
\end{itemize}
\setlength\parindent{0cm}

$\overline{F}$ et $\overline{S}$ désignent respectivement les évènements contraires des évènements $F$ et $S$.\index{probabilités}
\end{enumerate}

\begin{minipage}{0.6\linewidth} 
\textbf{a.~~}Donner la probabilité de l'évènement $S$.

\textbf{b.~~}Recopier et compléter les pointillés de l'arbre pondéré ci-contre sur
les quatre branches indiquées.\index{arbre pondéré}

\textbf{c.~~}Démontrer que la probabilité que la personne interrogée soit une femme ayant suivi le stage est égale à $0,208$.

\textbf{d.~~}On sait que la personne interrogée a suivi le stage. Quelle est la probabilité que ce soit une femme ?

\textbf{e.~~}Le directeur affirme que, parmi les hommes salariés de l'entreprise, moins de 10\,\% ont suivi le stage.
		
Justifier l'affirmation du directeur.
\end{minipage} \hfill
\begin{minipage}{0.4\linewidth} 
\begin{center}
\pstree[treemode=R,nodesepA=0pt,nodesepB=2.5pt,treesep=1cm,levelsep=2.5cm]{\TR{}}
{\pstree{\TR{$F$~}\taput{\ldots}}
	{\TR{$S$}\taput{\ldots}
	\TR{$\overline{S}$}\tbput{\ldots}
	}
\pstree{\TR{$\overline{F}$~}\tbput{\ldots}}
	{\TR{$S$}
	\TR{$\overline{S}$}
	}
}
\end{center}
\end{minipage}
\begin{enumerate}[resume]
\item On note $X$ la variable aléatoire qui à un échantillon de $20$ salariés de cette entreprise choisis au hasard associe le nombre de salariés de cet échantillon ayant suivi le stage. On suppose que l'effectif des salariés de l'entreprise est suffisamment important pour assimiler ce choix à un tirage avec remise.\index{loi binomiale}
	\begin{enumerate}
		\item Déterminer, en justifiant, la loi de probabilité suivie par la variable aléatoire $X$.\index{loi binomiale}
		\item Déterminer, à $10^{-3}$ près, la probabilité que $5$ salariés dans un échantillon de $20$ aient suivi le stage.
		\item Le programme ci-dessous, écrit en langage Python, utilise la fonction \textbf{binomiale$(i, n, p)$} créée pour l'occasion qui renvoie la valeur de la probabilité $P(X = i)$ dans le cas où la variable aléatoire $X$ suit une loi binomiale de paramètres $n$ et $p$.\index{script python}

\begin{center}
\begin{tabular}{|l|}\hline
def proba(k):\\
\quad  P=0\\
\quad for i in range(0,k+1) :\\
\qquad P=P+binomiale(i,20,0.25)\\
\quad return P\\ \hline
\end{tabular}
\end{center}

Déterminer, à $10^{-3}$ près, la valeur renvoyée par ce programme lorsque l'on saisit proba(5) dans la console Python.

Interpréter cette valeur dans le contexte de l'exercice.
		\item Déterminer, à $10^{-3}$ près, la probabilité qu'au moins $6$ salariés dans un échantillon de $20$ aient suivi le stage.
		\end{enumerate}	
\item Cette question est indépendante des questions 1 et 2.

Pour inciter les salariés à suivre le stage, l'entreprise avait décidé d'augmenter les salaires des salariés ayant suivi le stage de 5\,\%, contre 2\,\% d'augmentation pour les salariés n'ayant pas suivi le stage. 

Quel est le pourcentage moyen d'augmentation des salaires de cette entreprise dans ces conditions ?\index{pourcentage moyen}

\end{enumerate}

\bigskip

\textbf{\textsc{Exercice 4} (7 points)\hfill Thème: fonctions numériques}

\medskip

\emph{Cet exercice est un questionnaire à choix multiple.\\\index{QCM}
Pour chaque question, une seule des quatre réponses proposées est exacte. Le candidat indiquera sur sa copie le numéro de la question et la réponse choisie. \\
Aucune justification n'est demandée.\\
Une réponse fausse, une réponse multiple ou l'absence de réponse à une question ne rapporte ni n'enlève de point.}

\emph{Les six questions sont indépendantes}

\medskip

\begin{enumerate}
\item La courbe représentative de la fonction $f$ définie sur $\R$ par $f(x) = \dfrac{-2x^2 + 3x - 1}{x^2 + 1}$ admet pour asymptote
la droite d'équation:\index{asymptote}

\begin{center}
\begin{tabularx}{\linewidth}{*{2}{X}}
\textbf{a.~~}$x = -2$ ; &\textbf{b.~~} $y = -1$;\\
\textbf{c.~~} $y = - 2$;&\textbf{d.~~} $y = 0$
\end{tabularx}
\end{center}
\item Soit $f$ la fonction définie sur $\R$ par $f(x) = x\text{e}^{x^2}$.

La primitive $F$ de $f$ sur $\R$ qui vérifie $F(0) = 1$ est définie par :\index{primitive}

\begin{center}
\renewcommand\arraystretch{1.9}
\begin{tabularx}{\linewidth}{*{2}{X}}
\textbf{a.~~} $F(x) = \dfrac{x^2}{2}\text{e}^{x^2}$ ;&\textbf{b.~~} $F(x) = \dfrac{1}{2}\text{e}^{x^2}$\\
\textbf{c.~~} $F(x) = \left(1 + 2x^2\right)\text{e}^{x^2}$ ;&\textbf{d.~~} $F(x) = \dfrac{1}{2}\text{e}^{x^2} + \dfrac{1}{2}$
\end{tabularx}
\renewcommand\arraystretch{1}
\end{center}

\item ~

\begin{minipage}{0.4\linewidth}
On donne ci-contre la représentation graphique $\mathcal{C}_{f'}$ de \textbf{la fonction dérivée }\boldmath $f'$\unboldmath{} d'une fonction $f$ définie sur $\R$.

On peut affirmer que la fonction $f$ est :\index{convexité}

\textbf{a.~~} concave sur $]0~;~+\infty[$ ;

\textbf{b.~~} convexe sur $]0~;~+\infty[$ ;

\textbf{c.~~} convexe sur [0~;~2] ;

\textbf{d.~~} convexe sur $[2~;~+\infty[$.
\end{minipage}\hfill
\begin{minipage}{0.58\linewidth}
\psset{unit=0.75cm}
\begin{pspicture*}(-0.5,-5)(10,1)
\psgrid[gridlabels=0pt,subgriddiv=1,gridwidth=0.25pt](-0.5,-5)(10,1)
\psaxes[linewidth=1.25pt,Dx=11,Dy=11](0,0)(-0.5,-5)(10,1)
%\psecurve[linewidth=1.25pt,linecolor=blue](-0.3,-6)(-0.2,-5)(0,-4)(1,-1.2)(2,0)(3,0.45)(4,0.5)(5,0.45)(6,0.4)(10,0.1)(11,0.08)
\psplot[plotpoints=2000,linewidth=1.25pt,linecolor=blue]{-0.3}{11}{x 2 sub 2 mul 2.71828 x 2 div exp div}
\uput[d](1,0){\footnotesize 1}\uput[d](2,0){\footnotesize 2}\uput[dl](0,0){\footnotesize 0}\uput[d](0,1){\footnotesize 1}
\uput[r](0.4,-3.5){\blue$\mathcal{C}_{f'}$}
\end{pspicture*}
\end{minipage}

\bigskip
\item Parmi les primitives de la fonction $f$ définie sur $\R$  par $f(x) = 3\text{e}^{-x^2} + 2$ :\index{primitive}

\begin{center}
\begin{tabularx}{\linewidth}{*{2}{X}}
\textbf{a.~~} toutes sont croissantes sur $\R$ ;&\textbf{b.~~} toutes sont décroissantes sur $\R$ ;\\
\textbf{c.~~} certaines sont croissantes sur $\R$ et d'autres
décroissantes sur $\R$ ;& \textbf{d.~~} toutes sont croissantes sur $]-\infty~;~0]$ et
décroissantes sur $[0~;~+\infty[$.
\end{tabularx}
\end{center}

\item La limite en $+\infty$ de la fonction $f$ définie sur l'intervalle $]0~;~+\infty[$ par $f(x) = \dfrac{2\ln x}{3x^2 + 1}$ est égale à :\index{limite de fonction}

\begin{center}
\begin{tabularx}{\linewidth}{*{4}{X}}
\textbf{a.~~} $\dfrac23$ ;&\textbf{b.~~} $+ \infty$ ;&\textbf{c.~~} $- \infty$ ; &\textbf{d.~~} $0$.
\end{tabularx}
\end{center}

\item L'équation $\text{e}^{2x} + \text{e}^x - 12 = 0$ admet dans $\R$ :\index{equation avec exponentielle@équation avec exponentielle}

\begin{center}
\begin{tabularx}{\linewidth}{*{4}{X}}
\textbf{a.~~} trois solutions; &\textbf{b.~~} deux solutions; &\textbf{c.~~} une seule solution;&\textbf{d.~~} aucune solution.
\end{tabularx}
\end{center}
\end{enumerate}

\newpage
%%%%%%%%%%% Centresetrangers1 11 mai 2022
\phantomsection
\hypertarget{Centresetrangers1}{}

\label{Centresetrangers1}

\lfoot{\small{Centres étrangers}}
\rfoot{\small{11  mai 2022}}
\pagestyle{fancy}
\thispagestyle{empty}

\begin{center}{\Large\textbf{\decofourleft~Baccalauréat Centres étrangers\footnote{Arabie saoudite, Bahreïn, Chypre, Éthiopie, Grèce, Israël, Jordanie, Koweït, Qatar, Roumanie et Turquie} 11 mai 2022~\decofourright\\[7pt]  Sujet 1\\[7pt] ÉPREUVE D'ENSEIGNEMENT DE SPÉCIALITÉ}}
\end{center}

\vspace{0,25cm}

Le sujet propose 4 exercices

Le candidat choisit 3 exercices parmi les 4 exercices et \textbf{ne doit traiter que ces 3 exercices}

Chaque exercice est noté sur 7 points (le total sera ramené sur 20 points).
Les traces de recherche, même incomplètes ou infructueuses, seront prises en compte.

\bigskip

\textbf{\textsc{Exercice 1} \quad 7 points\hfill Thème: Fonction logarithme }

\medskip

\emph{Cet exercice est un questionnaire à choix multiples. Pour chacune des questions suivantes, une seule des quatre réponses proposées est exacte. Les six questions sont indépendantes.}\index{QCM}

\smallskip

\emph{Une réponse incorrecte, une réponse multiple ou l'absence de réponse à une question ne rapporte ni n'enlève de point. Pour répondre, indiquer sur la copie le numéro de la question et la lettre de la réponse choisie. \\
Aucune justification n'est demandée.}

\medskip

\begin{enumerate}
\item On considère la fonction $f$ définie pour tout réel $x$ par $f(x) = \ln \left(1  + x^2\right)$.\index{fonction logarithme}

Sur $\R$, l'équation $f(x) = \np{2022}$

\begin{center}
\begin{tabularx}{\linewidth}{*{2}{X}}
\textbf{a.~~} n'admet aucune solution. &\textbf{b.~~} admet exactement une solution.\\
\textbf{c.~~}admet exactement deux solutions.&\textbf{d.~~} admet une infinité de solutions.
\end{tabularx}
\end{center}\index{valeurs intermédiaires}

\item Soit la fonction $g$ définie pour tout réel $x$ strictement positif par: 
\[g(x) = x \ln (x) - x^2\]\index{fonction logarithme}

On note $\mathcal{C}_g$ sa courbe représentative dans un repère du plan.\index{convexité}\index{point d'inflexion}

\begin{center}
\begin{tabularx}{\linewidth}{*{2}{X}}
\textbf{a.~~}La fonction $g$ est convexe sur $]0~;~+\infty[$.&\textbf{b.~~}La fonction $g$ est concave sur $]0~;~+\infty[$.\\
\textbf{c.~~}La courbe $\mathcal{C}_g$ admet exactement un point d'inflexion sur $]0~;~+\infty[$.&\textbf{d.~~}La courbe $\mathcal{C}_g$ admet exactement
deux points d'inflexion sur $]0~;~+\infty[$.
\end{tabularx}
\end{center}

\item On considère la fonction $f$ définie sur $]- 1~;~1[$ par 

\[f(x) = \dfrac{x}{1 - x^2}\]

Une primitive de la fonction $f$ est la fonction $g$ définie sur l'intervalle $] - 1~;~1[$ par :\index{primitive}

\begin{center}
\begin{tabularx}{\linewidth}{*{2}{X}}
\textbf{a.~~} $g(x) = - \dfrac12 \ln \left(1 - x^2\right)$&\textbf{b.~~} $g(x) =  \dfrac{1 + x^2}{ \left(1 - x^2\right)^2}$\\
\textbf{c.~~} $g(x)= \dfrac{x^2}{2\left(x - \dfrac{x^3}{3}\right)}$&\textbf{d.~~} $g(x) = \dfrac{x^2}{2}\ln \left(1 - x^2\right)$
\end{tabularx}
\end{center}

\item La  fonction $x \longmapsto  \ln \left(-x^2- x + 6\right)$ est définie sur\index{ensemble de définition}\index{fonction logarithme}

\begin{center}
\begin{tabularx}{\linewidth}{*{2}{X}}
\textbf{a.~~} $]- 3~;~2[$&\textbf{b.~~} $]- \infty~;~6]$\\
\textbf{c.~~} $]0~;~+\infty[$&\textbf{d.~~} $]2~;~+\infty[$
\end{tabularx}
\end{center}

\item On considère la fonction $f$ définie sur $]0,5~;~+ \infty [$ par 

\[f(x) =x^2- 4x+ 3 \ln (2x - 1)\]\index{fonction logarithme}

Une équation de la tangente à la courbe représentative de $f$ au point d'abscisse 1 est:\index{equation de tangente@équation de tangente}

\begin{center}
\begin{tabularx}{\linewidth}{*{2}{X}}
\textbf{a.~~}$y = 4x- 7$ &\textbf{b.~~} $y = 2x - 4$\\
\textbf{c.~~} $y = -3(x - 1) + 4$ &\textbf{d.~~} $y = 2x - 1$
\end{tabularx}
\end{center}

\item L'ensemble $S$ des solutions dans $\R$ de l'inéquation $\ln (x + 3) < 2\ln (x + 1)$ est:\index{fonction logarithme}\index{inéquation}

\begin{center}
\begin{tabularx}{\linewidth}{*{2}{X}}
\textbf{a.~~}$S = ]- \infty~;~-2[ \cup ]1~;~+\infty[$&\textbf{b.~~}$S = ]1~;~+ \infty[$\\
\textbf{c.~~} $S = \emptyset$&\textbf{d.~~} $S = ]- 1~;~1[$
\end{tabularx}
\end{center}

\end{enumerate}

\bigskip

\textbf{\textsc{Exercice 2} \quad 7 points\hfill Thème: Géométrie dans l'espace}

\medskip

Dans l'espace, rapporté à un repère orthonormé \Oijk, on considère les points : 

\[\text{A}(2~;~0~;~3),\: \text{B}(0~;~2~;~1), \text{C}(-1~;~-1~;~2)\:\: \text{et D}(3~;~-3~;~ -1).\]
\smallskip

\begin{enumerate}
\item \textbf{Calcul d'un angle}
	\begin{enumerate}
		\item Calculer les coordonnées des vecteurs $\vect{\text{AB}}$ et $\vect{\text{AC}}$ et en déduire que les points A, B et C ne sont pas alignés.
		\item Calculer les longueurs AB et AC.
		\item À l'aide du produit scalaire $\vect{\text{AB}}\cdot \vect{\text{AC}}$, déterminer la valeur du cosinus de l'angle
$\widehat{\text{BAC}}$ puis donner une valeur approchée de la mesure de l'angle $\widehat{\text{BAC}}$ au dixième de degré.\index{produit scalaire}
	\end{enumerate}
\item \textbf{Calcul d'une aire}
	\begin{enumerate}
		\item Déterminer une équation du plan $\mathcal{P}$ passant par le point C et perpendiculaire à la droite (AB).\index{equation de plan@équation de plan}
		\item Donner une représentation paramétrique de la droite (AB).\index{equation de droite@équation de droite}
		\item En déduire les coordonnées du projeté orthogonal E du point C sur la droite
(AB), c'est-à-dire du point d'intersection de la droite (AB) et du plan $\mathcal{P}$
		\item Calculer l'aire du triangle ABC.
	\end{enumerate}
\item \textbf{Calcul d'un volume}
	\begin{enumerate}
		\item Soit le point F$(1~;~-1~;~3)$. Montrer que les points A, B, C et F sont coplanaires.
		\item Vérifier que la droite (FD) est orthogonale au plan (ABC).
		\item Sachant que le volume d'un tétraèdre est égal au tiers de l'aire de sa base
multiplié par sa hauteur, calculer le volume du tétraèdre ABCD.\index{volume tétraèdre}
	\end{enumerate}
\end{enumerate}

\bigskip

\textbf{\textsc{Exercice 3} \quad 7 points\hfill Thèmes: Fonction exponentielle et suite}

\bigskip

\textbf{Partie A :}

\medskip

Soit $h$ la fonction définie sur $\R$ par

\[h(x) = \text{e}^x - x\]\index{fonction exponentielle}

\smallskip

\begin{enumerate}
\item Déterminer les limites de $h$ en $-\infty$ et $+\infty$.\index{limite de fonction}
\item Étudier les variations de $h$ et dresser son tableau de variation.\index{tableau de variations}
\item En déduire que :

si $a$ et $b$ sont deux réels tels que $0 < a < b$ alors $h(a) - h(b) < 0$.
\end{enumerate}

\bigskip

\textbf{Partie B :}

\medskip

Soit $f$ la fonction définie sur $\R$ par

\[f(x) = \text{e}^x\]

On note $\mathcal{C}_f$ sa courbe représentative dans un repère \Oij.

\medskip

\begin{enumerate}
\item Déterminer une équation de la tangente $T$ à $\mathcal{C}_f$ au point d'abscisse 0.\index{equation de tangente@équation de tangente}
\end{enumerate}

Dans la suite de l'exercice on s'intéresse à l'écart entre $T$ et $\mathcal{C}_f$ au voisinage de $0$.

Cet écart est défini comme la différence des ordonnées des points de $T$ et $\mathcal{C}_f$ de même abscisse.

On s'intéresse aux points d'abscisse $\dfrac{1}{n}$, avec $n$ entier naturel non nul.

On considère alors la suite $\left(u_n\right)$ définie pour tout entier naturel non nul $n$ par : 

\[u_n = \text{exp} \left(\dfrac{1}{n}\right) - \dfrac{1}{n} - 1\]\index{suites}

\begin{enumerate}[resume]
\item Déterminer la limite de la suite $\left(u_n\right)$.\index{limite de suite}
\item 
	\begin{enumerate}
		\item Démontrer que, pour tout entier naturel non nul $n$,
		
\[u_{n+1} - u_n = h\left(\dfrac{1}{n + 1}\right) - h\left(\dfrac{1}{n}\right) \]

où $h$ est la fonction définie à la partie A.
		\item En déduire le sens de variation de la suite $\left(u_n\right)$.
	\end{enumerate}
\item Le tableau ci-dessous donne des valeurs approchées à $10^{-9}$ des premiers termes de la suite $\left(u_n\right)$.

\begin{center}
$\begin{array}{|l|c|}\hline
n &u_n\\ \hline
1 &\np{0,718281828}\\ \hline
2 &\np{0,148721271}\\ \hline
3 &\np{0,062279092}\\ \hline
4 &\np{0,034025417}\\ \hline
5 &\np{0,021402758}\\ \hline
6 &\np{0,014693746}\\ \hline
7 &\np{0,010707852}\\ \hline
8 &\np{0,008148453}\\ \hline
9 &\np{0,006407958}\\ \hline
10 &\np{0,005170918}\\ \hline
\end{array}$
\end{center}

Donner la plus petite valeur de l'entier naturel $n$ pour laquelle l'écart entre $T$ et $\mathcal{C}_f$ semble être inférieur à $10^{-2}$.
\end{enumerate}

\bigskip

\textbf{\textsc{Exercice 4} \quad 7 points\hfill Thème: Probabilités}

\bigskip

Les parties A et B peuvent être traitées de façon indépendante.

\medskip

Au cours de la fabrication d'une paire de lunettes, la paire de verres doit subir deux traitements notés T1 et T2.

\bigskip

\textbf{Partie A}

\medskip

On prélève au hasard une paire de verres dans la production.

On désigne par $A$ l'évènement : \og la paire de verres présente un défaut pour le traitement T1 \fg.

On désigne par $B$ l'évènement : \og la paire de verres présente un défaut pour le traitement T2 \fg.

On note respectivement $\overline{A}$ et $\overline{B}$ les évènements contraires de $A$ et $B$.

Une étude a montré que :

\begin{itemize}
\item la probabilité qu'une paire de verres présente un défaut pour le traitement T1 notée $P(A)$ est égale à 0,1.
\item la probabilité qu'une paire de verres présente un défaut pour le traitement T2 notée $P(B)$ est égale à 0,2.
\item la probabilité qu'une paire de verres ne présente aucun des deux défauts est 0,75.
\end{itemize}

\medskip

\begin{enumerate}
\item Recopier et compléter le tableau suivant avec les probabilités correspondantes.\index{tableau de probabilités}

\begin{center}
\begin{tabularx}{0.45\linewidth}{|*{4}{>{\centering \arraybackslash}X|}}\hline
				&$A$ 	&$\overline{A}$	&Total\\ \hline
$B$				&		&				&\\ \hline
$\overline{B}$	&		&				&\\ \hline
Total			&		&				&1\\ \hline
\end{tabularx}
\end{center}

\item
	\begin{enumerate}
		\item Déterminer, en justifiant la réponse, la probabilité qu'une paire de verres, prélevée au hasard dans la production, présente un défaut pour au moins un des deux traitements T1 ou T2.\index{probabilités}
		\item Donner la probabilité qu'une paire de verres, prélevée au hasard dans la production, présente deux défauts, un pour chaque traitement T1 et T2.
		\item Les évènements $A$ et $B$ sont-ils indépendants? Justifier la réponse.\index{evènements indépendants@évènements indépendants}
	\end{enumerate}	
\item Calculer la probabilité qu'une paire de verres, prélevée au hasard dans la production, présente un défaut pour un seul des deux traitements.
\item Calculer la probabilité qu'une paire de verres, prélevée au hasard dans la production, présente un défaut pour le traitement T2, sachant que cette paire de verres présente un défaut pour le traitement T1.
\end{enumerate}

\bigskip

\textbf{Partie B}

\medskip

On prélève, au hasard, un échantillon de $50$ paires de verres dans la production. On suppose que la production est suffisamment importante pour assimiler ce prélèvement à un tirage avec remise. 

On note $X$ la variable aléatoire qui, à chaque échantillon de ce type, associe le nombre de paires de verres qui présentent le défaut pour le traitement T1.

\medskip

\begin{enumerate}
\item Justifier que la variable aléatoire $X$ suit une loi binomiale et préciser les paramètres de cette loi.\index{loi binomiale}
\item Donner l'expression permettant de calculer la probabilité d'avoir, dans un tel échantillon, exactement $10$ paires de verres qui présentent ce défaut. 

Effectuer ce calcul et arrondir le résultat à $10^{-3}$.
\item En moyenne, combien de paires de verres ayant ce défaut peut-on trouver dans un échantillon de $50$ paires ?
\end{enumerate}

\newpage
%%%%%%%%%%% Métropole2 12 mai 2022
\phantomsection
\hypertarget{Metropole2}{}

\label{Metropole2}

\lfoot{\small{Métropole}}
\rfoot{\small{12 mai 2022}}
\pagestyle{fancy}
\thispagestyle{empty}

\begin{center}{\Large\textbf{\decofourleft~Baccalauréat Métropole 12 mai 2022~\decofourright\\[6pt]  Sujet 2\\[7pt] ÉPREUVE D'ENSEIGNEMENT DE SPÉCIALITÉ}}
\end{center}

\vspace{0,25cm}

Le sujet propose 4 exercices

Le candidat choisit 3 exercices parmi les 4 et \textbf{ne doit traiter que ces 3 exercices}

\bigskip

\textbf{\textsc{Exercice 1} \quad (7 points)\hfill Thème : probabilités}\index{probabilités}

\medskip

Le coyote est un animal sauvage proche du loup, qui vit en Amérique du Nord.

Dans l'état d'Oklahoma, aux États-Unis, 70\,\% des coyotes sont touchés par une maladie appelée ehrlichiose.

Il existe un test aidant à la détection de cette maladie. Lorsque ce test est appliqué à un coyote, son résultat est soit positif, soit négatif, et on sait que :

\setlength\parindent{1cm}
\begin{itemize}
\item[$\bullet~~$] Si le coyote est malade, le test est positif dans 97\,\% des cas.
\item[$\bullet~~$] Si le coyote n'est pas malade, le test est négatif dans 95\,\% des cas.
\end{itemize}
\setlength\parindent{0cm}

\bigskip

\textbf{Partie A}

\medskip

Des vétérinaires capturent un coyote d'Oklahoma au hasard et lui font subir un test pour l'ehrlichiose.

On considère les évènements suivants :

\setlength\parindent{1cm}
\begin{itemize}
\item[$\bullet~~$]$M$: \og le coyote est malade \fg;
\item[$\bullet~~$]$T$: \og le test du coyote est positif \fg.
\end{itemize}
\setlength\parindent{0cm}

On note $\overline{M}$ et $\overline{T}$ respectivement les évènements contraires de $M$ et $T$.

\medskip

\begin{enumerate}
\item Recopier et compléter l'arbre pondéré ci-dessous qui modélise la situation.\index{arbre pondéré}

\begin{center}
\pstree[treemode=R,nodesepA=0pt,nodesepB=2.5pt,treesep=1cm,levelsep=2.5cm]{\TR{}}
{\pstree{\TR{$M$~}\taput{\ldots}}
	{\TR{$T$}\taput{\ldots}
	\TR{$\overline{T}$}\tbput{\ldots}
	}
\pstree{\TR{$\overline{M}$~}\tbput{\ldots}}
	{\TR{$T$}\taput{\ldots}
	\TR{$\overline{T}$}\tbput{\ldots}
	}
}
\end{center}

\item Déterminer la probabilité que le coyote soit malade et que son test soit positif.
\item Démontrer que la probabilité de $T$ est égale à $0,694$.
\item On appelle \og valeur prédictive positive du test \fg{} la probabilité que le coyote soit effectivement malade sachant que son test est positif.

Calculer la valeur prédictive positive du test. On arrondira le résultat au millième.
\item  
\begin{enumerate}
\item Par analogie avec la question précédente, proposer une définition de la \og valeur prédictive négative du test \fg{} et calculer cette valeur en arrondissant au millième.
\item Comparer les valeurs prédictives positive et négative du test, et interpréter.
	\end{enumerate}	
\end{enumerate}

\bigskip

\textbf{Partie B}

\medskip

On rappelle que la probabilité qu'un coyote capturé au hasard présente un test positif est de $0,694$.

\medskip

\begin{enumerate}
\item Lorsqu'on capture au hasard cinq coyotes, on assimile ce choix à un tirage avec remise.

On note $X$ la variable aléatoire qui à un échantillon de cinq coyotes capturés au hasard associe le nombre de coyotes dans cet échantillon ayant un test positif.
\begin{enumerate}
\item Quelle est la loi de probabilité suivie par $X$ ? Justifier et préciser ses paramètres.\index{loi binomiale}
\item Calculer la probabilité que dans un échantillon de cinq coyotes capturés au hasard, un seul ait un test positif. On arrondira le résultat au centième.
\item Un vétérinaire affirme qu'il y a plus d'une chance sur deux qu'au moins quatre coyotes sur cinq aient un test positif : cette affirmation est-elle vraie ? Justifier la réponse.
	\end{enumerate}	
\item Pour tester des médicaments, les vétérinaires ont besoin de disposer d'un coyote présentant un test positif. Combien doivent-ils capturer de coyotes pour que la probabilité qu'au moins l'un d'entre eux présente un test positif soit supérieure à $0,99$ ?
\end{enumerate}

\bigskip

\textbf{\textsc{Exercice 2} \quad (7 points)\hfill Thèmes : fonctions numériques et suites}

\medskip

\emph{Cet exercice est un questionnaire à choix multiples. Pour chacune des questions suivantes, une seule des quatre réponses proposées est exacte. Une réponse fausse, une réponse multiple ou l'absence de réponse à une question ne rapporte ni n'enlève de point.\\
Pour répondre, indiquer sur la copie le numéro de la question et la lettre de la réponse choisie.\\Aucune justification n'est demandée.}\index{QCM}

\medskip

Pour les questions 1 à 3 ci-dessous, on considère une fonction $f$ définie et deux fois dérivable sur $\R$. La courbe de sa fonction dérivée $f'$ est donnée ci-dessous.

On admet que $f'$ admet un maximum en $- \dfrac{3}{2}$ et que sa courbe coupe l'axe des abscisses au point de coordonnées $\left(- \dfrac12~;~0\right)$.\index{maximum}

\begin{minipage}{0.5\linewidth}
\textbf{Question 1 }:

\textbf{a.~~} La fonction $f$ admet un maximum en $- \dfrac{3}{2}$ ;

\textbf{b.~~}  La fonction $f$ admet un maximum en $- \dfrac{1}{2}$ ;

\textbf{c.~~} La fonction $f$ admet un minimum en $- \dfrac{1}{2}$; 

\textbf{d.~~}  Au point d'abscisse $-1$, la courbe de la
fonction $f$ admet une tangente horizontale.
\end{minipage}\hfill
\begin{minipage}{0.48\linewidth}
\begin{center}
On rappelle que la courbe ci-dessous représente la fonction dérivée $f'$ de $f$.
\end{center}

\psset{unit=1.2cm}
\begin{pspicture*}(-5.2,-2.6)(1,1)
\psgrid[gridlabels=0pt,subgriddiv=4,gridwidth=0.25pt,subgridwidth=0.15pt]
\psaxes[linewidth=1.25pt,labelFontSize=\scriptstyle]{->}(0,0)(-5.2,-2.6)(1,1)
\psplot[plotpoints=2000,linewidth=1.25pt,linecolor=red]{-5}{3}{2 x mul 1 add 2.71828 x exp mul neg}
\end{pspicture*}
\end{minipage}

\medskip

\textbf{Question 2 }:

\begin{center}
\begin{tabularx}{\linewidth}{X X}
\textbf{a.~~}La fonction $f$ est convexe sur $\left]- \infty~;~- \dfrac32\right[$ ;&
\textbf{b.~~}La fonction $f$ est convexe sur $\left]- \infty~;~- \dfrac12\right[$ ;\\
\textbf{c.~~}La courbe $\mathcal{C}_f$ représentant la fonction $f$ n'admet pas de point d'inflexion ; &\textbf{d.~~}La fonction $f$ est concave sur $\left] - \infty~;~- \dfrac12\right[$.
\end{tabularx}\index{convexité}
\end{center}

\medskip

\textbf{Question 3}:

La dérivée seconde $f''$ de la fonction $f$ vérifie :

\begin{center}
\begin{tabularx}{\linewidth}{X X}
\textbf{a.~~} $f''(x) \geqslant  0$ pour $x \in \left]-\infty~;~- \dfrac12\right[$ ; &\textbf{b.~~} $f''(x) \geqslant  0$ pour $x \in [- 2~;~- 1]$ ;\\
\textbf{c.~~} $f''\left(- \dfrac32 \right) = 0$ ;&\textbf{d.~~} $f''(- 3) = 0$.
\end{tabularx}
\end{center}


\textbf{Question 4 :}

\medskip

On considère trois suites $\left(u_n\right)$, $\left(v_n\right)$ et $\left(w_n\right)$. On sait que, pour tout entier naturel $n$, on a : $u_n \leqslant v_n\leqslant  w_n$ et de plus: $\displaystyle\lim_{n \to + \infty} u_n= 1$ et $\displaystyle\lim_{n \to + \infty} w_n= 3$.

On peut alors affirmer que :\index{suites}

\begin{center}
\begin{tabularx}{\linewidth}{X X}
\textbf{a.~~} la suite $\left(v_n\right)$ converge ;&\textbf{b.~~} Si la suite 
$\left(u_n\right)$ est croissante alors la suite $\left(v_n\right)$ est minorée par $u_0$ ;\\
\textbf{c.~~} $1 \leqslant  v_0 \leqslant 3$ ;&\textbf{d.~~} la suite $\left(v_n\right)$ diverge.
\end{tabularx}
\end{center}

\medskip

\textbf{Question 5}:

\medskip

On considère une suite $\left(u_n\right)$ telle que, pour tout entier naturel $n$ non nul: $u_n \leqslant  u_{n+1}  \leqslant \dfrac1n$.\index{suites}

On peut alors affirmer que : 

\begin{center}
\begin{tabularx}{\linewidth}{X X}
\textbf{a.~~}la suite $\left(u_n\right)$ diverge ;&\textbf{b.~~}la suite $\left(u_n\right)$ converge ;\\
\textbf{c.~~}$\displaystyle\lim_{n \to + \infty} u_n =  0$ ;&\textbf{d.~~} $\displaystyle\lim_{n \to + \infty} u_n =  1$.
\end{tabularx}
\end{center}

\medskip

\textbf{Question 6}:

\medskip

On considère $\left(u_n\right)$ une suite réelle telle que pour tout entier naturel $n$, on a : $n < u_n < n + 1$. 

On peut affirmer que:

\begin{center}
\begin{tabularx}{\linewidth}{X X}
\textbf{a.~~}Il existe un entier naturel $N$ tel que $u_N$ est un entier ;&\textbf{b.~~} la suite $\left(u_n\right)$ est croissante ;\\
\textbf{c.~~} la suite $\left(u_n\right)$ est convergente ;&\textbf{d.~~}La suite $\left(u_n\right)$ n'a pas de limite.
\end{tabularx}
\end{center}

\bigskip

\textbf{\textsc{Exercice 3} \quad (7 points)\hfill Thème : géométrie dans l'espace}

\medskip

\begin{minipage}{0.6\linewidth}
On considère un cube ABCDEFGH et on appelle K le milieu du segment [BC].

On se place dans le repère $\left(\text{A}~;~\vect{\text{AB}},~\vect{\text{AD}},~\vect{\text{AE}}\right)$  et on considère  le tétraèdre EFGK.

On rappelle que le volume d'un tétraèdre est donné par: 

\[V = \dfrac13 \times \mathcal{B} \times h\]\index{volume tétraèdre}

où $\mathcal{B}$ désigne l'aire d'une base et $h$ la hauteur relative à cette base.
\end{minipage} \hfill
\begin{minipage}{0.35\linewidth}
\psset{unit=1cm}
\begin{pspicture}(5.5,5.8)
\psframe(0.2,0.2)(3.7,3.7)%ABFE
\psline(3.7,0.2)(5,1.9)(5,5.4)(3.7,3.7)(4.35,1.05)%BCGF
\psline(5,5.4)(1.5,5.4)(0.2,3.7)%GHE
\psline[linestyle=dashed](0.2,0.2)(1.5,1.9)(5,1.9)%ADC
\psline[linestyle=dashed](1.5,1.9)(1.5,5.4)%DH
\pspolygon[linestyle=dotted,linewidth=1.25pt](0.2,3.7)(5,5.4)(4.35,1.05)%EGK
\uput[dl](0.2,0.2){\small A}\uput[dr](3.7,0.2){\small B}\uput[r](5,1.9){\small C}
\uput[dr](1.5,1.9){\small D}\uput[l](0.2,3.7){\small E}\uput[r](3.7,3.7){\small F}
\uput[ur](5,5.4){\small G}\uput[ul](1.5,5.4){\small H}\uput[dr](4.35,1.05){\small K}
\end{pspicture}
\end{minipage}

\medskip

\begin{enumerate}
%%%%%%%
\item Préciser les coordonnées des points E, F{}, G et K.
\item Montrer que le vecteur $\vect{n}\begin{pmatrix}\phantom{-}2\\-2\\\phantom{-}1\end{pmatrix}$ est orthogonal au plan (EGK).\index{vecteur normal}
\item Démontrer que le plan (EGK) admet pour équation cartésienne : $2x - 2y + z - 1 = 0.$\index{equation de plan@équation de plan}\index{equation de plan@équation de plan}
\item Déterminer une représentation paramétrique de la droite $(d)$ orthogonale au plan (EGK)
passant par F{}.
\item Montrer que le projeté orthogonal L de F sur le plan (EGK) a pour coordonnées $\left(\frac59~;~\frac49~;~\frac79\right)$.
\item Justifier que la longueur LF est égale à $\dfrac23$.
\item Calculer l'aire du triangle EFG. En déduire que le volume du tétraèdre EFGK est égal à $\dfrac16$.
\item Déduire des questions précédentes l'aire du triangle EGK.
\item On considère les points P milieu du segment [EG], M milieu du segment [EK] et N milieu du segment[GK]. Déterminer le volume du tétraèdre FPMN.
\end{enumerate}

\bigskip

\textbf{\textsc{Exercice 4} \quad (7 points)\hfill Thèmes : fonctions numériques, fonction exponentielle}

\begin{center}

\textbf{Partie A : études de deux fonctions}

\end{center}

On considère les deux fonctions $f$ et $g$ définies sur l'intervalle $[0~;~+\infty[$ par:

\[f(x) = 0,06\left(-x^2 +13,7x\right)\quad  \text{et}\quad  g(x) = (-0,15x + 2,2)\text{e}^{0,2x} - 2,2.\]

On admet que les fonctions $f$ et $g$ sont dérivables et on note $f'$ et $g'$ leurs fonctions dérivées respectives.

\medskip

\begin{enumerate}
\item On donne le tableau de variations complet de la fonction $f$ sur l'intervalle $[0~;~+\infty[$.\index{tableau de variations}

\begin{center}
\psset{unit=1cm,arrowsize=2pt 3}
\begin{pspicture}(6.5,2)
\psframe(6.5,2)\psline(0,1.5)(6.5,1.5)\psline(1.5,0)(1.5,2)
\uput[u](0.75,1.4){$x$} \uput[u](1.6,1.4){$0$} \uput[u](4,1.4){$6,85$} \uput[u](6,1.4){$+ \infty$} 
\rput(0.75,0.75){$f(x)$}\uput[u](1.65,0){$0$}\uput[d](4,1.5){$f(6,85)$}\uput[u](6,0){$- \infty$}
\psline{->}(1.75,0.25)(3.4,1.25)\psline{->}(4.6,1.25)(5.9,0.35)
\end{pspicture}
\end{center}

	\begin{enumerate}
		\item Justifier la limite de $f$ en $+\infty$.\index{limite de fonction}
		\item Justifier les variations de la fonction $f$.
		\item Résoudre l'équation $f(x) = 0$.
\end{enumerate}
\item 
	\begin{enumerate}
		\item Déterminer la limite de $g$ en $+\infty$.\index{limite de fonction}
		\item Démontrer que, pour tout réel $x$ appartenant à $[0~;~+\infty[$ on a : 
		
$g'(x) = (- 0,03x + 0,29)\text{e}^{0,2x}$.\index{dérivée}
		\item Étudier les variations de la fonction $g$ et dresser son tableau de variations sur $[0~;~+\infty[$.
		
Préciser une valeur approchée à $10^{-2}$ près du maximum de $g$.\index{maximum}
		\item Montrer que l'équation $g(x) = 0$ admet une unique solution non nulle et déterminer, à $10^{-2}$ près, une valeur approchée de cette solution.\index{valeurs intermédiaires}
	\end{enumerate}
\end{enumerate}

\bigskip

\textbf{Partie B : trajectoires d'une balle de golf}

\medskip

Pour frapper la balle, un joueur de golf utilise un instrument appelé \og club\fg{} de golf.

On souhaite exploiter les fonctions $f$ et $g$ étudiées en partie A pour modéliser de deux façons différentes la trajectoire d'une balle de golf. On suppose que le terrain est parfaitement plat.

On admettra ici que $13,7$ est la valeur qui annule la fonction $f$ et une approximation de la valeur qui annule la fonction $g$.

On donne ci-dessous les représentations graphiques de $f$ et $g$ sur l'intervalle [0~;~13,7].

\begin{center}
\psset{unit=1cm}
\begin{pspicture*}(-1,-1)(14,3.5)
\psgrid[gridlabels=0pt,subgriddiv=1,gridwidth=0.2pt]
\psaxes[linewidth=1.25pt,Dx=20,Dy=20]{->}(0,0)(-1,-1)(14,3.5)
\psplot[plotpoints=1000,linewidth=1.25pt,linecolor=red]{0}{13.7}{13.7 x mul x dup mul sub 0.06 mul}\uput[ul](3,2){\red $\mathcal{C}_f$}
\psplot[plotpoints=1000,linewidth=1.25pt,linecolor=blue]{0}{13.7}{2.2 0.15 x mul sub 2.71828 0.2 x mul exp mul 2.2 sub}\uput[ur](12,2.2){\blue $\mathcal{C}_g$}
\uput[dl](0,0){0}\uput[d](1,0){1}\uput[l](0,1){1}\uput[d](13.7,0){13,7}
\end{pspicture*}
\end{center}

Pour $x$ représentant la distance horizontale parcourue par la balle en dizaine de yards après la frappe, (avec $0 < x < 13,7$), $f(x)$ (ou $g(x)$ selon le modèle) représente la hauteur correspondante de la balle par rapport au sol, en dizaine de yards (1 yard correspond à environ $0,914$ mètre).

On appelle \og angle de décollage \fg{} de la balle, l'angle entre l'axe des abscisses et la tangente à la courbe ($\mathcal{C}_f$ ou $\mathcal{C}_g$ selon le modèle) en son point d'abscisse $0$. Une mesure de l'angle de décollage de la balle est un nombre réel $d$ tel que $\tan (d)$ est égal au coefficient directeur de cette tangente.\index{tangente d'angle}

De même, on appelle \og angle d'atterrissage \fg{} de la balle, l'angle entre l'axe des abscisses et la tangente à la courbe ($\mathcal{C}_f$ ou $\mathcal{C}_g$ selon le modèle) en son point d'abscisse $13,7$. Une mesure de l'angle d'atterrissage de la balle est un nombre réel $a$ tel que $\tan (a)$ est égal à l'opposé du coefficient directeur de cette tangente.

Tous les angles sont mesurés en degré.

\begin{center}
\begin{tabularx}{\linewidth}{|p{5.75cm}|X|}\hline
Le schéma illustre les angles de décollage et d'atterrissage associés à la courbe 
$\mathcal{C}_f$&Le schéma illustre les angles de décollage et d'atterrissage
associés à la courbe $\mathcal{C}_g$.\\ \hline
\psset{unit=0.4cm}
\begin{pspicture}(-0.3,-0.75)(14,3.2)
%\psgrid
\psaxes[linewidth=1.25pt,Dx=20,Dy=10](0,0)(-0.3,-0.3)(14,3)
\psplot[plotpoints=1000]{0}{2}{0.822 x mul}
\psplot[plotpoints=1000,linewidth=1.25pt,linecolor=red]{0}{13.7}{13.7 x mul x dup mul sub 0.06 mul}\uput[ul](3,2){\red $\mathcal{C}_f$}
\psline(13.7,0)(11,2)
\psarc(0,0){0.7}{0}{40}\rput(1.2,0.5){\footnotesize $d$}
\psarc(13.7,0){0.7}{140}{180}\rput(12.4,0.5){\footnotesize $a$}
\uput[d](13.7,0){\footnotesize 13,7}
\end{pspicture}&\psset{unit=0.5cm}
\begin{pspicture}(-0.3,-0.75)(14,3.2)
%\psgrid
\psaxes[linewidth=1.25pt,Dx=20,Dy=10](0,0)(-0.3,-0.5)(14,3)
\psplot[plotpoints=1000,linewidth=1.25pt,linecolor=blue]{0}{13.7}{2.2 0.15 x mul sub 2.71828 0.2 x mul exp mul 2.2 sub}
\psline(0,0)(10.4,2.8)
\psline(13.7,0)(12,2.9)
\psarc(0,0){1}{0}{20}\rput(2,0.25){\footnotesize $d$}
\psarc(13.7,0){1}{120}{180}\rput(12.4,0.5){\footnotesize $a$}
\uput[d](13.7,0){\footnotesize 13,7}
\end{pspicture}\\ \hline
\end{tabularx}
\end{center}

\medskip

\begin{enumerate}
\item \emph{Première modélisation} : on rappelle qu'ici, l'unité étant la dizaine de yards, $x$ représente la distance horizontale parcourue par la balle après la frappe et $f(x)$ la hauteur correspondante de la balle.

Selon ce modèle :
	\begin{enumerate}
		\item Quelle est la hauteur maximale, en yard, atteinte par la balle au cours de sa trajectoire ?
		\item Vérifier que $f'(0) = 0,822$.
		\item Donner une mesure en degré de l'angle de décollage de la balle, arrondie au dixième. (On pourra éventuellement utiliser le tableau ci-dessous).
		\item Quelle propriété graphique de la courbe $\mathcal{C}_f$ permet de justifier que les angles de décollage et d'atterrissage de la balle sont égaux ?
	\end{enumerate}
\item \emph{Seconde modélisation } : on rappelle qu'ici, l'unité étant la dizaine de yards, $x$ représente la distance horizontale parcourue par la balle après la frappe et $g(x)$ la hauteur correspondante de la balle.

Selon ce modèle :

	\begin{enumerate}
		\item Quelle est la hauteur maximale, en yard, atteinte par la balle au cours de sa trajectoire ?\index{maximum}

On précise que $g'(0) = 0,29$ et $g'(13,7) \approx -1,87$.
		\item Donner une mesure en degré de l'angle de décollage de la balle, arrondie au dixième. (On pourra éventuellement utiliser le tableau ci-dessous).
		\item Justifier que $62$ est une valeur approchée, arrondie à l'unité près, d'une mesure en degré de l'angle d'atterrissage de la balle.
	\end{enumerate}
\medskip

\textbf{Tableau :} extrait d'une feuille de calcul donnant une mesure en degré d'un angle quand on connait sa tangente :

\begin{center}
\begin{tabularx}{\linewidth}{|c|*{13}{>{\centering \arraybackslash}X|}}\hline
&A	&B	&C	&D	&E	&F	&G	&H	&I	&J	&K	&L	&M\\ \hline
1&$\tan (\theta)$&0,815&0,816&0,817&0,818&0,819&0,82&0,821&0,822&0,823&0,824&0,825&0,826\\ \hline
2&\scriptsize $\theta$ en degrés	&39,18&39,21&39,25	&39,28&39,32&39,35&39,39&39,42&39,45&39,49&39,52&39,56\\ \hline
3		&	&		&	&	&	&	&	&	&	&	&	&	&\\ \hline
4&$\tan (\theta)$&0,285& 0,286& 0,287& 0,288& 0,289&0,29&0,291&0,292&
0,293&0,294&0,295 &0,296\\ \hline
5&\scriptsize  $\theta$ en degrés&15,91 &15,96&16,01& 16,07& 16,12& 16,17& 16,23& 16,28& 16,33& 16,38& 16,44& 16,49\\ \hline
\end{tabularx}
\end{center}
\end{enumerate}

\bigskip

\begin{center}
\textbf{Partie C : interrogation des modèles}
\end{center}

À partir d'un grand nombre d'observations des performances de joueurs professionnels, on a obtenu les résultats moyens suivants:

\begin{center}
\begin{tabularx}{\linewidth}{|*{4}{>{\centering \arraybackslash \footnotesize}X|}}\hline
Angle de décollage en degré&Hauteur maximale en yard&Angle d'atterrissage en degré&Distance horizontale en yard au point de chute\\ \hline
24&32&52&137\\ \hline
\end{tabularx}
\end{center}

Quel modèle, parmi les deux étudiés précédemment, semble le plus adapté pour décrire la frappe de la balle par un joueur professionnel ? La réponse sera justifiée.

\newpage
%%%%%%%%%%% Centresetrangers2 12 mai 2022
\phantomsection
\hypertarget{Centresetrangers2}{}

\label{Centresetrangers2}

\lfoot{\small{Centres étrangers}}
\rfoot{\small{12 mai 2022}}
\pagestyle{fancy}
\thispagestyle{empty}

\begin{center}{\Large\textbf{\decofourleft~Baccalauréat Centres étrangers\footnote{Arabie saoudite, Bahreïn, Chypre, Éthiopie, Grèce, Israël, Jordanie, Koweït, Qatar, Roumanie et Turquie} 12 mai 2022~\decofourright\\[7pt]  Sujet 2\\[7pt] ÉPREUVE D'ENSEIGNEMENT DE SPÉCIALITÉ}}
\end{center}

\vspace{0,25cm}

Le sujet propose 4 exercices

Le candidat choisit 3 exercices parmi les 4 exercices et \textbf{ne doit traiter que ces 3 exercices}

Chaque exercice est noté sur 7 points (le total sera ramené sur 20 points).

Les traces de recherche, même incomplètes ou infructueuses, seront prises en compte.

\bigskip

\textbf{\textsc{Exercice 1} \quad 7 points\hfill Thème: Fonction exponentielle }

\medskip

\emph{Cet exercice est un questionnaire à choix multiples. Pour chacune des questions suivantes, une seule des quatre réponses proposées est exacte.}\index{QCM}

\medskip

\emph{Une réponse incorrecte, une réponse multiple ou l'absence de réponse à une question ne rapporte ni n'enlève de point. Pour répondre, indiquer sur la copie le numéro de la question et la lettre de la réponse choisie.\\
Aucune justification n'est demandée.}

\medskip

\begin{enumerate}
\item Soit $f$ la fonction définie sur $\R$ par

\[f(x) =\dfrac{x}{\text{e}^x}\]

On suppose que $f$ est dérivable sur $\R$ et on note $f'$ sa fonction dérivée.\index{dérivée}

\begin{center}
\begin{tabularx}{\linewidth}{*{2}{X}}
\textbf{a.~~}$f'(x) = \text{e}^{-x}$& \textbf{b.~~} $f'(x) = x\text{e}^{-x}$\\
\textbf{c.~~}$f'(x) = (1- x)\text{e}^{-x}$& \textbf{d.~~} $f'(x) = (1+x) \text{e}^{-x}$
\end{tabularx}
\end{center}

\item Soit $f$ une fonction deux fois dérivable sur l'intervalle $[-3~;~1]$. On donne ci-dessous la représentation graphique de sa fonction dérivée seconde $f''$.

\begin{center}
\psset{xunit=1.5cm,yunit=0.75cm}
\begin{pspicture*}(-3.5,-3.5)(1.5,3.5)
\psgrid[gridlabels=0pt,subgriddiv=1,gridwidth=0.2pt]
\psaxes[linewidth=1.25pt]{->}(0,0)(-3.5,-3.5)(1.5,3.5)
\psplot[plotpoints=2000,linewidth=1.25pt,linecolor=red]{-3}{1}{x 3 exp x dup mul 3 mul add x sub 3 sub}
\end{pspicture*}

\end{center}

On peut alors affirmer que:

\begin{center}
\begin{tabularx}{\linewidth}{*{2}{X}}
\textbf{a.~~}La fonction $f$ est convexe sur l'intervalle $[-1~;~1]$&\textbf{b.~~} La fonction $f$ est concave sur l'intervalle $[- 2~;~0]$\\
\textbf{c.~~}La fonction $f'$ est décroissante sur l'intervalle $[-2~;~0]$&\textbf{d.~~}
La fonction $f'$ admet un maximum en 

$x = -1$
\end{tabularx}
\end{center}\index{convexité}\index{maximum}

Pour $x< - 1$, \: $f''(x)  > 0$, donc $f'$ est croissante et pour $x > -1$, \: $f''(x) < 0$, donc $f'$ est décroissante. La fonction $f'$ admet donc un maximum en $x = - 1$. Réponse \textbf{d.}
\item On considère la fonction $f$ définie sur $\R$ par:

\[f(x) = x^3\text{e}^{-x^2}\]

Si $F$ est une primitive de $f$ sur $\R$, 

\begin{center}
\renewcommand\arraystretch{1.9}
\begin{tabularx}{\linewidth}{*{2}{X}}
\textbf{a.~~}$F(x) = - \dfrac16\left(x^3 + 1\right)\text{e}^{-x^2}$ &
\textbf{b.~~}$F(x) = - \dfrac14 x^4\text{e}^{-x^2}$\\
\textbf{c.~~}$F(x)=-\dfrac12\left(x^2 + 1\right)\text{e}^{-x^2}$&
\textbf{d.~~}$F(x)= x^2\left(3 - 2x^2\right)\text{e}^{-x^2}$
\end{tabularx}
\renewcommand\arraystretch{1}
\end{center}

\item Que vaut :

\[\displaystyle\lim_{x \to + \infty} \dfrac{\text{e}^x + 1}{\text{e}^x - 1}\]\index{limite de fonction}

\begin{center}
\begin{tabularx}{\linewidth}{*{2}{X}}
\textbf{a.~~} $-1$&\textbf{b.~~}  1\\
\textbf{c.~~} $+ \infty$&\textbf{d.~~} n'existe pas
\end{tabularx}
\end{center}

\item On considère la fonction $f$ définie sur $\R$ par $f(x) = \text{e}^{2x +1}$.\index{fonction exponentielle}

La seule primitive $F$ sur $\R$ de la fonction $f$ telle que $F(0) = 1$ est la fonction :\index{primitive}

\begin{center}
\begin{tabularx}{\linewidth}{*{2}{X}}
\textbf{a.~~}$x \longmapsto  2\text{e}^{2x+1} - 2\text{e} + 1$&\textbf{b.~~}$x \longmapsto  2\text{e}^{2x+1} - \text{e}$\\
\textbf{c.~~} $x \longmapsto \dfrac12\text{e}^{2x+1} - \dfrac12 \text{e} + 1$&\textbf{d.~~}$x \longmapsto \text{e}^{x^2 + x}$
\end{tabularx}
\end{center}

\item ~

\begin{minipage}{0.46\linewidth}
Dans un repère, on a tracé ci-contre la courbe représentative d'une fonction $f$ définie et deux fois dérivable sur $[- 2 ~;~4]$
\end{minipage} \hfill \begin{minipage}{0.48\linewidth}
\psset{unit=0.9cm}
\begin{pspicture}(-3,-2)(5,3)
\psgrid[gridlabels=0pt,subgriddiv=1,gridwidth=0.2pt]
\psaxes[linewidth=1.25pt,Dx=10,Dy=10]{->}(0,0)(-3,-2)(5,3)
\psaxes[linewidth=1.25pt](0,0)(1,0)(0,1)
\psplot[plotpoints=2000,linewidth=1.25pt,linecolor=red]{-2}{4}{x 3 exp 3 div x dup mul sub 8 3 div add  3 mul 8 div}
\uput[d](4.6,0){$x$}\uput[l](0,2.6){$y$}
\end{pspicture}

\end{minipage}

Parmi les courbes suivantes, laquelle représente la fonction $f''$,
dérivée seconde de $f$ ?\index{dérivée seconde}

\begin{center}
\begin{tabularx}{\linewidth}{*{2}{X}}
\textbf{a.~~} \psset{unit=0.6cm}
\begin{pspicture}(-3,-3)(5,3)
\psgrid[gridlabels=0pt,subgriddiv=1,gridwidth=0.2pt]
\psaxes[linewidth=1.25pt,Dx=10,Dy=10]{->}(0,0)(-3,-3)(5,3)
\psplot[plotpoints=2000,linewidth=1.25pt,linecolor=blue]{-2}{4}{x 1 sub }
\psaxes[linewidth=1.25pt](0,0)(1,0)(0,1)
\uput[d](4.6,0){$x$}\uput[l](0,2.6){$y$}
\end{pspicture}&\textbf{b.~~} \psset{unit=0.6cm}
\begin{pspicture}(-3,-3)(5,3)
\psgrid[gridlabels=0pt,subgriddiv=1,gridwidth=0.2pt]
\psaxes[linewidth=1.25pt,Dx=10,Dy=10]{->}(0,0)(-3,-3)(5,3)
\psplot[plotpoints=2000,linewidth=1.25pt,linecolor=blue]{-2}{4}{1 x sub}
\psaxes[linewidth=1.25pt](0,0)(1,0)(0,1)
\uput[d](4.6,0){$x$}\uput[l](0,2.6){$y$}
\end{pspicture}\\
\textbf{c.~~} \psset{unit=0.6cm}
\begin{pspicture}(-3,-2)(5,4)
\psgrid[gridlabels=0pt,subgriddiv=1,gridwidth=0.2pt]
\psaxes[linewidth=1.25pt,Dx=10,Dy=10]{->}(0,0)(-3,-2)(5,4)
\psplot[plotpoints=2000,linewidth=1.25pt,linecolor=blue]{-2}{4}{x 3  div x 2 sub mul}
\psaxes[linewidth=1.25pt](0,0)(1,0)(0,1)
\uput[d](4.6,0){$x$}\uput[l](0,3.6){$y$}
\end{pspicture}& \textbf{d.~~}
\psset{unit=0.6cm}
\begin{pspicture}(-3,-2)(5,4)
\psgrid[gridlabels=0pt,subgriddiv=1,gridwidth=0.2pt]
\psaxes[linewidth=1.25pt,Dx=10,Dy=10]{->}(0,0)(-3,-2)(5,4)
\psplot[plotpoints=2000,linewidth=1.25pt,linecolor=blue]{-1.8}{3.8}{x 1  sub dup mul  2 div}
\psaxes[linewidth=1.25pt](0,0)(1,0)(0,1)
\uput[d](4.6,0){$x$}\uput[l](0,3.6){$y$}
\end{pspicture}
\end{tabularx}
\end{center}
\end{enumerate}

\bigskip

\textbf{\textsc{Exercice 2} \quad 7 points\hfill Thèmes : Fonction logarithme et suite}

\medskip

Soit $f$ la fonction définie sur l'intervalle $]0~;~ +\infty[$ par 

\[f(x) = x\ln (x) + 1\]\index{fonction logarithme}

On note $\mathcal{C}_f$ sa courbe représentative dans un repère du plan.

\medskip

\begin{enumerate}
\item Déterminer la limite de la fonction $f$ en $0$ ainsi que sa limite en $+\infty$.\index{limite de fonction}
\item 
	\begin{enumerate}
		\item On admet que $f$ est dérivable sur $]0~;~+\infty[$ et on notera $f'$ sa fonction dérivée.
		
Montrer que pour tout réel $x$ strictement positif :

\[f'(x) = 1 + \ln (x).\]\index{dérivée}

		\item En déduire le tableau de variation de la fonction $f$ sur $]0~;~ +\infty[$. On y fera figurer la valeur exacte de l'extremum de $f$ sur $]0~;~ +\infty[$ et les limites.\index{tableau de variations}
		\item Justifier que pour tout $x \in  ]0~;~1[,\:f(x) \in ]0~;~1[$.
	\end{enumerate}	
\item 
	\begin{enumerate}
		\item Déterminer une équation de la tangente $(T)$ à la courbe $\mathcal{C}_f$ au point d'abscisse 1.\index{equation de tangente@équation de tangente}
		\item Étudier la convexité de la fonction $f$ sur $]0~;~+\infty[$.\index{convexité}
		\item En déduire que pour tout réel $x$ strictement positif:
		
		\[f(x) \geqslant x\]
		
	\end{enumerate}
\item On définit la suite $\left(u_n\right)$ par son premier terme $u_0$ élément de l'intervalle ]0~;~1[ et pour tout entier naturel $n$ :\index{suites}

\[u_{n+1} = f\left(u_n\right)\]

	\begin{enumerate}
		\item Démontrer par récurrence que pour tout entier naturel $n$, on a : $0 < u_n < 1$.\index{récurrence}
		\item Déduire de la question 3. c. la croissance de la suite $\left(u_n\right)$.
		\item En déduire que la suite $\left(u_n\right)$ est convergente.\index{convergence de suite}
	\end{enumerate}
\end{enumerate}

\bigskip

\textbf{\textsc{Exercice 3} \quad 7 points\hfill Thème : Géométrie dans l'espace}

\medskip

L'espace est muni d'un repère orthonormé \Oijk.

On considère les points 

\[\text{A}(3~;~-2~;~2),\quad \text{B}(6~;~1~;~5),\quad \text{C}(6~;~-2~;~-1)\quad \text{et D}(0~;~4~;~-1).\]

\emph{On rappelle que le volume d'un tétraèdre est donné par la formule :}

\[V = \dfrac13 \mathcal{A} \times h\]\index{volume tétraèdre}

\emph{où $\mathcal{A}$ est l'aire de la base et $h$ la hauteur correspondante.}

\medskip

\begin{enumerate}
\item Démontrer que les points A, B, C et D ne sont pas coplanaires.
\item 
	\begin{enumerate}
		\item Montrer que le triangle ABC est rectangle.\index{triangle rectangle}
		\item Montrer que la droite (AD) est perpendiculaire au plan (ABC).\index{vecteur normal}
		\item En déduire le volume du tétraèdre ABCD.
	\end{enumerate}	
\item On considère le point H(5~;~0~;~1).
	\begin{enumerate}
		\item Montrer qu'il existe des réels $\alpha$ et $\beta$ tels que $\vect{\text{BH}} = \alpha \vect{\text{BC}} + \beta \vect{\text{BD}}$.
		\item Démontrer que H est le projeté orthogonal du point A sur le plan (BCD). 
		\item En déduire la distance du point A au plan (BCD).
	\end{enumerate}	
\item Déduire des questions précédentes l'aire du triangle BCD.
\end{enumerate}

\bigskip

\textbf{\textsc{Exercice 4} \quad 7 points\hfill Thème : Probabilités}

\medskip

Une urne contient des jetons blancs et noirs tous indiscernables au toucher.

Une partie consiste à prélever au hasard successivement et avec remise deux jetons de cette urne.

On établit la règle de jeu suivante:

\setlength\parindent{1cm}
\begin{itemize}
\item[$\bullet~~$] un joueur perd 9 euros si les deux jetons tirés sont de couleur blanche ;
\item[$\bullet~~$] un joueur perd 1 euro si les deux jetons tirés sont de couleur noire ;
\item[$\bullet~~$] un joueur gagne 5 euros si les deux jetons tirés sont de couleurs différentes.
\end{itemize}
\setlength\parindent{0cm}

\medskip

\begin{enumerate}
\item On considère que l'urne contient 2 jetons noirs et 3 jetons blancs.
	\begin{enumerate}
		\item Modéliser la situation à l'aide d'un arbre pondéré.\index{arbre pondéré}
		\item Calculer la probabilité de perdre $9$ \euro{} sur une partie.\index{probabilités}
	\end{enumerate}	
\item On considère maintenant que l'urne contient 3 jetons blancs et au moins deux jetons noirs mais on ne connait pas le nombre exact de jetons noirs. On appellera $N$ le nombre de jetons noirs.
	\begin{enumerate}
		\item Soit $X$ la variable aléatoire donnant le gain du jeu pour une partie.
		
Déterminer la loi de probabilité de cette variable aléatoire.\index{loi de probabilité}
		\item Résoudre l'inéquation pour $x$ réel:
		
\[-x^2  + 30x - 81 > 0\]\index{inéquation}

		\item En utilisant le résultat de la question précédente, déterminer le nombre de jetons noirs que l'urne doit contenir afin que ce jeu soit favorable au joueur.
		\item Combien de jetons noirs le joueur doit-il demander afin d'obtenir un gain moyen maximal ?\index{espérance}
	\end{enumerate}
\item On observe $10$ joueurs qui tentent leur chance en effectuant une partie de ce jeu, indépendamment les uns des autres. On suppose que 7 jetons noirs ont été placés dans l'urne (avec 3 jetons blancs). 

Quelle est la probabilité d'avoir au moins $1$ joueur gagnant $5$ euros?
\end{enumerate}
\newpage
%%%%%%%%%%% Asie1 17 mai 2022
\phantomsection
\hypertarget{Asie1}{}

\label{Asie1}

\lfoot{\small{Asie}}
\rfoot{\small{17 mai 2022}}
\pagestyle{fancy}
\thispagestyle{empty}

\begin{center}{\Large\textbf{\decofourleft~Baccalauréat Asie 17 mai 2022 Jour 1~\decofourright\\[6pt] ÉPREUVE D'ENSEIGNEMENT DE SPÉCIALITÉ}}

\vspace{0,25cm}

Le sujet propose 4 exercices.

Le candidat choisit 3 exercices parmi les 4 exercices et \textbf{ne doit traiter que ces 3 exercices.}

\medskip

Chaque exercice est noté sur 7 points (le total sera ramené sur 20 points). 

\medskip

Les traces de recherche, même incomplètes ou infructueuses, seront prises en compte.

\hrule
\end{center}

\bigskip

\textbf{\textsc{Exercice 1} \hfill 7 points}

\medskip

\emph{Principaux domaines abordés}: Probabilités conditionnelles et indépendance. Variables aléatoires.

\medskip

Lors d'une kermesse, un organisateur de jeux dispose, d'une part, d'une roue comportant quatre cases blanches et huit cases rouges et, d'autre part, d'un sac contenant cinq jetons portant les numéros 1, 2, 3, 4 et 5.

Le jeu consiste à faire tourner la roue, chaque case ayant la même probabilité d'être obtenue, puis à extraire un ou deux jetons du sac selon la règle suivante :

\setlength\parindent{1cm}
\begin{itemize}
\item[$\bullet~~$] si la case obtenue par la roue est blanche, alors le joueur extrait un jeton du sac ;
\item[$\bullet~~$] si la case obtenue par la roue est rouge, alors le joueur extrait successivement et sans remise deux jetons du sac.
\end{itemize}
\setlength\parindent{0cm}

Le joueur gagne si le ou les jetons tirés portent tous un numéro impair.

\medskip

\begin{enumerate}
\item Un joueur fait une partie et on note $B$ l'évènement \og la case obtenue est blanche \fg,
$R$ l'évènement \og la case obtenue est rouge\fg{} et $G$ l'évènement \og le joueur gagne la partie \fg. 
	\begin{enumerate}
		\item Donner la valeur de la probabilité conditionnelle $P_B(G)$.\index{probabilités}
		\item On admettra que la probabilité de tirer successivement et sans remise deux jetons impairs est égale à $0,3$.
		
Recopier et compléter l'arbre de probabilité suivant:\index{arbre pondéré}

\begin{center}
\pstree[treemode=R,nodesepB=3pt,levelsep=3cm]{\TR{}}
{\pstree{\TR{$B$~~} }
	{\TR{$G$}
	\TR{$\overline{G}$}
	}
\pstree{\TR{$R$~~} }
	{\TR{$G$}
	\TR{$\overline{G}$}
	}	
}
\end{center}

	\end{enumerate}

\item
	\begin{enumerate}
		\item Montrer que $P(G) = 0,4$.
		\item Un joueur gagne la partie.
		
Quelle est la probabilité qu'il ait obtenu une case blanche en lançant la roue ?
	\end{enumerate}
\item Les évènements $B$ et $G$ sont-ils indépendants ? Justifier.\index{evènements indépendants@évènements indépendants}
\item Un même joueur fait dix parties. Les jetons tirés sont remis dans le sac après chaque partie.

On note $X$ la variable aléatoire égale au nombre de parties gagnées.
	\begin{enumerate}
		\item Expliquer pourquoi $X$ suit une loi binomiale et préciser ses paramètres.\index{loi binomiale}
		\item Calculer la probabilité, arrondie à $10^{-3}$ près, que le joueur gagne exactement trois parties sur les dix parties jouées.
		\item Calculer $P (X \geqslant 4)$ arrondie à $10^{-3}$ près.
		
Donner une interprétation du résultat obtenu.
	\end{enumerate}
\item Un joueur fait $n$ parties et on note $p_n$ la probabilité de l'évènement \og le joueur gagne au moins une partie \fg.
	\begin{enumerate}
		\item Montrer que $p_n = 1 - 0,6^n$.
		\item Déterminer la plus petite valeur de l'entier $n$ pour laquelle la probabilité de gagner au moins une partie est supérieure ou égale à $0,99$.\index{inéquation}
	\end{enumerate}
\end{enumerate}

\bigskip

\textbf{\textsc{Exercice 2} \hfill 7 points}

\medskip

\emph{Principaux domaines abordés} : Suites numériques. Algorithmique et programmation.

\medskip

Un médicament est administré à un patient par voie intraveineuse. 

\bigskip

\textbf{Partie A : modèle discret de la quantité médicamenteuse}

\medskip

Après une première injection de 1 mg de médicament, le patient est placé sous perfusion.

On estime que, toutes les $30$~minutes, l'organisme du patient élimine 10\,\% de la quantité de médicament présente dans le sang et qu'il reçoit une dose supplémentaire de $0,25$ mg de la substance médicamenteuse.

On étudie l'évolution de la quantité de médicament dans le sang avec le modèle suivant :

pour tout entier naturel $n$, on note $u_n$ la quantité, en mg, de médicament dans le sang du patient au bout de $n$ périodes de trente minutes. On a donc $u_0 = 1$.

\medskip

\begin{enumerate}
\item Calculer la quantité de médicament dans le sang au bout d'une demi-heure.
\item Justifier que, pour tout entier naturel $n,\: u_{n+1} = 0,9u_n + 0,25$.\index{suites}
\item 
	\begin{enumerate}
		\item Montrer par récurrence sur $n$ que, pour tout entier naturel $n,\: u_n \leqslant  u_{n+1} < 5$.\index{récurrence}
		\item En déduire que la suite $\left(u_n\right)$ est convergente.\index{convergence de suite}
	\end{enumerate}
\item On estime que le médicament est réellement efficace lorsque sa quantité dans le sang
du patient est supérieure ou égale à $1,8$ mg.
	\begin{enumerate}
		\item Recopier et compléter le script écrit en langage Python suivant de manière à déterminer au bout de combien de périodes de trente minutes le médicament commence à être réellement efficace.\index{script python}
\begin{center}
\begin{tabular}{|l|}\hline
\texttt{\textbf{def} efficace():}\\
\quad \texttt{u=1}\\
\quad \texttt{n=0}\\
\quad \texttt{\textbf{while} \ldots\ldots :}\\
\quad\qquad \texttt{u=\ldots\ldots}\\
\quad\qquad \texttt{n = n+1}\\
\quad \texttt{\textbf{ return }n}\\ \hline
\end{tabular}
\end{center}

		\item Quelle est la valeur renvoyée par ce script ? Interpréter ce résultat dans le contexte de l'exercice.
	\end{enumerate}		
\item Soit $\left(v_n\right)$ la suite définie, pour tout entier naturel $n$, par $v_n = 2,5 - u_n$.
	\begin{enumerate}
		\item Montrer que $\left(v_n\right)$ est une suite géométrique dont on précisera la raison et le premier terme $\left(v_0\right)$.\index{suite géométrique}
		\item Montrer que, pour tout entier naturel $n,\: u_n = 2,5 - 1,5 \times 0,9^n$.
		\item Le médicament devient toxique lorsque sa quantité présente dans le sang du patient dépasse $3$ mg. 
		
D'après le modèle choisi, le traitement présente-t-il un risque pour le patient? Justifier.\index{inéquation}
	\end{enumerate}
\end{enumerate}

\bigskip

\textbf{Partie B : modèle continu de la quantité médicamenteuse}

\medskip

Après une injection initiale de $1$~mg de médicament, le patient est placé sous perfusion.

Le débit de la substance médicamenteuse administrée au patient est de $0,5$ mg par heure.

La quantité de médicament dans le sang du patient, en fonction du temps, est modélisée
par la fonction $f$, définie sur $[0~;~ +\infty[$, par 

\[f(t) = 2,5 - 1,5\text{e}^{-0,2t},\]\index{fonction exponentielle}

où $t$ désigne la durée de la perfusion exprimée en heure.

On rappelle que ce médicament est réellement efficace lorsque sa quantité dans le sang du patient est supérieure ou égale à $1,8$~mg.

\medskip

\begin{enumerate}
\item Le médicament est-il réellement efficace au bout de 3~h 45~min ?\index{inéquation}
\item Selon ce modèle, déterminer au bout de combien de temps le médicament devient réellement efficace.
\item Comparer le résultat obtenu avec celui obtenu à la question 4. b. du modèle discret de la Partie A.
\end{enumerate}

\bigskip

\textbf{\textsc{Exercice 3} \hfill 7 points}

\medskip

\emph{Principaux domaines abordés}: Manipulation des vecteurs, des droites et des plans de l'espace. Orthogonalité et distances dans l'espace. Représentations paramétriques et équations cartésiennes.

\bigskip

Le solide ABCDEFGH est un cube. On se place dans le repère orthonormé $\left(\text{A}~;~\vect{\imath},~\vect{\jmath},~\vect{k}\right)$ de l'espace dans lequel les coordonnées des points B, D et E sont : 

\[\text{B}(3~;~0 ~;~0), \text{D} (0~;~3~;~0)\: \text{et E}(0~;~0~;~3).\]

\begin{center}
\psset{unit=1cm}
\begin{pspicture}(6.5,5.8)
\pspolygon(0.5,1)(2.9,0.5)(2.9,3.7)(0.5,4.2)%ABFE
\psline(2.9,0.5)(5.6,1)(5.6,4.2)(2.9,3.7)%BCGF
\psline(5.6,4.2)(3.2,4.7)(0.5,4.2)%GHE
\psline[linestyle=dashed](0.5,1)(3.2,1.5)(5.6,1)%ADC
\psline[linestyle=dashed](3.2,1.5)(3.2,4.7)%DH
\psline[linewidth=1.3pt]{->}(0.5,1)(1.3,0.84)
\psline[linewidth=1.3pt]{->}(0.5,1)(1.4,1.16)
\psline[linewidth=1.3pt]{->}(0.5,1)(0.5,2.06)
\uput[dl](0.5,1){A} \uput[d](2.9,0.5){B} \uput[r](5.6,1){C} \uput[d](3.2,1.5){D}
\uput[l](0.5,4.2){E} \uput[u](2.9,3.7){F} \uput[r](5.6,4.2){G} \uput[u](3.2,4.7){H}
\uput[d](0.7,0.9){$\vect{\imath}$} \uput[u](0.95,1.08){$\vect{\jmath}$} \uput[l](0.5,1.53){$\vect{k}$}
\end{pspicture}
\end{center}

On considère les points P(0~;~0~;~1), Q(0~;~2~;~3) et R(1~;~0~;~3).

\medskip

\begin{enumerate}
\item Placer les points P{}, Q et R sur la figure en ANNEXE qui sera à rendre avec la copie.
\item Montrer que le triangle PQR est isocèle en R.
\item Justifier que les points P{}, Q et R définissent un plan.
\item On s'intéresse à présent à la distance entre le point E et le plan (PQR).
	\begin{enumerate}
		\item Montrer que le vecteur $\vect{u}(2~;~1~;~- 1)$ est normal au plan (PQR).\index{vecteur normal}
		\item En déduire une équation cartésienne du plan (PQR).\index{equation de plan@équation de plan}
		\item Déterminer une représentation paramétrique de la droite $(d)$ passant par le point E et orthogonale au plan (PQR).\index{equation paramétrique de droite@équation paramétrique de droite}
		\item Montrer que le point L$\left(\dfrac23~;~\dfrac13~;~\dfrac83\right)$ est le projeté orthogonal du point E sur le plan (PQR).
		\item Déterminer la distance entre le point E et le plan (PQR).
	\end{enumerate}
\item En choisissant le triangle EQR comme base, montrer que le volume du tétraèdre EPQR est $\dfrac23$.\index{volume tétraèdre}

On rappelle que le volume $V$ d'un tétraèdre est donné par la formule : 

\[V = \dfrac13 \times \text{aire d'une base} \times \text{hauteur correspondante}.\]

\item Trouver, à l'aide des deux questions précédentes, l'aire du triangle PQR.
\end{enumerate}

\bigskip

\textbf{\textsc{Exercice 4} \hfill 7 points}

\medskip

\emph{Principaux domaines abordés}: Étude de fonctions. Fonction logarithme.

\bigskip

Soit $f$ une fonction définie et dérivable sur $\R$. On considère les points A(1~;~3) et B(3~;~5).

On donne ci-dessous $\mathcal{C}_f$ la courbe représentative de $f$ dans un repère orthogonal du plan, ainsi que la tangente (AB) à la courbe $\mathcal{C}_f$ au point A.

\medskip

\begin{center}
\psset{unit=0.8cm,arrowsize=2pt 3}
\begin{pspicture*}(-7,-2.25)(8,6.25)
\psgrid[gridlabels=0pt,subgriddiv=1,gridwidth=0.15pt]
\psaxes[linewidth=1.25pt,labelFontSize=\scriptstyle]{->}(0,0)(-7,-2.25)(8,6.25)
\psplot[plotpoints=2000,linewidth=1.25pt,linecolor=red]{-7}{8}{x dup mul 1 add ln 3 add 2 ln sub}
\psplot[linestyle=dashed]{-7}{8}{x 2 add}
\uput[dr](1,3){A}\uput[ul](3,5){B}
\psdots[dotstyle=+,dotangle=45,dotscale=1.5](1,3)(3,5)
\uput[d](-5.5,5.75){\red $\mathcal{C}_f$}
\end{pspicture*}
\end{center}

\emph{Les trois parties de l'exercice peuvent être traitées de manière indépendante.}

\bigskip

\textbf{Partie A}

\medskip

\begin{enumerate}
\item Déterminer graphiquement les valeurs de $f(1)$ et $f'(1)$.\index{lecture graphique}
\item La fonction $f$ est définie par l'expression $f(x) = \ln \left(ax^2 + 1\right) + b$, où $a$ et $b$ sont des nombres réels positifs.\index{fonction logarithme}
	\begin{enumerate}
		\item Déterminer l'expression de $f'(x)$.\index{dérivée}
		\item Déterminer les valeurs de $a$ et $b$ à l'aide des résultats précédents.
	\end{enumerate}
\end{enumerate} 

\bigskip

\textbf{Partie B}

\medskip

On admet que la fonction $f$ est définie sur $\R$ par 

\[f(x) = \ln \left(x^2 + 1\right) + 3 - \ln (2).\]

\medskip

\begin{enumerate}
\item Montrer que $f$ est une fonction paire.\index{fonction paire}
\item Déterminer les limites de $f$ en $+\infty$ et en $-\infty$.\index{limite de fonction}
\item Déterminer l'expression de $f'(x)$.\index{dérivée}

Étudier le sens de variation de la fonction $f$ sur $\R$.

Dresser le tableau des variations de $f$ en y faisant figurer la valeur exacte du minimum ainsi que les limites de $f$ en $-\infty$ et $+\infty$.\index{tableau de variations}
\item À l'aide du tableau des variations de $f$, donner les valeurs du réel $k$ pour lesquelles l'équation $f(x) = k$ admet deux solutions.\index{valeurs intermédiaires}
\item Résoudre l'équation $f(x) = 3 + \ln 2$.\index{equation@équation}
\end{enumerate}

\bigskip

\textbf{Partie C}

\medskip

On rappelle que la fonction $f$ est définie sur $R$ par $f(x) = \ln \left(x^2 + 1\right) + 3 - \ln (2)$.

\medskip

\begin{enumerate}
\item Conjecturer, par lecture graphique, les abscisses des éventuels points d'inflexion
de la courbe $\mathcal{C}_f$.\index{lecture graphique}
\item Montrer que, pour tout nombre réel $x$, on a : $f''(x) = \dfrac{2\left(1 - x^2\right)}{\left(x^2 + 1\right)^2}$.\index{dérivée seconde}
\item En déduire le plus grand intervalle sur lequel la fonction $f$ est convexe.\index{convexité}
\end{enumerate}

\newpage

\begin{center}

\textbf{\Large ANNEXE à rendre avec la copie}

\vspace{5cm}

\psset{unit=1.5cm}
\begin{pspicture}(6.5,5.8)
\pspolygon(0.5,1)(2.9,0.5)(2.9,3.7)(0.5,4.2)%ABFE
\psline(2.9,0.5)(5.6,1)(5.6,4.2)(2.9,3.7)%BCGF
\psline(5.6,4.2)(3.2,4.7)(0.5,4.2)%GHE
\psline[linestyle=dashed](0.5,1)(3.2,1.5)(5.6,1)%ADC
\psline[linestyle=dashed](3.2,1.5)(3.2,4.7)%DH
\psline[linewidth=1.3pt]{->}(0.5,1)(1.3,0.84)
\psline[linewidth=1.3pt]{->}(0.5,1)(1.4,1.16)
\psline[linewidth=1.3pt]{->}(0.5,1)(0.5,2.06)
\uput[dl](0.5,1){A} \uput[d](2.9,0.5){B} \uput[r](5.6,1){C} \uput[d](3.2,1.5){D}
\uput[l](0.5,4.2){E} \uput[u](2.9,3.7){F} \uput[r](5.6,4.2){G} \uput[u](3.2,4.7){H}
\uput[d](0.7,0.9){$\vect{\imath}$} \uput[u](0.95,1.08){$\vect{\jmath}$} \uput[l](0.5,1.53){$\vect{k}$}
\end{pspicture}
\end{center}
\newpage
%%%%%%%%%%% Asie2 18 mai 2022
\phantomsection
\hypertarget{Asie2}{}

\label{Asie2}

\lfoot{\small{Asie}}
\rfoot{\small{18 mai 2022}}
\pagestyle{fancy}
\thispagestyle{empty}

\begin{center}{\Large\textbf{\decofourleft~Baccalauréat Asie 18 mai 2022 Jour 2~\decofourright\\[6pt] ÉPREUVE D'ENSEIGNEMENT DE SPÉCIALITÉ}}

\vspace{0,25cm}

Le sujet propose 4 exercices.

Le candidat choisit 3 exercices parmi les 4 exercices et \textbf{ne doit traiter que ces 3 exercices.}

\medskip

Chaque exercice est noté sur 7 points (le total sera ramené sur 20 points). 

\medskip

Les traces de recherche, même incomplètes ou infructueuses, seront prises en compte.

\hrule
\end{center}

\bigskip

\textbf{\textsc{Exercice 1} \hfill 7 points}

\medskip

\emph{Principaux domaines abordés}: Manipulation des vecteurs, des droites et des plans de l'espace. Orthogonalité et distances dans l'espace.
Représentations paramétriques et équations cartésiennes.

\bigskip

Dans un repère orthonormé \Oijk{} de l'espace,
on considère les points 

\[\text{A}(-3~;~1~;~3), \text{B}(2~;~2~;~3),\, \text{C}(1~;~7~;~-1), \,\text{D}(-4~;~6~;~-1)\: \text{et K}(-3~;~14~;~14).\]

\medskip

\begin{enumerate}
\item 
	\begin{enumerate}
		\item Calculer les coordonnées des vecteurs $\vect{\text{AB}},\: \vect{\text{DC}}$ et 
		$\vect{\text{AD}}$.
		\item Montrer que le quadrilatère ABCD est un rectangle.
		\item Calculer l'aire du rectangle ABCD.
	\end{enumerate}
\item 
	\begin{enumerate}
		\item Justifier que les points A, B et D définissent un plan.
		\item Montrer que le vecteur $\vect{n}(-2~;~10~;~13)$ est un vecteur normal au plan (ABD).\index{vecteur normal}
		\item En déduire une équation cartésienne du plan (ABD).\index{equation de plan@équation de plan}
	\end{enumerate}
\item
	\begin{enumerate}
		\item Donner une représentation paramétrique de la droite $\Delta$ orthogonale au plan (ABD) et qui passe par le point K.\index{equation paramétrique de droite@équation paramétrique de droite}
		\item Déterminer les coordonnées du point I, projeté orthogonal du point K sur le plan (ABD).
		\item Montrer que la hauteur de la pyramide KABCD de base ABCD et de sommet K
vaut $\sqrt{273}$.
	\end{enumerate}
\item Calculer le volume $V$ de la pyramide KABCD.\index{volume pyramide}

On rappelle que le volume $V$ d'une pyramide est donné par la formule:

\[V= \dfrac13 \times \text{aire de la base} \times \text{hauteur}.\]
\end{enumerate}

\newpage

\textbf{\textsc{Exercice 2} \hfill 7 points}

\medskip

\emph{Principaux domaines abordés}: Étude des fonctions.
Fonction logarithme.

\bigskip

\textbf{Partie A}

\begin{center}
\psset{unit=0.6cm,arrowsize=2pt 3}
\begin{pspicture*}(-4,-5)(15.5,8)
\psgrid[gridlabels=0pt,subgriddiv=1,gridwidth=0.15pt]
\psaxes[linewidth=1.25pt,labelFontSize=\scriptstyle]{->}(0,0)(-4,-5)(15.5,8)
\psplot[plotpoints=2000,linewidth=1.25pt,linecolor=blue]{3.001}{15.5}{x dup mul x sub 6 sub ln}
\psplot[plotpoints=2000,linewidth=1.25pt,linecolor=red,linestyle=dashed]{3.001}{15.5}{x 2 mul 1 sub x dup mul x sub 6 sub div}
\uput[r](3.1,7.7){\red $\mathcal{C}_2$}\uput[r](3.1,-4){\blue $\mathcal{C}_1$}
\end{pspicture*}
\end{center}

Dans le repère orthonormé ci-dessus, sont tracées les courbes représentatives d'une fonction $f$ et de sa fonction dérivée, notée $f'$, toutes deux définies sur $]3~;~+\infty[$.

\medskip

\begin{enumerate}
\item Associer à chaque courbe la fonction qu'elle représente. Justifier.\index{lecture graphique}
\item Déterminer graphiquement la ou les solutions éventuelles de l'équation $f(x) = 3$.\index{maximum}
\item Indiquer, par lecture graphique, la convexité de la fonction $f$.\index{convexité}
\end{enumerate}

\bigskip

\textbf{Partie B}

\medskip

\begin{enumerate}
\item Justifier que la quantité $\ln \left(x^2- x- 6\right)$ est bien définie pour les valeurs $x$ de l'intervalle $]3~;~ +\infty[$, que l'on nommera $I$ dans la suite.\index{fonction logarithme}
\item On admet que la fonction $f$ de la partie A est définie par $f(x) = \ln \left(x^2- x- 6\right)$ sur $I$. 

Calculer les limites de la fonction $f$ aux deux bornes de l'intervalle $I$.\index{limite de fonction}

En déduire une équation d'une asymptote à la courbe représentative de la fonction $f$ sur $I$.\index{asymptote}
\item 
	\begin{enumerate}
		\item Calculer $f'(x)$ pour tout $x$ appartenant à $I$.\index{dérivée}
		\item Étudier le sens de variation de la fonction $f$ sur $I$.
		
Dresser le tableau des variations de la fonction $f$ en y faisant figurer les limites aux bornes de $I$.\index{tableau de variations}
	\end{enumerate}
\item 
	\begin{enumerate}
		\item Justifier que l'équation $f(x) = 3$ admet une unique solution $\alpha$ sur l'intervalle ]5~;~ 6[.\index{equation@équation}
		\item Déterminer, à l'aide de la calculatrice, un encadrement de $\alpha$ à $10^{-2}$ près.
	\end{enumerate}
\item 
	\begin{enumerate}
		\item Justifier que $f''(x) = \dfrac{- 2x^2 + 2x - 13}{\left(x^2 - x - 6\right)^2}$.\index{dérivée seconde}
		\item Étudier la convexité de la fonction $f$ sur $I$.\index{convexité}
	\end{enumerate}
\end{enumerate}

\bigskip

\textbf{\textsc{Exercice 3} \hfill 7 points}

\medskip

\emph{Principaux domaines abordés}: Probabilités conditionnelles et indépendance. Variables aléatoires.

\medskip

\begin{center}\emph{Les deux parties de cet exercice sont indépendantes.}\end{center}

\textbf{Partie 1}

\medskip

Julien doit prendre l'avion; il a prévu de prendre le bus pour se rendre à l'aéroport. 

S'il prend le bus de $8$~h, il est sûr d'être à l'aéroport à temps pour son vol.

Par contre, le bus suivant ne lui permettrait pas d'arriver à temps à l'aéroport.

Julien est parti en retard de son appartement et la probabilité qu'il manque son bus est de $0,8$.

S'il manque son bus, il se rend à l'aéroport en prenant une compagnie de voitures privées; il a alors une probabilité de $0,5$ d'être à l'heure à l'aéroport.

\smallskip

On notera :

\setlength\parindent{12mm}
\begin{itemize}
\item[$\bullet~~$] $B$ l'évènement: \og Julien réussit à prendre son bus \fg ;
\item[$\bullet~~$] $V$ l'évènement: \og Julien est à l'heure à l'aéroport pour son vol \fg.
\end{itemize}
\setlength\parindent{0mm}

\medskip

\begin{enumerate}
\item Donner la valeur de $P_B(V)$.
\item Représenter la situation par un arbre pondéré.\index{arbre pondéré}
\item Montrer que $P(V) = 0,6$.
\item Si Julien est à l'heure à l'aéroport pour son vol, quelle est la probabilité qu'il soit arrivé à l'aéroport en bus ? Justifier.
\end{enumerate}

\bigskip

\textbf{Partie 2}

\medskip

Les compagnies aériennes vendent plus de billets qu'il n'y a de places dans les avions car certains passagers ne se présentent pas à l'embarquement du vol sur lequel ils ont réservé.
On appelle cette pratique le surbooking.

Au vu des statistiques des vols précédents, la compagnie aérienne estime que chaque passager a 5\,\% de chance de ne pas se présenter à l'embarquement.

Considérons un vol dans un avion de $200$~places pour lequel $206$~billets ont été vendus.
On suppose que la présence à l'embarquement de chaque passager est indépendante des autres passagers et on appelle $X$ la variable aléatoire qui compte le nombre de passagers se présentant à l'embarquement.

\medskip

\begin{enumerate}
\item Justifier que $X$ suit une loi binomiale dont on précisera les paramètres.\index{loi binomiale}
\item En moyenne, combien de passagers vont-ils se présenter à l'embarquement ?\index{espérance}
\item Calculer la probabilité que $201$ passagers se présentent à l'embarquement. Le résultat sera arrondi à $10^{-3}$ près.
\item Calculer $P(X \leqslant 200)$, le résultat sera arrondi à $10^{-3}$ près. Interpréter ce résultat dans le contexte de l'exercice.
\item La compagnie aérienne vend chaque billet à $250$ euros.

Si plus de $200$ passagers se présentent à l'embarquement, la compagnie doit rembourser le billet d'avion et payer une pénalité de $600$ euros à chaque passager lésé.

On appelle :

\setlength\parindent{12mm}
\begin{description}
\item[ ] $Y$ la variable aléatoire égale au nombre de passagers qui ne peuvent pas embarquer
bien qu'ayant acheté un billet;
\item[ ] $C$ la variable aléatoire qui totalise le chiffre d'affaire de la compagnie aérienne sur ce vol.
\end{description}

On admet que $Y$ suit la loi de probabilité donnée par le tableau suivant:

\begin{center}
\begin{tabularx}{\linewidth}{|c|*{7}{>{\centering \arraybackslash}X|}}\hline
$y_i$					&0	&1	&2	&3	&4	&5	&6\\ \hline
$P\left(Y = y_i\right)$	&\np{0,94775}&\np{0,03063}&\np{0,01441}&\np{0,00539}&\np{0,00151}&\np{0,00028}& \\ \hline
\end{tabularx}
\end{center}

	\begin{enumerate}
		\item Compléter la loi de probabilité donnée ci-dessus en calculant $P(Y = 6)$.\index{loi de probabilité}
		\item Justifier que: $C = \np{51500} - 850Y$.
		\item Donner la loi de probabilité de la variable aléatoire $C$ sous forme d'un tableau.
		
Calculer l'espérance de la variable aléatoire $C$ à l'euro près.\index{espérance}
		\item Comparer le chiffre d'affaires obtenu en vendant exactement $200$ billets et le chiffre d'affaires moyen obtenu en pratiquant le surbooking.
	\end{enumerate}
\end{enumerate}

\bigskip

\textbf{\textsc{Exercice 4} \hfill 7 points}

\medskip

\emph{Principaux domaines abordés} : Suites numériques.
Algorithmique et programmation.

\bigskip

On s'intéresse au développement d'une bactérie.

Dans cet exercice, on modélise son développement avec les hypothèses suivantes : cette bactérie a une probabilité $0,3$ de mourir sans descendance et une probabilité $0,7$ de se diviser en deux bactéries filles.

Dans le cadre de cette expérience, on admet que les lois de reproduction des bactéries sont les mêmes pour toutes les générations de bactéries qu'elles soient mère ou fille.

Pour tout entier naturel $n$, on appelle $p_n$ la probabilité d'obtenir au plus $n$ descendances pour une bactérie.

On admet que, d'après ce modèle, la suite $\left(p_n\right)$ est définie de la façon suivante :

$p_0 = 0,3$ et, pour tout entier naturel $n$,

\[p_{n+1} = 0,3 + 0,7p_n^2.\]\index{suites}

\begin{minipage}{0.68\linewidth}
\begin{enumerate}
\item La feuille de calcul ci-dessous donne des valeurs approchées de la suite $\left(p_n\right)$
	\begin{enumerate}
		\item Déterminer les valeurs exactes de $p_1$ et $p_2$ (masquées dans la feuille de calcul) et interpréter ces valeurs dans le contexte de l'énoncé.
		\item Quelle est la probabilité, arrondie à $10^{-3}$ près, d'obtenir au moins 11 générations de bactéries à partir d'une bactérie de ce type ?
		\item Formuler des conjectures sur les variations et la convergence de la suite $\left(p_n\right)$.\index{convergence de suite}
	\end{enumerate}
\item 
	\begin{enumerate}
		\item Démontrer par récurrence sur $n$ que,
pour tout entier naturel $n,\: 0 \leqslant p_n \leqslant p_{n+1} \leqslant0,5$.\index{récurrence}
		\item Justifier que la suite $\left(p_n\right)$ est convergente.
	\end{enumerate}
\item On appelle $L$ la limite de la suite $\left(p_n\right)$.\index{limite de suite}
	\begin{enumerate}
		\item Justifier que $L$ est solution de l'équation 
		
		\[0,7x^2 - x  + 0,3 = 0\]
		\item Déterminer alors la limite de la suite $\left(p_n\right)$.
	\end{enumerate}
\end{enumerate}
\end{minipage}\hfill
\begin{minipage}{0.31\linewidth}
\begin{tabularx}{\linewidth}{|c|c|>{\centering \arraybackslash}X|}\hline
&A &B\\ \hline
1&$n$&$p_n$\\ \hline
2& 0 &0,3\\ \hline
3&1&\\ \hline
4&2&\\ \hline
5&3& \np{0,40769562}\\ \hline 
6&4& \np{0,416351} \\ \hline 
7&5 &\np{0,42134371} \\ \hline 
8&6 &\np{0,42427137} \\ \hline 
9&7& \np{0,42600433}\\ \hline 
10&8& \np{0,42703578} \\ \hline 
11&9& \np{0,42765169} \\ \hline 
12& 10& \np{0,42802018} \\ \hline 
13& 11& \np{0,42824089} \\ \hline 
14& 12& \np{0,42837318} \\ \hline 
15& 13& \np{0,42845251} \\ \hline 
16& 14& \np{0,42850009} \\ \hline 
17& 15& \np{0,42852863} \\ \hline 
18& 16& \np{0,42854575} \\ \hline 
19& 17& \np{0,42855602}\\ \hline 
\end{tabularx}
\end{minipage}

\begin{enumerate}[resume]
\item La fonction suivante, écrite en langage Python, a pour objectif de renvoyer les $n$ premiers termes de la suite $\left(p_n\right)$.\index{script python}

\begin{center}
\begin{tabularx}{0.4\linewidth}{c |l|}\cline{2-2}
1 &\texttt{\textbf{def} suite(n) :}\\
2 &\quad \texttt{p= \ldots}\\
3 &\quad \texttt{s=[p]}\\
4 &\quad \texttt{for i in range (\ldots) :}\\
5 &\quad \qquad \texttt{p=\ldots}\\
6 &\quad \qquad \texttt{s.append(p)}\\
7 &\quad \texttt{\textbf{return} (s)}\\ \cline{2-2}
\end{tabularx}
\end{center}

Recopier, sur votre copie, cette fonction en complétant les lignes 2, 4 et 5 de façon à ce que la fonction \texttt{suite (n)} retourne, sous forme de liste, les $n$ premiers termes de la suite.
\end{enumerate}
\newpage
%%%%%%%%%%% GroupeI1 18 mai 2022
\phantomsection
\hypertarget{GroupeI1}{}

\label{GroupeI1}

\lfoot{\small{Centres étrangers}}
\rfoot{\small{18 mai 2022}}
\pagestyle{fancy}
\thispagestyle{empty}

\begin{center}{\Large\textbf{\decofourleft~Baccalauréat Centres étrangers Groupe 1  (D)\footnote{Afrique du Sud, Bulgarie, Comores, Djibouti, Kenya, Liban, Lituanie, Madagascar, Mozambique et Ukraine} 18 mai 2022~\decofourright\\[7pt]  Sujet 1\\[7pt] ÉPREUVE D'ENSEIGNEMENT DE SPÉCIALITÉ}}
\end{center}

\vspace{0,25cm}

Le sujet propose 4 exercices

Le candidat choisit 3 exercices parmi les 4 exercices et \textbf{ne doit traiter que ces 3 exercices}

Chaque exercice est noté sur 6 points (le total sera ramené sur 20 points)\footnote{Chaque exercice est noté sur 7 points au Liban.}

La clarté et la précision de l'argumentation ainsi que la qualité de la rédaction sont notées sur 2 points.

Les traces de recherche, même incomplètes ou infructueuses, seront prises en compte.

\bigskip

\textbf{\textsc{Exercice 1} \quad 6 points\hfill }

\medskip

\begin{tabularx}{\linewidth}{|X|}\hline
\textbf{Principaux domaines abordés :} Probabilités\\ \hline
\end{tabularx}

\medskip

Dans une station de ski, il existe deux types de forfait selon l'âge du skieur :

\begin{itemize}
\item un forfait JUNIOR pour les personnes de moins de 25 ans ;
\item un forfait SENIOR pour les autres.
\end{itemize}

 Par ailleurs, un usager peut choisir, en plus du forfait correspondant à son âge l'\emph{option  coupe-file} qui permet d'écourter le temps d'attente aux remontées mécaniques.

On admet que:
\begin{itemize}
\item[$\bullet~~$] 20\,\% des skieurs ont un forfait JUNIOR ;
\item[$\bullet~~$] 80\,\% des skieurs ont un forfait SENIOR ;
\item[$\bullet~~$] parmi les skieurs ayant un forfait JUNIOR, 6\,\% choisissent l'option coupe-file ;
\item[$\bullet~~$] parmi les skieurs ayant un forfait SENIOR, 12,5\,\% choisissent l'option coupe-file.
\end{itemize}

On interroge un skieur au hasard et on considère les évènements :

\begin{itemize}
\item[$\bullet~~$] $J$ : \og le skieur a un forfait JUNIOR \fg ;
\item[$\bullet~~$] $C$ : \og le skieur choisit l'option coupe-file \fg.
\end{itemize}

\medskip

\emph{Les deux parties peuvent être traitées de manière indépendante}

\bigskip

\textbf{Partie A}

\medskip

\begin{enumerate}
\item Traduire la situation par un arbre pondéré.\index{arbre pondéré}
\item Calculer la probabilité $P(J \cap C)$.
\item Démontrer que la probabilité que le skieur choisisse l'option coupe-file est égale à $0,112$.
\item Le skieur a choisi l'option coupe-file. Quelle est la probabilité qu'il s'agisse d'un
skieur ayant un forfait SENIOR ? Arrondir le résultat à $10^{-3}$.
\item Est-il vrai que les personnes de moins de vingt-cinq ans représentent moins de 15\,\% des skieurs ayant choisi l'option coupe-file ? Expliquer.
\end{enumerate}

\bigskip

\textbf{Partie B}

\medskip

On rappelle que la probabilité qu'un skieur choisisse l'option coupe-file est égale à 0,112.

On considère un échantillon de $30$ skieurs choisis au hasard.

Soit $X$ la variable aléatoire qui compte le nombre des skieurs de l'échantillon ayant
choisi t'option coupe-file.

\medskip

\begin{enumerate}
\item On admet que la variable aléatoire $X$ suit une loi binomiale.\index{loi binomiale}

Donner les paramètres de cette loi.
\item Calculer la probabilité qu'au moins un des $30$ skieurs ait choisi l'option coupe-file. Arrondir le résultat à $10^{-3}$.
\item Calculer la probabilité qu'au plus un des $30$ skieurs ait choisi l'option coupe-file. Arrondir le résultat à $10^{-3}$.
\item Calculer l'espérance mathématique de la variable aléatoire $X$.\index{espérance}
\end{enumerate}

\bigskip

\textbf{\textsc{Exercice 2} \quad 6 points\hfill Thème: Fonction exponentielle }

\medskip

\begin{tabularx}{\linewidth}{|X|}\hline
\textbf{Principaux domaines abordés:} Suites ; Fonctions, fonction logarithme.\\ \hline
\end{tabularx}

\medskip

\emph{Cet exercice est un questionnaire à choix multiples.\\
Pour chacune des questions suivantes, une seule des quatre réponses proposées
est exacte.\\
Une réponse exacte rapporte un point. Une réponse fausse, une réponse multiple ou
l'absence de réponse à une question ne rapporte ni n'enlève de point.\\
Pour répondre, indiquer sur la copie le numéro de la question et la lettre de la réponse choisie. Aucune justification n'est demandée.}\index{QCM}

\bigskip
\begin{enumerate}
\item Un récipient contenant initialement 1 litre d'eau est laissé au soleil.

Toutes les heures, le volume d'eau diminue de 15\,\%.

Au bout de quel nombre entier d'heures le volume d'eau devient-il inférieur à un quart de litre ?\index{inéquation}

\begin{center}
\begin{tabularx}{\linewidth}{*{4}{X}}
\textbf{a.~~} 2 heures &\textbf{b.~~} 8 heures .&\textbf{c.~~} 9 heures
&\textbf{d.~~} 13 heures 
\end{tabularx}
\end{center}

\item On considère la fonction $f$ définie sur l'intervalle $]0~;~+\infty[$ par $f(x) = 4\ln (3x)$.\index{fonction logarithme}

Pour tout réel $x$ de l'intervalle $]0~;~+\infty[$ , on a :

\begin{center}
\begin{tabularx}{\linewidth}{*{2}{X}}
\textbf{a.~~} $f(2x) = f(x) + \ln (24) - \ln \left(\frac32\right)$&\textbf{b.~~}  $f(2x) = f(x) + \ln (16)$\\
\textbf{c.~~} $f(2x) = \ln (2) + f(x)$& \textbf{d.~~} $f(2x) = 2f(x)$
\end{tabularx}
\end{center}
\item  On considère la fonction $g$ définie sur l'intervalle $]1~;~+\infty[$ par : 

\[g(x) = \dfrac{\ln (x)}{x - 1}.\]\index{fonction logarithme}


On note $\mathcal{C}_g$ la courbe représentative de la fonction $g$ dans un repère orthogonal. La courbe $\mathcal{C}_g$ admet :

\begin{center}
\begin{tabularx}{\linewidth}{*{2}{X}}
\textbf{a.~~} une asymptote verticale
et une asymptote horizontale.&\textbf{b.~~} une asymptote verticale
et aucune asymptote horizontale.\\
\textbf{c.~~} aucune asymptote verticale et une asymptote horizontale.&\textbf{d.~~} 
aucune asymptote verticale .et aucune asymptote horizontale.
\end{tabularx}
\end{center}
\end{enumerate}\index{asymptote}

Dans la suite de l'exercice, on considère la fonction $h$ définie sur l'intervalle ]0~;~2] par:

\[h(x) = x^2(1 + 2\ln (x)).\]\index{fonction logarithme}

On note $\mathcal{C}_h$ la courbe représentative de $h$ dans un repère du plan. 

On admet que $h$ est deux fois dérivable sur l'intervalle ]0~;~2].

On note $h'$ sa dérivée et $h''$ sa dérivée seconde.

On admet que, pour tout réel $x$ de l'intervalle ]0~;~2], on a : 

\[h'(x) = 4x(1 + \ln (x)).\]

\begin{enumerate}[resume]
\item Sur l'intervalle $\left]\dfrac{1}{\text{e}}~;~2\right]$, la fonction $h$ s'annule :

\begin{center}\index{lecture graphique}
\begin{tabularx}{\linewidth}{*{2}{X}}
\textbf{a.~~} exactement 0 fois. &\textbf{b.~~} exactement 1 fois.\\
 \textbf{c.~~} exactement 2 fois. &\textbf{d.~~} exactement 3 fois.
\end{tabularx}
\end{center}

\item Une équation de la tangente à $\mathcal{C}_h$ au point d'abscisse $\sqrt{\text{e}}$ est :\index{equation de tangente@équation de tangente}

\begin{center}
\begin{tabularx}{\linewidth}{*{2}{X}}
\textbf{a.~~} $y = \left(6\text{e}^{\frac12}\right) .{} x $&
\textbf{b.~~} $y = \left(6\sqrt{\text{e}}\right).{} x + 2\text{e}$\\
\textbf{c.~~} $y = 6\text{e}^{\frac{x}{2}}$&\textbf{d.~~} $y = \left(6\text{e}^{\frac12}\right) .{} x - 4\text{e}$.
\end{tabularx}
\end{center}

\item Sur l'intervalle ]0~;~2], le nombre de points d'inflexion de la courbe $\mathcal{C}_h$ est égal à :\index{point d'inflexion}

\begin{center}
\begin{tabularx}{\linewidth}{*{4}{X}}
\textbf{a.~~} 0&\textbf{b.~~} 1 &\textbf{c.~~}  2 &\textbf{d.~~}  3
\end{tabularx}
\end{center}

\item \footnote{Uniquement au Liban} On considère la suite $\left(u_n\right)$ définie pour tout entier naturel $n$ par 

\[u_{n+1} = \dfrac12u_n + 3\quad \text{et}\quad u_0 = 6.\]

On peut affirmer que :\index{suites}

\begin{center}
\begin{tabularx}{\linewidth}{*{2}{X}}
\textbf{a.~~} la suite $\left(u_n\right)$ est strictement croissante.&\textbf{b.~~} la suite $\left(u_n\right)$ est strictement décroissante.\\
\textbf{c.~~} la suite $\left(u_n\right)$ n'est pas monotone. &\textbf{d.~~} la suite $\left(u_n\right)$ est constante.
\end{tabularx}
\end{center}
\end{enumerate}

\bigskip

\textbf{\textsc{Exercice 3} \quad 6 points\hfill Thème: Fonction exponentielle }

\medskip

\begin{tabularx}{\linewidth}{|X|}\hline
\textbf{Principaux domaines abordés :} Suites;
Fonctions, Fonction exponentielle.\\ \hline
\end{tabularx}

\bigskip

\textbf{Partie A}

\medskip

On considère la fonction $f$ définie pour tout réel $x$ par:

\[ f(x) = 1+x - \text{e}^{0,5x - 2}.\]\index{fonction exponentielle}

On admet que la fonction $f$ est dérivable sur $\R$. On note $f'$ sa dérivée.

\medskip

\begin{enumerate}
\item 
	\begin{enumerate}
		\item Déterminer la limite de la fonction $f$ en $- \infty$.\index{limite de fonction}
		\item Démontrer que, pour tout réel $x$ non nul, $f(x) = 1+ 0,5x\left(2 - \dfrac{\text{e}^{0,5x}}{0,5x} \times \text{e}^{-2}\right)$.\index{dérivée}
		
En déduire la limite de la fonction $f$ en $+\infty$.\index{limite de fonction}
	\end{enumerate}
\item 
	\begin{enumerate}
		\item Déterminer $f'(x)$ pour tout réel $x$.\index{dérivée}
		\item Démontrer que l'ensemble des solutions de l'inéquation $f'(x) < 0$ est l'intervalle 
		
		$]4 + 2\ln (2)~;~+\infty[$.\index{inéquation}
	\end{enumerate}	
\item Déduire des questions précédentes le tableau de variations de la fonction $f$ sur $\R$. On fera figurer la valeur exacte de l'image de $4 + 2\ln (2)$ par $f$.\index{tableau de variations}
\item Montrer que l'équation $f(x) = 0$ admet une unique solution sur l'intervalle $[-1~;~0]$.\index{valeurs intermédiaires}
\end{enumerate}

\bigskip

\textbf{Partie B}

\medskip

On considère la suite $\left(u_n\right)$ définie par $u_0 = 0$ et, pour tout entier naturel $n$,
\: $u_{n+1} = f\left(u_n\right)$ où $f$ est la fonction définie à la partie A.\index{suites}

\medskip

\begin{enumerate}
\item 
	\begin{enumerate}
		\item Démontrer par récurrence que, pour tout entier naturel $n$, on a :\index{récurrence}
		
		\[u_n \leqslant u_{n+1} \leqslant 4.\]
		
		\item En déduire que la suite $\left(u_n\right)$ converge. On notera $\ell$ la limite.\index{limite de suite}
	\end{enumerate}	
\item 
	\begin{enumerate}
		\item On rappelle que $f$ vérifie la relation $\ell = f(\ell)$.
		
Démontrer que $\ell  = 4$.
		\item ~
		
\begin{minipage}{0.48\linewidth}On considère la fonction \texttt{valeur} écrite ci-contre dans le langage Python :\index{script python}
		\end{minipage}\hfill 
\begin{minipage}{0.48\linewidth}
\begin{tabular}{|l|}\hline
\texttt{def valeur (a) :}\\
\quad \texttt{u = 0}\\
\quad \texttt{n = 0}\\
\quad \texttt{while u $\leqslant$ a:}\\
\qquad \texttt{u=1 + u - exp(0.5*u - 2)}\\
\qquad \texttt{n = n+1}\\ 
\quad return \texttt{n}\\ \hline
\end{tabular}
\end{minipage}

L'instruction \texttt{valeur}(3.99) renvoie la valeur 12.

Interpréter ce résultat dans le contexte de l'exercice.
	\end{enumerate}
\end{enumerate}

\bigskip

\textbf{\textsc{Exercice 4} \quad 6 points\hfill Thème: Fonction exponentielle }

\medskip

\begin{tabularx}{\linewidth}{|X|}\hline
\textbf{Principaux domaines abordés :} Géométrie dans l'espace\\ \hline
\end{tabularx}

\medskip

L'espace est muni d'un repère orthonormé \Oijk.

On considère les points A$(5~;~0~;~-1)$, B$(1~;~4~;~-1)$, C(1~;~0~;~3), D(5~;~4~;~3) et E(10~;~9~;~8).

\medskip

\begin{enumerate}
\item 
	\begin{enumerate}
		\item Soit R le milieu du segment [AB].
		
Calculer les coordonnées du point R ainsi que les coordonnées du vecteur $\vect{\text{AB}}$.
		\item Soit $\mathcal{P}_1$ le plan passant par le point R et dont $\vect{\text{AB}}$ est un vecteur normal. Démontrer qu'une équation cartésienne du plan $\mathcal{P}_1$ est:

\[x - y - 1 = 0.\]\index{equation de plan@équation de plan}
		\item Démontrer que le point E appartient au plan $\mathcal{P}_1$ et que EA = EB .
	\end{enumerate}	
\item  On considère le plan $\mathcal{P}_2$ d'équation cartésienne $x - z - 2 = 0$. 
	\begin{enumerate}
		\item Justifier que les plans $\mathcal{P}_1$ et $\mathcal{P}_2$ sont sécants.
		\item On note $\Delta$ la droite d'intersection de $\mathcal{P}_1$ et $\mathcal{P}_2$.
		
Démontrer qu'une représentation paramétrique de la droite $\Delta$ est :\index{equation paramétrique de droite@équation paramétrique de droite}

\[\left\{\begin{array}{l c r}
x&=&2+t\\
y&=&1+t \\
z&=&t\end{array}\right.(t \in \R).\]

	\end{enumerate}
\item  On considère le plan $\mathcal{P}_3$ d'équation cartésienne $y + z - 3 = 0$.

Justifier que la droite $\Delta$ est sécante au plan $\mathcal{P}_3$ en un point $\Omega$ dont on déterminera les coordonnées.
\end{enumerate}

\medskip

Si S et T sont deux points distincts de l'espace, on rappelle que l'ensemble des points M de l'espace tels que MS = MT est un plan, appelé plan médiateur du segment [ST]. On admet que les plans $\mathcal{P}_1$,\: $\mathcal{P}_2$ et $\mathcal{P}_3$ sont les plans médiateurs respectifs des
segments [AB], [AC] et [AD].\index{plan médiateur}

\begin{enumerate}[resume]
\item
	\begin{enumerate}
		\item Justifier que $\Omega \text{A} = \Omega \text{B} = \Omega \text{C} = \Omega \text{D}$.
		\item En déduire que les points A, B, C et D appartiennent à une même sphère dont
on précisera le centre et le rayon.\index{sphère}
	\end{enumerate}
\end{enumerate}

\newpage
%%%%%%%%%%% Amérique Nord1 18 mai 2022
\phantomsection
\hypertarget{AmeriqueNord1}{}

\label{AmeriqueNord1}

\lfoot{\small{Amérique du Nord}}
\rfoot{\small{18 mai 2022}}
\pagestyle{fancy}
\thispagestyle{empty}

\begin{center}{\Large\textbf{\decofourleft~Baccalauréat Amérique du Nord Jour 1 18 mai 2022~\decofourright\\[6pt] ÉPREUVE D'ENSEIGNEMENT DE SPÉCIALITÉ}}
\end{center}

\vspace{0,25cm}

Le sujet propose 4 exercices

Le candidat choisit 3 exercices parmi les 4 exercices et \textbf{ne doit traiter que ces 3 exercices}

Chaque exercice est noté sur \textbf{7 points (le total sera ramené sur 20 points)}.

Les traces de recherche, même incomplètes ou infructueuses, seront prises en compte.

\bigskip

\textbf{\textsc{Exercice 1} \quad (7 points)\hfill Thème : probabilités}

\medskip

Chaque chaque jour où il travaille, Paul doit se rendre à la gare pour rejoindre son lieu de travail en train. Pour cela, il prend son vélo deux fois sur trois et, si il ne prend pas son vélo, il prend sa voiture.

\medskip

\begin{enumerate}
\item Lorsqu'il prend son vélo pour rejoindre la gare, Paul ne rate le train qu'une fois sur 50 alors que, lorsqu'il prend sa voiture pour rejoindre la gare Paul rate son train une fois sur 10.

On considère une journée au hasard lors de laquelle Paul sera à la gare pour prendre le train qui le conduira au travail.

On note:

\setlength\parindent{8mm}
\begin{itemize}
\item[$\bullet~~$] $V$ l'évènement \og Paul prend son vélo pour rejoindre la gare \fg{} ;
\item[$\bullet~~$] $R$ l'évènement \og Paul rate son train \fg.
\end{itemize}
\setlength\parindent{0mm}

	\begin{enumerate}
		\item Faire un arbre pondéré résumant la situation.\index{arbre pondéré}
		\item Montrer que la probabilité que Paul rate son train est égale à $\dfrac{7}{150}$.\index{probabilités}
		\item Paul a raté son train. Déterminer la valeur exacte de la probabilité qu'il ait pris son vélo pour rejoindre la gare.
	\end{enumerate}
\item On choisit au hasard un mois pendant lequel Paul s'est rendu $20$ jours à la gare pour rejoindre son lieu de travail selon les modalités décrites en préambule.

On suppose que, pour chacun de ces 20 jours, le choix entre le vélo et
la voiture est indépendant des choix des autres jours.

On note $X$ la variable aléatoire donnant le nombre de jours où Paul prend son vélo sur ces $20$ jours.
	\begin{enumerate}
		\item Déterminer la loi suivie par la variable aléatoire $X$. Préciser ses paramètres.\index{loi binomiale}
		\item Quelle est la probabilité que Paul prenne son vélo exactement $10$ jours sur ces $20$ jours pour se rendre à la gare ? On arrondira la probabilité cherchée à $10^{-3}$.
		\item Quelle est la probabilité que Paul prenne son vélo au moins $10$ jours sur ces 20 jours pour se rendre à la gare ?
On arrondira la probabilité cherchée à $10^{-3}$.
		\item En moyenne, combien de jours sur une période choisie au hasard de 20 jours pour se rendre à la gare, Paul prend-il son vélo ? On arrondira la réponse à l'entier.\index{espérance}
	\end{enumerate}
\item Dans le cas où Paul se rend à la gare en voiture, on note $T$ la variable aléatoire donnant le temps de trajet nécessaire pour se rendre à la gare. La durée du trajet est donnée en minutes, arrondie à la minute. La loi de probabilité de $T$ est donnée par le tableau ci-dessous :\index{loi de probabilité}

\begin{center}
\begin{tabularx}{\linewidth}{|m{2.3cm}|*{9}{>{\centering \arraybackslash}X|}}\hline
$k$ (en minutes)&10 &11&12 &13 &14 &15 &16 &17 &18\\ \hline
$P(T = k)$&0,14&0,13 &0,13&0,12 &0,12&0,11 &0,10 &0,08&0,07\\ \hline
\end{tabularx}
\end{center}

Déterminer l'espérance de la variable aléatoire $T$ et interpréter cette valeur dans le contexte de l'exercice.
\end{enumerate}

\bigskip

\textbf{\textsc{Exercice 2} \quad (7 points)\hfill Thème : suites}

\medskip

Dans cet exercice, on considère la suite $\left(T_n\right)$ définie par :

\[T_0 = 180 \:\text{et, \: pour tout entier naturel} \:n, \: T_{n+1} = 0,955T_n + 0,9\]

\begin{enumerate}
\item 
	\begin{enumerate}
		\item Démontrer par récurrence que, pour tout entier naturel $n$,\: $T_n \geqslant 20$.\index{récurrence}
		\item Vérifier que pour tout entier naturel $n$,\: $T_{n+1} -  T_n  = - 0,045\left(T_n - 20\right)$. En déduire le sens de variation de la suite $\left(T_n\right)$.
		\item Conclure de ce qui précède que la suite $\left(T_n\right)$ est convergente. Justifier.\index{suite convergente}
	\end{enumerate}	
\item Pour tout entier naturel $n$, on pose : $u_n =  T_n - 20$.
	\begin{enumerate}
		\item Montrer que la suite $\left(u_n\right)$ est une suite géométrique dont on précisera la raison.\index{suite géométrique}
		\item En déduire que pour tout entier naturel $n$, \:$T_n =  20 + 160 \times  0,955^n$.
		\item Calculer la limite de la suite $\left(T_n\right)$.\index{limite de suite}
		\item Résoudre l'inéquation $T_n \leqslant 120$ d'inconnue $n$ entier naturel.\index{inéquation}
	\end{enumerate}	
\item Dans cette partie, on s'intéresse à l'évolution de la température au centre d'un gâteau après sa sortie du four. 

On considère qu'à la sortie du four, la température au centre du gâteau est de $180 \degres$ C et celle de l'air ambiant de $20 \degres$ C.

La loi de refroidissement de Newton permet de modéliser la température au centre du gâteau par la suite précédente $\left(T_n\right)$. Plus précisément, $T_n$ représente la température au centre du gâteau, exprimée en degré Celsius, n minutes après sa sortie du four.
	\begin{enumerate}
		\item Expliquer pourquoi la limite de la suite $\left(T_n\right)$ déterminée à la question 2. c. était prévisible dans le contexte de l'exercice.
		\item On considère la fonction Python ci-dessous:\index{script python}
		\begin{center}
		\begin{tabular}{|l|}\hline
def temp(x) :\\
\quad T = 180\\
\quad n = 0\\
\quad while T $>$ x :\\
\qquad T=0.955*T+0.9\\
\qquad n=n+1\\
\quad return n\\ \hline
\end{tabular}
\end{center}
Donner le résultat obtenu en exécutant la commande temp(120).

Interpréter le résultat dans le contexte de l'exercice.
	\end{enumerate}
\end{enumerate}

\bigskip

\textbf{\textsc{Exercice 3} \quad (7 points)\hfill Thème : géométrie dans l'espace}

\medskip

Dans l'espace muni d'un repère orthonormé \Oijk{} d'unité 1 cm, on considère les points suivants :

\[\text{J}(2~;~0~;~1), \quad \text{K}( 1~;~2~;~1)\:\text{et} \quad \text{L}(-2~;~-2~;~-2)\]

\begin{enumerate}
\item 
	\begin{enumerate}
		\item Montrer que le triangle JKL est rectangle en J.\index{triangle rectangle}
		\item Calculer la valeur exacte de l'aire du triangle JKL en cm$^2$.
		\item Déterminer une valeur approchée au dixième près de l'angle géométrique $\widehat{\text{JKL}}$.\index{mesure d'angle}
	\end{enumerate}
		
\item
	\begin{enumerate}
		\item Démontrer que le vecteur $\vect{n}$ de coordonnées $\begin{pmatrix}6\\3\\-10\end{pmatrix}$ est un vecteur normal au plan (JKL).\index{vecteur normal}
		\item En déduire une équation cartésienne du plan (JKL).\index{equation de plan@équation de plan}
	\end{enumerate}
\end{enumerate}

Dans la suite, T désigne le point de coordonnées $(10~;~9~;~-6)$.

\begin{enumerate}[resume]
\item
	\begin{enumerate}
		\item Déterminer une représentation paramétrique de la droite $\Delta$
orthogonale au plan (JKL) et passant par T.\index{equation paramétrique de droite@équation paramétrique de droite}
		\item Déterminer les coordonnées du point H, projeté orthogonal du point T sur le plan (JKL).
		\item On rappelle que le volume $V$ d'un tétraèdre est donné par la formule :\index{volume tétraèdre}
		
		\[V = \dfrac13 \mathcal{B} \times h\:\: \text{où } \mathcal{B}\:\text{désigne l'aire d'une base et } \: h \: \text{la hauteur correspondante}\]
		
Calculer la valeur exacte du volume du tétraèdre JKLT en cm$^3$.
	\end{enumerate}
\end{enumerate}

\bigskip

\textbf{\textsc{Exercice 4} \quad (7 points)\hfill Thème : fonction exponentielle}

\medskip

Pour chacune des affirmations suivantes, indiquer si elle est vraie ou fausse. Justifier chaque réponse.\index{Vrai-Faux}

\medskip

\begin{enumerate}
\item \textbf{Affirmation 1} : Pour tout réel $x$ : $1 - \dfrac{1 - \text{e}^x}{1 + \text{e}^x} = \dfrac{2}{1 + \text{e}^{-x}}$.
\item On considère la fonction$g$ définie sur $\R$ par $g(x) = \dfrac{\text{e}^x}{\text{e}^x + 1}$.

\textbf{Affirmation 2 : } L'équation $g(x) = \dfrac12$ admet une unique solution dans $\R$.\index{valeurs intermédiaires}
\item On considère la fonction $f$ définie sur $\R$ par $f(x) = x^2\text{e}^{-x}$ et on note 
$\mathcal{C}$ sa courbe dans un repère orthonormé.

\textbf{Affirmation 3 : } L'axe des abscisses est tangent à la courbe $\mathcal{C}$ en un seul point.
\item On considère la fonction $h$ définie sur $\R$ par $h(x) = \text{e}^x\left(1 -  x^2\right)$.\index{tangente à la courbe}

\textbf{Affirmation 4 :} Dans le plan muni d'un repère orthonormé, la courbe représentative de la fonction $h$ n'admet pas de point d'inflexion.\index{point d'inflexion}

\item \textbf{Affirmation 5 :} $\displaystyle\lim_{x \to + \infty} \dfrac{\text{e}^x}{\text{e}^x + x} = 0$.\index{limite de fonction}

\item \textbf{Affirmation 6 :} Pour tout réel $x,\: 1 + \text{e}^{2x} \geqslant 2\text{e}^x$.
\end{enumerate}

\newpage
%%%%%%%%%%% GroupeI2 19 mai 2022
\phantomsection
\hypertarget{GroupeI2}{}

\label{GroupeI2}

\lfoot{\small{Centres étrangers}}
\rfoot{\small{19 mai 2022}}
\pagestyle{fancy}
\thispagestyle{empty}

\begin{center}{\Large\textbf{\decofourleft~Baccalauréat Centres étrangers Groupe 1 D\footnote{Afrique du Sud, Bulgarie, Comores, Djibouti, Kenya, Liban, Lituanie, Madagascar, Mozambique et Ukraine} 19 mai 2022~\decofourright\\[7pt]  Sujet 2\\[7pt] ÉPREUVE D'ENSEIGNEMENT DE SPÉCIALITÉ}}
\end{center}

\vspace{0,25cm}

Le sujet propose 4 exercices

Le candidat choisit 3 exercices parmi les 4 exercices et \textbf{ne doit traiter que ces 3 exercices}

Chaque exercice est noté sur 7 points (le total sera ramené sur 20 points).\footnote{Madagascar : chaque exercice est noté sur 6  points. La clarté et la précision de l'argumentation ainsi que la qualité de la rédaction sont notées sur 2 points.}

Les traces de recherche, même incomplètes ou infructueuses, seront prises en compte.

\bigskip

\textbf{\textsc{Exercice 1} \quad 6 points\hfill Thème: Probabilités}

\medskip

\emph{Les résultats seront arrondis si besoin à $10^{- 4}$ près}

\medskip

Une étude statistique réalisée dans une entreprise fournit les informations suivantes :

\setlength\parindent{1cm}
\begin{itemize}
\item[$\bullet~~$] 48\,\% des salariés sont des femmes. Parmi elles, $16,5$\,\% exercent une
profession de cadre ;
\item[$\bullet~~$] 52\,\% des salariés sont des hommes. Parmi eux, $21,5$\,\% exercent une
profession de cadre.
\end{itemize}
\setlength\parindent{0cm}

\medskip

On choisit une personne au hasard parmi les salariés. On considère les évènements suivants:

\setlength\parindent{1cm}
\begin{itemize}
\item[$\bullet~~$] $F$: \og la personne choisie est une femme \fg{} ;
\item[$\bullet~~$] $C$: \og la personne choisie exerce une profession de cadre \fg.
\end{itemize}
\setlength\parindent{0cm}

\medskip

\begin{enumerate}
\item Représenter la situation par un arbre pondéré.\index{arbre pondéré}
\item Calculer la probabilité que la personne choisie soit une femme qui exerce une profession de cadre.
\item 
	\begin{enumerate}
		\item Démontrer que la probabilité que la personne choisie exerce une profession de cadre est égale à $0,191$.
		\item Les évènements $F$ et $C$ sont-ils indépendants ? Justifier.\index{evènements indépendants@évènements indépendants}
	\end{enumerate}	
\item Calculer la probabilité de $F$ sachant $C$, notée $P_C(F)$. Interpréter le résultat dans le
contexte de l'exercice.
\item On choisit au hasard un échantillon de $15$ salariés. Le grand nombre de salariés
dans l'entreprise permet d'assimiler ce choix à un tirage avec remise.

On note $X$ la variable aléatoire donnant le nombre de cadres au sein de
l'échantillon de 15 salariés.

On rappelle que la probabilité qu'un salarié choisi au hasard soit un cadre est
égale à $0,191$.
	\begin{enumerate}
		\item Justifier que $X$ suit une loi binomiale dont on précisera les paramètres.\index{loi binomiale}
		\item Calculer la probabilité que l'échantillon contienne au plus 1 cadre.
		\item Déterminer l'espérance de la variable aléatoire $X$.\index{espérance}
	\end{enumerate}	
\item Soit $n$ un entier naturel.

On considère dans cette question un échantillon de $n$ salariés.

Quelle doit être la valeur minimale de $n$ pour que la probabilité qu'il y ait au moins un cadre au sein de l'échantillon soit supérieure ou égale à $0,99$ ?\index{inéquation}
\end{enumerate}

\bigskip

\textbf{\textsc{Exercice 2} \quad 6 points\hfill Thème: Géométrie dans l'espace}

\medskip

On considère le cube ABCDEFGH de côté 1 représenté ci-dessous.

\begin{center}
\psset{unit=1cm}
\begin{pspicture}(7.2,6.9)
\psframe(0.2,0.2)(5.2,5.2)%ABFE
\psline(5.2,0.2)(6.8,1.5)(6.8,6.5)(5.2,5.2)%BCGF
\psline(6.8,6.5)(1.8,6.5)(0.2,5.2)%GHE
\psline[linestyle=dashed](0.2,0.2)(1.8,1.5)(6.8,1.5)%ADC
\psline[linestyle=dashed](1.8,1.5)(1.8,6.5)%DH
\uput[dl](0.2,0.2){A} \uput[dr](5.2,0.2){B} \uput[r](6.8,1.5){C} \uput[l](1.8,1.5){D}
\uput[l](0.2,5.2){E} \uput[r](5.2,5.2){F} \uput[ur](6.8,6.5){G} \uput[ul](1.8,6.5){H}
\end{pspicture}
\end{center}

On munit l'espace du repère orthonormé $\left(\text{A}~;~\vect{\text{AB}},\, \vect{\text{AD}},\, \vect{\text{AE}}\right)$.

\medskip

\begin{enumerate}
\item 
	\begin{enumerate}
		\item Justifier que les droites (AH) et (ED) sont perpendiculaires.\index{droites perpendiculaires}
		\item Justifier que la droite (GH) est orthogonale au plan (EDH).\index{vecteur normal}
		\item En déduire que la droite (ED) est orthogonale au plan (AGH).
	\end{enumerate}	
\item Donner les coordonnées du vecteur $\vect{\text{ED}}$.

Déduire de la question 1. c. qu'une équation cartésienne du plan (AGH) est:\index{equation de plan@équation de plan}

\[y - z = 0.\]

\item On désigne par L le point de coordonnées $\left(\dfrac23~;~1~;~0\right)$.
	\begin{enumerate}
		\item Déterminer une représentation paramétrique de la droite (EL).\index{equation paramétrique de droite@équation paramétrique de droite}
		\item Déterminer l'intersection de la droite (EL) et du plan (AGH).
		\item Démontrer que le projeté orthogonal du point L sur le plan (AGH) est le
point K de coordonnées $\left(\dfrac23~;~\dfrac12~;~\dfrac12\right)$.
		\item Montrer que la distance du point L au plan (AGH) est égale à $\dfrac{\sqrt{2}}{2}$.
		\item Déterminer le volume du tétraèdre LAGH.\index{volume tétraèdre}
		
On rappelle que le volume $V$ d'un tétraèdre est donné par la formule :

\[V = \dfrac13 \times (\text{aire de la base}) \times \text{hauteur}.\]
	\end{enumerate}
\end{enumerate}

\bigskip

\textbf{\textsc{Exercice 3} \quad 6 points\hfill Thème: Fonctions; Suites}

\medskip

\emph{Cet exercice est un questionnaire à choix multiples.\\
Pour chacune des questions suivantes, une seule des quatre réponses proposées est exacte.\\
Une réponse fausse, une réponse multiple ou l'absence de réponse à une question ne rapporte ni n'enlève de point.\\
Pour répondre, indiquer sur la copie le numéro de la question et la lettre de la réponse choisie.\\Aucune justification n'est demandée.}\index{QCM}

\medskip

\begin{enumerate}
\item Soit $g$ la fonction définie sur $\R$ par $g(x) = x^{\np{1000}} + x$.\index{fonction exponentielle}

On peut affirmer que :
	\begin{enumerate}
		\item la fonction $g$ est concave sur $\R$.\index{convexité}
		\item la fonction $g$ est convexe sur $\R$.
		\item la fonction $g$ possède exactement un point d'inflexion.\index{point d'inflexion}
		\item la fonction $g$ possède exactement deux points d'inflexion.
	\end{enumerate}
\item On considère une fonction $f$ définie et dérivable sur $\R$. 

\begin{minipage}{0.48\linewidth}
On note $f'$ sa fonction dérivée.

On note $\mathcal{C}$ la courbe représentative de $f$.

On note $\Gamma$ la courbe représentative de $f'$.

On a tracé ci-contre la courbe $\Gamma$.
\end{minipage}\hfill
\begin{minipage}{0.4\linewidth}
\psset{unit=1.2cm}
\begin{pspicture*}(-2.1,-1.6)(3.2,1.4)
\psgrid[gridlabels=0pt,subgriddiv=1,gridwidth=0.15pt]
\psaxes[linewidth=1.25pt,labelFontSize=\scriptstyle]{->}(0,0)(-2,-1.6)(3.2,1.4)
\psplot[plotpoints=2000,linewidth=1.25pt,linecolor=red]{-1.8}{3.2}{x 1 add 2.71828 x exp div}
\uput[ur](2,0.45){\red $\Gamma$}
\end{pspicture*}
\end{minipage}

On note $T$ la tangente à la courbe $\mathcal{C}$ au point d'abscisse $0$.

On peut affirmer que la tangente $T$ est parallèle à la droite d'équation :\index{lecture graphique}

\begin{center}
\begin{tabularx}{\linewidth}{*{2}{X}}
\textbf{a.~~}$y =x$ &\textbf{b.~~}$y = 0$\\
\textbf{c.~~}$y = 1$&\textbf{d.~~}$x = 0$
\end{tabularx}
\end{center}
\item On considère la suite $\left(u_n\right)$ définie pour tout entier naturel $n$ par $u_n = \dfrac{(-1)^n}{n+1}$.\index{suites}

On peut affirmer que la suite $\left(u_n\right)$ est :

\begin{center}
\begin{tabularx}{\linewidth}{*{2}{X}}
\textbf{a.~~}majorée et non minorée.&\textbf{b.~~}minorée et non majorée.\\
\textbf{c.~~}bornée.				&\textbf{d.~~}non majorée et non minorée.\index{suite bornée}
\end{tabularx}
\end{center}

\item Soit $k$ un nombre réel non nul.

Soit $\left(v_n\right)$  une suite définie pour tout entier naturel $n$.

On suppose que $v_0 = k$ et que pour tout $n$, on a $v_n \times v_{n+1} < 0$.

On peut affirmer que $v_{10}$ est :

\begin{center}
\begin{tabularx}{\linewidth}{*{2}{X}}
\textbf{a.~~}positif. &\textbf{a.~~} négatif.\\
\textbf{c.~~}du signe de $k$.&\textbf{d.~~}du signe de $- k$.
\end{tabularx}
\end{center}
\item On considère la suite $\left(w_n\right)$ définie pour tout entier naturel $n$ par : 

\[w_{n+1} = 2w_n - 4\quad \text{et}\quad  w_2 = 8.\]

On peut affirmer que:

\begin{center}
\begin{tabularx}{\linewidth}{*{2}{X}}
\textbf{a.~~}$w_0 = 0$&\textbf{b.~~} $w_0 = 5$.\\
\textbf{c.~~}$w_0 = 10$.&\textbf{d.~~}Il n'est pas possible de calculer $w_0$.
\end{tabularx}
\end{center}

\item \footnote{Cette question ne fait pas partie du sujet donné à Madagascar} On considère la suite $\left(a_n\right)$ définie pour tout entier naturel $n$ par :

\[a_{n+1} = \dfrac{\text{e}^n}{\text{e}^n + 1}a_n\quad \text{et} \quad a_0 = 1.\]

On peut affirmer que :

\begin{center}
\begin{tabularx}{\linewidth}{*{2}{X}}
\textbf{a.~~} la suite $\left(a_n\right)$ est strictement croissante.&\textbf{b.~~}la suite $\left(a_n\right)$ est strictement décroissante.\\
\textbf{c.~~} la suite $\left(a_n\right)$ n'est pas monotone.&\textbf{d.~~} la suite $\left(a_n\right)$ est constante.
\end{tabularx}
\end{center}\index{fonction monotone}
\item Une cellule se reproduit en se divisant en deux cellules identiques, qui se divisent à leur tour, et ainsi de suite. 

On appelle temps de génération le temps nécessaire pour qu'une cellule donnée se divise en deux cellules.

On a mis en culture 1 cellule. Au bout de 4 heures, il y a environ \np{4000} cellules.

On peut affirmer que le temps de génération est environ égal à :

\begin{center}
\begin{tabularx}{\linewidth}{*{2}{X}}
\textbf{a.~~}moins d'une minute.&\textbf{b.~~}12 minutes.\\
\textbf{c.~~}20 minutes.		&\textbf{d.~~}1 heure.
\end{tabularx}
\end{center}

\end{enumerate}

\bigskip

\textbf{\textsc{Exercice 4} \quad 6 points\hfill Thème: Fonctions, Fonction exponentielle, Fonction logarithme; Suites}

\medskip

\textbf{Partie A}

\medskip

On considère la fonction $f$ définie pour tout réel $x$ de ]0~;~1] par:

\[f(x) = \text{e}^{-x} + \ln (x).\]\index{fonction logarithme}\index{fonction exponentielle}

\smallskip

\begin{enumerate}
\item Calculer la limite de $f$ en $0$.
\item On admet que $f$ est dérivable sur ]0~;~1]. On note $f'$ sa fonction dérivée.

Démontrer que, pour tout réel $x$ appartenant à ]0~;~1], on a :

\[f'(x) = \dfrac{1 - x\text{e}^{-x}}{x}\]\index{dérivée}

\item Justifier que, pour tout réel $x$ appartenant à ]0~;~1], on a $x\text{e}^{-x} < 1$.

En déduire le tableau de variation de $f$ sur ]0~;~1].\index{tableau de variations}
\item Démontrer qu'il existe un unique réel $\ell$ appartenant à ]0~;~1] tel que $f(\ell) = 0$.\index{valeurs intermédiaires}
\end{enumerate}

\medskip

\textbf{Partie B}

\medskip

\begin{enumerate}
\item On définit deux suites $\left(a_n\right)$ et $\left(b_n\right)$ par:\index{suites}

\[\left\{\begin{array}{l c l}
a_0& =& \dfrac{1}{10}\\
b_0& =& 1
\end{array}\right.\: \text{et, pour tout entier naturel }\: n,\left\{\begin{array}{l c l}
a_{n+1}&=&\text{e}^{-b_n}\\
b_{n+1}&=&\text{e}^{-a_n}
\end{array}\right.\]

	\begin{enumerate}
		\item Calculer $a_1$ et $b_1$. On donnera des valeurs approchées à $10^{-2}$ près.
		\item On considère ci-dessous la fonction \texttt{termes}, écrite en langage Python.\index{script python}
		
\begin{center}

\begin{tabular}{|l|}\hline
\texttt{def termes (n) :}\\
\quad \texttt{a=1/10}\\
\quad \texttt{b=1}\\
\quad \texttt{for k in range(0,n) :}\\
\qquad \texttt{c= \ldots}\\
\qquad \texttt{b  = \ldots}\\
\qquad \texttt{a = c}\\
\quad \texttt{return(a,b)}\\ \hline
\end{tabular}
\end{center}

Recopier et compléter sans justifier le cadre ci-dessus de telle sorte que la fonction termes calcule les termes des suites $\left(a_n\right)$ et $\left(b_n\right)$.
	\end{enumerate}
\item On rappelle que la fonction $x \longmapsto \text{e}^{-x}$ est décroissante sur $\R$.
	\begin{enumerate}
		\item Démontrer par récurrence que, pour tout entier naturel $n$, on a :\index{récurrence}
		
		\[0 < a_n \leqslant a_{n+1} \leqslant b_{n+1} \leqslant b_n \leqslant 1\]
		
		\item En déduire que les suites $\left(a_n\right)$ et $\left(b_n\right)$ sont convergentes.\index{suite convergente}
	\end{enumerate}	
\item On note $A$ la limite de $\left(a_n\right)$ et $B$ la limite de $\left(b_n\right)$.

On admet que $A$ et $B$ appartiennent à l'intervalle ]0~;~1], et que $A = \text{e}^{-B}$ et $B = \text{e}^{-A}$.
	\begin{enumerate}
		\item Démontrer que $f(A) = 0$.
		\item Déterminer $A - B$.
	\end{enumerate}
\end{enumerate}
%%%%%%%%%%%   fin Madagascar 19 mai 2022
\newpage
%%%%%%%%%%% Amérique Nord1 19 mai 2022
\phantomsection
\hypertarget{AmeriqueNord2}{}

\label{AmeriqueNord2}

\lfoot{\small{Amérique du Nord}}
\rfoot{\small{19 mai 2022}}
\pagestyle{fancy}
\thispagestyle{empty}

\begin{center}{\Large\textbf{\decofourleft~Baccalauréat Amérique du Nord Jour 2 19 mai 2022~\decofourright\\[6pt] ÉPREUVE D'ENSEIGNEMENT DE SPÉCIALITÉ}}
\end{center}

\vspace{0,25cm}

Le sujet propose 4 exercices

Le candidat choisit 3 exercices parmi les 4 exercices et \textbf{ne doit traiter que ces 3 exercices}

Chaque exercice est noté sur \textbf{7 points (le total sera ramené sur 20 points)}.

Les traces de recherche, même incomplètes ou infructueuses, seront prises en compte.

\bigskip

\textbf{\textsc{Exercice 1} \quad (7 points)\hfill Thème : probabilités, suites}

\medskip

Dans une région touristique, une société propose un service de location de vélos pour la journée.

\smallskip

La société dispose de deux points de location distinctes, le point A et le point B. Les vélos peuvent être empruntés et restitués indifféremment dans l'un où l'autre des deux points de location.

\smallskip

On admettra que le nombre total de vélos est constant et que tous les matins, à l'ouverture du service, chaque vélo se trouve au point A ou au point B.

\smallskip

D'après une étude statistique :

\smallskip

\setlength\parindent{1cm}
\begin{itemize}
\item[$\bullet~~$] Si un vélo se trouve au point A un matin, la probabilité qu'il se trouve au point A le matin suivant est égale à $0,84$ ;
\item[$\bullet~~$] Si un vélo se trouve au point B un matin la probabilité qu'il se trouve au point B le matin suivant est égale à $0,76$.
\end{itemize}
\setlength\parindent{0cm}

À l'ouverture du service le premier matin, la société a disposé la moitié de ses vélos au point A, l'autre moitié au point B.

\smallskip

On considère un vélo de la société pris au hasard.

\smallskip

Pour tout entier naturel non nul $n$, on définit les évènements suivants :

\setlength\parindent{1cm}
\begin{itemize}
\item[$\bullet~~$]$A_n$ : \og le vélo se trouve au point A le $n$-ième matin \fg{}
\item[$\bullet~~$]$B_n$ : \og le vélo se trouve au point B le $n$-ième matin \fg.
\end{itemize}
\setlength\parindent{0cm}

Pour tout entier naturel non nul $n$, on note $a_n$ la probabilité de l'évènement $A_n$ et $b_n$ la probabilité de l'évènement $B_n$. Ainsi $a_1 = 0,5$ et $b_1 =  0,5$.

\medskip

\begin{enumerate}
\item Recopier et compléter l'arbre pondéré ci-dessous qui modélise la situation pour les deux premiers matins :\index{arbre pondéré}

\begin{center}
\pstree[treemode=R,nodesepB=3pt,levelsep=3cm]{\TR{}}
{\pstree{\TR{$A_1~~$}\taput{\ldots}}
	{\TR{$A_2$}\taput{\ldots}
	\TR{$B_2$}\tbput{\ldots}
	}
\pstree{\TR{$B_1~~$}\tbput{\ldots}}
	{\TR{$A_2$}\taput{\ldots}
	\TR{$B_2$}\tbput{\ldots}
	}	
}
\end{center}

\item
	\begin{enumerate}
		\item Calculer $a_2$.
		\item Le vélo se trouve au point A le deuxième matin. Calculer la probabilité qu'il se soit trouvé au point B le premier matin. La probabilité sera arrondie au millième.
	\end{enumerate}
\item
	\begin{enumerate}
		\item Recopier et compléter l'arbre pondéré ci-dessous qui modélise la situation pour les $n$-ième et $n + 1$-ième  matins.\index{arbre pondéré}
		
\begin{center}
\pstree[treemode=R,nodesepB=3pt,levelsep=3cm]{\TR{}}
{\pstree{\TR{$A_n~~$}\taput{$a_n$}}
	{\TR{$A_{n+1}$}\taput{\ldots}
	\TR{$B_{n+1}$}\tbput{\ldots}
	}
\pstree{\TR{$B_n~~$}\tbput{\ldots}}
	{\TR{$A_{n+1}$}\taput{\ldots}
	\TR{$B_{n+1}$}\tbput{\ldots}
	}	
}
\end{center}
		\item Justifier que pour tout entier naturel non nul $n$,\: $a_{n+1} = 0,6a_n + 0,24$.
	\end{enumerate}
\item Montrer par récurrence que, pour tout entier naturel non nul $n$, \: $a_n = 0,6 - 0,1 \times 0,6^{n - 1}$.\index{récurrence}
\item Déterminer la limite de la suite $\left(a_n\right)$ et interpréter cette limite dans le contexte de l'exercice. \index{limite de suite}
\item Déterminer le plus petit entier naturel $n$ tel que $a_n \geqslant 0,599$ et interpréter le résultat obtenu dans le contexte de l'exercice.\index{inéquation}
\end{enumerate}

\bigskip

\textbf{\textsc{Exercice 2} \quad (7 points)\hfill Thème : fonctions, fonction exponentielle}

\bigskip

\textbf{Partie A}

\medskip 

Soit $p$ la fonction définie sur l'intervalle $[-3~;~4]$ par :

\[p(x) = x^3 - 3x^2 + 5x + 1\]\index{fonction polynôme}

\begin{enumerate}
\item Déterminer les variations de la fonction $p$ sur l'intervalle $[-3~;~4]$.
\item Justifier que l'équation $p(x) = 0$ admet dans l'intervalle $[-3~;~4]$ une unique solution qui sera notée $\alpha$.\index{valeurs intermédiaires}
\item Déterminer une valeur approchée du réel $\alpha$ au dixième près.
\item Donner le tableau de signes de la fonction $p$ sur l'intervalle $[-3~;~4]$.
\end{enumerate}

\bigskip

\textbf{Partie B}

\medskip 

Soit $f$ la fonction définie sur l'intervalle $[-3~;~4]$ par :

\[f(x) = \dfrac{\text{e}^x}{1 + x^2}\]\index{fonction exponentielle}

On note $\mathcal{C}_f$ sa courbe représentative dans un repère orthogonal.

\medskip

\begin{enumerate}
\item
	\begin{enumerate}
		\item Déterminer la dérivée de la fonction $f$ sur l'intervalle $[-3~;~4]$.\index{dérivée}
		\item Justifier que la courbe $\mathcal{C}_f$ admet une tangente horizontale au point d'abscisse 1.\index{equation de tangente@équation de tangente}
	\end{enumerate}
\item Les concepteurs d'un toboggan utilisent la courbe $\mathcal{C}_f$ comme profil d'un toboggan. Ils estiment que le toboggan assure de bonnes sensations si le profil possède au moins deux points d'inflexion.\index{point d'inflexion}

\medskip

\begin{minipage}{0.48\linewidth}
\psset{unit=0.75cm}
\begin{pspicture*}(-4,-1.95)(4.25,4.1)
\psgrid[gridlabels=0pt,subgriddiv=1,gridwidth=0.15pt](-4,0)(4,4)
\psaxes[linewidth=1.25pt,labelFontSize=\scriptstyle](0,0)(-4,-0)(4.25,4.1)
\psplot[plotpoints=2000,linewidth=1.25pt,linecolor=blue]{-3}{4}{2.71828 x exp x dup mul 1 add div}
\rput(0,-1){Représentation de la courbe $\mathcal{C}_f$}
\uput[ul](3,2){\blue \small $\mathcal{C}_f$}
\end{pspicture*}
\end{minipage} \hfill
\begin{minipage}{0.48\linewidth}
\psset{unit=0.75cm}
\begin{pspicture}(-4,-1.95)(4.25,4.1)
%\psgrid
\def\tobo{\psplot[plotpoints=2000,linewidth=1.25pt]{-3}{4}{2.71828 x exp x dup mul 1 add div}}
\psplot[plotpoints=2000,linewidth=1.25pt]{-3}{4}{2.71828 x exp x dup mul 1 add div}
\rput(-0.6,0.6){\tobo}
\psline(-3,0)(-3.6,0.6)\psline(4,3.2)(3.4,3.8)
\pscustom[fillstyle=solid,fillcolor=lightgray]
{\psplot[plotpoints=2000,linewidth=1.25pt,linecolor=red]{-3}{4}{2.71828 x exp x dup mul 1 add div}
\psline(4,3.2)(4,0)(-2,0)}
\rput(0,-1){Vue de profil du toboggan}
\end{pspicture}
\end{minipage}

	\begin{enumerate}
		\item D'après le graphique ci-dessus, le toboggan semble-t-il assurer de bonnes sensations ? Argumenter.\index{lecture graphique}
		\item On admet que la fonction $f''$, dérivée seconde de la fonction $f$, a pour expression pour tout réel $x$ de l'intervalle $[-3~;~4]$ :

		\[f''(x) = \dfrac{p(x)(x - 1)\text{e}^x}{\left(1 + x^2\right)^3}\]

où $p$ est la fonction définie dans la partie A.

En utilisant l'expression précédente de $f''$, répondre à la question : \og le toboggan assure-t-il de bonnes sensations ? \fg. Justifier.
	\end{enumerate}
\end{enumerate}

\bigskip

\textbf{\textsc{Exercice 3} \quad (7 points)\hfill Thème : géométrie dans l'espace}

\bigskip


Une exposition d'art contemporain a lieu dans une salle en forme de pavé droit de largeur 6 m, de longueur 8 m et de hauteur 4 m. 

Elle est représentée par le parallélépipède rectangle OBCDEFGH où
OB $= 6$~m, OD $= 8$~m et OE $= 4$~m.

On utilise le repère orthonormé \Oijk{} tel que $\vect{\imath} = \dfrac16\vect{\text{OB}}, \vect{\jmath} = \dfrac18\vect{\text{OD}}$ et $\vect{k} =\dfrac14\vect{\text{OE}}$.

\begin{center}
\psset{unit=1cm}
\begin{pspicture}(7.2,5.3)
%\psgrid
\pspolygon(0.2,1.2)(4.5,0.2)(4.5,2.7)(0.2,3.7)%BCGF
\psline(4.5,0.2)(6.9,1.2)(6.9,3.7)(4.5,2.7)%CDHG
\psline(6.9,3.7)(2.6,4.7)(0.2,3.7)%HEF
\psline(0.2,2.45)(1.8,3.33)(1.4,4.2)%ART
\psline[linestyle=dashed,linewidth=1.5pt](0.2,2.45)(1.4,4.2)
\psline[linestyle=dotted,linewidth=1.5pt](0.2,1.2)(2.6,2.2)(2.6,4.7)%BOE
\psline[linestyle=dotted,linewidth=1.5pt](2.6,2.2)(6.9,1.2)%OD
\psline{->}(2.6,2.2)(2.2,2.02)%vect i
\psline{->}(2.6,2.2)(3.4,2.03)%vect j
\psline{->}(2.6,2.2)(2.6,2.82)%\vect k
\psdot(2.7,1.5)%S
\uput[l](0.2,2.45){A} \uput[dl](0.2,1.2){B} \uput[d](4.5,0.2){C} \uput[dr](6.9,1.2){D}
\uput[u](2.6,4.7){E} \uput[ul](0.2,3.7){F} \uput[dr](4.5,2.7){G} \uput[ur](6.9,3.7){H}
\uput[d](2.6,2.2){O} \uput[ur](1.8,3.33){R} \uput[d](2.7,1.5){S} \uput[u](1.4,4.2){T}
\uput[ul](2.2,2){$\vect{\imath}$}\uput[ur](3.4,2.03){$\vect{\jmath}$}\uput[l](2.6,2.82){$\vect{k}$}
\end{pspicture}
\end{center}

Dans ce repère on a, en particulier C(6~;~8~;~0), F(6~;~0~;~4) et G(6~;~8~;~4).

Une des œuvres exposées est un triangle de verre représenté par le triangle ART qui a pour

sommets A(6~;~0~;~2), R(6~;~3~;~4) et T(3~;~0~;~4), Enfin, S est le point de coordonnées $\left(3~;~\dfrac52~;~0\right)$.

\begin{enumerate}
\item 
	\begin{enumerate}
		\item Vérifier que le triangle ART est isocèle en A.
		\item Calculer le produit scalaire $\vect{\text{AR}} \cdot \vect{\text{AT}}$.\index{produit scalaire}
		\item En déduire une valeur approchée à $0,1$ degré près de l'angle $\widehat{\text{RAT}}$.\index{calcul d'angle}
	\end{enumerate}
\item 
	\begin{enumerate}
		\item Justifier que le vecteur $\vect{n}\begin{pmatrix}2\\-2\\3\end{pmatrix}$ est un vecteur normal au plan (ART).\index{vecteur normal}
		\item En déduire une équation cartésienne du plan (ART).\index{equation de plan@équation de plan}
	\end{enumerate}
\item Un rayon laser dirigé vers le triangle ART est émis du plancher à partir du point S. On admet que ce rayon est orthogonal au plan (ART).
	\begin{enumerate}
		\item Soit $\Delta$ la droite orthogonale au plan (ART) et passant par le point S.
		
Justifier que le système ci-dessous est une représentation paramétrique de la droite $\Delta$ :\index{equation paramétrique de droite@équation paramétrique de droite}

\[\left\{\begin{array}{l c r}
x&=&3+2k\\
y&=& \dfrac52 - 2k\\
z &=& 3k
\end{array}\right.,\: \text{avec }\:k \in \R.\]

		\item Soit L le point d'intersection de la droite $\Delta$, avec le plan (ART).

Démontrer que L a pour coordonnées $\left(5~;~\dfrac12~;~3\right)$.
	\end{enumerate}
\item L'artiste installe un rail représenté  par le segment  [DK] ou K est le milieu du segment [EH].

Sur ce rail, il positionne une source lumineuse laser en un point N du segment [DK] et il oriente ce
second rayon laser vers le point S.

\begin{center}
\psset{unit=1cm,arrowsize=2pt 3}
\begin{pspicture}(7.2,5.3)
%\psgrid
\pspolygon(0.2,1.2)(4.5,0.2)(4.5,2.7)(0.2,3.7)%BCGF
\psline(4.5,0.2)(6.9,1.2)(6.9,3.7)(4.5,2.7)%CDHG
\psline(6.9,3.7)(2.6,4.7)(0.2,3.7)%HEF
\psline(0.2,2.45)(1.8,3.33)(1.4,4.2)%ART
\psline[linestyle=dashed,linewidth=1.5pt](0.2,2.45)(1.4,4.2)
\psline[linestyle=dotted,linewidth=1.5pt](0.2,1.2)(2.6,2.2)(2.6,4.7)%BOE
\psline[linestyle=dotted,linewidth=1.5pt](2.6,2.2)(6.9,1.2)%OD
\psline[linestyle=dotted,linewidth=1.5pt](4.75,4.2)(6.9,1.2)%KD
\psline{->}(2.6,2.2)(2.2,2.)%vect i
\psline{->}(2.6,2.2)(3.4,2.03)%vect j
\psline{->}(2.6,2.2)(2.6,2.82)%\vect k
\psline[ArrowInside=->](5.2,3.6)(2.7,1.5)(0.8,3.1)%NSL
\psdots(5.2,3.6)(2.7,1.5)(0.8,3.1)%NSL
\uput[l](0.2,2.45){A} \uput[dl](0.2,1.2){B} \uput[d](4.5,0.2){C} \uput[dr](6.9,1.2){D}
\uput[u](2.6,4.7){E} \uput[ul](0.2,3.7){F} \uput[dr](4.5,2.7){G} \uput[ur](6.9,3.7){H}
\uput[d](2.6,2.2){O} \uput[ur](1.8,3.33){R} \uput[d](2.7,1.5){S} \uput[u](1.4,4.2){T}
\uput[u](2.2,2){\small $\vect{\imath}$}\uput[u](3.4,2.03){\small $\vect{\jmath}$}\uput[u](2.6,2.82){\small $\vect{k}$}
\uput[ur](5.2,3.6){N}\uput[u](4.75,4.2){K}\uput[d](0.8,3.1){L}
\end{pspicture}
\end{center}

	\begin{enumerate}
		\item Montrer que, pour tout réel $t$ de l'intervalle [0~;~1], le point N de coordonnées $(0~;~8 - 4t~;~4t)$ est un point du segment [DK].
		\item  Calculer les coordonnées exactes du point N tel que les deux rayons laser représentés par les segments [SL] et [SN] soient perpendiculaires.\index{vecteurs orthogonaux}
	\end{enumerate}
\end{enumerate}

\bigskip

\textbf{\textsc{Exercice 4} \quad (7 points)\hfill Thème : fonction logarithme népérien, probabilités}

\bigskip

\emph{Cet exercice est un questionnaire à choix multiples (QCM)
 qui comprend six questions. Les six questions sont indépendantes. Pour chacune des questions, \textbf{une seule des quatre réponses est exacte}. Le candidat indiquera sur sa copie le numéro de la question suivi de la lettre correspondant à la réponse exacte. \\
 Aucune justification n'est demandée.\\
Une réponse fausse, une réponse multiple ou une absence de réponse ne rapporte ni n'enlève aucun point}\index{QCM}

\medskip

\textbf{Question 1}

\medskip

Le réel $a$ est définie par $a = \ln (9) + \ln \left(\dfrac{\sqrt{3}}{3} \right) + \ln \left(\dfrac19 \right)$ est égal à :\index{fonction logarithme}

\begin{center}
\begin{tabularx}{\linewidth}{*{4}{X}}
\textbf{a.~~}$1 - \dfrac12 \ln (3)$&\textbf{b.~~} $\dfrac12 \ln (3)$&\textbf{c.~~} $3 \ln (3) + \dfrac12$ &\textbf{d.~~} $- \dfrac12 \ln (3)$
\end{tabularx}
\end{center}
 
\medskip

\textbf{Question 2}

\medskip

On note $(E)$ l'équation suivante $\ln x + \ln (x - 10) = \ln 3 + \ln 7$ d'inconnue le réel $x$.\index{equation@équation}

\medskip
\textbf{a.~~}3 est solution de $(E)$.

\textbf{b.~~}$5 - \sqrt{46}$ est solution de $(E)$.

\textbf{c.~~}L'équation $(E)$ admet une unique solution réelle.

\textbf{d.~~}L'équation $(E)$ admet deux solutions réelles.

\medskip

\textbf{Question 3}

\medskip

La fonction $f$ est définie sur l'intervalle $]0~;~+ \infty[$ par l'expression $f(x) = x^2(- 1 + \ln x)$.

On note $\mathcal{C}_f$ sa courbe représentative dans le plan muni d'un repère.\index{fonction logarithme}

\textbf{a.~~} Pour tout réel $x$ de l'intervalle $]0~;~+ \infty[$, \: $f'(x) = 2x + \dfrac1x$.\index{dérivée}

\textbf{b.~~}La fonction $f$ est croissante sur l'intervalle $]0~;~+ \infty[$.

\textbf{c.~~}$f'\left(\sqrt{\text{e}} \right)$ est différent de $0$.

\textbf{d.~~}La droite d'équation $y = - \dfrac12 \text{e}$ est tangente à la courbe $\mathcal{C}_f$ au point d'abscisse $\sqrt{\text{e}}$.\index{equation de tangente@équation de tangente}

\medskip

\textbf{Question 4}

\medskip

Un sac contient 20 jetons jaunes et 30 jetons bleus. On tire successivement et avec remise 5 jetons du sac.

La probabilité de tirer exactement 2 jetons jaunes, arrondie au millième, est :\index{probabilités}

\begin{center}
\begin{tabularx}{\linewidth}{*{4}{X}}
\textbf{a.~~}0,683&\textbf{b.~~}0,346 &\textbf{c.~~}0,230&\textbf{d.~~} 0,165
\end{tabularx}
\end{center}

\medskip

\textbf{Question 5}

\medskip

Un sac contient 20 jetons jaunes et 30 jetons bleus. On tire successivement et avec remise 5 jetons du sac.

La probabilité de tirer au moins un jeton jaune, arrondie au millième, est :

\begin{center}
\begin{tabularx}{\linewidth}{*{4}{X}}
\textbf{a.~~}0,078&\textbf{b.~~}0,259 &\textbf{c.~~}0,337&\textbf{d.~~} 0,922
\end{tabularx}
\end{center}

\medskip

\textbf{Question 6}

\medskip

Un sac contient $20$ jetons jaunes et $30$ jetons bleus.

On réalise l'expérience aléatoire suivante : on tire successivement et avec remise cinq jetons du sac. 

On note le nombre de jetons jaunes obtenus après ces cinq tirages.

Si on répète cette expérience aléatoire un très grand nombre de fois alors, en moyenne, le nombre de jetons jaunes est égal à :

\begin{center}
\begin{tabularx}{\linewidth}{*{4}{X}}
\textbf{a.~~}0,4&\textbf{b.~~}1,2 &\textbf{c.~~}2&\textbf{d.~~} 2,5
\end{tabularx}
\end{center}
%%%%%%%%%  fin Amérique Nord1 19 mai 2022
\newpage
%%%%%%%%%%% Polynésie 30 août 2022
\phantomsection
\hypertarget{Polynesie3}{}

\label{Polynesie3}

\lfoot{\small{Polynésie}}
\rfoot{\small{30 août 2022}}
\pagestyle{fancy}
\thispagestyle{empty}

\begin{center}{\Large\textbf{\decofourleft~Baccalauréat Polynésie 30 août 2022~\decofourright\\[6pt] ÉPREUVE D'ENSEIGNEMENT DE SPÉCIALITÉ sujet \no 1}}

\bigskip

Durée de l'épreuve : \textbf{4 heures}

\medskip

L'usage de la calculatrice avec mode examen actif est autorisé

\medskip

Le sujet propose 4 exercices

Le candidat choisit 3 exercices parmi les 4 et \textbf{ne doit traiter que ces 3 exercices}
\end{center}

\bigskip

\textbf{\textsc{Exercice 1} \quad 7 points\hfill probabilités}

\medskip

Parmi les angines, un quart nécessite la prise d'antibiotiques, les autres non.

Afin d'éviter de prescrire inutilement des antibiotiques, les médecins disposent d'un test de diagnostic ayant les caractéristiques suivantes :

\begin{itemize}
\item[$\bullet~~$]lorsque l'angine nécessite la prise d'antibiotiques, le test est positif dans 90\,\% des cas;
\item[$\bullet~~$]lorsque l'angine ne nécessite pas la prise d'antibiotiques, le test est négatif dans 95\,\% des cas.
\end{itemize}

\medskip

Les probabilités demandées dans la suite de l'exercice seront arrondies à $10^{-4}$ près si nécessaire.

\bigskip

\textbf{Partie 1}

\medskip

Un patient atteint d'angine et ayant subi le test est choisi au hasard. 

On considère les évènements suivants :

\begin{itemize}
\item[$\bullet~~$]$A$ : \og le patient est atteint d'une angine nécessitant la prise d'antibiotiques \fg{} ;
\item[$\bullet~~$]$T$ : \og le test est positif \fg{} ;
\item[$\bullet~~$]$\overline{A}$ et $\overline{T}$ sont respectivement les évènements contraires de $A$ et $T$.
\end{itemize}

\medskip

\begin{enumerate}
\item Calculer $P(A \cap T)$. On pourra s'appuyer sur un arbre pondéré.\index{arbre pondéré}
\item Démontrer que $P(T) = \np{0,2625}$.\index{probabilités}
\item On choisit un patient ayant un test positif. Calculer la probabilité qu'il soit atteint d'une angine nécessitant la prise d'antibiotiques.
\item
	\begin{enumerate}
		\item Parmi les évènements suivants, déterminer ceux qui correspondent à un résultat erroné du test :  $A \cap T,\: \overline{A} \cap T,\:A \cap \overline{T},\: \overline{A} \cap \overline{T}$.
		\item On définit l'évènement $E$ : \og le test fournit un résultat erroné \fg.
		
Démontrer que $p(E) = \np{0,0625}$.
	\end{enumerate}
\end{enumerate}

\bigskip

\textbf{Partie 2}

\medskip

On sélectionne au hasard un échantillon de $n$ patients qui ont été testés.

On admet que l'on peut assimiler ce choix d'échantillon à un tirage avec remise.

On note $X$ la variable aléatoire qui donne le nombre de patients de cet échantillon ayant un test erroné.

\medskip

\begin{enumerate}
\item On suppose que $n = 50$.
	\begin{enumerate}
		\item Justifier que la variable aléatoire $X$ suit une loi binomiale $\mathcal{B}(n,\: p)$ de paramètres
		
$n = 50$ et $p = \np{0,0625}.$\index{loi binomiale}
		\item Calculer $P(X = 7)$.
		\item Calculer la probabilité qu'il y ait au moins un patient dans l'échantillon dont le test est erroné.
	\end{enumerate}	
\item Quelle valeur minimale de la taille de l'échantillon faut-il choisir pour que $P(X \geqslant 10)$ soit supérieure à $0,95$ ?\index{inéquation}
\end{enumerate}

\bigskip

\textbf{\textsc{Exercice 2} \quad 7 points\hfill suites, fonctions}

\medskip

Soit $k$ un nombre réel.

On considère la suite $\left(u_n\right)$ définie par son premier terme $u_0$ et pour tout entier naturel $n$,

\[u_{n+1} = ku_n\left(1- u_n\right).\]\index{suites}

Les deux parties de cet exercice sont indépendantes.

On y étudie deux cas de figure selon les valeurs de $k$.

\bigskip

\textbf{Partie 1}

\medskip

Dans cette partie, $k = 1,9$ et $u_0 = 0,1$.

On a donc, pour tout entier naturel $n$,\: $u_{n+1} = 1,9u_n\left(1 - u_n\right)$.

\medskip

\begin{enumerate}
\item On considère la fonction $f$ définie sur [0~;~1] par $f(x) = 1,9x(1 - x)$.\index{fonction polynôme}
	\begin{enumerate}
		\item Étudier les variations de $f$ sur l'intervalle [0~;~1].
		\item En déduire que si $x \in [0~;~1]$ alors $f(x) \in [0~;~1]$.
	\end{enumerate}	
\item Ci-dessous sont représentés les premiers termes de la suite $\left(u_n\right)$ construits à partir de la courbe $\mathcal{C}_f$ de la fonction $f$ et de la droite $D$ d'équation $y = x$.

Conjecturer le sens de variation de la suite $\left(u_n\right)$ et sa limite éventuelle.\index{lecture graphique}\index{limite de suite}

\medskip

%%début
\psset{unit=10cm,arrowsize=2pt 3,comma=true}
\def\xmin{-0.3} \def\xmax{1.05}
\def\ymin{-0.2} \def\ymax{0.65} 
\begin{pspicture*}(\xmin,\ymin)(\xmax,\ymax)

%%% grille 1cm x 1cm
\psgrid[unit=1cm,gridlabels=0,subgriddiv=1,gridcolor=gray](0,0)(1,1)
%%% axes
\psaxes[linewidth=1.5pt,Dx=0.1,Dy=0.1,labelFontSize=\scriptstyle]{->}(0,0)(0,0)(\xmax,\ymax)
%%% origine
\uput[dl](0,0){O}
%%% définition de la fonction
\def\f{1.9 x mul  1 x sub mul}
%%% tracé de la représentation graphique de la fonction
\psplot[plotpoints=2000,linewidth=1.25pt,linecolor=blue]{0}{1}{\f}
%%%% nom de la courbe
\uput[10](0.8,0.3){\blue $\mathscr{C}_f$}
%%% droite d'équation y = x
\psplot[plotpoints=1000,linecolor=red!80]{0}{1}{x}
\uput[d]{45}(0.65,0.65){\red $D : y=x$}
%%% \pstfixpoint[options]{abscisse début}{fonction}{nb itérations}
\psFixpoint[linecolor=red,ArrowInside=-<]{0.1}{\f}{3}
\psline[linestyle=dashed](0.1,0.171)(0,0.171)\uput[l](0,0.171){\small $u_1 = f\left(u_0\right)$}
\uput[d](0.1,-0.075){\small $u_0$}
\psline[linestyle=dashed](0.171,0.171)(0.171,0)\uput[d](0.171,-0.075){\small $u_1$}
\psline[linestyle=dashed](0.269,0.269)(0.269,0)\uput[d](0.269,-0.075){\small $u_2$}
\psline[linestyle=dashed](0.374,0.374)(0.374,0)\uput[d](0.374,-0.075){\small $u_3$}
\psline[linecolor=red](0.374,0.374)(0.374,0.445)
\end{pspicture*}
%%fin
\item 
	\begin{enumerate}
		\item En utilisant les résultats de la question 1, démontrer par récurrence que pour tout entier naturel $n$ :\index{récurrence}
		
\[0 \leqslant u_n \leqslant u_{n+1} \leqslant \dfrac12.\]

		\item En déduire que la suite $\left(u_n\right)$ converge.\index{suite convergente}
		\item Déterminer sa limite.\index{limite de suite}
	\end{enumerate}
\end{enumerate}

\bigskip

\textbf{Partie 2}

\medskip

Dans cette partie, $k= \dfrac12$  et $u_0 = \dfrac14$.

On a donc, pour tout entier naturel $n$,\: $u_{n+1} = \dfrac12 u_n \left(1 - u_n\right)$ et $u_0 = \dfrac14$.

On admet que pour tout entier naturel $n :\: 0 \leqslant u_n \leqslant \left(\dfrac12\right)^n$.

\medskip

\begin{enumerate}
\item Démontrer que la suite $\left(u_n\right)$ converge et déterminer sa limite.\index{limite de suite}
\item On considère la fonction Python \texttt{algo (p)} où \texttt{p} désigne un entier naturel non nul:\index{script python}

\begin{center}
\begin{tabular}{|l|}\hline
\texttt{def algo(p) :}\\
\quad  \texttt{u =1/4}\\
\quad \texttt{n = 0}\\
\quad \texttt{while u > 10**(-p):}\\
\quad \quad \texttt{u = 1/2*u*(1 - u)}\\
\quad \quad \texttt{n = n+1}\\
\quad \texttt{ return(n)}\\ \hline
\end{tabular}
\end{center}
Expliquer pourquoi, pour tout entier naturel non nul \texttt{p}, la boucle while ne tourne pas indéfiniment, ce qui permet à la commande \texttt{algo (p)} de renvoyer une valeur.
\end{enumerate}

\bigskip

\textbf{\textsc{Exercice 3} \quad 7 points\hfill fonctions}

\bigskip

\textbf{Partie 1}

\medskip

Soit $g$ la fonction définie pour tout nombre réel $x$ de l'intervalle $]0~;~+ \infty[$ par:

\[g(x) = \dfrac{2 \ln x}{ x}.\]\index{fonction logarithme}

\smallskip

\begin{enumerate}
\item On note $g'$ la dérivée de $g$. Démontrer que pour tout réel $x$ strictement positif : 

\[g'(x) = \dfrac{2 - 2\ln x}{x^2}.\]\index{dérivée}

\item On dispose de ce tableau de variations de la fonction $g$ sur l'intervalle $]0~;~+ \infty[$ : 

\begin{center}
\psset{unit=1cm}
\begin{pspicture}(9,3)
%\psgrid
\psframe(9,3)\psline(0,2.5)(9,2.5)
\psline(2.5,0)(2.5,3)
\uput[u](1.25,2.4){$x$}\uput[u](3,2.4){$0$}\uput[u](5.5,2.4){1}\uput[u](7,2.4){e}\uput[u](8.5,2.4){$+ \infty$}
\psline(2.95,0)(2.95,2.5)\psline(3.05,0)(3.05,2.5)
\uput[u](3.5,0){$- \infty$}\rput(5.5,1.25){0}\uput[d](7,2.5){$\frac{2}{\text{e}}$}\uput[u](8.8,0){0}
\rput(1.25,1.5){Variations }
\rput(1.25,1.){de $g$}
\psline(4,0.32)(5.3,1.1)\psline{->}(5.6,1.3)(6.8,2.)\psline{->}(7.2,2)(8.6,0.4)
\end{pspicture}
\end{center}\index{tableau de variations}

\smallskip

Justifier les informations suivantes lues dans ce tableau:

	\begin{enumerate}
		\item la valeur $\dfrac{2}{\text{e}}$ ;
		\item les variations de la fonction $g$ sur son ensemble de définition ;
		\item les limites de la fonction $g$ aux bornes de son ensemble de définition.
	\end{enumerate}
\item En déduire le tableau de signes de la fonction $g$ sur l'intervalle $]0~;~+ \infty[$.
\end{enumerate}

\bigskip

\textbf{Partie 2}

\medskip

Soit $f$ la fonction définie sur l'intervalle $]0~;~ + \infty[$ par 

\[f(x) = [\ln (x)]^2.\]\index{fonction logarithme}

 Dans cette partie, chaque étude est effectuée sur l'intervalle $]0~;~+ \infty[$.

\medskip

\begin{enumerate}
\item Démontrer que sur l'intervalle $]0~;~+ \infty[$, la fonction $f$ est une primitive de la fonction $g$.\index{primitive}
\item À l'aide de la partie 1, étudier :
	\begin{enumerate}
		\item la convexité de la fonction $f$ ;\index{convexité}
		\item les variations de la fonction $f$.
\end{enumerate}
\item 
	\begin{enumerate}
		\item Donner une équation de la tangente à la courbe représentative de la fonction $f$ au point d'abscisse e.\index{equation de la tangente@équation de tangente}
		\item En déduire que, pour tout réel $x$ dans $]0~;~\text{e}]$ :
		
\[[\ln (x)]^2 \geqslant \dfrac{2}{\text{e}}x - 1.\]

	\end{enumerate}
\end{enumerate}
\bigskip

\textbf{\textsc{Exercice 4} \quad 7 points\hfill géométrie dans le plan et dans l'espace}

\medskip

On considère le cube ABCDEFGH.

On note I le milieu du segment[EH] et on considère le triangle CFI.

L'espace est muni du repère orthonormé $\left(\text{A}~;~ \vect{\text{AB}},\: \vect{\text{AD}},\: \vect{\text{AE}}\right)$ et on admet que le point I a pour coordonnées $\left(0~;~\dfrac12;~1\right)$ dans ce repère.

\begin{center}
\psset{unit=1cm}
\begin{pspicture}(7.4,8)
\pspolygon(0.2,0.8)(4.6,0.2)(4.6,4.6)(0.2,5.2)%BCGF
\psline(4.6,0.2)(7.2,2.6)(7.2,7)(4.6,4.6)%CDHG
\psline(7.2,7)(2.8,7.6)(0.2,5.2)%HEF
\psline[linestyle=dashed](0.2,0.8)(2.8,3.2)(7.2,2.6)%BAD
\psline[linestyle=dashed](2.8,3.2)(2.8,7.6)%AE
\psline(5,7.3)(0.2,5.2)(4.6,0.2)%IFC
\psline[linestyle=dashed](5,7.3)(4.6,0.2)%IC
\psdots(2.8,3.2)(0.2,0.8)(4.6,0.2)(7.2,2.6)(2.8,7.6)(0.2,5.2)(4.6,4.6)(7.2,7)(5,7.3)(2.4,4.9)
\uput[l](2.8,3.2){A} \uput[l](0.2,0.8){B} \uput[dr](4.6,0.2){C} \uput[r](7.2,2.6){D}
\uput[u](2.8,7.6){E} \uput[l](0.2,5.2){F} \uput[dl](4.6,4.6){G} \uput[ur](7.2,7){H}
\uput[u](5,7.3){I} \uput[u](2.4,4.9){J}
\end{pspicture}
\end{center}

\smallskip

\begin{enumerate}
\item 
	\begin{enumerate}
		\item Donner sans justifier les coordonnées des points C, F et G.\index{lecture graphique}
		\item Démontrer que le vecteur $\vect{n}\begin{pmatrix}1\\2\\2\end{pmatrix}$ est normal au plan (CFI).\index{vecteur normal}
		\item Vérifier qu'une équation cartésienne du plan (CFI) est : $x + 2y + 2z - 3 = 0$.\index{equation de plan@équation de plan}
	\end{enumerate}	
\item  On note $d$ la droite passant par G et orthogonale au plan (CFI).
	\begin{enumerate}
		\item Déterminer une représentation paramétrique de la droite $d$.\index{equation paramétrique de droite@équation paramétrique de droite}
		\item Démontrer que le point K$\left(\dfrac79~;~\dfrac59~;~\dfrac59\right)$ est le projeté orthogonal du point G sur le plan (CFI).
		\item Déduire des questions précédentes que la distance du point G au plan (CFI) est
égale à $\dfrac23$.
	\end{enumerate}
\item On considère la pyramide GCFI.

\emph{On rappelle que le volume $V$ d'une pyramide est donné par la formule} 

\[V = \frac13 \times b \times h,\]\index{volume pyramide}

\emph{où $b$ est l'aire d'une base et $h$ la hauteur associée à cette base}.
	\begin{enumerate}
		\item Démontrer que le volume de la pyramide GCFI est égal à $\dfrac16$, exprimé en unité de volume.
		\item En déduire l'aire du triangle CFI, en unité d'aire.
	\end{enumerate}
\end{enumerate}

\newpage
%%%%%%%%%%% Métropole3 9 septembre 2022
\phantomsection
\hypertarget{Metropole3}{}

\label{Metropole3}

\rfoot{\small Métropole, Antilles-Guyane}
\lfoot{\small 8 septembre 2022}
\pagestyle{fancy}
\thispagestyle{empty}
\begin{center} {\Large \textbf{\decofourleft~Baccalauréat spécialité Jour 1 \decofourright\\[7pt]
Métropole Antilles-Guyane 8  septembre 2022}}
\end{center}

\bigskip

\textbf{Exercice 1  \quad 7 points\hfill Thèmes : fonctions, suites}

\medskip

\emph{Cet exercice est un questionnaire à choix multiples.\\
Pour chacune des questions suivantes, une seule des quatre réponses proposées est exacte. Une réponse fausse, une réponse multiple ou l'absence de réponse à une question ne rapporte ni n'enlève de point.\\
Pour répondre, indiquer sur la copie le numéro de la question et la lettre de la réponse choisie. Aucune justification n'est demandée.}\index{QCM}

\medskip

\begin{enumerate}
\item On considère la fonction $g$ définie sur $\R$ par :

\[ g(x) = \dfrac{2\text{e}^x}{\text{e}^x +1}.\]\index{fonction exponentielle}

\smallskip

La courbe représentative de la fonction $ g$ admet pour asymptote en $+ \infty$ la droite d'équation :\index{asymptote}

\begin{center}
\begin{tabularx}{\linewidth}{X X X X}
\textbf{a.~} $x = 2$ ;&\textbf{b.~} $y = 2$ ;&\textbf{c.~} $y = 0$ ;&\textbf{d.~} $x = - 1$\\
\end{tabularx}
\end{center}
\item ~

\begin{minipage}{0.38\linewidth}
On considère une fonction $f$ définie et deux fois dérivable sur $\R$.

On appelle $\mathcal{C}$ sa représentation graphique.

On désigne par $f''$ la dérivée seconde de $f$.

On a représenté sur le graphique ci-contre la courbe de $f''$, notée 
$\mathcal{C}''$.
\end{minipage}\hfill
\begin{minipage}{0.58\linewidth}
\begin{center}
\psset{unit=0.8cm,arrowsize=2pt 3}
\begin{pspicture*}(-7,-6.02)(3.02,4)
\psgrid[gridlabels=0pt,subgriddiv=1,gridwidth=0.15pt]
\psaxes[linewidth=1.25pt,Dx=10,Dy=10]{->}(0,0)(-7,-6)(3,4)
\multido{\n=-1+1}{4}{\uput[d](\n,0){\small \n}}
\uput[l](0,1){1}
\psplot[plotpoints=2000,linewidth=1.25pt,linecolor=red]{-7}{2.25}{x 1 add 2.71828 x exp mul x 2 sub mul}
\uput[ul](-5,0.25){\red $\mathcal{C}''$}
\end{pspicture*}
\end{center}
\end{minipage}

\begin{center}
\begin{tabularx}{\linewidth}{*{2}{X}}
\textbf{a.~} $\mathcal{C}$ admet un unique point d'inflexion ; &\textbf{b.~} $f$ est convexe sur l'intervalle $[- 1~;~2 ]$ ; \\
\textbf{c.~} $f$ est convexe sur $]- \infty~;~- 1$] et sur $[2~;~+ \infty[$ ;&\textbf{d.~} $f$ est convexe sur $\R$.\\
\end{tabularx}
\end{center}\index{point d'inflexion}\index{convexité}

\item On donne la suite $\left(u_n\right)$ définie par : $u_0 = 0$ et pour tout entier naturel $n,\: u_{n+1} = \dfrac12 u_n + 1$.\index{suites}

La suite $\left(v_n\right)$, définie pour tout entier naturel $n$ par $v_n = u_n  - 2$, est :

\begin{center}
\begin{tabularx}{\linewidth}{*{2}{X}}
\textbf{a.~} arithmétique de raison $- 2$ ;&\textbf{b.~} géométrique de raison $- 2$ ;\\
\textbf{c.~} arithmétique de raison 1 ;&\textbf{d.~}géométrique de raison  $\dfrac12$.\\
\end{tabularx}
\end{center}\index{suite géométrique}

\item  On considère une suite $\left(u_n\right)$ telle que, pour tout entier naturel, on a :

\[1  + \left(\dfrac14\right)^n \leqslant u_n \leqslant 2 - \dfrac{n}{n+1}\]

On peut affirmer que la suite $\left(u_n\right)$ :

\begin{center}
\begin{tabularx}{\linewidth}{*{2}{X}}
\textbf{a.~} converge vers 2 ; &\textbf{b.~} converge vers 1 ;\\
\textbf{c.~} diverge vers $+ \infty$ ;&\textbf{d.~} n'a pas de limite.\\
\end{tabularx}
\end{center}\index{suite convergente}

\item Soit $f$ la fonction définie sur $]0~;~+\infty[$ par $f(x) = x^2 \ln x$.\index{fonction logarithme}

Une primitive $F$ de $f$ sur $]0~;~+\infty[$ est définie par :\index{primitive}

\begin{center}
\begin{tabularx}{\linewidth}{*{2}{X}}
\textbf{a.~} $F(x) = \dfrac13 x^3 \left(\ln x - \dfrac13 \right)$ ;&\textbf{b.~} $F(x) = \dfrac13 x^3 (\ln x - 1)$ ;\\[8pt]
\textbf{c.~} $F(x) = \dfrac13 x^2$ ;&\textbf{d.~} $F(x) = \dfrac13 x^2 (\ln x - 1)$.\\
\end{tabularx}
\end{center}

\item Pour tout réel $x$, l'expression $2 + \dfrac{3\text{e}^{-x} - 5}{\text{e}^{-x} + 1}$ est égale à : 

\begin{center}
\begin{tabularx}{\linewidth}{*{2}{X}}
\textbf{a.~} $\dfrac{5 - 3\text{e}^x}{1 + \text{e}^x}$ ;&\textbf{b.~} $\dfrac{5 + 3\text{e}^x}{1 - \text{e}^x}$ ;\\[8pt]
\textbf{c.~} $\dfrac{5 + 3\text{e}^x}{1 + \text{e}^x}$ ;&\textbf{d.~} $\dfrac{5 - 3\text{e}^x}{1 - \text{e}^x}$.
\end{tabularx}
\end{center}
\end{enumerate}

\bigskip

\textbf{Exercice 2 \quad  7 points\hfill Thème probabilités}

\medskip

Un hôtel situé à proximité d'un site touristique dédié à la préhistoire propose deux visites dans les environs, celle d'un musée et celle d'une grotte.

\medskip

Une étude a montré que 70\,\% des clients de l'hôtel visitent le musée. De plus, parmi les clients visitant le musée, 60\,\% visitent la grotte.

Cette étude montre aussi que 6\,\% des clients de l'hôtel ne font aucune visite.

On interroge au hasard un client de l'hôtel et on note :

\begin{itemize}
\item[$\bullet~~$] $M$ l'évènement : \og le client visite le musée \fg{} ;
\item[$\bullet~~$] $G$ l'évènement : \og le client visite la grotte \fg.
\end{itemize}

On note $\overline{M}$ l'évènement contraire de $M$,\: $\overline{G}$ l'évènement contraire de $G$, et pour tout évènement $E$, on note $p(E)$ la probabilité de $E$.

Ainsi, d'après l'énoncé, on a : $p\left(\overline{M} \cap \overline{G}\right) = 0,06$.

\medskip

\begin{minipage}{0.53\linewidth}
\begin{enumerate}
\item 
	\begin{enumerate}
		\item Vérifier que $p_{\overline{M}}\left(\overline{G}\right) = 0,2$, où $p_{\overline{M}}\left(\overline{G}\right)$ désigne la probabilité que le client interrogé
ne visite pas la grotte sachant qu'il ne visite pas le musée.
		\item L'arbre pondéré ci-contre modélise la situation. Recopier et compléter cet arbre en indiquant sur chaque branche la probabilité associée.\index{arbre pondéré}
		\item Quelle est la probabilité de l'évènement \og le client visite la grotte et ne visite pas le musée \fg{} ?\index{probabilités}
		\item Montrer que $p(G) = 0,66$.
	\end{enumerate}
\end{enumerate}
\end{minipage}\hfill
\begin{minipage}{0.43\linewidth}
\begin{center}
\pstree[treemode=R,nodesepA=0pt,nodesepB=2.5pt,treesep = 1cm,levelsep=2.5cm]{\TR{}}
{\pstree{\TR{$M$~}\taput{\ldots}}
{\TR{$G$}\taput{\ldots}
\TR{$\overline{G}$}\tbput{\ldots}
}
\pstree{\TR{$\overline{M}$~}\tbput{\ldots}}
{\TR{$G$}\taput{\ldots}
\TR{$\overline{G}$}\tbput{\ldots}
}}
\end{center}
\end{minipage}

\begin{enumerate}[resume,start=2]
\item Le responsable de l'hôtel affirme que parmi les clients qui visitent la grotte, plus de la moitié visitent également le musée. Cette affirmation est-elle exacte ?
\item Les tarifs pour les visites sont les suivants:

\begin{itemize}
\item[$\bullet~~$] visite du musée : 12 euros;
\item[$\bullet~~$] visite de la grotte : 5 euros.
\end{itemize}

On considère la variable aléatoire $T$ qui modélise la somme dépensée par un client de l'hôtel pour ces visites.
	\begin{enumerate}
		\item Donner la loi de probabilité de $T$. On présentera les résultats sous la forme d'un tableau.\index{loi de probabilité}
		\item Calculer l'espérance mathématique de $T$.\index{espérance}
		\item Pour des questions de rentabilité, le responsable de l'hôtel estime que le montant
moyen des recettes des visites doit être supérieur à $700$~euros par jour.

Déterminer le nombre moyen de clients par journée permettant d'atteindre cet objectif.
	\end{enumerate}
\item Pour augmenter les recettes, le responsable souhaite que l'espérance de la variable aléatoire modélisant la somme dépensée par un client de l'hôtel pour ces visites passe à $15$~euros, sans modifier le prix de visite du musée qui demeure à $12$ euros. 

Quel prix faut-il fixer pour la visite de la grotte afin d'atteindre cet objectif ? (On admettra que l'augmentation du prix d'entrée de la grotte ne modifie pas la fréquentation des deux sites).
\item On choisit au hasard $100$ clients de l'hôtel, en assimilant ce choix à un tirage avec remise.

Quelle est la probabilité qu'au moins les trois quarts de ces clients aient visité la grotte à l'occasion de leur séjour à l'hôtel ? 

On donnera une valeur du résultat à $10^{-3}$ près.
\end{enumerate}

\newpage

\textbf{Exercice 3 \quad  7 points\hfill Thème fonctions logarithmes et exponentielles, suites}

\bigskip

\textbf{Partie A}

\medskip

On considère la fonction $f$ définie sur l'intervalle $[1~;~+\infty[$ par 

\[f(x) = \dfrac{\ln x}{x},\]\index{fonction logarithme}

où $\ln$ désigne la fonction logarithme népérien.

%\medskip

\begin{enumerate}
\item Donner la limite de la fonction $f$ en $+\infty$.\index{limite de fonction}
\item On admet que la fonction $f$ est dérivable sur l'intervalle $[1~;~+\infty[$ et on note $f'$ sa fonction dérivée.
	\begin{enumerate}
		\item Montrer que, pour tout nombre réel $x \geqslant 1,\:f'(x) = \frac{1 - \ln x}{x^2}$.\index{dérivée}
		\item Justifier le tableau de signes suivant, donnant le signe de $f'(x)$ suivant les valeurs de $x$.
\begin{center}
\psset{unit=1cm}
\begin{pspicture}(6,1)
\psframe(6,1)
\psline(0,0.5)(6,0.5)\psline(2,0)(2,1)
\uput[u](1,0.4){$x$}\uput[u](2.1,0.4){$1$}\uput[u](4,0.4){e}\uput[u](5.5,0.4){$+ \infty$}
\rput(1,0.25){$f'(x)$}\rput(3,0.25){$+$}\rput(4,0.25){0} \rput(5,0.25){$-$}
\end{pspicture}
\end{center}
		\item Dresser le tableau de variations complet de la fonction $f$.\index{tableau de variations}
	\end{enumerate}
\item Soit $k$ un nombre réel positif ou nul.
	\begin{enumerate}
		\item Montrer que, si $0 \leqslant k \leqslant \dfrac{1}{\text{e}}$, l'équation $f(x) = k$ admet une unique solution sur l'intervalle $[1~;~\text{e}]$.\index{valeurs intermédiaires}
		\item Si $k > \dfrac{1}{\text{e}}$, l'équation $f(x) = k$ admet-elle des solutions sur l'intervalle $[1~;~+\infty[$ ?

Justifier.
	\end{enumerate}
\end{enumerate}

\bigskip

\textbf{Partie B}

\medskip

Soit $g$ la fonction définie sur $\R$ par :

\[g(x) = \text{e}^{\frac{x}{4}}.\]\index{fonction exponentielle}

On considère la suite $\left(u_n\right)$ définie par $u_0 = 1$ et, pour tout entier naturel $n$ :

$u_{n+1} = \text{e}^{\frac{u_n}{4}}$ c'est-à-dire : $u_{n+1} = g\left(u_n\right)$.

\medskip

\begin{enumerate}
\item Justifier que la fonction $g$ est croissante sur $\R$.
\item Montrer par récurrence que, pour tout entier naturel $n$, on a : $u_n \leqslant  u_{n+1} \leqslant \text{e}$.\index{récurrence}
\item En déduire que la suite $\left(u_n\right)$ est convergente.\index{suite convergente}
\end{enumerate}

On note $\ell$ la limite de la suite $\left(u_n\right)$ et on admet que $f$ est solution de l'équation : 
\[\text{e}^{\frac{x}{4}} = x.\]

\begin{enumerate}[resume]
\item En déduire que $\ell$ est solution de l'équation $f(x) = \dfrac14$, où $f$ est la fonction étudiée dans
la partie A.
\item Donner une valeur approchée à $10^{-2}$ près de la limite $\ell$ de la suite $\left(u_n\right)$.
\end{enumerate}

\bigskip

\textbf{Exercice 4 \quad  7 points\hfill Thème : géométrie dans l'espace}

\medskip

Dans l'espace rapporté à un repère orthonormé \Oijk, on considère les points

\[\text{A}(-1~;~-1~;~3),\qquad \text{B}(1~;~1~;~2),\qquad  \text{C}(1~;~-1~;~7)\]

On considère également la droite $\Delta$ passant par les points D$(-1~;~6~;~8)$ et E$(11~;~- 9~;~2)$.

\medskip

\begin{enumerate}
\item 
	\begin{enumerate}
		\item Vérifier que la droite $\Delta$ admet pour représentation paramétrique:\index{equation paramétrique de droite@équation paramétrique de droite}
		
\[\left\{\begin{array}{l c l}
x &=& -1 + 4t\\
y &=& \phantom{-}6 - 5t \quad\text{avec}\: t \in \R\\
z &=& \phantom{-}8 - 2t
\end{array}\right.\]

		\item Préciser une représentation paramétrique de la droite $\Delta'$ parallèle à $\Delta$ et passant par l'origine O du repère.
		\item Le point F$(1,36~;~-1,7~;~-0,7)$ appartient-il à la droite $\Delta'$ ?
	\end{enumerate}	
\item
	\begin{enumerate}
		\item Montrer que les points A, B et C définissent un plan.
		\item Montrer que la droite $\Delta$ est perpendiculaire au plan (ABC).\index{vecteur normal}
		\item Montrer qu'une équation cartésienne du plan (ABC) est: $4x - 5y - 2z + 5 = 0$.\index{equation de plan@équation de plan}
	\end{enumerate}
\item
	\begin{enumerate}
		\item Montrer que le point G(7~;~-4~;~4) appartient à la droite $\Delta$.
		\item Déterminer les coordonnées du point H, projeté orthogonal du point G sur le plan (ABC).
		\item En déduire que la distance du point G au plan (ABC) est égale à $3\sqrt 5$.
	\end{enumerate}
\item
	\begin{enumerate}
		\item Montrer que le triangle ABC est rectangle en A.\index{triangle rectangle}
		\item Calculer le volume $V$ du tétraèdre ABCG.\index{volume tétraèdre}
		
\emph{On rappelle que le volume $V$ d'un tétraèdre est donné par la formule $V = \dfrac13 \times  B \times h$ où $B$ est l'aire d'une base et $h$ la hauteur correspondant à cette base.}
	\end{enumerate}
\end{enumerate}

\newpage
%%%%%%%%%%% Métropole4 9 septembre 2022
\phantomsection
\hypertarget{Metropole4}{}

\label{Metropole4}

\rfoot{\small Métropole, Antilles-Guyane}
\lfoot{\small 9 septembre 2022}
\pagestyle{fancy}
\thispagestyle{empty}
\begin{center} {\Large \textbf{\decofourleft~Baccalauréat spécialité Jour 2 \decofourright\\[7pt]
Métropole Antilles-Guyane 9 septembre 2022}}
\end{center}

\bigskip

\textbf{Exercice 1 \quad 7 points\hfill Thèmes : probabilités}

\medskip

Dans le magasin d'Hugo, les clients peuvent louer deux types de vélos : vélos de route ou bien vélos tout terrain. 

Chaque type de vélo peut être loué dans sa version électrique ou non.

On choisit un client du magasin au hasard, et on admet que:
\begin{itemize}
\item Si le client loue un vélo de route, la probabilité que ce soit un vélo électrique est de $0,4$ ;
\item Si le client loue un vélo tout terrain, la probabilité que ce soit un vélo électrique est de $0,7$ ; 
\item La probabilité que le client loue un vélo électrique est de $0,58$.
\end{itemize}

\medskip

On appelle $\alpha$ la probabilité que le client loue un vélo de route, avec $0 \leqslant \alpha \leqslant 1$.

On considère les évènements suivants:

\begin{itemize}
\item[$\bullet~~$] $R$ : \og le client loue un vélo de route \fg{} ;
\item[$\bullet~~$] $E$ : \og le client loue un vélo électrique \fg{} ;
\item[$\bullet~~$] $\overline{R}$ et $\overline{E}$ , évènements contraires de $R$ et $E$.
\end{itemize}

\medskip

\begin{minipage}{0.48\linewidth}

On modélise cette situation aléatoire à l'aide de l'arbre reproduit ci-contre:\index{arbre pondéré}

Si $F$ désigne un évènement quelconque, on notera $p(F)$ la probabilité de $F$.
\end{minipage} \hfill
\begin{minipage}{0.48\linewidth}
\begin{center}
\pstree[treemode=R,nodesepA=0pt,nodesepB=2.5pt,treesep = 1cm,levelsep=2.5cm]{\TR{}}
{\pstree{\TR{$R$~}\taput{$\alpha$}}
{\TR{$E$}\taput{\ldots}
\TR{$\overline{E}$}\tbput{\ldots}
}
\pstree{\TR{$\overline{R}$~}\tbput{$1 - \alpha$}}
{\TR{$E$}\taput{\ldots}
\TR{$\overline{E}$}\tbput{\ldots}
}}
\end{center}
\end{minipage} 

\medskip

\begin{enumerate}
\item Recopier cet arbre sur la copie et le compléter.
\item 
	\begin{enumerate}
		\item Montrer que $p(E) = 0,7 - 0,3\alpha$.\index{probabilités}
		\item En déduire que : $\alpha = 0,4$.
	\end{enumerate}
\item On sait que le client a loué un vélo électrique.

Déterminer la probabilité qu'il ait loué un vélo tout terrain. On donnera le résultat arrondi au centième.
\item Quelle est la probabilité que le client loue un vélo tout terrain électrique ?
\item Le prix de la location à la journée d'un vélo de route non électrique est de $25$ euros, celui d'un vélo tout terrain non électrique de $35$~euros.

Pour chaque type de vélo, le choix de la version électrique augmente le prix de location à la journée de $15$~euros.

On appelle $X$ la variable aléatoire modélisant le prix de location d'un vélo à la journée.
	\begin{enumerate}
		\item Donner la loi de probabilité de $X$. On présentera les résultats sous forme d'un tableau.
		\item Calculer l'espérance mathématique de $X$ et interpréter ce résultat.\index{espérance}
	\end{enumerate}	
\item Lorsqu'on choisit $30$ clients d'Hugo au hasard, on assimile ce choix à un tirage avec remise.

On note $Y$ la variable aléatoire associant à un échantillon de $30$ clients choisis au hasard le nombre de clients qui louent un vélo électrique.

On rappelle que la probabilité de l'événement $E$ est : $p(E) = 0,58$.
	\begin{enumerate}
		\item Justifier que $Y$ suit une loi binomiale dont on précisera les paramètres.\index{loi binomiale}
		\item Déterminer la probabilité qu'un échantillon contienne exactement $20$ clients qui louent un vélo électrique. On donnera le résultat arrondi au millième.
		\item Déterminer la probabilité qu'un échantillon contienne au moins $15$ clients qui louent un vélo électrique. On donnera le résultat arrondi au millième.
	\end{enumerate}
\end{enumerate}

\bigskip

\textbf{Exercice 2 \quad 7 points\hfill Thèmes : suites, fonctions}

\medskip

\emph{Cet exercice est un questionnaire à choix multiples. Pour chacune des questions suivantes, une seule des quatre réponses proposées est exacte.\\
Une réponse fausse, une réponse multiple ou l'absence de réponse à une question ne rapporte ni n'enlève de point.\\
Pour répondre, indiquer sur la copie le numéro de la question et la lettre de la réponse choisie. Aucune justification n'est demandée.}\index{QCM}

\medskip

\begin{enumerate}
\item On considère les suites $\left(a_n\right)$ et $\left(b_n\right)$ définie par $a_0 = 1$ et, pour tout entier naturel $n$,\: $a_{n+1} = 0,5a_n + 1$ et $b_n = a_n - 2$.\index{suites}

On peut affirmer que:

\begin{center}
\begin{tabularx}{\linewidth}{*{2}{X}}
\textbf{a.~} $\left(a_n\right)$ est arithmétique ;&\textbf{b.~} $\left(b_n\right)$ est géométrique;\\
\textbf{c.~} $\left(a_n\right)$ est géométrique;&\textbf{d.~} $\left(b_n\right)$ est arithmétique. 
\end{tabularx}
\end{center}\index{suite arithmétique }\index{suite géométrique}

Dans les questions 2. et 3., on considère les suites $\left(u_n\right)$ et $\left(v_n\right)$ définies par :

\[u_0 = 2,\: v_0 = 1 \:\: \text{et, pour tout entier naturel }\:\:n : \left\{\begin{array}{l c l}
u_{n+1}&=&u_n + 3v_n\\
v_{n+1}&=&u_n + v_n.
\end{array}\right.\]

\item On peut affirmer que :

\begin{center}
\begin{tabularx}{\linewidth}{*{4}{X}}
\textbf{a.~}$\left\{\begin{array}{l c l}
u_2& =& 5\\ v_2 &=& 3
\end{array}\right.$&\textbf{b.~} $u_2^2 - 3v_2^2 = - 2^2$&\textbf{c.~} $\dfrac{u_2}{v_2} = 1,75$&\textbf{d.~} $5u_1 = 3v_1$.\\
\end{tabularx}
\end{center}

\item On considère le programme ci-dessous écrit en langage Python :\index{script python}

\begin{center}
\begin{tabular}{|>{\texttt}l|}\hline
\texttt{def valeurs() :}\\
\quad \texttt{u = 2}\\
\quad \texttt{v = 1}\\
\quad \texttt{for k in range(1,11)}\\
\quad \quad \texttt{c = u}\\
\quad \quad \texttt{u = u + 3*v}\\
\quad \quad \texttt{v = c + v}\\
\quad \texttt{return (u, v)}\\ \hline
\end{tabular}
\end{center}

Ce programme renvoie :

\begin{center}
\begin{tabularx}{\linewidth}{*{4}{X}}
\textbf{a.~} $u_{11}$ et $v_{11}$ ;&
\textbf{b.~} $u_{10}$ et $v_{11}$ ;&
\textbf{c.~} les valeurs de $u_n$ et $v_n$ pour $n$ allant de 1 à 10;&
\textbf{d.~} $u_{0 }$ et $v_{10}$.
\end{tabularx}
\end{center}
\end{enumerate}

Pour les questions 4. et 5., on considère une fonction $f$ deux fois dérivable sur l'intervalle
$[-4~;~2]$. On note $f'$ la fonction dérivée de $f$ et $f''$  la dérivée seconde de $f$.

On donne ci-dessous la courbe représentative $\mathcal{C}'$ de la fonction dérivée $f'$ dans un repère du plan. On donne de plus les points A$(-2~;~0)$, B(1~;~0) et C(0~;~5).

\begin{minipage}{0.52\linewidth}
\begin{enumerate}[resume]
\item La fonction $f$ est :

\begin{center}
\begin{tabularx}{\linewidth}{*{2}{X}}
\textbf{a.~}concave sur $[-2~;~1]$ ; 	&\textbf{b.~}convexe sur $[-4~;~0]$;\\
\textbf{c.~}convexe sur $[-2~;~1]$;		&\textbf{d.~}convexe sur [0~;~2].
\end{tabularx}
\end{center}\index{convexité}

\item On admet que la droite (BC) est la tangente à la courbe $\mathcal{C}'$ au point B.
On a :\index{lecture graphique}

\begin{center}
\begin{tabularx}{\linewidth}{*{2}{X}}
\textbf{a.~} $f'(1) < 0$ ;	&\textbf{b.~} $f'(1) = 5$ ;\\
\textbf{c.~} $f''(1) > 0$;	&\textbf{d.~} $f''(1) = - 5$.
\end{tabularx}
\end{center}
\end{enumerate}
\end{minipage}\hfill
\begin{minipage}{0.45\linewidth}
\begin{center}
\psset{unit=1cm,arrowsize=2pt 3}
\begin{pspicture*}(-4,-3)(2,6)
\psaxes[linewidth=1.25pt,Dx=10,Dy=10]{->}(0,0)(-4,-3)(2,6)
\psaxes[linewidth=1.25pt](0,0)(1,1)
\psplot[plotpoints=2000,linewidth=1.25pt,linecolor=blue]{-4}{2}{x 2 add 1 x  sub mul 2.71828 x 0.5 mul exp mul}\uput[dr](-4,-1.5){\blue $\mathcal{C}'$}
\psline(-4,0.1)(-4,-0.1)
\uput[dr](-2,0){A} \uput[ur](1,0){B} \uput[l](0,5){C} 
\psplot[plotpoints=2000,linewidth=1.25pt]{-4}{2}{5 5 x  mul sub}
\uput[d](-4,0){$-4$}
\end{pspicture*}
\end{center}

\end{minipage}

\begin{enumerate}[resume,start=6]
\item Soit $f$ la fonction définie sur $\R$ par $f(x) = \left(x^2 + 1\right)\text{e}^x$.

La primitive $F$ de $f$ sur $\R$  telle que $F(0) = 1$ est définie par:\index{primitive}

\begin{center}
\begin{tabularx}{\linewidth}{*{2}{X}}
\textbf{a.~} $F(x) = \left(x^2 - 2x +3\right)\text{e}^x$ ;&\textbf{b.~} $F(x) = \left(x^2 - 2x + 3\right)\text{e}^x - 2;$\\
\textbf{c.~} $F(x) = \left(\dfrac13 x^3 + x\right)\text{e}^x + 1$ ;&\textbf{d.~} $F(x) = \left(\dfrac13 x^3 + x \right) \text{e}^x$.\\
\end{tabularx}
\end{center}
\end{enumerate}

\bigskip

\textbf{Exercice 3 \quad 7 points\hfill Thèmes : fonction logarithme, suites}

\medskip

\textbf{Les parties B et C sont indépendantes}

\medskip

On considère la fonction $f$  définie sur $]0~;~+\infty[$ par 

\[f(x) = x - x \ln x,\]\index{fonction logarithme}

où ln désigne la fonction logarithme népérien.

\bigskip

\textbf{Partie A}

\medskip

\begin{enumerate}
\item Déterminer la limite de $f(x)$ quand $x$ tend vers $0$.\index{limite de fonction}
\item Déterminer la limite de $f(x)$ quand $x$ tend vers $+\infty$.
\item On admet que la fonction $f$ est dérivable sur $]0~;~+\infty[$ et on note $f'$ sa fonction dérivée.
	\begin{enumerate}
		\item Démontrer que, pour tout réel $x > 0$, on a : $f'(x) = - \ln x$.\index{dérivée}
		\item En déduire les variations de la fonction $f$ sur $]0~;~+\infty[$ et dresser son tableau de variations.\index{tableau de variations}
	\end{enumerate}
\item Résoudre l'équation $f(x) = x$ sur $]0~;~+\infty[$.\index{equation@équation}
\end{enumerate}

\bigskip

\textbf{Partie B}

\medskip

Dans cette partie, on pourra utiliser avec profit certains résultats de la partie A. 

On considère la suite $\left(u_n\right)$ définie par:\index{suites}

\[\left\{\begin{array}{l c l}
u_0 &=& 0,5\\
u_{n+1} &=& u_n - u_n \ln u_n \:\text{ pour tout entier naturel } n,
\end{array}\right.\]

Ainsi, pour tout entier naturel $n$, on a : $u_{n+1} = f\left(u_n\right)$.

\medskip

\begin{enumerate}
\item On rappelle que la fonction $f$ est croissante sur l'intervalle [0,5~;~1].

Démontrer par récurrence que, pour tout entier naturel $n$, on a : $0,5 \leqslant  u_n \leqslant u_{n+1} \leqslant 1$.\index{récurrence}
\item 
	\begin{enumerate}
		\item Montrer que la suite $\left(u_n\right)$ est convergente.\index{suite convergente}
		\item On note $\ell$ la limite de la suite $\left(u_n\right)$. Déterminer la valeur de $\ell$.\index{limite de suite}
	\end{enumerate}
\end{enumerate}

\bigskip

\textbf{Partie C}

\medskip

Pour un nombre réel $k$ quelconque, on considère la fonction $f_k$ définie sur $]0~;~+\infty[$ par:

\[f_k(x) = kx - x \ln x.\]\index{fonction logarithme}

\begin{enumerate}
\item Pour tout nombre réel $k$, montrer que $f_k$ admet un maximum $y_k$ atteint en $x_k = \text{e}^{k- 1}$.\index{dérivée}
\item Vérifier que, pour tout nombre réel $k$, on a : $x_k = y_k$.
\end{enumerate}

\bigskip

\textbf{Exercice 4 \quad 7 points\hfill Thèmes : géométrie dans l'espace }

\medskip

Dans l'espace rapporté à un repère orthonormé \Oijk, on considère:

\begin{itemize}
\item[$\bullet~~$] la droite $\mathcal{D}$ passant par le point A(2~;~4~;~0) et dont un vecteur directeur est $\vect{u}\begin{pmatrix}1\\2\\0\end{pmatrix}$ ;
\item[$\bullet~~$] la droite $\mathcal{D}'$ dont une représentation paramétrique est : $\left\{\begin{array}{l c l}
x&=&3\\
y&=& 3 + t\\
z&=&3 + t
\end{array}\right. ,\: t \in \R$.
\end{itemize}

\medskip

\begin{enumerate}
\item 

	\begin{enumerate}
		\item Donner les coordonnées d'un vecteur directeur $\vect{u'}$ de la droite $\mathcal{D}'$.
		\item Montrer que les droites $\mathcal{D}$ et $\mathcal{D}'$ ne sont pas parallèles.
		\item Déterminer une représentation paramétrique de la droite $\mathcal{D}$.
	\end{enumerate}
\end{enumerate}

On admet dans la suite de cet exercice qu'il existe une unique droite $\Delta$ perpendiculaire aux droites $\mathcal{D}$ et $\mathcal{D}'$. Cette droite $\Delta$ coupe chacune des droites $\mathcal{D}$ et $\mathcal{D}'$. 

On appellera M le point d'intersection de $\Delta$ et $\mathcal{D}$, et M$'$ le point d'intersection de $\Delta$ et $\mathcal{D}'$.

On se propose de déterminer la distance MM$'$ appelée \og distance entre les droites $\mathcal{D}$ et $\mathcal{D}'$ \fg.
\begin{enumerate}[resume]
\item Montrer que le vecteur $\vect{v}\begin{pmatrix}2\\- 1\\1\end{pmatrix}$ est un vecteur directeur de la droite $\Delta$.\index{vecteur directeur}
\item On note $\mathcal{P}$ le plan contenant les droites $\mathcal{D}$ et $\Delta$, c'est-à-dire le plan passant par le point A et de vecteurs directeurs $\vect{u}$ et $\vect{v}$.
	\begin{enumerate}
		\item Montrer que le vecteur $\vect{n}\begin{pmatrix}2\\-1\\-5 \end{pmatrix}$ est un vecteur normal au plan $\mathcal{P}$.\index{vecteur normal}
		\item En déduire qu'une équation du plan $\mathcal{P}$ est : $2x - y - 5z = 0$.\index{equation de plan@équation de plan}
		\item On rappelle que M$'$ est le point d'intersection des droites $\Delta$ et $\mathcal{D}'$.
		
Justifier que M$'$ est également le point d'intersection de $\mathcal{D}'$ et du plan $\mathcal{P}$.
		
En déduire que les coordonnées du point M$'$ sont (3~;~1~;~1).
	\end{enumerate}
\item 
	\begin{enumerate}
		\item Déterminer une représentation paramétrique de la droite $\Delta$.\index{equation paramétrique de droite@équation paramétrique de droite}
		\item Justifier que le point M a pour coordonnées (1~;~2~;~0).
		\item Calculer la distance MM$'$.
	\end{enumerate}
\item On considère la droite $d$ de représentation paramétrique $\left\{\begin{array}{l c l}
x &=& 5t\\y &=& 2 + 5t \\z&=&1 + t\end{array}\right.$ avec $t \in \R$.
	\begin{enumerate}
		\item Montrer que la droite $d$ est parallèle au plan $\mathcal{P}$.
		\item On note $\ell$ la distance d'un point N de la droite $d$ au plan $\mathcal{P}$ . 
		
Exprimer le volume du  tétraèdre ANMM$'$ en fonction de $\ell$.\index{volume tétraèdre}

On rappelle que le volume d'un tétraèdre est donné par : $V = \dfrac13 \times B \times h$ où $B$ désigne l'aire d'une base et $h$ la hauteur relative à cette base.
		\item Justifier que, si N$_1$ et N$_2$ sont deux points quelconques de la droite $d$, les tétraèdres AN$_1$MM$'$ et AN$_2$MM$'$ ont le même volume.
	\end{enumerate}
\end{enumerate}
\newpage
%%%%%%%%%%% AmeriSud1 26 septembre 2022
\phantomsection
\hypertarget{AmeriSud1}{}

\label{AmeriSud1}

\lfoot{\small{Amérique du Sud}}
\rfoot{\small{26 septembre 2022}}
\pagestyle{fancy}
\thispagestyle{empty}

\begin{center}{\Large\textbf{\decofourleft~Baccalauréat Amérique du Sud 26 septembre 2022~\decofourright\\[6pt] ÉPREUVE D'ENSEIGNEMENT DE SPÉCIALITÉ Jour 1}}
\end{center}

\vspace{0,25cm}

Le sujet propose 4 exercices

Le candidat choisit 3 exercices parmi les 4 exercices et \textbf{ne doit traiter que ces 3 exercices}

\medskip

Chaque exercice est noté sur 7 points (le total sera ramené sur 20 points).

Les traces de recherche, même incomplètes ou infructueuses, seront prises en compte.

\bigskip

\textbf{\textsc{Exercice 1 Probabilités} \hfill 7 points}

\medskip

\textbf{PARTIE A}

\medskip

Le système d'alarme d'une entreprise fonctionne de telle sorte que, si un danger se présente, l'alarme s'active avec une probabilité de $0,97$. 

La probabilité qu'un danger se présente est de $0,01$ et la probabilité que l'alarme s'active est de \np{0,01465}.

On note $A$ l'évènement \og  l'alarme s'active \fg{}  et $D$ l'événement \og  un danger se présente \fg . 

On note $\overline{M}$ l'évènement contraire d'un évènement $M$ et $P(M)$ la probabilité de l'évènement $M$.

\medskip

\begin{enumerate}
\item Représenter la situation par un arbre pondéré qui sera complété au fur et à mesure de l'exercice.\index{arbre pondéré}
\item  
	\begin{enumerate}
		\item Calculer la probabilité qu'un danger se présente et que l'alarme s'active.
		\item En déduire la probabilité qu'un danger se présente sachant que l'alarme s'active.
Arrondir le résultat à $10^{-3}$.
	\end{enumerate}
\item  Montrer que la probabilité que l'alarme s'active sachant qu'aucun danger ne s'est présenté est $0,005$.
\item On considère qu'une alarme ne fonctionne pas normalement lorsqu'un danger se présente et qu'elle ne s'active pas ou bien lorsqu'aucun danger ne se présente et qu'elle s'active.

Montrer que la probabilité que l'alarme ne fonctionne pas normalement est inférieure à $0,01$.
\end{enumerate}

\medskip

\textbf{PARTIE B}

\medskip

Une usine fabrique en grande quantité des systèmes d'alarme. On prélève successivement et au hasard $5$ systèmes d'alarme dans la production de l'usine. Ce prélèvement est assimilé à un tirage avec remise.

On note $S$ l'évènement \og  l'alarme ne fonctionne pas normalement \fg{}  et on admet que

$P(S) = \np{0,00525}$.

On considère $X$ la variable aléatoire qui donne le nombre de systèmes d'alarme ne fonctionnant pas normalement parmi les $5$ systèmes d'alarme prélevés.

Les résultats seront arrondis à $10^{-4}$.

\medskip

\begin{enumerate}
\item Donner la loi de probabilité suivie par la variable aléatoire $X$ et préciser ses paramètres.\index{loi binomiale}
\item Calculer la probabilité que, dans le lot prélevé, un seul système d'alarme ne fonctionne pas normalement.
\item Calculer la probabilité que, dans le lot prélevé, au moins un système d'alarme ne fonctionne pas normalement.
\end{enumerate}

\medskip

\textbf{PARTIE C}

\medskip

Soit $n$ un entier naturel non nul. On prélève successivement et au hasard $n$ systèmes d'alarme. Ce prélèvement est assimilé à un tirage avec remise.

Déterminer le plus petit entier $n$ tel que la probabilité d'avoir, dans le lot prélevé, au moins un système d'alarme qui ne fonctionne pas normalement soit supérieure à $0,07$.

\bigskip

\textbf{\textsc{Exercice 2 Suites} \hfill 7 points}

\medskip

Soit $\left(u_n\right)$ la suite définie par $u_0 = 4$ et, pour tout entier naturel $n$,\: $u_{n+1} = \dfrac15 u_n^2$.\index{suites}

\medskip

\begin{enumerate}
\item 
	\begin{enumerate}
		\item Calculer $u_1$ et $u_2$.
		\item Recopier et compléter la fonction ci-dessous écrite en langage Python.
Cette fonction est nommée \emph{suite\_u} et prend pour paramètre l'entier naturel $p$.

Elle renvoie la valeur du terme de rang $p$ de la suite $\left(u_n\right)$.\index{script python}

\begin{center}
\begin{tabular}{l}
\texttt{def suite\_u(p) :}\\
	\quad \texttt{u= \ldots}\\
	\quad \texttt{for i in range(1,\ldots) :}\\
	\quad \quad \texttt{u =\ldots}\\
	\quad \texttt{return u}\\
	\end{tabular}
\end{center}

	\end{enumerate}
\item 
	\begin{enumerate}
		\item Démontrer par récurrence que pour tout entier naturel $n$,\: $0 < u_n \leqslant 4$.\index{récurrence}
		\item Démontrer que la suite $\left(u_n\right)$ est décroissante.
		\item En déduire que la suite $\left(u_n\right)$ est convergente.\index{suite convergente}
	\end{enumerate}
\item 
	\begin{enumerate}
		\item Justifier que la limite $\ell$ de la suite $\left(u_n\right)$ vérifie l'égalité $\ell = \dfrac15 \ell^2$.
		\item En déduire la valeur de $\ell$.\index{limite de suite}
	\end{enumerate}
\item Pour tout entier naturel $n$, on pose $v_n = \ln \left(u_n\right)$ et $w_n = v_n - \ln (5)$.\index{fonction logarithme}
	\begin{enumerate}
		\item Montrer que, pour tout entier naturel $n$,\: $v_{n+1} = 2v_n - \ln (5)$.
		\item Montrer que la suite $\left(w_n\right)$ est géométrique de raison 2.\index{suite géométrique}
		\item Pour tout entier naturel $n$, donner l'expression de $w_n$ en fonction de $n$ et montrer que $v_n = \ln \left(\dfrac45 \right) \times 2^n + \ln (5)$.
	\end{enumerate}
\item Calculer $\displaystyle\lim_{n \to + \infty} v_n$ et retrouver $\displaystyle\lim_{n \to + \infty} u_n$.\index{limite de suite}
\end{enumerate}

\bigskip

\textbf{\textsc{Exercice 3 Fonctions, fonction logarithme} \hfill 7 points}

\medskip

Soit $g$ la fonction définie sur l'intervalle $]0~;~+\infty[$ par 

\[g(x) = 1+ x^2[1 - 2 \ln (x)].\]\index{fonction logarithme}

La fonction $g$ est dérivable sur l'intervalle $]0~;~+\infty[$ et on note $g'$ sa fonction dérivée.

On appelle $\mathcal{C}$ la courbe représentative de la fonction $g$ dans un repère orthonormé du plan.

\medskip

\textbf{PARTIE A}

\medskip

\begin{enumerate}
\item Justifier que $g(\text{e})$ est strictement négatif.
\item Justifier que $\displaystyle\lim_{x \to + \infty} g(x) = - \infty$.\index{limite de fonction}
\item 
	\begin{enumerate}
		\item Montrer que, pour tout $x$ appartenant à l'intervalle $]0~;~+\infty[$,\: $g'(x) = -4x \ln (x)$.\index{dérivée}
		\item Étudier le sens de variation de la fonction $g$ sur l'intervalle $]0~;~+\infty[$ 
		\item Montrer que l'équation $g(x) = 0$ admet une unique solution, notée $\alpha$, sur l'intervalle $[1~;~+\infty[$.\index{valeurs intermédiaires}
		\item Donner un encadrement de $\alpha$ d'amplitude $10^{-2}$
	\end{enumerate}
\item Déduire de ce qui précède le signe de la fonction $g$ sur l'intervalle $[1~;~+\infty[$.
\end{enumerate}

\medskip

\textbf{PARTIE B}

\medskip

\begin{enumerate}
\item On admet que, pour tout $x$ appartenant à l'intervalle $[1~;~\alpha]$, $g''(x) = - 4[\ln (x) + 1]$.

Justifier que la fonction $g$ est concave sur l'intervalle $[1~;~\alpha]$.\index{convexité}
\end{enumerate}

\begin{minipage}{0.48\linewidth}
\begin{enumerate}[resume]
\item Sur la figure ci-contre, A et B sont les points de la
courbe $\mathcal{C}$ d'abscisses respectives 1 et $\alpha$.
	\begin{enumerate}
		\item Déterminer l'équation réduite de la droite (AB).
		\item En déduire que pour tout réel $x$ appartenant à
l'intervalle $[1~;~\alpha]$,\: $g(x) \geqslant \dfrac{- 2}{\alpha - 1} x + \dfrac{2\alpha}{\alpha - 1}$.
	\end{enumerate}
\end{enumerate}
\end{minipage} \hfill
\begin{minipage}{0.48\linewidth}
\begin{center}
\psset{unit=1.3cm,arrowsize=2pt 3}
\begin{pspicture}(-0.2,-0.2)(2.5,3)
\psgrid[gridlabels=0pt,subgriddiv=1,gridwidth=0.15pt]
\psaxes[linewidth=1.25pt,labelFontSize=\scriptstyle]{->}(0,0)(0,0)(2.5,3)
\psplot[plotpoints=1000,linewidth=1.25pt,linecolor=red]{1}{1.9}{1 x ln 2 mul sub x dup mul mul 1 add}
\uput[ul](1,2){\red A}\uput[ur](1.9,0){\red B}\uput[d](1.9,0.05){$\alpha$}
\uput[ur](1.5,1.5){\red $\mathcal{C}$}
\end{pspicture}
\end{center}
\end{minipage}

\bigskip

\textbf{\textsc{Exercice 4 Géométrie dans l'espace} \hfill 7 points}

\medskip

Dans la figure ci-dessous, ABCDEFGH est un parallélépipède rectangle tel que 

AB $= 5$, AD $= 3$ et AE $= 2$.

L'espace est muni d'un repère orthonormé d'origine A dans lequel les points B, D et E ont respectivement pour coordonnées (5~;~0~;~0), (0~;~3~;~0) et (0~;~0~;~2).

\begin{center}
\psset{unit=1cm}
\begin{pspicture}(9,5)
%\psgrid
\psframe(0.2,0.2)(7.2,3)%ABFE
\psline(7.2,0.2)(8.4,1.6)(8.4,4.4)(7.2,3)%BCGF
\psline(8.4,4.4)(1.6,4.4)(0.2,3)%GHE
\psline[linestyle=dashed](0.2,0.2)(1.6,1.6)(8.4,1.6)%ADC
\psline[linestyle=dashed](1.6,1.6)(1.6,4.4)%DH
\pspolygon[linestyle=dotted,linewidth=1.5pt](0.2,0.2)(8.4,1.6)(3,4.4)%ACM
\uput[dl](0.2,0.2){A} \uput[dr](7,0.2){B} \uput[r](8.4,1.6){C}
\uput[ur](1.6,1.6){D} \uput[l](0.2,3){E} \uput[ul](7.2,3){F}
\uput[ur](8.4,4.4){G} \uput[ul](1.6,4.4){H} \uput[u](3,4.4){M}
\end{pspicture}
\end{center}

\begin{enumerate}
\item 
	\begin{enumerate}
		\item Donner, dans le repère considéré, les coordonnées des points H et G.\index{lecture graphique}
		\item Donner une représentation paramétrique de la droite (GH).\index{equation paramétrique de droite@équation paramétrique de droite}
	\end{enumerate}	
\item Soit M un point du segment [GH] tel que $\vect{\text{HM}} =k\vect{\text{HG}}$ avec $k$ un nombre réel de l'intervalle [0~;~1].\index{vecteurs colinéaires}
	\begin{enumerate}
		\item Justifier que les coordonnées de M sont $(5k~;~3~;~2)$.
		\item En déduire que $\vect{\text{AM}} \cdot \vect{\text{CM}} = 25k^2  - 25k + 4$.\index{produit scalaire}
		\item Déterminer les valeurs de $k$ pour lesquelles AMC est un triangle rectangle en
M.\index{triangle rectangle}
	\end{enumerate}
\end{enumerate}	

Dans toute la suite de l'exercice, on considère que le point M a pour coordonnées (1~;~3~;~2).

On admet que le triangle AMC est rectangle en M .

On rappelle que le volume d'un tétraèdre est donné par la formule $\dfrac13 \times\text{Aire de la base}  \times h$ où $h$ est la hauteur relative à la base.

\begin{enumerate}[resume]
\item On considère le point K de coordonnées (1~;~3~;~0).
	\begin{enumerate}
		\item Déterminer une équation cartésienne du plan (ACD).\index{equation de plan@équation de plan}
		\item Justifier que le point K est le projeté orthogonal du point M sur le plan (ACD).
		\item En déduire le volume du tétraèdre MACD.\index{volume tétraèdre}
	\end{enumerate}	
\item On note P le projeté orthogonal du point D sur le plan (AMC).

Calculer la distance DP ; en donner une valeur arrondie à $10^{-1}$.
\end{enumerate}
\newpage
%%%%%%%%%%% AmeriSud1 27 septembre 2022
\phantomsection
\hypertarget{AmeriSud2}{}

\label{AmeriSud2}

\lfoot{\small{Amérique du Sud}}
\rfoot{\small{27 septembre 2022 Jour 2}}
\pagestyle{fancy}
\thispagestyle{empty}

\begin{center}{\Large\textbf{\decofourleft~Baccalauréat Amérique du Sud 27 septembre 2022~\decofourright\\[6pt] ÉPREUVE D'ENSEIGNEMENT DE SPÉCIALITÉ Jour 2}}
\end{center}

\vspace{0,25cm}

Le sujet propose 4 exercices

Le candidat choisit 3 exercices parmi les 4 exercices et \textbf{ne doit traiter que ces 3 exercices}

\medskip

Chaque exercice est noté sur 7 points (le total sera ramené sur 20 points).

Les traces de recherche, même incomplètes ou infructueuses, seront prises en compte.

\bigskip

\textbf{\textsc{Exercice 1 Probabilités} \hfill 7 points}

\medskip

Une entreprise fabrique des composants pour l'industrie automobile. Ces composants sont conçus sur trois chaînes de montage numérotées de 1 à 3.
\begin{itemize}
\item[$\bullet~~$] La moitié des composants est conçue sur la chaîne \no 1 ;
\item[$\bullet~~$]  30\,\% des composants sont conçus sur la chaîne \no 2;
\item[$\bullet~~$]  les composants restant sont conçus sur la chaîne \no 3.
\end{itemize}

À l'issue du processus de fabrication, il apparaît que 1\,\% des pièces issues de la chaîne \no 1 présentent un défaut, de même que 0,5\,\% des pièces issues de la chaîne \no 2 et 4\,\% des pièces issues de la chaîne \no 3.

On prélève au hasard un de ces composants. On note :

\begin{itemize}
\item[$\bullet~~$] $C_1$ l'évènement \og le composant provient de la chaîne \no 1 \fg{} ;
\item[$\bullet~~$] $C_2$ l'évènement \og le composant provient de la chaîne \no 2 \fg{} ;
\item[$\bullet~~$] $C_3$ l'évènement \og le composant provient de la chaîne \no 3 \fg{} ;
\item[$\bullet~~$] $D$ l'évènement \og le composant est défectueux\fg et $\overline{D}$ son évènement contraire.
\end{itemize}

\emph{Dans tout l'exercice, les calculs de probabilité seront donnés en valeur décimale exacte ou arrondie à $10^{-4}$ si nécessaire}.

\medskip

\textbf{PARTIE A}

\medskip

\begin{enumerate}
\item Représenter cette situation par un arbre pondéré.\index{arbre pondéré}
\item Calculer la probabilité que le composant prélevé provienne de la chaîne \no 3 et soit
défectueux.\index{probabilités}
\item Montrer que la probabilité de l'évènement $D$ est $P(D) = \np{0,0145}$.
\item Calculer la probabilité qu'un composant défectueux provienne de la chaîne \no 3.
\end{enumerate}

\medskip

\textbf{PARTIE B}

\medskip

L'entreprise décide de conditionner les composants produits en constituant des lots de $n$ unités. On note $X$ la variable aléatoire qui, à chaque lot de $n$ unités, associe le nombre de composants défectueux de ce lot.

Compte tenu des modes de production et de conditionnement de l'entreprise, on peut considérer que $X$ suit la loi binomiale de paramètres $n$ et $p = \np{0,0145}$.

\medskip

\begin{enumerate}
\item Dans cette question, les lots possèdent $20$ unités. On pose $n = 20$.
	\begin{enumerate}
		\item Calculer la probabilité pour qu'un lot possède exactement trois composants défectueux.
		\item Calculer la probabilité pour qu'un lot ne possède aucun composant défectueux.
		
En déduire la probabilité qu'un lot possède au moins un composant défectueux.
	\end{enumerate}
\item  Le directeur de l'entreprise souhaite que la probabilité de n'avoir aucun composant
défectueux dans un lot de $n$ composants soit supérieure à $0,85$.

Il propose de former des lots de 11 composants au maximum. A-t-il raison ? Justifier la réponse.
\end{enumerate}

\medskip

\textbf{PARTIE C}

\medskip

Les coûts de fabrication des composants de cette entreprise sont de $15$ euros s'ils proviennent de la chaîne de montage \no 1, 12 euros s'ils proviennent de la chaîne de montage \no 2 et $9$ euros s'ils proviennent de la chaîne de montage \no 3.

Calculer le coût moyen de fabrication d'un composant pour cette entreprise.

\bigskip

\textbf{\textsc{Exercice 2 Fonctions, fonction logarithme} \hfill 7 points}

\medskip

Le but de cet exercice est d'étudier la fonction $f$, définie sur $]0~;~ +\infty[$ ,par:

\[f(x) = 3x - x \ln (x) - 2 \ln (x).\]\index{fonction logarithme}

\smallskip

\textbf{PARTIE A: Étude d'une fonction auxiliaire }\boldmath $g$ \unboldmath

\medskip

Soit $g$ la fonction définie sur $]0~;~+\infty[$ par 

\[g(x) = 2(x - 1) - x \ln (x).\]\index{fonction logarithme}

On note $g'$ la fonction dérivée de $g$. On admet que $\displaystyle\lim_{x \to + \infty} g(x) = - \infty$.

\medskip

\begin{enumerate}
\item Calculer $g(1)$ et $g(\text{e})$.
\item Déterminer $\displaystyle\lim_{x \to + 0} g(x)$ en justifiant votre démarche.\index{limite de fonction}
\item Montrer que, pour tout $x > 0$,\: $g'(x) = 1 - \ln (x)$.\index{dérivée}

En déduire le tableau des variations de $g$ sur $]0~;~ +\infty[$.\index{tableau de variations}
\item Montrer que l'équation $g(x) = 0$ admet exactement deux solutions distinctes sur
$]0~;~ +\infty[$ : 1 et $\alpha$ avec $\alpha$ appartenant à l'intervalle $[\text{e}~;~+\infty[$.

On donnera un encadrement de $\alpha$ à 0,01 près.\index{valeurs intermédiaires}
\item En déduire le tableau de signes de $g$ sur $]0~;~ +\infty[$.
\end{enumerate}

\bigskip

\textbf{PARTIE B : Étude de la fonction} \boldmath $f$ \unboldmath

\medskip

On considère dans cette partie la fonction $f$, définie sur $]0~;~ +\infty[$,par 

\[f(x) = 3x - x \ln (x)- 2\ln (x).\]\index{fonction logarithme}

On note $f'$ la fonction dérivée de $f$.

La représentation graphique $\mathcal{C}_f$ de cette fonction $f$ est donnée dans le repère \Oij{} ci-dessous. On admet que : $\displaystyle\lim_{x \to 0}f(x) = + \infty$.

\begin{center}
\psset{unit=0.82cm,arrowsize=2pt 3}
\begin{pspicture*}(-1,-2.01)(16,6)
\psgrid[gridlabels=0pt,subgriddiv=1,gridwidth=0.15pt](0,-2)(16,6)
\psaxes[linewidth=1.25pt,labelFontSize=\scriptstyle]{->}(0,0)(0,-1.95)(16,5.95)
\psaxes[linewidth=1.25pt,labelFontSize=\scriptstyle](0,0)(0,-1.95)(16,5.95)
\psplot[plotpoints=2000,linewidth=1.25pt,linecolor=blue]{0.01}{16}{x 3 mul x ln x 2 add mul sub}
\uput[ur](10.5,2){\blue $\mathcal{C}_f$}
\uput[u](15.75,0){$x$}
\uput[l](0,5.75){$y$}
\end{pspicture*}
\end{center}

\begin{enumerate}
\item Déterminer la limite de $f$ en $+ \infty$ en justifiant votre démarche.\index{limite de fonction}
\item 
	\begin{enumerate}
		\item Justifier que pour tout $x > 0$,\: $f'(x) = \dfrac{g(x)}{x}$.\index{dérivée}
		\item En déduire le tableau des variations de $f$ sur $]0~;~ +\infty[$.\index{tableau de variations}
	\end{enumerate}	
\item On admet que, pour tout $x > 0$, la dérivée seconde de $f$, notée $f''$, est définie par $f''(x) = \dfrac{2 - x}{x^2}$.

Étudier la convexité de $f$ et préciser les coordonnées du point
 d'inflexion de $\mathcal{C}_f$.\index{convexité}\index{point d'inflexion}
\end{enumerate}

\bigskip

\textbf{\textsc{Exercice 3 Suites} \hfill 7 points}

\medskip

La population d'une espèce en voie de disparition est surveillée de près dans une réserve naturelle.

Les conditions climatiques ainsi que le braconnage font que cette population diminue de 10\,\% chaque année.

Afin de compenser ces pertes, on réintroduit dans la réserve $100$ individus à la fin de chaque année.

On souhaite étudier l'évolution de l'effectif de cette population au cours du temps. Pour cela, on modélise l'effectif de la population de l'espèce par la suite $\left(u_n\right)$ où $u_n$ représente l'effectif de la population au début de l'année $2020 + n$.\index{suites}

On admet que pour tout entier naturel $n$,\: $u_n \geqslant 0$.

Au début de l'année 2020, la population étudiée compte \np{2000} individus, ainsi $u_0 = \np{2000}$.

\medskip

\begin{enumerate}
\item Justifier que la suite $\left(u_n\right)$ vérifie la relation de récurrence:

\[u_{n+1} = 0,9u_n + 100.\]\index{récurrence}

\item Calculer $u_1$ puis $u_2$.
\item Démontrer par récurrence que pour tout entier naturel $n$ :\: $\np{1000} < u_{n+1}  \leqslant u_n$.\index{récurrence}
\item La suite $\left(u_n\right)$ est-elle convergente ? Justifier la réponse.\index{suite convergente}
\item On considère la suite $\left(v_n\right)$ définie pour tout entier naturel $n$ par $v_n = u_n - \np{1000}$.
	\begin{enumerate}
		\item Montrer que la suite $\left(v_n\right)$ est géométrique de raison $0,9$.\index{suite géométrique}
		\item En déduire que, pour tout entier naturel $n$,\: $u_n = \np{1000} \left(1 + 0,9^n\right)$.
		\item Déterminer la limite de la suite $\left(u_n\right)$.\index{limite de suite}
		
En donner une interprétation dans le contexte de cet exercice.
	\end{enumerate}

\item On souhaite déterminer le nombre d'années nécessaires pour que l'effectif de la population passe en dessous d'un certain seuil $S$ (avec $S > \np{1000}$).
	\begin{enumerate}
		\item Déterminer le plus petit entier $n$ tel que $u_n \leqslant \np{1020}$.

Justifier la réponse par un calcul.
	\end{enumerate}
\begin{minipage}{0.58\linewidth}
	\begin{enumerate}[resume]
		\item Dans le programme Python ci-contre, la variable $n$ désigne le nombre d'années écoulées depuis 2020, la variable $u$ désigne l'effectif de la population. 
		
Recopier et compléter ce programme afin qu'il retourne le nombre d'années nécessaires pour que l'effectif de la population passe en dessous du seuil $S$.
	\end{enumerate}\index{script python}

\end{minipage}\hfill
\begin{minipage}{0.38\linewidth}
\begin{tabular}{|c l|}\hline
1&\texttt{def population(S) :}\\
2&\quad \texttt{n=0}\\
3&\quad \texttt{u=2000}\\
4& \\
5&\quad \texttt{while \ldots\ldots} :\\
6&\quad \quad \texttt{u= \ldots}\\
7&\quad \quad \texttt{n = \ldots}\\
8&\quad \texttt{return \ldots}\\ \hline
\end{tabular}
\end{minipage}
\end{enumerate}

\bigskip

\textbf{\textsc{Exercice 4 Géométrie dans l'espace} \hfill 7 points}

\bigskip

\begin{minipage}{0.41\linewidth}
Dans l'espace muni d'un repère orthonormé \Oijk, on considère les points 

\[\text{A}(0~;~8~;~6),\quad  \text{B}(6~;~4~;~4 )\quad \text{et C}(2~;~4~;~0).\]

\end{minipage}\hfill
\begin{minipage}{0.55\linewidth}
\psset{unit=1cm,arrowsize=2pt 3}
\begin{pspicture}(-1.8,-1.8)(5.4,3.8)
\pspolygon[fillstyle=solid,fillcolor=lightgray](1.7,-0.6)(4.8,3)(0.4,0.4)
\psaxes[linewidth=1.25pt,Dx=20,Dy=20]{->}(0,0)(0,0)(5.4,3.8)
\psline[linewidth=1.25pt]{->}(0,0)(-1.8,-1.8)
\multido{\n=0+-0.3}{7}{\psdots[dotstyle=+,dotangle=45](\n,\n)}
\multido{\n=0+0.6}{9}{\psline(\n,-0.05)(\n,0.05)}
\multido{\n=0+0.5}{8}{\psline(-0.05,\n)(0.05,\n)}
\psline[linestyle=dashed](4.8,0)(4.8,3)(0,3)
\psframe[linestyle=dashed](-1.8,-1.8)(0.4,0.4)
\psframe[linestyle=dashed](2.4,2)
\psline[linestyle=dashed](-1.8,0.4)(0,2)
\psline[linestyle=dashed](0.4,-1.8)(2.4,0)
\psline[linestyle=dashed](0.4,0.4)(2.4,2)
\psline[linestyle=dashed](-0.6,-0.6)(1.8,-0.6)
\uput[d](0,0){\small O} \uput[ul](-0.15,-0.2){\footnotesize $\vect{\imath}$}
\uput[d](0.3,0){\footnotesize $\vect{\jmath}$}
\uput[l](0,0.3){\footnotesize $\vect{k}$}
\uput[l](-1.8,-1.8){$x$}\uput[d](5.3,0){$y$}\uput[l](0,3.7){$z$}

\uput[ur](4.8,3){A} \uput[ul](0.4,0.4){B} \uput[dr](1.7,-0.6){C}
\end{pspicture}
\end{minipage}

\bigskip

\begin{enumerate}
\item 
	\begin{enumerate}
		\item Justifier que les points A, B et C ne sont pas alignés.
		\item Montrer que le vecteur $\vect{n}(1~;~2~;~-1)$ est
un vecteur normal au plan (ABC).\index{vecteur normal}
		\item Déterminer une équation cartésienne du plan (ABC).\index{equation de plan@équation de plan}
	\end{enumerate}
\item Soient D et E les points de coordonnées respectives (0~;~0~;~6) et (6~;~6~;~0).
	\begin{enumerate}
		\item Déterminer une représentation paramétrique de la droite (DE).\index{equation paramétrique de droite@équation paramétrique de droite}
		\item Montrer que le milieu I du segment [BC] appartient à la droite (DE).
	\end{enumerate}	
\item On considère le triangle ABC.
	\begin{enumerate}
		\item Déterminer la nature du triangle ABC.
		\item Calculer l'aire du triangle ABC en unité d'aire.
		\item Calculer $\vect{\text{AB}} \cdot \vect{\text{AC}}$.\index{produit scalaire}
		\item En déduire une mesure de l'angle $\widehat{\text{BAC}}$ arrondie à $0,1$ degré.\index{mesure d'angle}
	\end{enumerate}	
\item On considère le point H de coordonnées $\left(\dfrac53~;~\dfrac{10}{3}~;~- \dfrac53\right)$.

Montrer que H est le projeté orthogonal du point O sur le plan (ABC). 

En déduire la distance du point O au plan (ABC).
\end{enumerate}
\newpage
%%%%%%%%%%% Nouvelle-Calédonie 1 26 octobre 2022
\phantomsection
\hypertarget{NCaledo1}{}

\label{NCaledo1}

\lfoot{\small{Nouvelle-Calédonie Jour 1}}
\rfoot{\small{26 octobre 2022}}
\pagestyle{fancy}
\thispagestyle{empty}

\begin{center}{\Large\textbf{\decofourleft~Baccalauréat Nouvelle-Calédonie 26 octobre 2022 Jour 1~\decofourright\\[7pt] ÉPREUVE D'ENSEIGNEMENT DE SPÉCIALITÉ }}

\bigskip

Durée de l'épreuve : \textbf{4 heures}

\medskip

L'usage de la calculatrice avec mode examen actif est autorisé

\medskip

Le sujet propose 4 exercices

Le candidat choisit 3 exercices parmi les 4 et \textbf{ne doit traiter que ces 3 exercices}
\end{center}

\bigskip

\textbf{\textsc{Exercice 1} \quad 7 points\hfill}

\medskip

\textbf{Principaux domaines abordés :} fonctions, fonction logarithme; convexité.

\medskip

On considère la fonction $f$ définie sur l'intervalle $]0~;~+\infty[$ par

\[f(x) = x^2 - 6x + 4\ln (x).\]\index{fonction logarithme}

On admet que la fonction $f$ est deux fois dérivable sur l'intervalle $]0~;~+ \infty[$.

On note $f$' sa dérivée et $f''$ sa dérivée seconde.

On note $\mathcal{C}_f$ la courbe représentative de la fonction $f$ dans un repère orthogonal.

\medskip

\begin{enumerate}
\item 
	\begin{enumerate}
		\item Déterminer $\displaystyle\lim_{x \to 0} f(x)$.\index{limite de fonction}
	
Interpréter graphiquement ce résultat.
		\item Déterminer $\displaystyle\lim_{x \to + \infty} f(x)$.\index{limite de fonction}
	\end{enumerate}	
\item
	\begin{enumerate}
		\item Déterminer $f'(x)$ pour tout réel $x$ appartenant à $]0~;~+ \infty[$.\index{dérivée}

		\item Étudier le signe de $f'(x)$ sur l'intervalle $]0~;~+ \infty[$.
		
En déduire le tableau de variations de $f$.\index{tableau de variations}
	\end{enumerate}
\item Montrer que l'équation $f(x) = 0$ admet une unique solution dans l'intervalle [4~;~5].\index{valeurs intermédiaires}
\item On admet que, pour tout $x$ de $]0~;~+ \infty[$, on a :

\[f''(x) = \dfrac{2x^2 - 4}{x^2}.\]

	\begin{enumerate}
		\item Étudier la convexité de la fonction $f$ sur $]0~;~+ \infty[$.\index{convexité}
		
On précisera les valeurs exactes des coordonnées des éventuels points d'inflexion de $\mathcal{C}_f$.\index{point d'inflexion}
		\item On note A le point de coordonnées $\left(\sqrt 2~;~f\left(\sqrt 2~\right)\right)$.
		
Soit $t$ un réel strictement positif tel que $t \ne \sqrt 2$. Soit $M$ le point de coordonnées $(t~;~ f(t))$.

En utilisant la question 4. a, indiquer, selon la valeur de $t$, les positions relatives du segment [A$M$] et de la courbe $\mathcal{C}_f$.
	\end{enumerate}
\end{enumerate}

\bigskip

\textbf{\textsc{Exercice 2} \quad 7 points\hfill}

\medskip

\textbf{Principaux domaines abordés :}
suites ;
fonctions, fonction exponentielle.

\medskip

On considère la fonction $f$ définie sur $\R$ par

\[f(x) = x^3\e^x.\]\index{fonction exponentielle}

On admet que la fonction $f$ est dérivable sur $\R$ et on note $f'$ sa fonction dérivée.

\medskip

\begin{enumerate}
\item On définit la suite $\left(u_n\right)$ par $u_0 = - 1$ et, pour tout entier naturel $n,\: u_{n+1} = f\left(u_n\right)$.\index{suites}
	\begin{enumerate}
		\item Calculer $u_1$ puis $u_2$.
		
On donnera les valeurs exactes, puis les valeurs approchées à $10^{-3}$.
		\item On considère la fonction \texttt{fonc}, écrite en langage Python ci-dessous.

\smallskip

\begin{minipage}{0.524\linewidth}
On rappelle qu'en langage Python,\index{script python}

\og \texttt{i in range (n)}\fg{} signifie que 

\texttt{i} varie de 0 à \texttt{n -1}.
\end{minipage}\hfill
\begin{minipage}{0.40\linewidth}
\begin{tabular}{|l l|}\hline
\texttt{def}& \texttt{fonc (n)} :\\
&\texttt{u =- 1}\\
&\texttt{for i in range(n) :}\\
&\quad \texttt{u=u**3*exp(u)}\\
&\texttt{return u}\\\hline
\end{tabular}
\end{minipage}

Déterminer, sans justifier, la valeur renvoyée par \texttt{fonc (2)} arrondie à $10^{-3}$.
	\end{enumerate}
\item 
	\begin{enumerate}
		\item Démontrer que, pour tout $x$ réel, on a $f'(x) = x^2\e^x(x + 3)$.\index{dérivée}
		\item Justifier que le tableau de variations de $f$ sur $\R$ est celui représenté
ci-dessous :\index{tableau de variations}

\begin{center}
\psset{unit=1cm,arrowsize=2pt 3}
\begin{pspicture}(7,2.5)
\psframe(7,2.5)
\psline(0,2)(7,2)\psline(1,0)(1,2.5)
\uput[u](0.5,1.9){$x$} \uput[u](1.2,1.9){$- \infty$} \uput[u](4,1.9){$-3$} \uput[u](6.5,1.9){$+ \infty$} 
\rput(0.5,1){$f$}\uput[d](1.2,2){0}\uput[u](4,0){$- 27\e^{-3}$}\uput[d](6.5,2){$+ \infty$}
\psline{->}(1.5,1.5)(3.5,0.5)\psline{->}(4.5,0.52)(6.5,1.5)
\end{pspicture}
\end{center}

		\item Démontrer, par récurrence, que pour tout entier naturel $n$, on a :
		
\[- 1 \leqslant u_n \leqslant u_{n+1} \leqslant 0.\]\index{récurrence}
		
		\item En déduire que la suite $\left(u_n\right)$ est convergente.
		\item On note $\ell$ la limite de la suite $\left(u_n\right)$.

On rappelle que $\ell$ est solution de l'équation $f(x) = x$.

Déterminer $\ell$. (Pour cela, on admettra que l'équation $x^2\e^x - 1 = 0$ possède une seule solution dans $\R$ et que celle-ci est strictement supérieure à $\dfrac12$).\index{limite de suite}
	\end{enumerate}
\end{enumerate}

\bigskip

\textbf{\textsc{Exercice 3} \quad 7 points\hfill }

\medskip

\textbf{Principaux domaines abordés :} géométrie dans l'espace.

\medskip

Une maison est constituée d'un parallélépipède rectangle ABCDEFGH surmonté d'un prisme EFIHGJ dont une base est le triangle EIF isocèle en I.

Cette maison est représentée ci-dessous.

\begin{center}
\psset{unit=1cm,arrowsize=2pt 3}
\begin{pspicture}(12.2,6)
\pspolygon(0.4,3.4)(0.4,1.5)(1.9,0.4)(1.9,2.3)%HDAE
\psline(1.9,0.4)(7.9,0.8)(7.9,2.7)(1.9,2.3)%ABFE
\psline[linestyle=dashed](0.4,1.5)(6.4,1.9)(6.4,3.8)(0.4,3.4)%DCGH
\psline[linestyle=dashed](7.9,0.8)(6.4,1.9)%BC
\pspolygon(7.9,2.7)(6.4,3.8)(3.4,5.5)(4.9,4.4)%FGJI
\psline(4.9,4.4)(1.9,2.3)%IE
\psline(3.4,5.5)(0.4,3.4)%JH
\psline[linewidth=1.5pt]{->}(1.9,0.4)(3.9,0.533)
\psline[linewidth=1.5pt]{->}(1.9,0.4)(1.15,0.95)
\psline[linewidth=1.5pt]{->}(1.9,0.4)(1.9,2.3)
\psline(11.6,0.3)(7.9,1.2)
\psline[linestyle=dashed](7.9,1.2)(7.15,1.35)
\uput[d](1.9,0.4){A} \uput[dr](7.9,0.8){B} \uput[r](6.4,1.9){C} \uput[l](0.4,1.5){D}
\uput[u](1.9,2.3){E} \uput[r](7.9,2.7){F} \uput[ur](6.4,3.8){G} \uput[ul](0.4,3.4){H}
\uput[l](4.9,4.4){I} \uput[u](3.4,5.5){J} \uput[r](11.6,0.3){R} \uput[dl](1.7,0.6){$\vect{\jmath}$}
\uput[dr](2.9,0.45){$\vect{\imath}$} \uput[r](1.9,1.35){$\vect{k}$}
\psdots(11.6,0.3)
\end{pspicture}
\end{center}

On a AB $= 3$,\quad AD $= 2$,\quad AE $= 1$.

On définit les vecteurs $\vect{\imath}= \dfrac13\vect{\text{AB}},\:\vect{\jmath}= \dfrac12\vect{\text{AD}}, \:\vect{k} = \vect{\text{AE}}$.

On munit ainsi l'espace du repère orthonormé $\left(\text{A}~;~\vect{\imath},~\vect{\jmath},~\vect{k}\right)$.

\medskip

\begin{enumerate}
\item Donner les coordonnées du point G.
\item Le vecteur $\vect{n}$ de coordonnées $(2~;~0~;~-3)$ est vecteur normal au plan (EHI).

Déterminer une équation cartésienne du plan (EHI).\index{equation de plan@équation de plan}
\item Déterminer les coordonnées du point I.
\item Déterminer une mesure au degré près de l'angle $\widehat{\text{EIF}}$.\index{calcul d'angle}
\item Afin de raccorder la maison au réseau électrique, on souhaite creuser une tranchée rectiligne depuis un relais électrique situé en contrebas de la maison.

Le relais est représenté par le point R de coordonnées $(6~;~- 3~;~- 1)$.

La tranchée est assimilée à un segment d'une droite $\Delta$ passant par R et dirigée par le vecteur $\vect{u}$ de coordonnées $(-3~;~4~;~1)$. On souhaite vérifier que la tranchée atteindra la maison au niveau de l'arête [BC].
	\begin{enumerate}
		\item Donner une représentation paramétrique de la droite $\Delta$.\index{equation paramétrique de droite@équation paramétrique de droite}
		\item On admet qu'une équation du plan (BFG) est $x = 3$.
		
Soit K le point d'intersection de la droite $\Delta$ avec le plan (BFG).

Déterminer les coordonnées du point K.
		\item Le point K appartient-il bien à l'arête [BC] ?
	\end{enumerate}
\end{enumerate}

\bigskip

\textbf{\textsc{Exercice 4} \quad 7 points\hfill}

\medskip

\textbf{Principaux domaines abordés :} 
probabilités.

\medskip

\emph{Cet exercice est un questionnaire à choix multiples.\\
Pour chacune des questions suivantes, une seule des quatre réponses proposées est exacte.\\
Une réponse fausse, une réponse multiple ou l'absence de réponse à une question ne rapporte ni n'enlève de point.\\
Pour répondre, indiquer sur la copie le numéro de la question et la lettre de la réponse choisie. Aucune justification n'est demandée.}\index{QCM}

\medskip

On considère un système de communication binaire transmettant des $0$
et des $1$.

Chaque $0$ ou $1$ est appelé bit.

En raison d'interférences, il peut y avoir des erreurs de transmission :

un $0$ peut être reçu comme un $1$ et, de même, un $1$ peut être reçu comme un $0$.

Pour un bit choisi au hasard dans le message, on note les évènements :

\begin{minipage}{0.48\linewidth}

\begin{itemize}
\item[$\bullet~~$] $E_0$ : \og le bit envoyé est un $0$ \fg{} ;
\item[$\bullet~~$] $E_1$ : \og le bit envoyé est un 1 \fg{} ;
\item[$\bullet~~$] $R_0$ : \og le bit reçu est un $0$\fg{} 
\item[$\bullet~~$] $R_1$ : \og le bit reçu est un $1$ \fg.
\end{itemize}
\end{minipage}\hfill
\begin{minipage}{0.48\linewidth}
\begin{center}
\pstree[treemode=R,nodesepA=0pt,nodesepB=2.5pt,treesep = 1cm,levelsep=2.5cm]{\TR{}}
{\pstree{\TR{$E_0$~}\taput{0,4}}
{\TR{$R_0$}\taput{\ldots}
\TR{$R_1$}\tbput{0,01}
}
\pstree{\TR{$E_1$~}\tbput{\ldots}}
{\TR{$R_0$}\taput{0,02}
\TR{$R_1$}\tbput{\ldots}
}}
\end{center}
\end{minipage}

On sait que:

$p\left(E_0\right) = 0,4 \:;\quad p_{E_0}\left(R_1\right) = 0,01 \:;\quad p_{E_1}\left(R_0\right) = 0,02$.

On rappelle que la probabilité conditionnelle de $A$ sachant $B$ est notée $p_B(A)$.

On peut ainsi représenter la situation par l'arbre de probabilités ci-dessus.

\medskip

\begin{enumerate}
\item La probabilité que le bit envoyé soit un $0$ et que le bit reçu soit un $0$ est égale à :\index{probabilités}

\begin{center}
\begin{tabularx}{\linewidth}{*{4}{X}}
\textbf{a.~~}0,99 &\textbf{b.~~}0,396 &\textbf{c.~~}0,01 &\textbf{d.~~} 0,4
\end{tabularx}
\end{center}

\item La probabilité  $p\left(R_0\right)$ est égale à :
\begin{center}
\begin{tabularx}{\linewidth}{*{4}{X}}
\textbf{a.~~}0,99 &\textbf{b.~~}0,02 &\textbf{c.~~}0,408 &\textbf{d.~~}0,931
\end{tabularx}
\end{center}
\item Une valeur, approchée au millième, de la probabilité $p_{R_1}\left(E_0\right)$ est égale 
\begin{center}
\begin{tabularx}{\linewidth}{*{4}{X}}
\textbf{a.~~}0,004 &\textbf{b.~~}0,001 &\textbf{c.~~}0,007 &\textbf{d.~~}0,010
\end{tabularx}
\end{center}
\item La probabilité de l'évènement \og  il y a une erreur de transmission \fg{} est égale à :
\begin{center}
\begin{tabularx}{\linewidth}{*{4}{X}}
\textbf{a.~~}0,03 &\textbf{b.~~}0,016 &\textbf{c.~~}0,16 &\textbf{d.~~}0,015
\end{tabularx}
\end{center}
\end{enumerate}

Un message de longueur huit bits est appelé un octet.

On admet que la probabilité qu'un octet soit transmis sans erreur est égale à 0,88.

\begin{enumerate}[resume]
\item On transmet successivement 10 octets de façon indépendante.

La probabilité, à $10^{-3}$ près, qu'exactement 7 octets soient transmis sans erreur est égale à :

\begin{center}
\begin{tabularx}{\linewidth}{*{4}{X}}
\textbf{a.~~}0,915 &\textbf{b.~~}0,109 &\textbf{c.~~}0,976 &\textbf{d.~~}0,085
\end{tabularx}
\end{center}
\item On transmet successivement 10 octets de façon indépendante.

La probabilité qu'au moins 1 octet soit transmis sans erreur est égale à :

\begin{center}
\begin{tabularx}{\linewidth}{*{4}{X}}
\textbf{a.~~} $1 - 0,12^{10}$ &\textbf{b.~~}$0,12^{10}$ &\textbf{c.~~}$0,88^{10}$ &\textbf{d.~~} $1- 0,88^{10}$
\end{tabularx}
\end{center}
\item Soit $N$ un entier naturel. On transmet successivement $N$ octets de façon indépendante. 

Soit $N_0$ la plus grande valeur de $N$ pour laquelle la probabilité que les $N$ octets soient tous transmis sans erreur est supérieure ou égale à $0,1$.

On peut affirmer que:

\begin{center}
\begin{tabularx}{\linewidth}{*{4}{X}}
\textbf{a.~~}$N_0 = 17$ &\textbf{b.~~}$N_0 = 18$ &\textbf{c.~~}$N_0 = 19$ &\textbf{d.~~}$N_0 = 20$
\end{tabularx}
\end{center}
\end{enumerate}
\newpage
%%%%%%%%%%% Nouvelle-Calédonie 27 octobre 2022
\phantomsection
\hypertarget{NCaledo2}{}

\label{NCaledo2}

\lfoot{\small{Nouvelle-Calédonie Jour 2}}
\rfoot{\small{27 octobre 2022}}
\pagestyle{fancy}
\thispagestyle{empty}

\begin{center}{\Large\textbf{\decofourleft~Baccalauréat Nouvelle-Calédonie 27 octobre 2022 Jour 2~\decofourright\\[7pt] ÉPREUVE D'ENSEIGNEMENT DE SPÉCIALITÉ}}
\end{center}

\vspace{0,25cm}

\textbf{Le candidat traite 4 exercices : les exercices 1, 2 et 3 communs à tous les candidats et un seul des deux exercices A ou B.}

\bigskip

\textbf{\textsc{Exercice 1} \hfill 7 points}

\textbf{Principaux domaines abordés :} Probabilités

\medskip

Au basket-ball, il existe deux sortes de tir :

\begin{itemize}
\item les tirs à deux points.

Ils sont réalisés près du panier et rapportent deux points s'ils sont réussis.
\item les tirs à trois points.

Ils sont réalisés loin du panier et rapportent trois points s'ils sont réussis.
\end{itemize}

\smallskip

Stéphanie s'entraîne au tir. On dispose des données suivantes :

\begin{itemize}
\item[$\bullet~~$] Un quart de ses tirs sont des tirs à deux points. Parmi eux, 60\,\% sont réussis.
\item[$\bullet~~$] Trois quarts de ses tirs sont des tirs à trois points. Parmi eux, 35\,\% sont réussis.
\end{itemize}

\medskip

\begin{enumerate}
\item Stéphanie réalise un tir.

On considère les évènements suivants :

\begin{description}
\item[ ] $D$ : \og Il s'agit d'un tir à deux points \fg.
\item[ ] $R$ : \og le tir est réussi \fg.
\end{description}
	\begin{enumerate}
		\item Représenter la situation à l'aide d'un arbre de probabilités.\index{arbre pondéré}
		\item Calculer la probabilité $p\left(\overline{D} \cap R\right)$.\index{probabilités}
		\item Démontrer que la probabilité que Stéphanie réussisse un tir est égale à $0,4125$.
		\item Stéphanie réussit un tir. Calculer la probabilité qu'il s'agisse d'un tir à trois points. Arrondir le résultat au centième.
	\end{enumerate}	
\item Stéphanie réalise à présent une série de $10$ tirs à trois points.

On note $X$ la variable aléatoire qui compte le nombre de tirs réussis.

On considère que les tirs sont indépendants. On rappelle que la probabilité que Stéphanie réussisse un tir à trois points est égale à $0,35$.
	\begin{enumerate}
		\item Justifier que $X$ suit une loi binomiale. Préciser ses paramètres.\index{loi binomiale}
		\item Calculer l'espérance de $X$. Interpréter le résultat dans le contexte de l'exercice.\index{espérance}
		\item Déterminer la probabilité que Stéphanie rate $4$ tirs ou plus. Arrondir le résultat au centième.
		\item Déterminer la probabilité que Stéphanie rate au plus $4$ tirs. Arrondir le résultat au centième.
	\end{enumerate}
\item Soit $n$ un entier naturel non nul.

Stéphanie souhaite réaliser une série de $n$ tirs à trois points.

On considère que les tirs sont indépendants. On rappelle que la probabilité qu'elle réussisse un tir à trois points est égale à $0,35$.

Déterminer la valeur minimale de $n$ pour que la probabilité que Stéphanie réussisse au moins un tir parmi les $n$ tirs soit supérieure ou égale à $0,99$.\index{inéquation}
\end{enumerate}

\bigskip

\textbf{\textsc{Exercice 2} \hfill 7 points}

\textbf{Principaux domaines abordés :} fonctions, fonction logarithme.

\medskip

Soit $f$ la fonction définie sur l'intervalle $]0~;~+\infty[$ par :

\[f(x) = x \ln (x) - x - 2.\]\index{fonction logarithme}

On admet que la fonction $f$ est deux fois dérivable sur $]0~;~+\infty[$.

On note $f'$ sa dérivée, $f''$ sa dérivée seconde et $\mathcal{C}_f$ sa courbe représentative dans un repère.

\medskip

\begin{enumerate}
\item 
	\begin{enumerate}
		\item Démontrer que, pour tout $x$ appartenant à $]0~;~+\infty[$, on a $f'(x) = \ln (x)$.\index{dérivée}
		\item Déterminer une équation de la tangente $T$ à la courbe $\mathcal{C}_f$ au point
d'abscisse $x = $e.\index{tangente à la courbe}
		\item Justifier que la fonction $f$ est convexe sur l'intervalle $]0~;~+\infty[$.\index{convexité}
		\item En déduire la position relative de la courbe $\mathcal{C}_f$ et de la tangente $T$.
	\end{enumerate}	
\item 
	\begin{enumerate}
		\item Calculer la limite de la fonction $f$ en $0$.\index{limite de fonction}
		\item Démontrer que la limite de la fonction $f$ en $+\infty$ est égale à $+\infty$.\index{limite de fonction}
	\end{enumerate}
\item Dresser le tableau de variations de la fonction $f$ sur l'intervalle $]0~;~+\infty[$.\index{tableau de variations}
\item 
	\begin{enumerate}
		\item Démontrer que l'équation $f(x) = 0$ admet une unique solution dans
l'intervalle $]0~;~+\infty[$. On note $\alpha$ cette solution.\index{valeurs intermédiaires}
		\item Justifier que le réel $\alpha$ appartient à l'intervalle ]4,3~;~4,4[.
		\item En déduire le signe de la fonction $f$ sur l'intervalle $]0~;~+\infty[$.
	\end{enumerate}	
\item On considère la fonction \texttt{seuil} suivante écrite dans le langage Python :\index{script python}

On rappelle que la fonction \texttt{log} du module \texttt{math} (que l'on suppose importé)
désigne la fonction logarithme népérien ln.

\begin{center}
\begin{tabular}{|l l|} \hline
\texttt{def}& \texttt{seuil(pas) :}\\
&\texttt{x=4.3}\\
&\texttt{while x*log (x) - x - 2 < 0:}\\
&\quad \texttt{x=x+pas}\\
&\texttt{return x}\\ \hline
\end{tabular}
\end{center}

Quelle est la valeur renvoyée à l'appel de la fonction \texttt{seuil(0.01)} ?

Interpréter ce résultat dans le contexte de l'exercice.
\end{enumerate}

\bigskip

\textbf{\textsc{Exercice 3} \hfill 7 points}

\textbf{Principaux domaines abordés :} géométrie dans l'espace

\medskip

\begin{minipage}{0.5\linewidth}
Une maison est modélisée par un parallélépipède rectangle ABCDEFGH surmonté d'une pyramide EFGHS.

On a DC $= 6$,\:\: DA = DH $= 4$.

Soit les points I, J et K tels que

$\vect{\text{DI}} = \dfrac16\vect{\text{DC}}, \quad \vect{\text{DJ}} = \dfrac14\vect{\text{DA}},\quad \vect{\text{DK}} = \dfrac14\vect{\text{DH}}$.

On note $\vect{\imath} = \vect{\text{DI}},\:\vect{\jmath} = \vect{\text{DJ}},\: \vect{k} = \vect{\text{DK}}$.

On se place dans le repère orthonormé $\left(\text{D}~;~\vect{\imath},\:\vect{\jmath}, \: \vect{k}\right)$.

On admet que le point S a pour coordonnées (3~;~2~;~6).
\end{minipage} \hfill
\begin{minipage}{0.46\linewidth}
\begin{center}
\psset{unit=1cm,arrowsize=2pt 3}
\begin{pspicture}(6.3,6)
\pspolygon(0.3,3.9)(0.3,0.7)(2.7,0.3)(2.7,3.5)%GCBF
\psline(2.7,0.3)(6,1.1)(6,4.3)(2.7,3.5)%BAEF
\psline[linestyle=dotted,linewidth=1.25pt](0.3,0.7)(3.6,1.5)(6,1.1)%CDA
\psline[linestyle=dotted,linewidth=1.25pt](3.6,1.5)(3.6,4.7)(3.1,5.6)%DHS
\psline[linestyle=dotted,linewidth=1.25pt](0.3,3.9)(3.6,4.7)(6,4.3)%GHE
\psline(6,4.3)(3.1,5.6)(2.7,3.5)%ESF
\psline(3.1,5.6)(0.3,3.9)%SG
\psline(4.3,4.8)(4.3,5.35)%PQ
\psline[linewidth=1.25pt]{->}(3.6,1.5)(3.05,1.37)
\psline[linewidth=1.25pt]{->}(3.6,1.5)(4.2,1.4)
\psline[linewidth=1.25pt]{->}(3.6,1.5)(3.6,2.3)
\uput[r](6,1.1){A} \uput[d](2.7,0.3){B} \uput[dl](0.3,0.7){C} \uput[d](3.6,1.5){D}
\uput[ur](6,4.3){E} \uput[dl](2.7,3.5){F} \uput[l](0.3,3.9){G} \uput[ur](3.6,4.7){H}
\uput[u](3.1,5.6){S} \uput[d](4.3,4.8){P} \uput[u](4.3,5.35){Q} \uput[u](3.05,1.37){I}
\uput[ur](4.2,0.9){J} \uput[r](3.6,2.3){K} \uput[d](3.32,1.42){$\vect{\imath}$} \uput[u](3.9,1.3){$\vect{\jmath}$}
\uput[l](3.6,1.9){$\vect{k}$}
\end{pspicture}
\end{center}
\end{minipage}

\medskip

\begin{enumerate}
\item Donner, sans justifier, les coordonnées des points B, E, F et G.\index{lecture graphique}
\item Démontrer que le volume de la pyramide EFGHS représente le septième du volume total de la maison.\index{volume pyramide}

On rappelle que le volume $V$ d'un tétraèdre est donné par la formule :

\[V = \dfrac13 \times (\text{aire de la base}) \times  \text{hauteur}.\]
\item 
	\begin{enumerate}
		\item Démontrer que le vecteur $\vect{n}$ de coordonnées $\begin{pmatrix}0\\1\\1\end{pmatrix}$ est normal au plan (EFS).\index{vecteur normal}
		\item En déduire qu'une équation cartésienne du plan (EFS) est $y + z - 8 = 0$.\index{equation de plan@équation de plan}
	\end{enumerate}	
\item On installe une antenne sur le toit, représentée par le segment [PQ]. On dispose des données suivantes:

\begin{itemize}
\item[$\bullet~~$] le point P appartient au plan (EFS) ;
\item[$\bullet~~$] le point Q a pour coordonnées (2~;~3~;~5,5) ;
\item[$\bullet~~$] la droite (PQ) est dirigée par le vecteur $\vect{k}$.
\end{itemize}
	\begin{enumerate}
		\item Justifier qu'une représentation paramétrique de la droite (PQ) est :\index{equation paramétrique de droite@équation paramétrique de droite}

\[\left\{\begin{array}{l c l}
x &=&2\\
y &=&3\\
z &=& 5,5 + t
\end{array}\right. \quad (t \in \R)\]
		\item En déduire les coordonnées du point P.
		\item En déduire la longueur PQ de l'antenne.
	\end{enumerate}	
\item Un oiseau vole en suivant une trajectoire modélisée par la droite $\Delta$ dont une représentation paramétrique est :

\[\left\{\begin{array}{l c r}
x & =& -4 +6s\\ y &=& 7 - 4s\\z &=& 2 + 4s
\end{array}\right.\quad (s \in \R)\]

Déterminer la position relative des droites (PQ) et $\Delta$.

L'oiseau va-t-il percuter l'antenne représentée par le segment [PQ] ?

\end{enumerate}

\bigskip

\textbf{\textsc{Exercice 4} \hfill 7 points}

\textbf{Principaux domaines abordés :} suites, fonctions, primitives

\medskip

\emph{Cet exercice est un questionnaire à choix multiples.\\
Pour chacune des questions suivantes, une seule des quatre réponses proposées est exacte.\\
Une réponse fausse, une réponse multiple ou l'absence de réponse à une question ne rapporte ni n'enlève de point.\\
Pour répondre, indiquer sur la copie le numéro de la question et la lettre de la réponse choisie.\\
Aucune justification n'est demandée.}\index{QCM}

\medskip

\begin{enumerate}
\item On considère la suite $\left(u_n\right)$ définie pour tout entier naturel $n$ par

\[u_n  = \dfrac{(- 1)^n}{n + 1}.\]\index{suites}

On peut affirmer que:
\begin{center}
\begin{tabularx}{\linewidth}{X X}
\textbf{a.~~} la suite $\left(u_n\right)$ diverge vers $+\infty$. &\textbf{b.~~} la suite $\left(u_n\right)$ diverge vers $-\infty$.\\
\textbf{c.~~}  la suite $\left(u_n\right)$ n'a pas de limite. &\textbf{d.~~} la suite $\left(u_n\right)$ converge.
\end{tabularx}
\end{center}\index{suite convergente}

\begin{center} \decosix \decosix \decosix  \end{center}
\end{enumerate}

Dans les questions 2 et 3, on considère deux suites $\left(v_n\right)$  et $\left(w_n\right)$ vérifiant la relation : 

\[w_n = \e^{- 2v_n} + 2.\]

\begin{enumerate}[resume]
\item  Soit $a$ un nombre réel strictement positif. On a $v_0 = \ln (a)$.

\begin{center}
\begin{tabularx}{\linewidth}{X X}
\textbf{a.~~} $w_0 = \dfrac{1}{a^2}  +2$&\textbf{b.~~}$w_0 = \dfrac{1}{a^2  +2}$\\
\textbf{c.~~} $w_0 = -2a +2$&\textbf{d.~~}$w_0 = \dfrac{1}{- 2a} + 2$
\end{tabularx}
\end{center}

\item On sait que la suite $\left(v_n\right)$ est croissante. On peut affirmer que la suite $\left(w_n\right)$ est :

\begin{center}
\begin{tabularx}{\linewidth}{X X}
\textbf{a.~~}décroissante et majorée par 3.&\textbf{b.~~}décroissante et minorée par 2 .\\
\textbf{c.~~}croissante et majorée par 3 .&\textbf{d.~~}croissante et minorée par 2.
\end{tabularx}
\end{center}\index{suite monotone}

\item On considère la suite $\left(a_n\right)$ ainsi définie :

\[a_0 = 2 \:\:\text{et, pour tout entier naturel}\: n, \quad a_{n+1} = \dfrac13a_n + \dfrac83.\]


Pour tout entier naturel $n$, on a :
\begin{center}
\begin{tabularx}{\linewidth}{X X}
\textbf{a.~~}$a_n = 4 \times \left(\dfrac13\right)^n - 2$&\textbf{b.~~}$a_n = - \dfrac{2}{3^n} + 4$\\
\textbf{c.~~}$a_n = 4 - \left(\dfrac13\right)^n$ & \textbf{d.~~} $a_n = 2 \times \left(\dfrac13\right)^n + \dfrac{8n}{3}$
\end{tabularx}
\end{center}
\item On considère une suite $\left(b_n\right)$ telle que, pour tout entier naturel $n$, on a :

\[b_{n+1} = b_n + \ln \left(\dfrac{2}{\left(b_n \right)^2 + 3}\right).\]\index{fonction logarithme}

On peut affirmer que :

\begin{center}
\begin{tabularx}{\linewidth}{X X}
\textbf{a.~~} la suite $\left(b_n\right)$ est croissante.&\textbf{b.~~}la suite $\left(b_n\right)$ est décroissante.\\
\textbf{c.~~} la suite $\left(b_n\right)$ n'est pas monotone.&\textbf{d.~~}le sens de variation de la suite $\left(b_n\right)$ dépend de $b_0$.
\end{tabularx}
\end{center}

\item On considère la fonction $g$ définie sur l'intervalle $]0~;~+\infty[$ par : 

\[g(x) = \dfrac{\e^x}{x}.\]\index{fonction exponentielle}

On note $\mathcal{C}_g$ la courbe représentative de la fonction $g$ dans un repère orthogonal.

La courbe $\mathcal{C}_g$ admet :

\begin{center}
\begin{tabularx}{\linewidth}{X X}
\textbf{a.~~}une asymptote verticale
et une asymptote horizontale.&\textbf{b.~~}une asymptote verticale
et aucune asymptote horizontale.\\
\textbf{c.~~}aucune asymptote verticale et une asymptote horizontale.&\textbf{d.~~}aucune asymptote verticale et aucune asymptote horizontale.
\end{tabularx}
\end{center}\index{asymptote}
\item Soit $f$ la fonction définie sur $\R$ par

\[f(x) = x\e^{x^2+1}.\]\index{fonction exponentielle}

Soit $F$ une primitive sur $\R$ de la fonction $f$. Pour tout réel $x$, on a :\index{primitive}

\begin{center}
\begin{tabularx}{\linewidth}{X X}
\textbf{a.~~}$F(x) = \dfrac12x^2\e^{x^2+1}$ 	& \textbf{b.~~}$F(x) = \left(1 + 2x^2 \right)\e^{x^2+1}$ \\
\textbf{c.~~}$F(x) = \e^{x^2+1}$				& \textbf{d.~~}$F(x) = \dfrac12\e^{x^2+1}$
\end{tabularx}
\end{center}
\end{enumerate}
%%%%%%%%%
\phantomsection
\hypertarget{Index}{}
\setlength{\columnsep}{1.5cm}
\printindex
\end{document}