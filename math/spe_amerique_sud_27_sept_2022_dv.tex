\documentclass[11pt]{article}
\usepackage[T1]{fontenc}
\usepackage[utf8]{inputenc}
\usepackage{fourier}
\usepackage[scaled=0.875]{helvet}
\renewcommand{\ttdefault}{lmtt}
\usepackage{makeidx}
\usepackage{amsmath,amssymb}
\usepackage{fancybox}
\usepackage[normalem]{ulem}
\usepackage{pifont}
\usepackage{lscape}
\usepackage{multicol}
\usepackage{mathrsfs}
\usepackage{tabularx}
\usepackage{multirow}
\usepackage{enumitem}
\usepackage{textcomp} 
\newcommand{\euro}{\eurologo{}}
%Tapuscrit : Denis Vergès
%Relecture : 
\usepackage{pst-plot,pst-tree,pstricks,pst-node,pst-text}
\usepackage{pst-eucl}
\usepackage{pstricks-add}
\newcommand{\R}{\mathbb{R}}
\newcommand{\N}{\mathbb{N}}
\newcommand{\D}{\mathbb{D}}
\newcommand{\Z}{\mathbb{Z}}
\newcommand{\Q}{\mathbb{Q}}
\newcommand{\C}{\mathbb{C}}
\usepackage[left=3.5cm, right=3.5cm, top=3cm, bottom=3cm]{geometry}
\newcommand{\vect}[1]{\overrightarrow{\,\mathstrut#1\,}}
\renewcommand{\theenumi}{\textbf{\arabic{enumi}}}
\renewcommand{\labelenumi}{\textbf{\theenumi.}}
\renewcommand{\theenumii}{\textbf{\alph{enumii}}}
\renewcommand{\labelenumii}{\textbf{\theenumii.}}
\def\Oij{$\left(\text{O}~;~\vect{\imath},~\vect{\jmath}\right)$}
\def\Oijk{$\left(\text{O}~;~\vect{\imath},~\vect{\jmath},~\vect{k}\right)$}
\def\Ouv{$\left(\text{O}~;~\vect{u},~\vect{v}\right)$}
\usepackage{fancyhdr}
\usepackage[dvips]{hyperref}
\hypersetup{%
pdfauthor = {APMEP},
pdfsubject = {Baccalauréat spécialité},
pdftitle = {Amérique du Sud 27 septembre 2021},
allbordercolors = white,
pdfstartview=FitH} 
\usepackage[french]{babel}
\usepackage[np]{numprint}
\begin{document}
\setlength\parindent{0mm}
\rhead{\textbf{A. P{}. M. E. P{}.}}
\lhead{\small Baccalauréat spécialité}
\lfoot{\small{Amérique du Sud}}
\rfoot{\small{27 septembre 2022 Jour 2}}
\pagestyle{fancy}
\thispagestyle{empty}

\begin{center}{\Large\textbf{\decofourleft~Baccalauréat Amérique du Sud 27 septembre 2022~\decofourright\\[6pt] ÉPREUVE D'ENSEIGNEMENT DE SPÉCIALITÉ Jour 2}}
\end{center}

\vspace{0,25cm}

Le sujet propose 4 exercices

Le candidat choisit 3 exercices parmi les 4 exercices et \textbf{ne doit traiter que ces 3 exercices}

\medskip

Chaque exercice est noté sur 7 points (le total sera ramené sur 20 points).

Les traces de recherche, même incomplètes ou infructueuses, seront prises en compte.

\bigskip

\textbf{\textsc{Exercice 1 Probabilités} \hfill 7 points}

\medskip

Une entreprise fabrique des composants pour l'industrie automobile. Ces composants sont conçus sur trois chaînes de montage numérotées de 1 à 3.
\begin{itemize}
\item[$\bullet~~$] La moitié des composants est conçue sur la chaîne \no 1 ;
\item[$\bullet~~$]  30\,\% des composants sont conçus sur la chaîne \no 2;
\item[$\bullet~~$]  les composants restant sont conçus sur la chaîne \no 3.
\end{itemize}

À l'issue du processus de fabrication, il apparaît que 1\,\% des pièces issues de la chaîne \no 1 présentent un défaut, de même que 0,5\,\% des pièces issues de la chaîne \no 2 et 4\,\% des pièces issues de la chaîne \no 3.

On prélève au hasard un de ces composants. On note :

\begin{itemize}
\item[$\bullet~~$] $C_1$ l'évènement \og le composant provient de la chaîne \no 1 \fg{} ;
\item[$\bullet~~$] $C_2$ l'évènement \og le composant provient de la chaîne \no 2 \fg{} ;
\item[$\bullet~~$] $C_3$ l'évènement \og le composant provient de la chaîne \no 3 \fg{} ;
\item[$\bullet~~$] $D$ l'évènement \og le composant est défectueux\fg et $\overline{D}$ son évènement contraire.
\end{itemize}

\emph{Dans tout l'exercice, les calculs de probabilité seront donnés en valeur décimale exacte ou arrondie à $10^{-4}$ si nécessaire}.

\medskip

\textbf{PARTIE A}

\medskip

\begin{enumerate}
\item Représenter cette situation par un arbre pondéré.
\item Calculer la probabilité que le composant prélevé provienne de la chaîne \no 3 et soit
défectueux.
\item Montrer que la probabilité de l'évènement $D$ est $P(D) = \np{0,0145}$.
\item Calculer la probabilité qu'un composant défectueux provienne de la chaîne \no 3.
\end{enumerate}

\medskip

\textbf{PARTIE B}

\medskip

L'entreprise décide de conditionner les composants produits en constituant des lots de $n$ unités. On note $X$ la variable aléatoire qui, à chaque lot de $n$ unités, associe le nombre de composants défectueux de ce lot. 

Compte tenu des modes de production et de conditionnement de l'entreprise, on peut considérer que $X$ suit la loi binomiale de paramètres $n$ et $p = \np{0,0145}$.

\medskip

\begin{enumerate}
\item Dans cette question, les lots possèdent $20$ unités. On pose $n =20$.
	\begin{enumerate}
		\item Calculer la probabilité pour qu'un lot possède exactement trois composants défectueux.
		\item Calculer la probabilité pour qu'un lot ne possède aucun composant défectueux.
		
En déduire la probabilité qu'un lot possède au moins un composant défectueux.
	\end{enumerate}
\item  Le directeur de l'entreprise souhaite que la probabilité de n'avoir aucun composant
défectueux dans un lot de $n$ composants soit supérieure à $0,85$. 

Il propose de former des lots de 11 composants au maximum. A-t-il raison ? Justifier la réponse.
\end{enumerate}

\medskip

\textbf{PARTIE C}

\medskip

Les coûts de fabrication des composants de cette entreprise sont de $15$ euros s'ils proviennent de la chaîne de montage \no 1, 12 euros s'ils proviennent de la chaîne de montage \no 2 et 9 euros s'ils proviennent de la chaîne de montage \no 3.

Calculer le coût moyen de fabrication d'un composant pour cette entreprise.

\bigskip

\textbf{\textsc{Exercice 2 Fonctions, fonction logarithme} \hfill 7 points}

\medskip

Le but de cet exercice est d'étudier la fonction $f$, définie sur $]0~;~ +\infty[$ ,par:

\[f(x) =3x - x \ln (x) - 2 \ln (x).\]

\smallskip

\textbf{PARTIE A: Étude d'une fonction auxiliaire }\boldmath $g$ \unboldmath

\medskip

Soit $g$ la fonction définie sur $]0~;~ +\infty[$ par 

\[g(x) = 2(x - 1) - x \ln (x).\]

On note $g'$ la fonction dérivée de $g$. On admet que $\displaystyle\lim_{x \to + \infty} g(x) = - \infty$.

\medskip

\begin{enumerate}
\item Calculer $g(1)$ et $g(\text{e})$.
\item Déterminer $\displaystyle\lim_{x \to + 0} g(x)$ en justifiant votre démarche.
\item Montrer que, pour tout $x > 0$,\: $g'(x) = 1 - \ln (x)$.

En déduire le tableau des variations de $g$ sur $]0~;~ +\infty[$.
\item Montrer que l'équation $g(x) = 0$ admet exactement deux solutions distinctes sur
$]0~;~ +\infty[$ : 1 et $\alpha$ avec $\alpha$ appartenant à l'intervalle $[\text{e}~;~+\infty[$.

On donnera un encadrement de $\alpha$ à 0,01 près.
\item En déduire le tableau de signes de $g$ sur $]0~;~ +\infty[$. 
\end{enumerate}

\bigskip

\textbf{PARTIE B : Étude de la fonction} \boldmath $f$ \unboldmath

\medskip

On considère dans cette partie la fonction $f$, définie sur $]0~;~ +\infty[$,par 

\[f(x) = 3x - x \ln (x)- 2\ln (x).\]

On note $f'$ la fonction dérivée de $f$.

La représentation graphique $\mathcal{C}_f$ de cette fonction $f$ est donnée dans le repère \Oij{} ci-dessous. On admet que : $\displaystyle\lim_{x \to 0}f(x) = + \infty$.

\begin{center}
\psset{unit=0.82cm,arrowsize=2pt 3}
\begin{pspicture*}(-1,-2.01)(16,6)
\psgrid[gridlabels=0pt,subgriddiv=1,gridwidth=0.15pt](0,-2)(16,6)
\psaxes[linewidth=1.25pt,labelFontSize=\scriptstyle]{->}(0,0)(0,-1.95)(16,5.95)
\psaxes[linewidth=1.25pt,labelFontSize=\scriptstyle](0,0)(0,-1.95)(16,5.95)
\psplot[plotpoints=2000,linewidth=1.25pt,linecolor=blue]{0.01}{16}{x 3 mul x ln x 2 add mul sub}
\uput[ur](10.5,2){\blue $\mathcal{C}_f$}
\uput[u](15.75,0){$x$}
\uput[l](0,5.75){$y$}
\end{pspicture*}
\end{center}

\begin{enumerate}
\item Déterminer la limite de $f$ en $+ \infty$ en justifiant votre démarche.
\item 
	\begin{enumerate}
		\item Justifier que pour tout $x > 0$,\: $f'(x) = \dfrac{g(x)}{x}$.
		\item En déduire le tableau des variations de $f$ sur $]0~;~ +\infty[$.
	\end{enumerate}	
\item On admet que, pour tout $x > 0$, la dérivée seconde de $f$, notée $f''$, est définie par $f''(x) = \dfrac{2 - x}{x^2}$.

Étudier la convexité de $f$ et préciser les coordonnées du point
 d'inflexion de $\mathcal{C}_f$.
\end{enumerate}

\bigskip

\textbf{\textsc{Exercice 3 Suites} \hfill 7 points}

\medskip

La population d'une espèce en voie de disparition est surveillée de près dans une réserve naturelle.

Les conditions climatiques ainsi que le braconnage font que cette population diminue de 10\,\% chaque année.

Afin de compenser ces pertes, on réintroduit dans la réserve $100$ individus à la fin de chaque année.

On souhaite étudier l'évolution de l'effectif de cette population au cours du temps. Pour cela, on modélise l'effectif de la population de l'espèce par la suite $\left(u_n\right)$ où $u_n$ représente l'effectif de la population au début de l'année $2020 + n$. 

On admet que pour tout entier naturel $n$,\: $u_n \geqslant 0$.

Au début de l'année 2020, la population étudiée compte \np{2000} individus, ainsi $u_0 = \np{2000}$.

\medskip

\begin{enumerate}
\item Justifier que la suite $\left(u_n\right)$ vérifie la relation de récurrence:

\[u_{n+1} = 0,9u_n + 100.\]

\item Calculer $u_1$ puis $u_2$.
\item Démontrer par récurrence que pour tout entier naturel $n$ :\: $\np{1000} < u_{n+1}  \leqslant u_n$.
\item La suite $\left(u_n\right)$ est-elle convergente ? Justifier la réponse.
\item On considère la suite $\left(v_n\right)$ définie pour tout entier naturel $n$ par $v_n = u_n - \np{1000}$.
	\begin{enumerate}
		\item Montrer que la suite $\left(v_n\right)$ est géométrique de raison $0,9$.
		\item En déduire que, pour tout entier naturel $n$,\: $u_n = \np{1000} \left(1 + 0,9^n\right)$.
		\item Déterminer la limite de la suite $\left(u_n\right)$.
		
En donner une interprétation dans le contexte de cet exercice.
	\end{enumerate}

\item On souhaite déterminer le nombre d'années nécessaires pour que l'effectif de la population passe en dessous d'un certain seuil $S$ (avec $S > \np{1000}$).
	\begin{enumerate}
		\item Déterminer le plus petit entier $n$ tel que $u_n \leqslant \np{1020}$.

Justifier la réponse par un calcul.
	\end{enumerate}
\begin{minipage}{0.58\linewidth}
	\begin{enumerate}[resume]
		\item Dans le programme Python ci-contre, la variable $n$ désigne le nombre d'années écoulées depuis 2020, la variable $u$ désigne l'effectif de la population. 
		
Recopier et compléter ce programme afin qu'il retourne le nombre d'années nécessaires pour que l'effectif de la population passe en dessous du seuil $S$.
	\end{enumerate}

\end{minipage}\hfill
\begin{minipage}{0.38\linewidth}
\begin{tabular}{|c l|}\hline
1&\texttt{def population(S) :}\\
2&\quad \texttt{n=0}\\
3&\quad \texttt{u=2000}\\
4& \\
5&\quad \texttt{while \ldots\ldots} :\\
6&\quad \quad \texttt{u= \ldots}\\
7&\quad \quad \texttt{n = \ldots}\\
8&\quad \texttt{return \ldots}\\ \hline
\end{tabular}
\end{minipage}
\end{enumerate}

\bigskip

\textbf{\textsc{Exercice 4 Géométrie dans l'espace} \hfill 7 points}

\bigskip

\begin{minipage}{0.41\linewidth}
Dans l'espace muni d'un repère orthonormé \Oijk, on considère les points 

\[\text{A}(0~;~8~;~6),\quad  \text{B}(6~;~4~;~4 )\quad \text{et C}(2~;~4~;~0).\]

\end{minipage}\hfill
\begin{minipage}{0.55\linewidth}
\psset{unit=1cm,arrowsize=2pt 3}
\begin{pspicture}(-1.8,-1.8)(5.4,3.8)
\pspolygon[fillstyle=solid,fillcolor=lightgray](1.7,-0.6)(4.8,3)(0.4,0.4)
\psaxes[linewidth=1.25pt,Dx=20,Dy=20]{->}(0,0)(0,0)(5.4,3.8)
\psline[linewidth=1.25pt]{->}(0,0)(-1.8,-1.8)
\multido{\n=0+-0.3}{7}{\psdots[dotstyle=+,dotangle=45](\n,\n)}
\multido{\n=0+0.6}{9}{\psline(\n,-0.05)(\n,0.05)}
\multido{\n=0+0.5}{8}{\psline(-0.05,\n)(0.05,\n)}
\psline[linestyle=dashed](4.8,0)(4.8,3)(0,3)
\psframe[linestyle=dashed](-1.8,-1.8)(0.4,0.4)
\psframe[linestyle=dashed](2.4,2)
\psline[linestyle=dashed](-1.8,0.4)(0,2)
\psline[linestyle=dashed](0.4,-1.8)(2.4,0)
\psline[linestyle=dashed](0.4,0.4)(2.4,2)
\psline[linestyle=dashed](-0.6,-0.6)(1.8,-0.6)
\uput[d](0,0){\small O} \uput[ul](-0.15,-0.2){\footnotesize $\vect{\imath}$}
\uput[d](0.3,0){\footnotesize $\vect{\jmath}$}
\uput[l](0,0.3){\footnotesize $\vect{k}$}
\uput[l](-1.8,-1.8){$x$}\uput[d](5.3,0){$y$}\uput[l](0,3.7){$z$}

\uput[ur](4.8,3){A} \uput[ul](0.4,0.4){B} \uput[dr](1.7,-0.6){C}
\end{pspicture}
\end{minipage}

\bigskip

\begin{enumerate}
\item 
	\begin{enumerate}
		\item Justifier que les points A, B et C ne sont pas alignés.
		\item Montrer que le vecteur $\vect{n}(1~;~2~;~-1)$ est
un vecteur normal au plan (ABC).
		\item Déterminer une équation cartésienne du plan (ABC).
	\end{enumerate}
\item Soient D et E les points de coordonnées respectives (0~;~0~;~6) et (6~;~6~;~0).
	\begin{enumerate}
		\item Déterminer une représentation paramétrique de la droite (DE).
		\item Montrer que le milieu I du segment [BC] appartient à la droite (DE).
	\end{enumerate}	
\item On considère le triangle ABC.
	\begin{enumerate}
		\item Déterminer la nature du triangle ABC.
		\item Calculer l'aire du triangle ABC en unité d'aire.
		\item Calculer $\vect{\text{AB}}  \cdot \vect{\text{AC}}$.
		\item En déduire une mesure de l'angle $\widehat{\text{BAC}}$ arrondie à $0,1$ degré.
	\end{enumerate}	
\item On considère le point H de coordonnées $\left(\dfrac53~;~\dfrac{10}{3}~;~- \dfrac53\right)$.

Montrer que H est le projeté orthogonal du point O sur le plan (ABC). 

En déduire la distance du point O au plan (ABC).
\end{enumerate}
\end{document}