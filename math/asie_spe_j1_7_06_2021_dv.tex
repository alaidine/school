\documentclass[11pt]{article}
\usepackage[T1]{fontenc}
\usepackage[utf8]{inputenc}
\usepackage{fourier}
\usepackage[scaled=0.875]{helvet}
\renewcommand{\ttdefault}{lmtt}
\usepackage{makeidx}
\usepackage{amsmath,amssymb}
\usepackage{fancybox}
\usepackage[normalem]{ulem}
\usepackage{pifont}
\usepackage{lscape}
\usepackage{multicol}
\usepackage{mathrsfs}
\usepackage{tabularx}
\usepackage{multirow}
\usepackage{enumitem}
\usepackage{textcomp} 
\newcommand{\euro}{\eurologo{}}
%Tapuscrit : Denis Vergès
%Relecture : 
\usepackage{pst-plot,pst-tree,pstricks,pst-node,pst-text}
\usepackage{pst-eucl}
\usepackage{pstricks-add}
\newcommand{\R}{\mathbb{R}}
\newcommand{\N}{\mathbb{N}}
\newcommand{\D}{\mathbb{D}}
\newcommand{\Z}{\mathbb{Z}}
\newcommand{\Q}{\mathbb{Q}}
\newcommand{\C}{\mathbb{C}}
\usepackage[left=3.5cm, right=3.5cm, top=3cm, bottom=3cm]{geometry}
\newcommand{\vect}[1]{\overrightarrow{\,\mathstrut#1\,}}
\renewcommand{\theenumi}{\textbf{\arabic{enumi}}}
\renewcommand{\labelenumi}{\textbf{\theenumi.}}
\renewcommand{\theenumii}{\textbf{\alph{enumii}}}
\renewcommand{\labelenumii}{\textbf{\theenumii.}}
\def\Oij{$\left(\text{O}~;~\vect{\imath},~\vect{\jmath}\right)$}
\def\Oijk{$\left(\text{O}~;~\vect{\imath},~\vect{\jmath},~\vect{k}\right)$}
\def\Ouv{$\left(\text{O}~;~\vect{u},~\vect{v}\right)$}
\newcommand{\e}{\text{e}}
\usepackage{fancyhdr}
\usepackage[dvips]{hyperref}
\hypersetup{%
pdfauthor = {APMEP},
pdfsubject = {Baccalauréat spécialité},
pdftitle = {Métropole 7 juin 2021},
allbordercolors = white,
pdfstartview=FitH} 
\usepackage[french]{babel}
\usepackage[np]{numprint}
\begin{document}
\setlength\parindent{0mm}
\rhead{\textbf{A. P{}. M. E. P{}.}}
\lhead{\small Baccalauréat spécialité}
\lfoot{\small{Asie}}
\rfoot{\small{7 juin 2021}}
\pagestyle{fancy}
\thispagestyle{empty}

\begin{center}{\Large\textbf{\decofourleft~Baccalauréat Asie 7 juin 2021 Jour 1~\decofourright\\[6pt] ÉPREUVE D'ENSEIGNEMENT DE SPÉCIALITÉ}}
\end{center}

\vspace{0,25cm}

Le candidat traite 4 exercices : les exercices 1, 2 et 3 communs à tous les candidats et un seul des deux exercices A ou B.

\bigskip

\textbf{\textsc{Exercice 1} \hfill 5 points}

\textbf{Commun à tous les candidats}

\medskip

En 2020, une influenceuse sur les réseaux sociaux compte \np{1000}~abonnés à son profil. On modélise le nombre d'abonnés ainsi: chaque année, elle perd 10\,\% de ses abonnés auxquels s'ajoutent $250$ nouveaux abonnés.

Pour tout entier naturel $n$, on note $u_n$ le nombre d'abonnés à son profil en l'année $(2020 + n)$, suivant cette modélisation. Ainsi $u_0 = \np{1000}$.

\medskip

\begin{enumerate}
\item Calculer $u_1$.
\item Justifier que pour tout entier naturel $n,\,  u_{n+1} = 0,9u_n + 250$.
\item La fonction Python nommée \og suite \fg{} est définie ci-dessous. Dans le contexte de l'exercice, interpréter la valeur renvoyée par suite(10).

\begin{center}
\begin{tabular}{|l|}\hline
def suite( n) :\\
\quad u = \np{1000}\\
\quad for i in range(n) :\\
\qquad  u = 0,9*u + 250\\
\quad return u\\ \hline
\end{tabular}
\end{center}

\item 
	\begin{enumerate}
		\item Montrer, à l'aide d'un raisonnement par récurrence, que pour tout entier naturel $n$,
$u_n \leqslant  \np{2500}$.
		\item Démontrer que la suite $\left(u_n\right)$ est croissante.
		\item Déduire des questions précédentes que la suite $\left(u_n\right)$ est convergente.
	\end{enumerate}
\item Soit $\left(v_n\right)$ la suite définie par $v_n = u_n - \np{2500}$ pour tout entier naturel $n$.
	\begin{enumerate}
		\item Montrer que la suite $\left(v_n\right)$ est une suite géométrique de raison $0,9$ et de terme initial $v_0 = \np{- 1500}$.
		\item Pour tout entier naturel $n$, exprimer $v_n$ en fonction de $n$ et montrer que :
		
\[u_n = - \np{1500} \times  0,9^n + \np{2500}.\]
		\item Déterminer la limite de la suite $\left(u_n\right)$ et interpréter dans le contexte de l'exercice.
	\end{enumerate}
\item Écrire un programme qui permet de déterminer en quelle année le nombre d'abonnés dépassera \np{2200}.

Déterminer cette année.
\end{enumerate}

\bigskip

\textbf{EXERCICE 2 commun à tous les candidats \hfill 5 points}

\medskip

On considère un cube ABCDEFGH d'arête 8 cm et de centre $\Omega$.

Les points P, Q et R sont définis par $\vect{\text{AP}} = \dfrac{3}{4}\vect{\text{AB}},\, \vect{\text{AQ}} = ~\dfrac{3}{4}\vect{\text{AE}}$ et $\vect{\text{FR}} = \dfrac{1}{4}\vect{\text{FG}}$.

On se place dans le repère orthonormé $\left(\text{A}~;\vect{\imath},~\vect{\jmath},~\vect{k}\right)$ avec : $\vect{\imath} = \dfrac{1}{8}\vect{\text{AB}},\, \vect{\jmath}= \dfrac{1}{8}\vect{\text{AD}}$ et 

$\vect{k} = \dfrac{1}{8}\vect{\text{AE}}$.

\begin{center}
\psset{unit=0.85cm,arrowsize=2pt 4}
\begin{pspicture}(11,12)
%\psgrid
\pspolygon(0,1.7)(6.5,0)(6.5,7.2)(0,8.9)%BCGF
\uput[d](0,1.7){B}\uput[d](6.5,0){C}\uput[u](6.5,7.2){G}\uput[u](0,8.9){F}
\psline(6.5,0)(10,2.6)(10,9.8)(6.5,7.2)%CDHG
\uput[r](10,2.6){D}\uput[u](10,9.8){H}
\psline(10,9.8)(3.5,11.5)(0,8.9)%HEF
\uput[u](3.5,11.5){E}
\psline[linestyle=dashed](0,1.7)(3.5,4.3)(3.5,11.5)%BAE
\uput[d](3.5,4.3){A}\uput[r](5,5.73){$\Omega$}
\psline[linestyle=dashed](3.5,4.3)(10,2.6)%AD
\psline[linestyle=dashed](3.5,4.3)(6.5,7.2)%AG
\psline[linestyle=dashed](3.5,11.5)(6.5,0)%EC
\psdots(0.88,2.35)(3.5,9.7)(5,5.73)(1.625,8.475)
\uput[u](0.88,2.35){P}\uput[l](3.5,9.7){Q}\uput[u](1.625,8.475){R}
\uput[u](3.0625,3.975){$\vect{\imath}$}\uput[u](4.3125,4.0875){$\vect{\jmath}$}
\uput[l](3.5,5.2){$\vect{k}$}
\psline{->}(3.5,4.3)(3.0625,3.975) \psline{->}(3.5,4.3)(4.3125,4.0875) \psline{->}(3.5,4.3)(3.5,5.2)
\end{pspicture}
\end{center}

\textbf{Partie I}

\medskip

\begin{enumerate}
\item Dans ce repère, on admet que les coordonnées du point R sont (8~;~2~;~8). 

Donner les coordonnées des points P et Q.
\item Montrer que le vecteur $\vect{n}(1~;~-5~;~1)$ est un vecteur normal au plan (PQR).
\item Justifier qu'une équation cartésienne du plan (PQR) est $x - 5y + z - 6 = 0$.
\end{enumerate}

\bigskip

\textbf{Partie II}

\medskip

On note L le projeté orthogonal du point $\Omega$ sur le plan (PQR).

\medskip

\begin{enumerate}
\item Justifier que les coordonnées du point $\Omega$ sont (4~;~4~;~4).
\item Donner une représentation paramétrique de la droite $d$ perpendiculaire au plan (PQR) et passant par $\Omega$.
\item Montrer que les coordonnées du point L sont $\left(\dfrac{14}{3}~;~ \dfrac{2}{3}~;~\dfrac{14}{3}\right)$
\item Calculer la distance du point $\Omega$ au plan (PQR).
\end{enumerate}


\bigskip

\textbf{EXERCICE 3 commun à tous les candidats \hfill 5 points}

\medskip

Un sac contient les huit lettres suivantes: A B C D E F G H (2 voyelles et 6 consonnes).

Un jeu consiste à tirer simultanément au hasard deux lettres dans ce sac. 

On gagne si le tirage est constitué d'une voyelle \textbf{et} d'une consonne.

\medskip

\begin{enumerate}
\item Un joueur extrait simultanément deux lettres du sac.
	\begin{enumerate}
		\item Déterminer le nombre de tirages possibles.
		\item Déterminer la probabilité que le joueur gagne à ce jeu.
	\end{enumerate}
\end{enumerate}

\medskip
	
\textbf{Les questions 2 et 3 de cet exercice sont indépendantes.}

\medskip

Pour la suite de l'exercice, on admet que la probabilité que le joueur gagne est égale à $\dfrac{3}{7}$.

\begin{enumerate}[resume]
\item Pour jouer, le joueur doit payer $k$ euros, $k$ désignant un entier naturel non nul. 

Si le joueur gagne, il remporte la somme de $10$ euros, sinon il ne remporte rien.

On note $G$ la variable aléatoire égale au gain algébrique d'un joueur (c'est-à-dire la somme remportée à laquelle on soustrait la somme payée).
	\begin{enumerate}
		\item Déterminer la loi de probabilité de $G$.
		\item Quelle doit être la valeur maximale de la somme payée au départ pour que le jeu reste
favorable au joueur ?	
	\end{enumerate}
\item Dix joueurs font chacun une partie. Les lettres tirées sont remises dans le sac après chaque partie.

On note $X$ la variable aléatoire égale au nombre de joueurs gagnants.
	\begin{enumerate}
		\item Justifier que $X$ suit une loi binomiale et donner ses paramètres.
		\item Calculer la probabilité, arrondie à $10^{-3}$, qu'il y ait exactement quatre joueurs gagnants. 
		\item Calculer $P(X \geqslant 5)$ en arrondissant à $10^{-3}$. Donner une interprétation du résultat obtenu. 
		\item Déterminer le plus petit entier naturel $n$ tel que $P(X \leqslant n) \geqslant 0,9$.
	\end{enumerate}
\end{enumerate}

\bigskip

\begin{center}\textbf{EXERCICE au choix du candidat \hfill 5 points}\end{center}

\medskip

\textbf{Le candidat doit traiter UN SEUL des deux exercices A ou B}

\textbf{Il indique sur sa copie l'exercice choisi: exercice A ou exercice B}

\medskip

\textbf{EXERCICE -- A}

\medskip

\begin{tabular}{|l|}\hline
\textbf{Principaux domaines abordés}\\
-- convexité\\
-- fonction logarithme\\ \hline
\end{tabular}

\bigskip

\textbf{Partie I : lectures graphiques}

\medskip

$f$ désigne une fonction définie et dérivable sur $\R$.

On donne ci-dessous la courbe représentative de la fonction dérivée $f'$.

\begin{center}
\psset{xunit=0.9cm,yunit=3cm}
\begin{pspicture}(-7,-0.8)(7,1.2)
\multido{\n=-7+1}{15}{\psline[linestyle=dashed,linewidth=0.1pt](\n,-0.8)(\n,1.2)}
\multido{\n=-0.8+0.2}{11}{\psline[linestyle=dashed,linewidth=0.1pt](-7,\n)(7,\n)}
\psaxes[linewidth=1.25pt,labelFontSize=\scriptstyle]{->}(0,0)(-7,-0.8)(7,1.2)
\psplot[plotpoints=2000,linewidth=1.25pt,linecolor=red]{-7}{7}{x 2 mul 1 add x dup mul x add 2.5 add div}
\uput[u](0,1){\red Courbe de la fonction dérivée $f'$}
\end{pspicture}
\end{center}

\medskip

\emph{Avec la précision permise par le graphique, répondre aux questions suivantes}

\medskip

\begin{enumerate}
\item Déterminer le coefficient directeur de la tangente à la courbe de la fonction $f$ en $0$.
\item 
	\begin{enumerate}
		\item Donner les variations de la fonction dérivée $f'$.
		\item En déduire un intervalle sur lequel $f$ est convexe.
	\end{enumerate}
\end{enumerate}

\bigskip

\textbf{Partie II : étude de fonction}

\medskip

La fonction $f$ est définie sur $\R$ par 

\[f(x) = \ln \left(x^2 + x + \dfrac{5}{2}\right).\]

\medskip

\begin{enumerate}
\item Calculer les limites de la fonction $f$ en $+\infty$ et en $-\infty$.
\item Déterminer une expression $f'(x)$ de la fonction dérivée de $f$ pour tout $x \in \R$.
\item En déduire le tableau des variations de $f$. On veillera à placer les limites dans ce tableau.
\item 
	\begin{enumerate}
		\item Justifier que l'équation $f(x) = 2$ a une unique solution $\alpha$ dans l'intervalle $\left[-\dfrac{1}{2}~;~+ \infty\right[$. 
		\item Donner une valeur approchée de $\alpha$ à $10^{-1}$ près.
 	\end{enumerate}
\item La fonction $f'$ est dérivable sur $\R$. On admet que, pour tout $x \in  \R$,\,  $f''(x) = \dfrac{-2x^2 - 2x + 4}{\left(x^2 + x + \dfrac{5}{2}\right)^2}$.

Déterminer le nombre de points d'inflexion de la courbe représentative de $f$.
\end{enumerate}

\bigskip

\textbf{EXERCICE -- B}

\medskip

\begin{tabular}{|l|}\hline
\textbf{Principaux domaines abordés}\\
-- Étude de fonction, fonction exponentielle\\
-- Équations différentielles\\ \hline
\end{tabular}

\bigskip

\textbf{Partie I}

\medskip

Considérons l'équation différentielle

\[y'= -0,4y + 0,4\]

où $y$ désigne une fonction de la variable $t$, définie et dérivable sur $[0~;~ + \infty[$.

\medskip

\begin{enumerate}
\item 
	\begin{enumerate}
		\item Déterminer une solution particulière constante de cette équation différentielle. 
		\item En déduire l'ensemble des solutions de cette équation différentielle.
\item Déterminer la fonction $g$, solution de cette équation différentielle, qui vérifie $g(0) = 10$.
	\end{enumerate}
\end{enumerate}

\bigskip

\textbf{Partie II}

\medskip

Soit $p$ la fonction définie et dérivable sur l'intervalle $[0~;~+ \infty[$ par 

\[p(t) = \dfrac{1}{g(t)} = \dfrac{1}{1 + 9\e^{-0,4t}}.\]

\smallskip

\begin{enumerate}
\item Déterminer la limite de $p$ en $+ \infty$. 
\item Montrer que $p'(t) = \dfrac{3,6\e^{-0,4t}}{ \left(1 + 9\e^{-0,4t}\right)^2}$ pour tout $t \in  [0~;~+ \infty[$.

\item  
	\begin{enumerate}
		\item Montrer que l'équation $p(t) = \dfrac{1}{2}$ admet une unique solution $\alpha$ sur $[0~;~+ \infty[$. 
		\item Déterminer une valeur approchée de $\alpha$ à $10^{-1}$ près à l'aide d'une calculatrice.
	\end{enumerate}
\end{enumerate}

\bigskip

\textbf{Partie III}

\medskip

\begin{enumerate}
\item $p$ désigne la fonction de la partie II.

Vérifier que $p$ est solution de l'équation différentielle $y' = 0,4y(1 - y)$ avec la condition initiale 
$y(0) = \dfrac{1}{10}$ où $y$ désigne une fonction définie et dérivable sur $[0~;~ + \infty[$.
\item Dans un pays en voie de développement, en l'année 2020, 10\,\% des écoles ont accès à internet. 

Une politique volontariste d'équipement est mise en œuvre et on s'intéresse à l'évolution de la proportion des écoles ayant accès à internet. 

On note $t$ le temps écoulé, exprimé en année, depuis l'année 2020.

La proportion des écoles ayant accès à internet à l'instant $t$ est modélisée par $p(t)$.

Interpréter dans ce contexte la limite de la question II 1 puis la valeur approchée de $\alpha$ de la question II 3. b. ainsi que la valeur $p(0)$.
\end{enumerate}
\end{document}