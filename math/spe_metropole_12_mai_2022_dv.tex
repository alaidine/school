\documentclass[11pt,a4paper,french]{article}
\usepackage[T1]{fontenc}
\usepackage[utf8]{inputenc}
\usepackage{fourier}
\usepackage[scaled=0.875]{helvet}
\renewcommand{\ttdefault}{lmtt}
\usepackage{makeidx}
\usepackage{amsmath,amssymb}
\usepackage{fancybox}
\usepackage[normalem]{ulem}
\usepackage{pifont}
\usepackage{lscape}
\usepackage{multicol}
\usepackage{mathrsfs}
\usepackage{tabularx}
\usepackage{multirow}
\usepackage{enumitem}
\usepackage{textcomp} 
\newcommand{\euro}{\eurologo{}}
%Tapuscrit : Denis Vergès
%Relecture : Jean-Claude Souque et Jean-Paul Blaineau
\usepackage{pst-plot,pst-tree,pstricks,pst-node,pst-text}
\usepackage{pst-eucl,pst-3dplot,pstricks-add}
\usepackage{esvect}
\newcommand{\R}{\mathbb{R}}
\newcommand{\N}{\mathbb{N}}
\newcommand{\D}{\mathbb{D}}
\newcommand{\Z}{\mathbb{Z}}
\newcommand{\Q}{\mathbb{Q}}
\newcommand{\C}{\mathbb{C}}
\usepackage[left=2.5cm, right=2.5cm, top=2cm, bottom=3cm]{geometry}
\headheight15 mm
\newcommand{\vect}[1]{\overrightarrow{\,\mathstrut#1\,}}
\renewcommand{\theenumi}{\textbf{\arabic{enumi}}}
\renewcommand{\labelenumi}{\textbf{\theenumi.}}
\renewcommand{\theenumii}{\textbf{\alph{enumii}}}
\renewcommand{\labelenumii}{\textbf{\theenumii.}}
\def\Oij{$\left(\text{O}~;~\vect{\imath},~\vect{\jmath}\right)$}
\def\Oijk{$\left(\text{O}~;~\vect{\imath},~\vect{\jmath},~\vect{k}\right)$}
\def\Ouv{$\left(\text{O}~;~\vect{u},~\vect{v}\right)$}
\newcommand{\e}{\text{e}}
\usepackage{fancyhdr}
\usepackage[dvips]{hyperref}
\hypersetup{%
pdfauthor = {APMEP},
pdfsubject = {Baccalauréat Spécialité},
pdftitle = {Métropole Sujet 2  12  mai 2022},
allbordercolors = white,
pdfstartview=FitH} 
\usepackage{babel}
\usepackage[np]{numprint}
\renewcommand\arraystretch{1.3}
\frenchsetup{StandardLists=true}
\begin{document}
\setlength\parindent{0mm}
\rhead{\textbf{A. P{}. M. E. P{}.}}
\lhead{\small Baccalauréat spécialité sujet 2}
\lfoot{\small{Métropole}}
\rfoot{\small{12 mai 2022}}
\pagestyle{fancy}
\thispagestyle{empty}

\begin{center}{\Large\textbf{\decofourleft~Baccalauréat Métropole 12 mai 2022~\decofourright\\[6pt]  Sujet 2\\[7pt] ÉPREUVE D'ENSEIGNEMENT DE SPÉCIALITÉ}}
\end{center}

\vspace{0,25cm}

Le sujet propose 4 exercices

Le candidat choisit 3 exercices parmi les 4 et \textbf{ne doit traiter que ces 3 exercices}

\bigskip

\textbf{\textsc{Exercice 1} \quad (7 points)\hfill Thème : probabilités}

\medskip

Le coyote est un animal sauvage proche du loup, qui vit en Amérique du Nord.

Dans l'état d'Oklahoma, aux États-Unis, 70\,\% des coyotes sont touchés par une maladie appelée ehrlichiose.

Il existe un test aidant à la détection de cette maladie. Lorsque ce test est appliqué à un coyote, son résultat est soit positif, soit négatif, et on sait que :

\setlength\parindent{1cm}
\begin{itemize}
\item[$\bullet~~$] Si le coyote est malade, le test est positif dans 97\,\% des cas.
\item[$\bullet~~$] Si le coyote n'est pas malade, le test est négatif dans 95\,\% des cas.
\end{itemize}
\setlength\parindent{0cm}

\bigskip

\textbf{Partie A}

\medskip

Des vétérinaires capturent un coyote d'Oklahoma au hasard et lui font subir un test pour l'ehrlichiose.

On considère les évènements suivants :

\setlength\parindent{1cm}
\begin{itemize}
\item[$\bullet~~$]$M$: \og le coyote est malade \fg;
\item[$\bullet~~$]$T$: \og le test du coyote est positif \fg.
\end{itemize}
\setlength\parindent{0cm}

On note $\overline{M}$ et $\overline{T}$ respectivement les évènements contraires de $M$ et $T$.

\medskip

\begin{enumerate}
\item Recopier et compléter l'arbre pondéré ci-dessous qui modélise la situation.

\begin{center}
\pstree[treemode=R,nodesepA=0pt,nodesepB=2.5pt,treesep=1cm,levelsep=2.5cm]{\TR{}}
{\pstree{\TR{$M$~}\taput{\ldots}}
	{\TR{$T$}\taput{\ldots}
	\TR{$\overline{T}$}\tbput{\ldots}
	}
\pstree{\TR{$\overline{M}$~}\tbput{\ldots}}
	{\TR{$T$}\taput{\ldots}
	\TR{$\overline{T}$}\tbput{\ldots}
	}
}
\end{center}

\item Déterminer la probabilité que le coyote soit malade et que son test soit positif.
\item Démontrer que la probabilité de $T$ est égale à $0,694$.
\item On appelle \og valeur prédictive positive du test \fg{} la probabilité que le coyote soit effectivement malade sachant que son test est positif.

Calculer la valeur prédictive positive du test. On arrondira le résultat au millième.
\item  
\begin{enumerate}
\item Par analogie avec la question précédente, proposer une définition de la \og valeur prédictive négative du test \fg{} et calculer cette valeur en arrondissant au millième.
\item Comparer les valeurs prédictives positive et négative du test, et interpréter.
	\end{enumerate}	
\end{enumerate}

\bigskip

\textbf{Partie B}

\medskip

On rappelle que la probabilité qu'un coyote capturé au hasard présente un test positif est de $0,694$.

\medskip

\begin{enumerate}
\item Lorsqu'on capture au hasard cinq coyotes, on assimile ce choix à un tirage avec remise.

On note $X$ la variable aléatoire qui à un échantillon de cinq coyotes capturés au hasard associe le nombre de coyotes dans cet échantillon ayant un test positif.
\begin{enumerate}
\item Quelle est la loi de probabilité suivie par $X$ ? Justifier et préciser ses paramètres.
\item Calculer la probabilité que dans un échantillon de cinq coyotes capturés au hasard, un seul ait un test positif. On arrondira le résultat au centième.
\item Un vétérinaire affirme qu'il y a plus d'une chance sur deux qu'au moins quatre coyotes sur cinq aient un test positif : cette affirmation est-elle vraie ? Justifier la réponse.
	\end{enumerate}	
\item Pour tester des médicaments, les vétérinaires ont besoin de disposer d'un coyote présentant un test positif. Combien doivent-ils capturer de coyotes pour que la probabilité qu'au moins l'un d'entre eux présente un test positif soit supérieure à $0,99$ ?
\end{enumerate}

\bigskip

\textbf{\textsc{Exercice 2} \quad (7 points)\hfill Thèmes : fonctions numériques et suites}

\medskip

\emph{Cet exercice est un questionnaire à choix multiples. Pour chacune des questions suivantes, une seule des quatre réponses proposées est exacte. Une réponse fausse, une réponse multiple ou l'absence de réponse à une question ne rapporte ni n'enlève de point.\\
Pour répondre, indiquer sur la copie le numéro de la question et la lettre de la réponse choisie.\\Aucune justification n'est demandée.}

\medskip

Pour les questions 1 à 3 ci-dessous, on considère une fonction $f$ définie et deux fois dérivable sur $\R$. La courbe de sa fonction dérivée $f'$ est donnée ci-dessous.

On admet que $f'$ admet un maximum en $- \dfrac{3}{2}$ et que sa courbe coupe l'axe des abscisses au point de coordonnées $\left(- \dfrac12~;~0\right)$.

\begin{minipage}{0.5\linewidth}
\textbf{Question 1 }:

\textbf{a.~~} La fonction $f$ admet un maximum en $- \dfrac{3}{2}$ ;

\textbf{b.~~}  La fonction $f$ admet un maximum en $- \dfrac{1}{2}$ ;

\textbf{c.~~} La fonction $f$ admet un minimum en $- \dfrac{1}{2}$; 

\textbf{d.~~}  Au point d'abscisse $-1$, la courbe de la
fonction $f$ admet une tangente horizontale.
\end{minipage}\hfill
\begin{minipage}{0.48\linewidth}
\begin{center}
On rappelle que la courbe ci-dessous représente la fonction dérivée $f'$ de $f$.
\end{center}

\psset{unit=1.2cm}
\begin{pspicture*}(-5.2,-2.6)(1,1)
\psgrid[gridlabels=0pt,subgriddiv=4,gridwidth=0.25pt,subgridwidth=0.15pt]
\psaxes[linewidth=1.25pt,labelFontSize=\scriptstyle]{->}(0,0)(-5.2,-2.6)(1,1)
\psplot[plotpoints=2000,linewidth=1.25pt,linecolor=red]{-5}{3}{2 x mul 1 add 2.71828 x exp mul neg}
\end{pspicture*}
\end{minipage}

\medskip

\textbf{Question 2 }:

\begin{center}
\begin{tabularx}{\linewidth}{X X}
\textbf{a.~~}La fonction $f$ est convexe sur $\left]- \infty~;~- \dfrac32\right[$ ;&
\textbf{b.~~}La fonction $f$ est convexe sur $\left]- \infty~;~- \dfrac12\right[$ ;\\
\textbf{c.~~}La courbe $\mathcal{C}_f$ représentant la fonction $f$ n'admet pas de point d'inflexion ; &\textbf{d.~~}La fonction $f$ est concave sur $\left] - \infty~;~- \dfrac12\right[$.
\end{tabularx}
\end{center}

\medskip

\textbf{Question 3}:

La dérivée seconde $f''$ de la fonction $f$ vérifie :

\begin{center}
\begin{tabularx}{\linewidth}{X X}
\textbf{a.~~} $f''(x) \geqslant  0$ pour $x \in \left]-\infty~;~- \dfrac12\right[$ ; &\textbf{b.~~} $f''(x) \geqslant  0$ pour $x \in [- 2~;~- 1]$ ;\\
\textbf{c.~~} $f''\left(- \dfrac32 \right) = 0$ ;&\textbf{d.~~} $f''(- 3) = 0$.
\end{tabularx}
\end{center}

%Le nombre dérivé $f''\left(- \dfrac32 \right) = 0$ (la tangente à la courbe en ce point est horizontale). Réponse \textbf{c.}
\textbf{Question 4 :}

\medskip

On considère trois suites $\left(u_n\right)$, $\left(v_n\right)$ et $\left(w_n\right)$. On sait que, pour tout entier naturel $n$, on a : $u_n \leqslant v_n\leqslant  w_n$ et de plus: $\displaystyle\lim_{n \to + \infty} u_n= 1$ et $\displaystyle\lim_{n \to + \infty} w_n= 3$.

On peut alors affirmer que :

\begin{center}
\begin{tabularx}{\linewidth}{X X}
\textbf{a.~~} la suite $\left(v_n\right)$ converge ;&\textbf{b.~~} Si la suite 
$\left(u_n\right)$ est croissante alors la suite $\left(v_n\right)$ est minorée par $u_0$ ;\\
\textbf{c.~~} $1 \leqslant  v_0 \leqslant 3$ ;&\textbf{d.~~} la suite $\left(v_n\right)$ diverge.
\end{tabularx}
\end{center}

%Si $\left(u_n\right)$ est croissante elle est minorée par $u_0$, donc quel que soit $n \in \N$, \: $u_0< u_n \leqslant v_n$, donc $u_0 \leqslant v_n$. Réponse \textbf{b.}
\medskip

\textbf{Question 5}:

\medskip

On considère une suite $\left(u_n\right)$ telle que, pour tout entier naturel $n$ non nul: $u_n \leqslant  u_{n+1}  \leqslant \dfrac1n$.

On peut alors affirmer que : 

\begin{center}
\begin{tabularx}{\linewidth}{X X}
\textbf{a.~~}la suite $\left(u_n\right)$ diverge ;&\textbf{b.~~}la suite $\left(u_n\right)$ converge ;\\
\textbf{c.~~}$\displaystyle\lim_{n \to + \infty} u_n =  0$ ;&\textbf{d.~~} $\displaystyle\lim_{n \to + \infty} u_n =  1$.
\end{tabularx}
\end{center}
%La suite $\left(u_n\right)$ est croissante et comme $\displaystyle\lim_{n \to + \infty} \dfrac{1}{n} = 0$, elle est majorée par 0 : elle converge donc vers une limite $\ell$ telle que $\ell \leqslant 0$, mais on ne connait pas cette limite. Réponse \textbf{b.}
\medskip

\textbf{Question 6}:

\medskip

On considère $\left(u_n\right)$ une suite réelle telle que pour tout entier naturel $n$, on a : $n < u_n < n + 1$. 

On peut affirmer que:

\begin{center}
\begin{tabularx}{\linewidth}{X X}
\textbf{a.~~}Il existe un entier naturel $N$ tel que $u_N$ est un entier ;&\textbf{b.~~} la suite $\left(u_n\right)$ est croissante ;\\
\textbf{c.~~} la suite $\left(u_n\right)$ est convergente ;&\textbf{d.~~}La suite $\left(u_n\right)$ n'a pas de limite.
\end{tabularx}
\end{center}
%Au rang $n$ : $n < u_n < n + 1$ ;
%
%Au rang $n + 1$ : $n+1 < u_{n+1} < n + 2$ ; donc $u_n < n+1 < u_{n+1}$ : la suite est croissante. Réponse \textbf{b.}
\bigskip

\textbf{\textsc{Exercice 3} \quad (7 points)\hfill Thème : géométrie dans l'espace}

\medskip

\begin{minipage}{0.6\linewidth}
On considère un cube ABCDEFGH
et on appelle K le milieu du segment [BC].

On se place dans le repère 
$\left(\text{A}~;~\vect{\text{AB}},~\vect{\text{AD}},~\vect{\text{AE}}\right)$ 
et on considère  le tétraèdre EFGK.

On rappelle que le volume d'un tétraèdre est donné par: 

\[V = \dfrac13 \times \mathcal{B} \times h\]

où $\mathcal{B}$ désigne l'aire d'une base et $h$ la hauteur relative à cette base.
\end{minipage} \hfill
\begin{minipage}{0.35\linewidth}
\psset{unit=1cm}
\begin{pspicture}(5.5,5.8)
\psframe(0.2,0.2)(3.7,3.7)%ABFE
\psline(3.7,0.2)(5,1.9)(5,5.4)(3.7,3.7)(4.35,1.05)%BCGF
\psline(5,5.4)(1.5,5.4)(0.2,3.7)%GHE
\psline[linestyle=dashed](0.2,0.2)(1.5,1.9)(5,1.9)%ADC
\psline[linestyle=dashed](1.5,1.9)(1.5,5.4)%DH
\pspolygon[linestyle=dotted,linewidth=1.25pt](0.2,3.7)(5,5.4)(4.35,1.05)%EGK
\uput[dl](0.2,0.2){\small A}\uput[dr](3.7,0.2){\small B}\uput[r](5,1.9){\small C}
\uput[dr](1.5,1.9){\small D}\uput[l](0.2,3.7){\small E}\uput[r](3.7,3.7){\small F}
\uput[ur](5,5.4){\small G}\uput[ul](1.5,5.4){\small H}\uput[dr](4.35,1.05){\small K}
\end{pspicture}
\end{minipage}

\medskip

\begin{enumerate}
%%%%%%%
\item Préciser les coordonnées des points E, F{}, G et K.
\item Montrer que le vecteur $\vect{n}\begin{pmatrix}\phantom{-}2\\-2\\\phantom{-}1\end{pmatrix}$ est orthogonal au plan (EGK).
\item Démontrer que le plan (EGK) admet pour équation cartésienne : $2x - 2y + z - 1 = 0.$
\item Déterminer une représentation paramétrique de la droite $(d)$ orthogonale au plan (EGK)
passant par F{}.
\item Montrer que le projeté orthogonal L de F sur le plan (EGK) a pour coordonnées $\left(\frac59~;~\frac49~;~\frac79\right)$.
\item Justifier que la longueur LF est égale à $\dfrac23$.
\item Calculer l'aire du triangle EFG. En déduire que le volume du tétraèdre EFGK est égal à $\dfrac16$.
\item Déduire des questions précédentes l'aire du triangle EGK.
\item On considère les points P milieu du segment [EG], M milieu du segment [EK] et N milieu du segment[GK]. Déterminer le volume du tétraèdre FPMN.
\end{enumerate}

\bigskip

\textbf{\textsc{Exercice 4} \quad (7 points)\hfill Thèmes : fonctions numériques, fonction exponentielle}

\begin{center}

\textbf{Partie A : études de deux fonctions}

\end{center}

On considère les deux fonctions $f$ et $g$ définies sur l'intervalle $[0~;~+\infty[$ par:

\[f(x) = 0,06\left(-x^2 +13,7x\right)\quad  \text{et}\quad  g(x) = (-0,15x + 2,2)\text{e}^{0,2x} - 2,2.\]

On admet que les fonctions $f$ et $g$ sont dérivables et on note $f'$ et $g'$ leurs fonctions dérivées respectives.

\medskip

\begin{enumerate}
\item On donne le tableau de variations complet de la fonction $f$ sur l'intervalle $[0~;~+\infty[$.

\begin{center}
\psset{unit=1cm,arrowsize=2pt 3}
\begin{pspicture}(6.5,2)
\psframe(6.5,2)\psline(0,1.5)(6.5,1.5)\psline(1.5,0)(1.5,2)
\uput[u](0.75,1.4){$x$} \uput[u](1.6,1.4){$0$} \uput[u](4,1.4){$6,85$} \uput[u](6,1.4){$+ \infty$} 
\rput(0.75,0.75){$f(x)$}\uput[u](1.65,0){$0$}\uput[d](4,1.5){$f(6,85)$}\uput[u](6,0){$- \infty$}
\psline{->}(1.75,0.25)(3.4,1.25)\psline{->}(4.6,1.25)(5.9,0.35)
\end{pspicture}
\end{center}

	\begin{enumerate}
		\item Justifier la limite de $f$ en $+\infty$.
		\item Justifier les variations de la fonction $f$.
		\item Résoudre l'équation $f(x) = 0$.
\end{enumerate}
\item 
	\begin{enumerate}
		\item Déterminer la limite de $g$ en $+\infty$.
		\item Démontrer que, pour tout réel $x$ appartenant à $[0~;~+\infty[$ on a : $g'(x) = (- 0,03x + 0,29)\text{e}^{0,2x}$.
		\item Étudier les variations de la fonction $g$ et dresser son tableau de variations sur $[0~;~+\infty[$.
		
Préciser une valeur approchée à $10^{-2}$ près du maximum de $g$.
		\item Montrer que l'équation $g(x) = 0$ admet une unique solution non nulle et déterminer, à $10^{-2}$ près, une valeur approchée de cette solution.
	\end{enumerate}
\end{enumerate}

\bigskip

\textbf{Partie B : trajectoires d'une balle de golf}

\medskip

Pour frapper la balle, un joueur de golf utilise un instrument appelé \og club\fg{} de golf.

On souhaite exploiter les fonctions $f$ et $g$ étudiées en Partie A pour modéliser de deux façons différentes la trajectoire d'une balle de golf. On suppose que le terrain est parfaitement plat.

On admettra ici que $13,7$ est la valeur qui annule la fonction $f$ et une approximation de la valeur qui annule la fonction $g$.

On donne ci-dessous les représentations graphiques de $f$ et $g$ sur l'intervalle [0~;~13,7].

\begin{center}
\psset{unit=1cm}
\begin{pspicture*}(-1,-1)(14,3.5)
\psgrid[gridlabels=0pt,subgriddiv=1,gridwidth=0.2pt]
\psaxes[linewidth=1.25pt,Dx=20,Dy=20]{->}(0,0)(-1,-1)(14,3.5)
\psplot[plotpoints=1000,linewidth=1.25pt,linecolor=red]{0}{13.7}{13.7 x mul x dup mul sub 0.06 mul}\uput[ul](3,2){\red $\mathcal{C}_f$}
\psplot[plotpoints=1000,linewidth=1.25pt,linecolor=blue]{0}{13.7}{2.2 0.15 x mul sub 2.71828 0.2 x mul exp mul 2.2 sub}\uput[ur](12,2.2){\blue $\mathcal{C}_g$}
\uput[dl](0,0){0}\uput[d](1,0){1}\uput[l](0,1){1}\uput[d](13.7,0){13,7}
\end{pspicture*}
\end{center}

Pour $x$ représentant la distance horizontale parcourue par la balle en dizaine de yards après la frappe, (avec $0 < x < 13,7$), $f(x)$ (ou $g(x)$ selon le modèle) représente la hauteur correspondante de la balle par rapport au sol, en dizaine de yards (1 yard correspond à environ $0,914$ mètre).

On appelle \og angle de décollage \fg{} de la balle, l'angle entre l'axe des abscisses et la tangente à la courbe ($\mathcal{C}_f$ ou $\mathcal{C}_g$ selon le modèle) en son point d'abscisse $0$. Une mesure de l'angle de décollage de la balle est
un nombre réel $d$ tel que $\tan (d)$ est égal au coefficient directeur de cette tangente.

De même, on appelle \og angle d'atterrissage \fg{} de la balle, l'angle entre l'axe des abscisses et la tangente
à la courbe ($\mathcal{C}_f$ ou $\mathcal{C}_g$ selon le modèle) en son point d'abscisse $13,7$. Une mesure de l'angle d'atterrissage de la balle est un nombre réel $a$ tel que $\tan (a)$ est égal à l'opposé du coefficient directeur de cette tangente.

Tous les angles sont mesurés en degré.

\begin{center}
\begin{tabularx}{\linewidth}{|p{5.75cm}|X|}\hline
Le schéma illustre les angles de décollage et d'atterrissage associés à la courbe 
$\mathcal{C}_f$&Le schéma illustre les angles de décollage et d'atterrissage
associés à la courbe $\mathcal{C}_g$.\\ \hline
\psset{unit=0.4cm}
\begin{pspicture}(-0.3,-0.75)(14,3.2)
%\psgrid
\psaxes[linewidth=1.25pt,Dx=20,Dy=10](0,0)(-0.3,-0.3)(14,3)
\psplot[plotpoints=1000]{0}{2}{0.822 x mul}
\psplot[plotpoints=1000,linewidth=1.25pt,linecolor=red]{0}{13.7}{13.7 x mul x dup mul sub 0.06 mul}\uput[ul](3,2){\red $\mathcal{C}_f$}
\psline(13.7,0)(11,2)
\psarc(0,0){0.7}{0}{40}\rput(1.2,0.5){\footnotesize $d$}
\psarc(13.7,0){0.7}{140}{180}\rput(12.4,0.5){\footnotesize $a$}
\uput[d](13.7,0){\footnotesize 13,7}
\end{pspicture}&\psset{unit=0.5cm}
\begin{pspicture}(-0.3,-0.75)(14,3.2)
%\psgrid
\psaxes[linewidth=1.25pt,Dx=20,Dy=10](0,0)(-0.3,-0.5)(14,3)
\psplot[plotpoints=1000,linewidth=1.25pt,linecolor=blue]{0}{13.7}{2.2 0.15 x mul sub 2.71828 0.2 x mul exp mul 2.2 sub}
\psline(0,0)(10.4,2.8)
\psline(13.7,0)(12,2.9)
\psarc(0,0){1}{0}{20}\rput(2,0.25){\footnotesize $d$}
\psarc(13.7,0){1}{120}{180}\rput(12.4,0.5){\footnotesize $a$}
\uput[d](13.7,0){\footnotesize 13,7}
\end{pspicture}\\ \hline
\end{tabularx}
\end{center}

\medskip

\begin{enumerate}
\item \emph{Première modélisation} : on rappelle qu'ici, l'unité étant la dizaine de yards, $x$ représente la distance horizontale parcourue par la balle après la frappe et $f(x)$ la hauteur correspondante de la balle.

Selon ce modèle :
	\begin{enumerate}
		\item Quelle est la hauteur maximale, en yard, atteinte par la balle au cours de sa trajectoire ?
		\item Vérifier que $f'(0) = 0,822$.
		\item Donner une mesure en degré de l'angle de décollage de la balle, arrondie au dixième. (On pourra éventuellement utiliser le tableau ci-dessous).
		\item Quelle propriété graphique de la courbe $\mathcal{C}_f$ permet de justifier que les angles de décollage et d'atterrissage de la balle sont égaux ?
	\end{enumerate}
\item \emph{Seconde modélisation } : on rappelle qu'ici, l'unité étant la dizaine de yards, $x$ représente la distance horizontale parcourue par la balle après la frappe et $g(x)$ la hauteur correspondante de la balle.

Selon ce modèle :

	\begin{enumerate}
		\item Quelle est la hauteur maximale, en yard, atteinte par la balle au cours de sa trajectoire ?

On précise que $g'(0) = 0,29$ et $g'(13,7) \approx -1,87$.
		\item Donner une mesure en degré de l'angle de décollage de la balle, arrondie au dixième. (On pourra éventuellement utiliser le tableau ci-dessous).
		\item Justifier que $62$ est une valeur approchée, arrondie à l'unité près, d'une mesure en degré de l'angle d'atterrissage de la balle.
	\end{enumerate}
\medskip

\textbf{Tableau :} extrait d'une feuille de calcul donnant une mesure en degré d'un angle quand on connait sa tangente :

\begin{center}
\begin{tabularx}{\linewidth}{|c|*{13}{>{\centering \arraybackslash}X|}}\hline
&A	&B	&C	&D	&E	&F	&G	&H	&I	&J	&K	&L	&M\\ \hline
1&$\tan (\theta)$&0,815&0,816&0,817&0,818&0,819&0,82&0,821&0,822&0,823&0,824&0,825&0,826\\ \hline
2&\scriptsize $\theta$ en degrés&39,18&39,21&39,25&39,28&39,32&39,35&39,39&39,42&39,45&39,49&39,52&39,56\\ \hline
3&	&&&&&&&&&&&&\\ \hline
4&$\tan (\theta)$&0,285& 0,286& 0,287& 0,288& 0,289&0,29&0,291&0,292&
0,293&0,294&0,295 &0,296\\ \hline
5&\scriptsize  $\theta$ en degrés&15,91 &15,96&16,01& 16,07& 16,12& 16,17& 16,23& 16,28& 16,33& 16,38& 16,44& 16,49\\ \hline
\end{tabularx}
\end{center}
\end{enumerate}

\bigskip

\begin{center}
\textbf{Partie C : interrogation des modèles}
\end{center}

À partir d'un grand nombre d'observations des performances de joueurs professionnels, on a obtenu les résultats moyens suivants:

\begin{center}
\begin{tabularx}{\linewidth}{|*{4}{>{\centering \arraybackslash \footnotesize}X|}}\hline
Angle de décollage en degré&Hauteur maximale en yard&Angle d'atterrissage en degré&Distance horizontale en yard au point de chute\\ \hline
24&32&52&137\\ \hline
\end{tabularx}
\end{center}

Quel modèle, parmi les deux étudiés précédemment, semble le plus adapté pour décrire la frappe de la balle par un joueur professionnel ? La réponse sera justifiée.
\end{document}