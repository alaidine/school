\documentclass[11pt]{article}
\usepackage[T1]{fontenc}
\usepackage[utf8]{inputenc}
\usepackage{fourier}
\usepackage[scaled=0.875]{helvet}
\renewcommand{\ttdefault}{lmtt}
\usepackage{makeidx}
\usepackage{amsmath,amssymb}
\usepackage{fancybox}
\usepackage[normalem]{ulem}
\usepackage{pifont}
\usepackage{lscape}
\usepackage{multicol}
\usepackage{mathrsfs}
\usepackage{tabularx}
\usepackage{multirow}
\usepackage{enumitem}
\usepackage{textcomp} 
\newcommand{\euro}{\eurologo{}}
%Tapuscrit : Denis Vergès
%Relecture : François Hache
\usepackage{pst-plot,pst-tree,pstricks,pst-node,pst-text}
\usepackage{pst-eucl}
\usepackage{pstricks-add}
\newcommand{\R}{\mathbb{R}}
\newcommand{\N}{\mathbb{N}}
\newcommand{\D}{\mathbb{D}}
\newcommand{\Z}{\mathbb{Z}}
\newcommand{\Q}{\mathbb{Q}}
\newcommand{\C}{\mathbb{C}}
\usepackage[left=3.5cm, right=3.5cm, top=3cm, bottom=3cm]{geometry}
\newcommand{\vect}[1]{\overrightarrow{\,\mathstrut#1\,}}
\renewcommand{\theenumi}{\textbf{\arabic{enumi}}}
\renewcommand{\labelenumi}{\textbf{\theenumi.}}
\renewcommand{\theenumii}{\textbf{\alph{enumii}}}
\renewcommand{\labelenumii}{\textbf{\theenumii.}}
\def\Oij{$\left(\text{O}~;~\vect{\imath},~\vect{\jmath}\right)$}
\def\Oijk{$\left(\text{O}~;~\vect{\imath},~\vect{\jmath},~\vect{k}\right)$}
\def\Ouv{$\left(\text{O}~;~\vect{u},~\vect{v}\right)$}
\usepackage{fancyhdr}
\usepackage[dvips]{hyperref}
\hypersetup{%
pdfauthor = {APMEP},
pdfsubject = {Baccalauréat spécialité},
pdftitle = {Polynésie 2 juin 2021},
allbordercolors = white,
pdfstartview=FitH} 
\usepackage[french]{babel}
\usepackage[np]{numprint}
\begin{document}
\setlength\parindent{0mm}
\rhead{\textbf{A. P{}. M. E. P{}.}}
\lhead{\small Baccalauréat spécialité}
\lfoot{\small{Polynésie}}
\rfoot{\small{2 juin 2021}}
\pagestyle{fancy}
\thispagestyle{empty}

\begin{center}{\Large\textbf{\decofourleft~Baccalauréat Polynésie 2 juin 2021~\decofourright\\[6pt] ÉPREUVE D'ENSEIGNEMENT DE SPÉCIALITÉ \no 2}}
\end{center}

\vspace{0,25cm}

Le candidat traite \textbf{4 exercices} : les exercices 1, 2 et 3 communs à tous les candidats et un seul des deux exercices A ou B.

\bigskip

\textbf{\textsc{Exercice 1} \hfill 5 points}

\textbf{Commun à tous les candidats}

\medskip

On considère la suite $\left(u_n\right)$ définie par $u_0 = \np{10000}$ et pour tout entier naturel $n$ : 

\[u_{n+1} = 0,95u_n + 200.\]

\begin{enumerate}
\item Calculer $u_1$ et vérifier que $u_2 = \np{9415}$. 
\item 
	\begin{enumerate}
		\item Démontrer, à l'aide d'un raisonnement par récurrence, que pour tout entier naturel $n$ :

\[u_n > \np{4000}.\]

		\item On admet que la suite $\left(u_n\right)$ est décroissante. Justifier qu'elle converge.
	\end{enumerate}
\item  Pour tout entier naturel $n$, on considère la suite $\left(v_n\right)$ définie par: $v_n = u_n - \np{4000}$.
	\begin{enumerate}
		\item Calculer $v_0$.
		\item Démontrer que la suite $\left(v_n\right)$ est géométrique de raison égale à $0,95$.
		\item En déduire que pour tout entier naturel $n$ :

\[u_n = \np{4000} + \np{6000} \times 0,95^n.\]

		\item Quelle est la limite de la suite $\left(u_n\right)$ ? Justifier la réponse.
	\end{enumerate}
\item En 2020, une espèce animale comptait \np{10000} individus. L'évolution observée les années précédentes conduit à estimer qu'à partir de l'année 2021, cette population baissera de 5\,\% chaque début d'année.

Pour ralentir cette baisse, il a été décidé de réintroduire $200$ individus à la fin de chaque année, à partir de 2021.

Une responsable d'une association soutenant cette stratégie affirme que : \og l'espèce ne devrait pas s'éteindre, mais malheureusement, nous n'empêcherons pas une disparition de plus de la moitié de la population \fg.

Que pensez-vous de cette affirmation ? Justifier la réponse.
\end{enumerate}

\bigskip

\textbf{\textsc{Exercice 2} \hfill 5 points}

\textbf{Commun à tous les candidats}

\medskip

Un test est mis au point pour détecter une maladie dans un pays.

Selon les autorités sanitaires de ce pays, 7\,\% des habitants sont infectés par cette maladie. 

Parmi les individus infectés, 20\,\% sont déclarés négatifs.

Parmi les individus sains, 1\,\% sont déclarés positifs.

Une personne est choisie au hasard dans la population.

On note :

\setlength\parindent{9mm}
\begin{itemize}
\item[$\bullet~~$]$M$ l'évènement : \og la personne est infectée par la maladie\fg{} ;
\item[$\bullet~~$]$T$ l'évènement: \og le test est positif \fg.
\end{itemize}
\setlength\parindent{0mm}

\medskip

\begin{enumerate}
\item Construire un arbre pondéré modélisant la situation proposée.
\item 
	\begin{enumerate}
		\item Quelle est la probabilité pour que la personne soit infectée par la maladie et que son test soit positif?
		\item Montrer que la probabilité que son test soit positif est de \np{0,0653}.
	\end{enumerate}
\item On sait que le test de la personne choisie est positif. 

Quelle est la probabilité qu'elle soit infectée ?

On donnera le résultat sous forme approchée à $10^{-2}$ près.
\item On choisit dix personnes au hasard dans la population. La taille de la population de ce pays permet d'assimiler ce prélèvement à un tirage avec remise.

On note $X$ la variable aléatoire qui comptabilise le nombre d'individus ayant un test positif parmi les dix personnes.
	\begin{enumerate}
		\item Quelle est la loi de probabilité suivie par $X$ ? Préciser ses paramètres.
		\item Déterminer la probabilité pour qu'exactement deux personnes aient un test positif.

On donnera le résultat sous forme approchée à $10^{-2}$ près.
	\end{enumerate}
\item Déterminer le nombre minimum de personnes à tester dans ce pays pour que la probabilité qu'au moins une de ces personnes ait un test positif, soit supérieure à 99\,\%.
\end{enumerate}

\bigskip

\textbf{\textsc{Exercice 3} \hfill 5 points}

\textbf{Commun à tous les candidats}

Dans l'espace, on considère le cube ABCDEFGH d'arête de longueur égale à 1.

On munit l'espace du repère orthonormé $\left(\text{A}~;~\vect{\text{AB}},~\vect{\text{AD}},~\vect{\text{AE}}\right)$.

On considère le point M tel que $\vect{\text{BM}} = \dfrac{1}{3}\vect{\text{BH}}$.

\begin{center}
\psset{unit=1cm}
\begin{pspicture}(-4,-2.75)(4.5,5.5)
\psline[linewidth=1.25pt]{->}(0,5.5)\uput[u](0,5.5){$z$}
\psline[linewidth=1.25pt]{->}(4.5,-1.8)\uput[dr](4.5,-1.8){$y$}
\psline[linewidth=1.25pt]{->}(-4,-2)\uput[dl](-4,-2){$x$}
\pspolygon[linewidth=1.25pt](-2.4,-1.16)(0.5,-2.4)(2.85,-1.1)(2.85,2.6)(0,3.6)(-2.4,2.3)%BCDHEF
\psline[linewidth=1.25pt](-2.4,2.3)(0.5,1.06)(2.85,2.6)%FGH
\psline[linewidth=1.25pt](0.5,1.06)(0.5,-2.4)%GC
\psline[linestyle=dashed,linewidth=1.25pt](-2.4,-1.16)(2.85,2.6)
\psdots(-0.65,0.093) \uput[ul](-0.65,0.093){M}
\uput[d](-2.4,-1.16){B} \uput[d](0.5,-2.4){C} \uput[d](2.85,-1.1){D} 
\uput[r](2.85,2.6){H} \uput[ur](0,3.6){E} \uput[l](-2.4,2.3){F} 
\uput[d](0,0){A} \uput[u](0.5,1.06){G}
\end{pspicture}
\end{center}


\begin{enumerate}
\item Par lecture graphique, donner les coordonnées des points B, D, E, G et H.
\item 
	\begin{enumerate}
		\item Quelle est la nature du triangle EGD ? Justifier la réponse.
		\item On admet que l'aire d'un triangle équilatéral de côté $c$ est égale à $\dfrac{\sqrt{3}}{4}c^2$.
		
Montrer que l'aire du triangle EGD est égale à $\dfrac{\sqrt{3}}{2}$.
	\end{enumerate}
\item Démontrer que les coordonnées de M sont $\left(~\dfrac{2}{3}~;~\dfrac{1}{3}~; ~\dfrac{1}{3}\right)$.
\item
	\begin{enumerate}
		\item Justifier que le vecteur $\vect{n}(-1~;~1~;~1)$ est normal au plan (EGD).
		\item En déduire qu'une équation cartésienne du plan (EGD) est: $- x + y + z - 1 = 0$.
		\item Soit $\mathcal{D}$ la droite orthogonale au plan (EGD) et passant par le point M. 
		
Montrer qu'une représentation paramétrique de cette droite est:

\renewcommand\arraystretch{1.7}
\[\mathcal{D} \, :\, \left\{\begin{array}{l c l}
x&=&\dfrac{2}{3} - t\\
y&=&\dfrac{1}{3} + t\\
z&=&\dfrac{1}{3} + t
\end{array}\right., \, t \in \R\]
\renewcommand\arraystretch{1}
	\end{enumerate}
\item Le cube ABCDEFGH est représenté ci-dessus selon une vue qui permet de mieux
percevoir la pyramide GEDM, en gris sur la figure :

\begin{center}
\psset{unit=1cm}
\begin{pspicture}(-7,-0.5)(2,6.8)
\psline[linewidth=2pt,fillstyle=solid,fillcolor=lightgray](-4.2,0.6)(-0.5,2.6)(0,4.2)(-2.9,6.2)%DMEG
\pspolygon[linestyle=dotted,linewidth=2pt,fillstyle=solid,fillcolor=gray](-2.9,6.2)(-4.2,0.6)(-0.5,2.6)%GDM
\psline[linewidth=2pt](-4.2,0.6)(0,4.2)(-2.9,6.2)%DEG
\psline[linewidth=1.25pt]{->}(1.9,2.3)\uput[ur](1.9,2.3){$x$}
\psline[linewidth=1.25pt]{->}(-6.4,0.9)\uput[ul](-6.4,0.9){$y$}
\psline[linewidth=1.25pt]{->}(0,6.2)\uput[u](0,6.2){$z$}
\pspolygon[linewidth=1.25pt](0,0)(1.3,1.55)(1.3,5.8)(-2.9,6.2)(-4.2,4.6)(-4.2,0.6)%ABFGHDA
\psline[linewidth=1.25pt](-4.2,4.6)(0,4.2)(1.3,5.8)%HEF

\psline[linestyle=dashed,linewidth=1.25pt](1.3,1.55)(-2.9,2.1)(-2.9,6.2)%BCG
\psline[linestyle=dashed,linewidth=1.25pt](-2.9,2.1)(-4.2,0.6)%CD

\psline[linestyle=dotted,linewidth=2pt](-0.5,2.6)(0,4.2)%ME

\uput[d](-0.5,2.6){M}
\uput[r](1.3,1.55){B} \uput[ul](-2.9,2.1){C} \uput[d](-4.2,0.6){D} 
\uput[l](-4.2,4.6){H} \uput[r](0,4.2){E} \uput[r](1.3,5.8){F} 
\uput[d](0,0){A} \uput[u](-2.9,6.2){G}
\end{pspicture}
\end{center}

Le but de cette question est de calculer le volume de la pyramide GEDM.

	\begin{enumerate}
		\item Soit K, le pied de la hauteur de la pyramide GEDM issue du point M.
		
Démontrer que les coordonnées du point K sont $\left(\dfrac{1}{3}~;~\dfrac{2}{3}~;~\dfrac{2}{3}\right)$.
		\item En déduire le volume de la pyramide GEDM.
		
\emph{On rappelle que le volume $V$ d'une pyramide est donné par la formule }

\emph{$V = \dfrac{b \times h}{3}$ où 
$b$ désigne l'aire d'une base et $h$ la hauteur associée}.
	\end{enumerate}
\end{enumerate}

\bigskip

\textbf{EXERCICE au choix du candidat \hfill 5 points}

\medskip

\textbf{Le candidat doit traiter un seul des deux exercices A ou B.}

\textbf{Il indique sur sa copie l'exercice choisi: exercice A ou exercice B.}

\textbf{Pour éclairer son choix, les principaux domaines abordés par chaque exercice sont indiqués dans un encadré.}

\medskip

\textbf{EXERCICE A}

\medskip

\begin{tabularx}{\linewidth}{|X|}\hline
\textbf{Principaux domaines abordés : Fonction exponentielle, convexité, dérivation, }\\ \textbf{équations différentielles}\\ \hline
\end{tabularx}

\medskip

Cet exercice est composé de trois parties indépendantes.

\medskip

On a représenté ci-dessous, dans un repère orthonormé, une portion de la courbe
représentative $\mathcal{C}$ d'une fonction $f$ définie sur $\R$ :

\begin{center}
\psset{unit=1.25cm}
\begin{pspicture*}(-3,-0.95)(3.5,3.5)
\psgrid[gridlabels=0pt,subgriddiv=1,griddots=8]
\psaxes[linewidth=1.25pt,labelFontSize=\scriptstyle]{->}(0,0)(-3,-0.95)(3.5,3.5)
\psplot[plotpoints=2000,linewidth=1.25pt,linecolor=red]{-2.25}{3.5}{x 2 add 2.71828 x exp div}
\psplot[plotpoints=200,linewidth=1.5pt,linestyle=dotted]{-1.5}{2.5}{ 2 x sub}
\uput[ul](-1.6,2){\red $\mathcal{C}$}\uput[ur](0,2){A}\uput[ur](2,0){B}
\uput[d](3.4,0){\small $x$}\uput[l](0,3.4){\small $y$}
\end{pspicture*}
\end{center}

On considère les points A(0~;~2) et B(2~;~0).

\bigskip

\begin{center}\textbf{Partie 1}\end{center}

Sachant que la courbe $\mathcal{C}$ passe par A et que la droite (AB) est la tangente à la courbe $\mathcal{C}$ au point A, donner par lecture graphique : 

\medskip

\begin{enumerate}
\item La valeur de $f(0)$ et celle de $f'(0)$.
\item Un intervalle sur lequel la fonction $f$ semble convexe.
\end{enumerate}

\bigskip

\begin{center}\textbf{Partie 2}\end{center}

On note $(E)$ l'équation différentielle 

\[y' = -y + \text{e}^{-x}.\]

On admet que $g :\,  x \longmapsto  x\text{e}^{-x}$ est une solution particulière de $(E)$.

\medskip

\begin{enumerate}
\item Donner toutes les solutions sur $\R$ de l'équation différentielle $(H)$ : 
$y' = -y$.
\item En déduire toutes les solutions sur $\R$ de l'équation différentielle $(E)$.
\item Sachant que la fonction $f$ est la solution particulière de $(E)$ qui vérifie $f(0) = 2$, déterminer une expression de $f(x)$ en fonction de $x$.
\end{enumerate}

\bigskip

\begin{center}\textbf{Partie 3}\end{center}

On admet que pour tout nombre réel $x$,\, $f(x) = (x + 2)\text{e}^{-x}$.

\medskip

\begin{enumerate}
\item On rappelle que $f'$ désigne la fonction dérivée de la fonction $f$.
	\begin{enumerate}
		\item Montrer que pour tout $x \in \R,\, f'(x) = (-x - 1) \text{e}^{-x}$.
		\item Étudier le signe de $f'(x)$ pour tout $x \in \R$ et dresser le tableau des variations de $f$ sur $\R$.

On ne précisera ni la limite de $f$ en $- \infty$ ni la limite de $f$ en $+ \infty$.

On calculera la valeur exacte de l'extremum de $f$ sur $\R$.
	\end{enumerate}
\item On rappelle que $f''$ désigne la fonction dérivée seconde de la fonction $f$.
	\begin{enumerate}
		\item Calculer pour tout $x \in \R,\, f''(x)$.
		\item Peut-on affirmer que $f$ est convexe sur l'intervalle $[0~;~+\infty[$?
	\end{enumerate}
\end{enumerate}

\bigskip

\textbf{EXERCICE B}

\medskip

\fbox{\textbf{Principaux domaines abordés: Fonction logarithme népérien, dérivation}}

\medskip

Cet exercice est composé de deux parties.

\medskip

Certains résultats de la première partie seront utilisés dans la deuxième.

\medskip

\begin{center}\textbf{Partie 1 : Étude d'une fonction auxiliaire}\end{center}

Soit la fonction $f$ définie sur l'intervalle [1~;~4] par : 

\[f(x) = - 30x + 50 + 35\ln x.\]

\begin{enumerate}
\item On rappelle que $f'$ désigne la fonction dérivée de la fonction $f$.
	\begin{enumerate}
		\item Pour tout nombre réel $x$ de l'intervalle [1~;~4], montrer que: 
		
\[f'(x) = \dfrac{35- 30x}{x}.\]

		\item Dresser le tableau de signe de $f'(x)$ sur l'intervalle [1~;~4].
		\item En déduire les variations de $f$ sur ce même intervalle.
	\end{enumerate}
\item Justifier que l'équation $f(x) = 0$ admet une unique solution, notée $\alpha$, sur l'intervalle [1~;~4] puis donner une valeur approchée de $\alpha$ à $10^{-3}$ près.
\item  Dresser le tableau de signe de $f(x)$ pour $x \in [1~;~4]$.
\end{enumerate}

\bigskip

\textbf{Partie 2 : Optimisation}

\medskip

Une entreprise vend du jus de fruits. Pour $x$ milliers de litres vendus, avec $x$ nombre réel de l'intervalle [1~;~4], l'analyse des ventes conduit à modéliser le bénéfice $B(x)$ par l'expression donnée en milliers d'euros par :

\[B(x) = - 15x^2 + 15x +35x \ln x.\]

\begin{enumerate}
\item D'après le modèle, calculer le bénéfice réalisé par l'entreprise lorsqu'elle vend \np{2500}~litres de jus de fruits.

On donnera une valeur approchée à l'euro près de ce bénéfice.
\item Pour tout $x$ de l'intervalle [1~;~4], montrer que $B'(x) = f(x)$ où $B'$ désigne la fonction dérivée de $B$.
\item 
	\begin{enumerate}
		\item À l'aide des résultats de la \textbf{partie 1}, donner les variations de la fonction $B$ sur l'intervalle [1~;~4].
		\item En déduire la quantité de jus de fruits, au litre près, que l'entreprise doit vendre afin de réaliser un bénéfice maximal.
	\end{enumerate}
\end{enumerate}
\end{document}