\documentclass[11pt]{article}
\usepackage[T1]{fontenc}
\usepackage[utf8]{inputenc}
\usepackage{fourier}
\usepackage[scaled=0.875]{helvet}
\renewcommand{\ttdefault}{lmtt}
\usepackage{makeidx}
\usepackage{amsmath,amssymb}
\usepackage{fancybox}
\usepackage[normalem]{ulem}
\usepackage{pifont}
\usepackage{lscape}
\usepackage{multicol}
\usepackage{mathrsfs}
\usepackage{tabularx}
\usepackage{multirow}
\usepackage{enumitem}
\usepackage{textcomp} 
\newcommand{\euro}{\eurologo{}}
%Tapuscrit : Denis Vergès
%Relecture : 
\usepackage{pst-plot,pst-tree,pstricks,pst-node,pst-text}
\usepackage{pst-eucl}
\usepackage{pstricks-add}
\newcommand{\R}{\mathbb{R}}
\newcommand{\N}{\mathbb{N}}
\newcommand{\D}{\mathbb{D}}
\newcommand{\Z}{\mathbb{Z}}
\newcommand{\Q}{\mathbb{Q}}
\newcommand{\C}{\mathbb{C}}
\usepackage[left=3.5cm, right=3.5cm, top=3cm, bottom=3cm]{geometry}
\newcommand{\vect}[1]{\overrightarrow{\,\mathstrut#1\,}}
\renewcommand{\theenumi}{\textbf{\arabic{enumi}}}
\renewcommand{\labelenumi}{\textbf{\theenumi.}}
\renewcommand{\theenumii}{\textbf{\alph{enumii}}}
\renewcommand{\labelenumii}{\textbf{\theenumii.}}
\def\Oij{$\left(\text{O}~;~\vect{\imath},~\vect{\jmath}\right)$}
\def\Oijk{$\left(\text{O}~;~\vect{\imath},~\vect{\jmath},~\vect{k}\right)$}
\def\Ouv{$\left(\text{O}~;~\vect{u},~\vect{v}\right)$}
\usepackage{fancyhdr}
\usepackage[dvips]{hyperref}
\hypersetup{%
pdfauthor = {APMEP},
pdfsubject = {Baccalauréat spécialité},
pdftitle = {Amérique du Sud 26 septembre 2021},
allbordercolors = white,
pdfstartview=FitH} 
\usepackage[french]{babel}
\usepackage[np]{numprint}
\begin{document}
\setlength\parindent{0mm}
\rhead{\textbf{A. P{}. M. E. P{}.}}
\lhead{\small Baccalauréat spécialité}
\lfoot{\small{Amérique du Sud}}
\rfoot{\small{26 septembre 2022}}
\pagestyle{fancy}
\thispagestyle{empty}

\begin{center}{\Large\textbf{\decofourleft~Baccalauréat Amérique du Sud 26 septembre 2022~\decofourright\\[6pt] ÉPREUVE D'ENSEIGNEMENT DE SPÉCIALITÉ Jour 1}}
\end{center}

\vspace{0,25cm}

Le sujet propose 4 exercices

Le candidat choisit 3 exercices parmi les 4 exercices et \textbf{ne doit traiter que ces 3 exercices}

\medskip

Chaque exercice est noté sur 7 points (le total sera ramené sur 20 points).

Les traces de recherche, même incomplètes ou infructueuses, seront prises en compte.

\bigskip

\textbf{\textsc{Exercice 1 Probabilités} \hfill 7 points}

\medskip

\textbf{PARTIE A}

\medskip

Le système d'alarme d'une entreprise fonctionne de telle sorte que, si un danger se présente, l'alarme s'active avec une probabilité de $0,97$. 

La probabilité qu'un danger se présente est de $0,01$ et la probabilité que l'alarme s'active est de \np{0,01465}.

On note $A$ l'évènement \og  l'alarme s'active \fg{}  et $D$ l'événement \og  un danger se présente \fg . 

On note $\overline{M}$ l'évènement contraire d'un évènement $M$ et $P(M)$ la probabilité de l'évènement $M$.

\medskip

\begin{enumerate}
\item Représenter la situation par un arbre pondéré qui sera complété au fur et à mesure de l'exercice.
\item  
	\begin{enumerate}
		\item Calculer la probabilité qu'un danger se présente et que l'alarme s'active.
		\item En déduire la probabilité qu'un danger se présente sachant que l'alarme s'active.
Arrondir le résultat à $10^{-3}$.
	\end{enumerate}
\item  Montrer que la probabilité que l'alarme s'active sachant qu'aucun danger ne s'est présenté est $0,005$.
\item On considère qu'une alarme ne fonctionne pas normalement lorsqu'un danger se présente et qu'elle ne s'active pas ou bien lorsqu'aucun danger ne se présente et qu'elle s'active.

Montrer que la probabilité que l'alarme ne fonctionne pas normalement est inférieure à $0,01$.
\end{enumerate}

\medskip

\textbf{PARTIE B}

\medskip

Une usine fabrique en grande quantité des systèmes d'alarme. On prélève successivement et au hasard $5$ systèmes d'alarme dans la production de l'usine. Ce prélèvement est assimilé à un tirage avec remise.

On note $S$ l'évènement \og  l'alarme ne fonctionne pas normalement \fg{}  et on admet que

$P(S) = \np{0,00525}$.

On considère $X$ la variable aléatoire qui donne le nombre de systèmes d'alarme ne fonctionnant pas normalement parmi les $5$ systèmes d'alarme prélevés.

Les résultats seront arrondis à $10^{-4}$.

\medskip

\begin{enumerate}
\item Donner la loi de probabilité suivie par la variable aléatoire X et préciser ses paramètres.
\item Calculer la probabilité que, dans le lot prélevé, un seul système d'alarme ne fonctionne pas normalement.
\item Calculer la probabilité que, dans le lot prélevé, au moins un système d'alarme ne fonctionne pas normalement.
\end{enumerate}

\medskip

\textbf{PARTIE C}

\medskip

Soit $n$ un entier naturel non nul. On prélève successivement et au hasard $n$ systèmes d'alarme. Ce prélèvement est assimilé à un tirage avec remise.

Déterminer le plus petit entier $n$ tel que la probabilité d'avoir, dans le lot prélevé, au moins un système d'alarme qui ne fonctionne pas normalement soit supérieure à $0,07$.

\bigskip

\textbf{\textsc{Exercice 2 Suites} \hfill 7 points}

\medskip

Soit $\left(u_n\right)$ la suite définie par $u_0 = 4$ et, pour tout entier naturel $n$,\: $u_{n+1} = \dfrac15 u_n^2$.

\medskip

\begin{enumerate}
\item 
	\begin{enumerate}
		\item Calculer $u_1$ et $u_2$.
		\item Recopier et compléter la fonction ci-dessous écrite en langage Python.
Cette fonction est nommée \emph{suite\_u} et prend pour paramètre l'entier naturel $p$.

Elle renvoie la valeur du terme de rang $p$ de la suite $\left(u_n\right)$.

\begin{center}
\begin{tabular}{l}
\texttt{def suite\_u(p) :}\\
	\quad \texttt{u= \ldots}\\
	\quad \texttt{for i in range(1,\ldots) :}\\
	\quad \quad \texttt{u =\ldots}\\
	\quad \texttt{return u}\\
	\end{tabular}
\end{center}

	\end{enumerate}
\item 
	\begin{enumerate}
		\item Démontrer par récurrence que pour tout entier naturel $n$,\: $0 < u_n \leqslant 4$.
		\item Démontrer que la suite $\left(u_n\right)$ est décroissante.
		\item En déduire que la suite $\left(u_n\right)$ est convergente.
	\end{enumerate}
\item 
	\begin{enumerate}
		\item Justifier que la limite $\ell$ de la suite $\left(u_n\right)$ vérifie l'égalité $\ell = \dfrac15 \ell^2$.
		\item En déduire la valeur de $\ell$.
	\end{enumerate}
\item Pour tout entier naturel $n$, on pose $v_n = \ln \left(u_n\right)$ et $w_n = v_n - \ln (5)$.
	\begin{enumerate}
		\item Montrer que, pour tout entier naturel $n$,\: $v_{n+1} = 2v_n - \ln (5)$.
		\item Montrer que la suite $\left(w_n\right)$ est géométrique de raison 2.
		\item Pour tout entier naturel $n$, donner l'expression de $w_n$ en fonction de $n$ et montrer que $v_n = \ln \left(\dfrac45 \right) \times 2^n + \ln (5)$.
	\end{enumerate}
\item Calculer $\displaystyle\lim_{n \to + \infty} v_n$ et retrouver $\displaystyle\lim_{n \to + \infty} u_n$.
\end{enumerate}

\bigskip

\textbf{\textsc{Exercice 3 Fonctions, fonction logarithme} \hfill 7 points}

\medskip

Soit $g$ la fonction définie sur l'intervalle $]0~;~+\infty[$ par 

\[g(x) = 1+ x^2[1 - 2 \ln (x)].\]

La fonction $g$ est dérivable sur l'intervalle $]0~;~+\infty[$ et on note $g'$ sa fonction dérivée.

On appelle $\mathcal{C}$ la courbe représentative de la fonction $g$ dans un repère orthonormé du plan.

\medskip

\textbf{PARTIE A}

\medskip

\begin{enumerate}
\item Justifier que $g(\text{e})$ est strictement négatif.
\item Justifier que $\displaystyle\lim_{x \to + \infty} g(x) = - \infty$.
\item 
	\begin{enumerate}
		\item Montrer que, pour tout $x$ appartenant à l'intervalle $]0~;~+\infty[$,\: $g'(x) = -4x \ln (x)$.
		\item Étudier le sens de variation de la fonction $g$ sur l'intervalle $]0~;~+\infty[$ 
		\item Montrer que l'équation $g(x) = 0$ admet une unique solution, notée $\alpha$, sur l'intervalle $[1~;~+\infty[$.
		\item Donner un encadrement de $\alpha$ d'amplitude $10^{-2}$
	\end{enumerate}
\item Déduire de ce qui précède le signe de la fonction $g$ sur l'intervalle $[1~;~+\infty[$.
\end{enumerate}

\medskip

\textbf{PARTIE B}

\medskip

\begin{enumerate}
\item On admet que, pour tout $x$ appartenant à l'intervalle $[1~;~\alpha]$, $g''(x) = - 4[\ln (x) + 1]$.

Justifier que la fonction $g$ est concave sur l'intervalle $[1~;~\alpha]$.
\end{enumerate}

\begin{minipage}{0.48\linewidth}
\begin{enumerate}[resume]
\item Sur la figure ci-contre, A et B sont les points de la
courbe $\mathcal{C}$ d'abscisses respectives 1 et $\alpha$.
	\begin{enumerate}
		\item Déterminer l'équation réduite de la droite (AB).
		\item En déduire que pour tout réel $x$ appartenant à
l'intervalle $[1~;~\alpha]$,\: $g(x) \geqslant \dfrac{- 2}{\alpha - 1} x + \dfrac{2\alpha}{\alpha - 1}$.
	\end{enumerate}
\end{enumerate}
\end{minipage} \hfill
\begin{minipage}{0.48\linewidth}
\begin{center}
\psset{unit=1.3cm,arrowsize=2pt 3}
\begin{pspicture}(-0.2,-0.2)(2.5,3)
\psgrid[gridlabels=0pt,subgriddiv=1,gridwidth=0.15pt]
\psaxes[linewidth=1.25pt,labelFontSize=\scriptstyle]{->}(0,0)(0,0)(2.5,3)
\psplot[plotpoints=1000,linewidth=1.25pt,linecolor=red]{1}{1.9}{1 x ln 2 mul sub x dup mul mul 1 add}
\uput[ul](1,2){\red A}\uput[ur](1.9,0){\red B}\uput[d](1.9,0.05){$\alpha$}
\uput[ur](1.5,1.5){\red $\mathcal{C}$}
\end{pspicture}
\end{center}
\end{minipage}

\bigskip

\textbf{\textsc{Exercice 4 Géométrie dans l'espace} \hfill 7 points}

\medskip

Dans la figure ci-dessous, ABCDEFGH est un parallélépipède rectangle tel que 

AB $= 5$, AD $= 3$ et AE $= 2$.

L'espace est muni d'un repère orthonormé d'origine A dans lequel les points B, D et E ont respectivement pour coordonnées (5~;~0~;~0), (0~;~3~;~0) et (0~;~0~;~2).

\begin{center}
\psset{unit=1cm}
\begin{pspicture}(9,5)
%\psgrid
\psframe(0.2,0.2)(7.2,3)%ABFE
\psline(7.2,0.2)(8.4,1.6)(8.4,4.4)(7.2,3)%BCGF
\psline(8.4,4.4)(1.6,4.4)(0.2,3)%GHE
\psline[linestyle=dashed](0.2,0.2)(1.6,1.6)(8.4,1.6)%ADC
\psline[linestyle=dashed](1.6,1.6)(1.6,4.4)%DH
\pspolygon[linestyle=dotted,linewidth=1.5pt](0.2,0.2)(8.4,1.6)(3,4.4)%ACM
\uput[dl](0.2,0.2){A} \uput[dr](7,0.2){B} \uput[r](8.4,1.6){C}
\uput[ur](1.6,1.6){D} \uput[l](0.2,3){E} \uput[ul](7.2,3){F}
\uput[ur](8.4,4.4){G} \uput[ul](1.6,4.4){H} \uput[u](3,4.4){M}
\end{pspicture}
\end{center}

\begin{enumerate}
\item 
	\begin{enumerate}
		\item Donner, dans le repère considéré, les coordonnées des points H et G.
		\item Donner une représentation paramétrique de la droite (GH).
	\end{enumerate}	
\item Soit M un point du segment [GH] tel que $\vect{\text{HM}} =k\vect{\text{HG}}$ avec $k$ un nombre réel de l'intervalle [0~;~1].
	\begin{enumerate}
		\item Justifier que les coordonnées de M sont $(5k~;~3~;~2)$.
		\item En déduire que $\vect{\text{AM}} \cdot \vect{\text{CM}} = 25k^2  - 25k + 4$.
		\item Déterminer les valeurs de $k$ pour lesquelles AMC est un triangle rectangle en
M.
	\end{enumerate}
\end{enumerate}	

Dans toute la suite de l'exercice, on considère que le point M a pour coordonnées (1~;~3~;~2).

On admet que le triangle AMC est rectangle en M .

On rappelle que le volume d'un tétraèdre est donné par la formule $\dfrac13 \times\text{Aire de la base}  \times h$ où $h$ est la hauteur relative à la base.

\begin{enumerate}[resume]
\item On considère le point K de coordonnées (1~;~3~;~0).
	\begin{enumerate}
		\item Déterminer une équation cartésienne du plan (ACD).
		\item Justifier que le point K est le projeté orthogonal du point M sur le plan (ACD).
		\item En déduire le volume du tétraèdre MACD.
	\end{enumerate}	
\item On note P le projeté orthogonal du point D sur le plan (AMC).

Calculer la distance DP ; en donner une valeur arrondie à $10^{-1}$.
\end{enumerate}

\end{document}