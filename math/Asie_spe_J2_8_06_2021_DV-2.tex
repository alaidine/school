\documentclass[11pt]{article}
\usepackage[T1]{fontenc}
\usepackage[utf8]{inputenc}
\usepackage{fourier}
\usepackage[scaled=0.875]{helvet}
\renewcommand{\ttdefault}{lmtt}
\usepackage{makeidx}
\usepackage{amsmath,amssymb,MnSymbol}
\usepackage{fancybox}
\usepackage[normalem]{ulem}
\usepackage{pifont}
\usepackage{lscape}
\usepackage{multicol}
\usepackage{mathrsfs}
\usepackage{tabularx}
\usepackage{multirow}
\usepackage{enumitem}
\usepackage{textcomp} 
\newcommand{\euro}{\eurologo{}}
%Tapuscrit : Denis Vergès
%Relecture : 
\usepackage{pst-plot,pst-tree,pstricks,pst-node,pst-text}
\usepackage{pst-eucl}
\usepackage{pstricks-add}
\newcommand{\R}{\mathbb{R}}
\newcommand{\N}{\mathbb{N}}
\newcommand{\D}{\mathbb{D}}
\newcommand{\Z}{\mathbb{Z}}
\newcommand{\Q}{\mathbb{Q}}
\newcommand{\C}{\mathbb{C}}
\usepackage[left=3.5cm, right=3.5cm, top=3.1cm, bottom=3cm]{geometry}
\setlength\headheight{13.6pt}
\newcommand{\vect}[1]{\overrightarrow{\,\mathstrut#1\,}}
\renewcommand{\theenumi}{\textbf{\arabic{enumi}}}
\renewcommand{\labelenumi}{\textbf{\theenumi.}}
\renewcommand{\theenumii}{\textbf{\alph{enumii}}}
\renewcommand{\labelenumii}{\textbf{\theenumii.}}
\def\Oij{$\left(\text{O}~;~\vect{\imath},~\vect{\jmath}\right)$}
\def\Oijk{$\left(\text{O}~;~\vect{\imath},~\vect{\jmath},~\vect{k}\right)$}
\def\Ouv{$\left(\text{O}~;~\vect{u},~\vect{v}\right)$}
\newcommand{\e}{\text{e}}
\usepackage{fancyhdr}
\usepackage[dvips]{hyperref}
\hypersetup{%
pdfauthor = {APMEP réf : 21MATJ2JAI},
pdfsubject = {Baccalauréat spécialité},
pdftitle = {Polynésie juin 2021},
allbordercolors = white,
pdfstartview=FitH} 
\usepackage[french]{babel}
\usepackage[np]{numprint}
\begin{document}
\setlength\parindent{0mm}
\rhead{\textbf{A. P{}. M. E. P{}.}}
\lhead{\small Baccalauréat spécialité}
\lfoot{\small{Asie}}
\rfoot{\small{8 juin 2021}}
\pagestyle{fancy}
\thispagestyle{empty}

\begin{center}{\Large\textbf{\decofourleft~Baccalauréat Asie 8 juin 2021 Jour 2~\decofourright\\[6pt] ÉPREUVE D'ENSEIGNEMENT DE SPÉCIALITÉ}}
\end{center}

\vspace{0,25cm}

\textbf{Le candidat traite 4 exercices : les exercices 1, 2 et 3 communs à tous les candidats et un seul des deux exercices A ou B.}

\bigskip

\textbf{\textsc{Exercice 1} \hfill 5 points}

\textbf{Commun à tous les candidats}

\medskip

\textbf{Cet exercice est un questionnaire à choix multiples (QCM)}

\medskip


\emph{Pour chaque question, trois affirmations sont proposées, une seule de ces affirmations est exacte.\\
Le candidat recopiera sur sa copie le numéro de chaque question et la lettre de la réponse choisie pour celle-ci.\\
 AUCUNE JUSTIFICATION n'est demandée. Une réponse fausse ou l'absence de réponse n'enlève aucun point.}

\medskip

\begin{enumerate}
\item On considère la fonction $f$ définie sur $\R$ par 

\[f(x) = \left(x^2 - 2x - 1\right)\e^x.\]

\textbf{A.} La fonction dérivée de $f$ est la fonction définie par $f'(x) = (2x - 2)\e^x$.

\textbf{B.} La fonction $f$ est décroissante sur l'intervalle $]-\infty~;~2]$.

\textbf{C.} $\displaystyle\lim_{x \to - \infty} f(x) = 0$.
\item  On considère la fonction $f$ définie sur $\R$ par $f(x) = \dfrac{3}{5 + \e^x}$.

Sa courbe représentative dans un repère admet :

\textbf{A.} une seule asymptote horizontale;

\textbf{B.} une asymptote horizontale et une asymptote verticale; 

\textbf{C.} deux asymptotes horizontales.
\item  On donne ci-dessous la courbe $\mathcal{C}_{f''}$ représentant la fonction dérivée seconde $f''$  d'une fonction $f$ définie et deux fois dérivable sur l'intervalle $[-3,5~;~6]$.

\begin{center}
\psset{unit=1cm}
\begin{pspicture}(-4,-3)(7,5)
\psgrid[gridlabels=0pt,subgriddiv=1,gridwidth=0.1pt]
\psaxes[linewidth=1.25pt,labelFontSize=\scriptstyle]{->}(0,0)(-4,-3)(7,5)
\psplot[plotpoints=2000,linewidth=1.25pt,linecolor=blue]{-3.5}{6}{x 3 exp 0.1 mul  x dup mul 0.4 mul sub 1.1 x mul sub 3 add}
\uput[u](6.8,0){$x$} \uput[r](0,4.8){$y$}
\uput[u](1.5,4){Courbe de la fonction dérivée seconde $f''$}
\end{pspicture}
\end{center}
\textbf{A.} La fonction $f$ est convexe sur l'intervalle $[-3~;~3]$.

\textbf{B.}  La fonction $f$ admet trois points d'inflexion.

\textbf{C.} La fonction dérivée $f'$ de $f$ est décroissante sur l'intervalle [0~;~2].


\item  On considère la suite $\left(u_n\right)$ définie pour tout entier naturel $n$ par $u_n = n^2 - 17n + 20$. 

\textbf{A.} La suite $\left(u_n\right)$ est minorée.

\textbf{B.} La suite $\left(u_n\right)$ est décroissante.

\textbf{C.} L'un des termes de la suite $\left(u_n\right)$ est égal à \np{2021}.
\item On considère la suite $\left(u_n\right)$ définie par $u_0 = 2$ et, pour tout entier naturel $n$,\,

 $u_{n+1} = 0,75u_n +5$.

On considère la fonction \og seuil \fg{} suivante écrite en Python :

\begin{center}
\begin{tabular}{|l|}\hline
def seuil() :\\
\quad u = 2\\
\quad n = 0\\
\quad while u < 45 :\\
\qquad u = 0,75*u + 5\\
\qquad n = n+1\\
\quad return n\\ \hline
\end{tabular}
\end{center}

Cette fonction renvoie :

\textbf{A.} la plus petite valeur de $n$ telle que $u_n \geqslant  45$ ;
 
\textbf{B.} la plus petite valeur de $n$ telle que $u_n  < 45$ ;
 
\textbf{C.} la plus grande valeur de $n$ telle que $u_n \geqslant 45$.
\end{enumerate}

\bigskip

\textbf{\textsc{Exercice 2} \hfill 5 points}

\textbf{Commun à tous les candidats}

\medskip

On considère un pavé droit ABCDEFGH tel que AB = AD = 1 et AE = 2, représenté ci- dessous.

Le point I est le milieu du segment [AE]. Le point K est le milieu du segment [DC]. Le point L
est défini par: $\vect{\text{DL}} = \dfrac{3}{2}\vect{\text{AI}}$. N est le projeté orthogonal du point D sur le plan (AKL).

\begin{center}
\psset{unit=1cm,arrowsize=2pt 4}
\begin{pspicture}(6,10)
\pspolygon(0.5,0.5)(4,0)(4,7.2)(0.5,7.7)%BCGF
\uput[d](0.5,0.5){B} \uput[d](4,0){C} \uput[ul](4,7.2){G} \uput[ul](0.5,7.7){F} 
\psline(4,0)(5,1.4)(5,8.6)(4,7.2)%CDHG
\uput[r](5,1.4){D} \uput[ur](5,8.6){H} \uput[d](1.5,1.9){A} \uput[dr](4.5,0.7){K}\uput[l](1.5,5.5){I}
\uput[r](5,7.3){L}
\psline(5,8.6)(1.5,9.1)(0.5,7.7)%HEF
\uput[u](1.5,9.1){E}
\psline[linestyle=dashed]{->}(1.5,1.9)(1.5,5.5)%AI
\psline[linestyle=dashed]{->}(1.5,1.9)(0.5,0.5)%AB
\psline[linestyle=dashed]{->}(1.5,1.9)(5,1.4)%AD
\psline[linestyle=dashed](4.5,0.7)(1.5,1.9)(5,7.3)%KAL
\psline[linestyle=dashed](1.5,5.5)(1.5,9.1)
\psline(4.5,0.7)(5,7.3)
\end{pspicture}
\end{center}

\bigskip

On se place dans le repère orthonormé $\left(\text{A}~;~\vect{\text{AB}},~\vect{\text{AD}},~\vect{\text{AI}}\right)$. 

On admet que le point L a pour coordonnées $\left(0~;~1~;~\dfrac{3}{2}\right)$.

\medskip

\begin{enumerate}
\item Déterminer les coordonnées des vecteurs $\vect{\text{AK}}$ et $\vect{\text{AL}}$.
\item  
	\begin{enumerate}
		\item Démontrer que le vecteur $\vect{n}$ de coordonnées $(6~;~-3~;~2)$ est un vecteur normal au plan (AKL).
		\item En déduire une équation cartésienne du plan (AKL).
		\item Déterminer un système d'équations paramétriques de la droite $\Delta$ passant par D et
perpendiculaire au plan (AKL).
		\item En déduire que le point N de coordonnées $\left(\dfrac{18}{49}~;~\dfrac{40}{49}~;~\dfrac{6}{49}\right)$ est le projeté orthogonal du point D sur le plan (AKL).
	\end{enumerate}
\end{enumerate}

On rappelle que le volume $\mathcal{V}$ d'un tétraèdre est donné par la formule : 

\[\mathcal{V} = \dfrac{1}{3}\times  (\text{aire de la base}) \times \text{hauteur}.\]

\begin{enumerate}[resume]
\item 
	\begin{enumerate}
		\item Calculer le volume du tétraèdre ADKL en utilisant le triangle ADK comme base. 
		\item Calculer la distance du point D au plan (AKL).
		\item Déduire des questions précédentes l'aire du triangle AKL.
	\end{enumerate}
\end{enumerate}

\bigskip

\textbf{\textsc{Exercice 3} \hfill 5 points}

\textbf{Commun à tous les candidats}

\medskip
Une société de jeu en ligne propose une nouvelle application pour smartphone nommée \og Tickets coeurs! \fg.

Chaque participant génère sur son smartphone un ticket comportant une grille de taille $3 \times 3$ sur laquelle sont placés trois cœurs répartis au hasard, comme par exemple ci-dessous.

\begin{center}
\psset{unit=1cm}
\begin{pspicture}(3,3)
\multido{\n=0+1}{4}{\psline(\n,0)(\n,3)}
\multido{\n=0+1}{4}{\psline(0,\n)(3,\n)}
\rput(1.5,2.5){$\heartsuit$}\rput(0.5,1.5){$\heartsuit$}\rput(2.5,0.5){$\heartsuit$}
\end{pspicture}
\end{center}

Le ticket est gagnant si les trois cœurs sont positionnés côte à côte sur une même ligne, sur une
même colonne ou sur une même diagonale.

\medskip

\begin{enumerate}
\item Justifier qu'il y a exactement $84$ façons différentes de positionner les trois cœurs sur une grille.
\item Montrer que la probabilité qu'un ticket soit gagnant est égale à $\dfrac{2}{21}$.
\item Lorsqu'un joueur génère un ticket, la société prélève 1~\euro{} sur son compte en banque. Si le ticket est gagnant, la société verse alors au joueur $5$~\euro. Le jeu est-il favorable au joueur?
\item Un joueur décide de générer $20$ tickets sur cette application. On suppose que les générations des tickets sont indépendantes entre elles.
	\begin{enumerate}
		\item Donner la loi de probabilité de la variable aléatoire $X$ qui compte le nombre de tickets gagnants parmi les $20$ tickets générés.
		\item Calculer la probabilité, arrondie à $10^{-3}$, de l'évènement $(X  =  5)$.
		\item Calculer la probabilité, arrondie à $10^{-3}$, de l'évènement $(X \geqslant 1)$ et interpréter le résultat dans le contexte de l'exercice.
	\end{enumerate}
\end{enumerate}

\bigskip

\textbf{EXERCICE au choix du candidat \hfill5 points}

\medskip

\textbf{Le candidat doit traiter UN SEUL des deux exercices A ou B}

\medskip

\textbf{Il indique sur sa copie l'exercice choisi : exercice A ou exercice B}

\medskip

\textbf{EXERCICE -- A}

\medskip

\begin{tabular}{|l|}\hline
\textbf{Principaux domaines abordés}\\
-- Suites\\
-- Équations différentielles\\ \hline
\end{tabular}

\medskip

Dans cet exercice, on s'intéresse à la croissance du bambou Moso de taille maximale 20 mètres. 

Le modèle de croissance de Ludwig von Bertalanffy suppose que la vitesse de croissance pour un tel bambou est proportionnelle à l'écart entre sa taille et la taille maximale.

\bigskip

\textbf{Partie I : modèle discret}

\medskip

Dans cette partie, on observe un bambou de taille initiale $1$~mètre.

Pour tout entier naturel $n$, on note $u_n$ la taille, en mètre, du bambou $n$ jours après le début de l'observation. On a ainsi $u_0 = 1$.

Le modèle de von Bertalanffy pour la croissance du bambou entre deux jours consécutifs se traduit par l'égalité :

\[u_{n+1} = u_n + 0,05\left(20 - u_n\right)\,  \text{pour tout entier naturel}\,  n.\]

\medskip

\begin{enumerate}
\item Vérifier que $u_1 = 1,95$.
\item 
	\begin{enumerate}
		\item Montrer que pour tout entier naturel $n$,\,  $u_{n+1} = 0,95u_n + 1$.
		\item On pose pour tout entier naturel $n$,\,  $v_n = 20 - u_n$. 
		
Démontrer que la suite $\left(v_n\right)$ est une suite géométrique dont on précisera le terme initial $v_0$ et la raison.
		\item En déduire que, pour tout entier naturel $n$,\,  $u_n = 20 - 19 \times 0,95^n$.
	\end{enumerate}
\item Déterminer la limite de la suite $\left(u_n\right)$.
\end{enumerate}

\bigskip

\textbf{Partie II : modèle continu}

\medskip

Dans cette partie, on souhaite modéliser la taille du même bambou Moso par une fonction donnant sa taille, en mètre, en fonction du temps $t$ exprimé en jour. 

D'après le modèle de von Bertalanffy, cette fonction est solution de l'équation différentielle

\[(E) \qquad y' = 0,05(20 - y)\]

où $y$ désigne une fonction de la variable $t$, définie et dérivable sur $[0~;~+\infty[$ et $y'$ désigne sa fonction dérivée.

Soit la fonction $L$ définie sur l'intervalle $[0~;~+\infty[$ par 

\[L(t) = 20 - 19\e^{-0,05t}.\]

\smallskip

\begin{enumerate}
\item Vérifier que la fonction $L$ est une solution de $(E)$ et qu'on a également $L(0) = 1$.
\item On prend cette fonction $L$ comme modèle et on admet que, si on note $L'$ sa fonction dérivée,
$L'(t)$ représente la vitesse de croissance du bambou à l'instant $t$.
	\begin{enumerate}
		\item Comparer $L'(0)$ et $L'(5)$.
		\item Calculer la limite de la fonction dérivée $L'$ en $+\infty$. 
		
Ce résultat est-il en cohérence avec la description du modèle de croissance exposé au début de l'exercice ?
	\end{enumerate}
\end{enumerate}

\bigskip

\textbf{EXERCICE -- B}

\medskip

\begin{tabular}{|l|}\hline
\textbf{Principaux domaines abordés}\\
-- Suites, étude de fonction\\
-- Fonction logarithme\\ \hline
\end{tabular}

\medskip

Soit la fonction $f$ définie sur l'intervalle $]1~;~ +\infty[$ par 

\[f(x) = x - \ln (x - 1).\]

On considère la suite $\left(u_n\right)$ de terme initial $u_0 = 10$ et telle que $u_{n+1} = f\left(u_n\right)$ pour tout entier naturel $n$.

\bigskip

\textbf{Partie I :}

\medskip

La feuille de calcul ci-dessous a permis d'obtenir des valeurs approchées des premiers termes de la suite $\left(u_n\right)$.

\begin{center}
\begin{tabularx}{0.6\linewidth}{|c|*{2}{>{\centering \arraybackslash}X|}}\hline
&A &B\\ \hline
1 &$n$&$u_n$\\ \hline
2 &0&10\\ \hline
3& 1&\np{7,80277542}\\ \hline
4& 2&\np{5,88544474}\\ \hline
5& 3&\np{4,29918442}\\ \hline
6& 4&\np{3,10550913}\\ \hline
7& 5&\np{2,36095182}\\ \hline
8& 6&\np{2,0527675}\\ \hline
9& 7&\np{2,00134509}\\ \hline
10& 8&\np{2,0000009}\\ \hline
\end{tabularx}
\end{center}

\medskip

\begin{enumerate}
\item Quelle formule a été saisie dans la cellule B3 pour permettre le calcul des valeurs approchées de $\left(u_n\right)$ par recopie vers le bas ?
\item À l'aide de ces valeurs, conjecturer le sens de variation et la limite de la suite $\left(u_n\right)$.
\end{enumerate}

\bigskip

\textbf{Partie II :}

\medskip

On rappelle que la fonction $f$ est définie sur l'intervalle $]1~;~ +\infty[$ par 

\[f(x) = x - \ln (x - 1).\]

\medskip

\begin{enumerate}
\item Calculer $\displaystyle\lim_{x \to 1} f(x)$. On admettra que $\displaystyle\lim_{x \to + \infty} f(x) = + \infty$.
\item  
	\begin{enumerate}
		\item Soit $f'$ la fonction dérivée de $f$. Montrer que pour tout $x \in ]1~;~ +\infty[$,\,  $f'(x) = \dfrac{x - 2}{x - 1}$.
		\item En déduire le tableau des variations de $f$ sur l'intervalle $]1~;~ +\infty[$, complété par les limites.
		\item Justifier que pour tout $x
\geqslant  2$,\,  $f(x) \geqslant  2$.
	\end{enumerate}
\end{enumerate}

\bigskip

\textbf{Partie III :}

\medskip

\begin{enumerate}
\item En utilisant les résultats de la partie II, démontrer par récurrence que $u_n \geqslant  2$ pour tout entier naturel $n$.
\item Montrer que la suite $\left(u_n\right)$ est décroissante.
\item En déduire que la suite $\left(u_n\right)$ est convergente. On note $\ell$ sa limite.
\item On admet que $\ell$ vérifie $f(\ell) = \ell$. Donner la valeur de $\ell$.
\end{enumerate}
\end{document}